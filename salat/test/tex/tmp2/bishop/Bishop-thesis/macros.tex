% This is in a separate file so that it can be included by the figures when psfragging etc, so that the notation in the figures matches the main text
%\usepackage[l2tabu, orthodox, abort]{nag}               % check for common latex errors
%\ChNameVar{\Large\sffamily}\ChTitleVar{\Large\sffamily} % to avoid nag warnings
\usepackage[error, all]{onlyamsmath}                    % check for a different class of errors
%\usepackage{braket}                                     % meh. I like my own version of Dirac notation
%\usepackage[morose]{cmdtrack}                           % for checking which of these commands are actually used!

\newcommand*{\figthanks}[1]{ {(Figure used with permission from~\cite{#1}.
    See \nameref{ch:copyperm}.)}}

\newcommand{\vx}{{\vphantom{y}x}}

\newcommand*{\ie}{\textit{i.e.}}
\newcommand*{\eg}{e.g.}
\newcommand*{\dc}{d.c.}
\newcommand*{\etal}{\textit{et al.}}
\newcommand*{\hc}{\text{h.c.}}

%\newcommand*{\SSS}{\textsw{S}}
%\newcommand*{\RRR}{\textsw{R}}
\newcommand*{\SSS}{\ensuremath{\mathcal{S}}} % or \CMcal{}
\newcommand*{\RRR}{\ensuremath{\mathcal{R}}}

\newcommand*{\abs}[1]{\left\lvert #1\right\rvert}
\newcommand*{\bra}[1]{\left\langle #1 \right\rvert}
\newcommand*{\ket}[1]{\left\lvert #1 \right\rangle}
\newcommand*{\bket}[2]{\left\langle \, #1 \,|\, #2 \, \right\rangle}
\newcommand*{\boket}[3]{\langle\, #1 \,|\, #2 \,|\, #3 \,\rangle}
\newcommand*{\com}[2]{\left[#1,#2\right]}

% hmm. should do this right but I can't be bothered. Pick the one that works with the style in use...
\newcommand{\onlinecite}{\citenum}
%\newcommand{\onlinecite}{\cite}
%\newcommand{\onlinecite}{\citealp}

% Easily-modified referencing style commands.
% Now superceded by cleveref? (cleveref had bugs when I started, but now seems to be fully working)
\newcommand*{\eref}[1]{(\ref{#1})} % NB Not APS style
\newcommand*{\Eref}[1]{Equation~(\ref{#1})}
\newcommand*{\aref}[1]{\hyperref[#1]{appendix~\ref*{#1}}}
\newcommand*{\Aref}[1]{\hyperref[#1]{Appendix~\ref*{#1}}}
\newcommand*{\chref}[1]{\hyperref[#1]{chapter~\ref*{#1}}}
\newcommand*{\Chref}[1]{\hyperref[#1]{Chapter~\ref*{#1}}}
\newcommand*{\sref}[1]{\hyperref[#1]{section~\ref*{#1}}}
\newcommand*{\Sref}[1]{\hyperref[#1]{Section~\ref*{#1}}}
\newcommand*{\fref}[1]{\hyperref[#1]{figure~\ref*{#1}}}
\newcommand*{\Fref}[1]{\hyperref[#1]{Figure~\ref*{#1}}}
\newcommand*{\tref}[1]{\hyperref[#1]{table~\ref*{#1}}}
\newcommand*{\Tref}[1]{\hyperref[#1]{Table~\ref*{#1}}}
%%%%%%

\newcommand*{\software}[1]{\textsf{#1}}
\newcommand*{\mma}{\software{Mathematica}}
\newcommand*{\umfpack}{\software{UMFPACK}}% Damn, no latin modern sans small caps

\newcommand{\stree}{\mathsf{T}}
\newcommand{\tPhi}{\tilde{\Phi}}

\DeclareMathOperator{\tr}{tr}
\newcommand*{\DD}{\mathcal{D}}
\newcommand*{\Tt}{T'}
\newcommand*{\Dp}{\Delta}
\newcommand*{\sigt}{\tilde{\sigma}}
\newcommand*{\dat}{\downarrow}
\newcommand*{\uat}{\uparrow}
%\newcommand*{\be}{\begin{equation}} % but this doesnt work if equation is actually gather....
%\newcommand*{\ee}{\end{equation}}
\def\be#1\ee{%
%\begin{gridenv}%GGG
\begin{gather}#1\end{gather}%
%\end{gridenv}%GGG
} % ...ouch

\newcommand*{\amp}{V_0} %% XXX FIXME pdfx

\newcommand*{\EC}{E_{\text{C}}}
\newcommand*{\EJ}{E_{\text{J}}}
\newcommand*{\ngate}{n_{\text{g}}}
\newcommand*{\op}[1]{{#1}} % maybe \hat{} maybe not??
\newcommand*{\ph}{\op{\phi}}
\newcommand*{\vph}{\op{\varphi}}
\newcommand*{\nh}{\op{n}}
\newcommand*{\omr}{\omega_{\text{r}}}
\newcommand*{\omd}{\omega_{\text{d}}}
\newcommand*{\Hd}{\op{H}_{\text{d}}}
\newcommand*{\aop}{\op{a}}

\newcommand*{\Cc}[1]{\ensuremath{C_{#1}}} % Was \C
\newcommand*{\Ct}{\ensuremath{C_\text{t}}}
\newcommand*{\ec}[1]{\ensuremath{E_{\text{C}#1}}}
\newcommand*{\ect}{\ensuremath{E_\text{Ct}}}
\newcommand*{\ej}[1]{\ensuremath{E_{\text{J}#1}}}
\newcommand*{\nng}[1]{\ensuremath{n_{\text{g}#1}}}
\newcommand*{\eC}[1]{\ensuremath{\xi_{#1}}}
\newcommand*{\eCt}{\ensuremath{\xi_\text{t}}}
\newcommand*{\nn}[1]{\ensuremath{n_{#1}}}
\newcommand*{\pho}[1]{\ensuremath{\varphi_{#1}}}
\newcommand*{\vg}[1]{\ensuremath{V'_{#1}}}

\newcommand*{\dg}{^{\dag}}
\newcommand*{\pdg}{^{\vphantom{\dag}}}

\newcounter{fig}
\newenvironment{romenu}{\begin{list}{(\roman{fig})}{\usecounter{fig} \setlength{\labelwidth}{1.5cm}}}{\end{list}}

\newcommand{\MM}{\ensuremath{\langle M\rangle}}

\newcommand*{\pp}{\partial}
\newcommand*{\vc}[1]{\mathbf{#1}}
\DeclareMathOperator{\sgn}{sgn}

\newcommand*{\rmi}{\mathrm{i}}
\newcommand*{\rme}{\mathrm{e}}
\newcommand*{\rmd}{\mathrm{d}}

\newcommand*{\pd}{\partial}

% Define a semibold maths version, for use in the captions
% No semibold MnSymbol, so use bold or normal?
\makeatletter
\DeclareMathVersion{semiboldproportional}
\SetSymbolFont{operators}{semiboldproportional}{T1} {\Mn@Math@Family} {sb}{n}
\SetSymbolFont{letters}  {semiboldproportional}{OML}{MinionPro-TOsF}  {sb}{\Mn@Math@LetterShape}
\SetMathAlphabet\mathit  {semiboldproportional}{T1} {\Mn@Math@Family} {sb}{it}

\DeclareMathVersion{semiboldtabular}
\SetSymbolFont{operators}{semiboldtabular}     {T1} {\Mn@Math@TFamily}{sb}{n}
\SetSymbolFont{letters}  {semiboldtabular}     {OML}{MinionPro-TOsF}  {sb}{\Mn@Math@LetterShape}
\SetMathAlphabet\mathit  {semiboldtabular}     {T1} {\Mn@Math@TFamily}{sb}{it}
\def\semiboldmath{\@nomath\semiboldmath
    \mathweight{semibold}}
\makeatother

\newcommand{\captitle}[1]{\textbf{\semiboldmath #1}}
% Use extra bold for caption letters, for easier spotting...
\newcommand{\capl}[1]{{\fontseries{eb}\selectfont{#1}}}

\newenvironment{subal}[1]{%
      \subequations #1%
%      \gridenv%GGG
      \align
    }{%
      \endalign
%      \endgridenv%GGG
      \endsubequations
      \ignorespacesafterend
    }
