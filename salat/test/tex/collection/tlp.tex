\documentclass[oneside,openany,12pt]{book}
%
% For tracking changes
\newcommand{\version}{0.23}
%
% For English and German hyphenation patterns
\usepackage[austrian,english]{babel}
%
%
% Tells it to use type-1 font encoding:
\usepackage[T1]{fontenc}
%
%
% Font choice:
\usepackage{fouriernc}
%
%
% Sets page dimensions and orientation:
\usepackage[letterpaper,landscape,top=.5in,bottom=.5in,left=.5in,right=.5in]{geometry}
%
%
% For multi-page tables
\usepackage{longtable}
%
%
% For the columns of Russell's intro
\usepackage{multicol}
\usepackage{enumitem}
%
%
% For the basic column-based set-up of the book:
\usepackage{parcolumns}
%
%
% For greater control over lines in truth-tables:
\usepackage{hhline}
%
%
% Used for s p a c e d German emphasis:
\usepackage{soul}
%
%
% For greater math options:
\usepackage{amsmath}
%
%
% Improves typography in cramped quarters:
\usepackage[kerning=true]{microtype}
%
%
% Creative commons icons:
\usepackage{ccicons}
%
% 
% Page number: move down a tad
\renewcommand{\thepage}{\protect\raisebox{-1.5em}{\arabic{page}}}
%
% For graphics:
\usepackage{tikz}
\usetikzlibrary{arrows}
%
\usepackage{hyperref}
\hypersetup{
   pdfauthor={Ludwig Wittgenstein},%
   pdftitle={Tractatus Logico-Philosophicus},%
   pdfsubject={Philosophy,Logic},%
   pdfcreator={Kevin C. Klement},
   colorlinks,%
   linkcolor=blue
}
%
\title{Tractatus Logico-Philosophicus}
\author{Ludwig Wittgenstein}
\date{1922}
%------------------------------
% New commands for basic set-up
%------------------------------
% Section (proposition) numbers:
\newcommand{\pn}[1]{\zeroout\colchunk[1]{\hypertarget{prop#1}{\textbf{#1}}}}
% Paragraph indentation:
\newcommand{\pbk}{\hspace{1.5em}}
% Unindents if needed
\newcommand{\negpbk}{\hspace*{-1.6em}}
% Blanks out paragraph number for second+ paragraphs in a section
\newcommand{\pnskip}{\zeroout\colchunk[1]{}}
% German column entry: 
\newcommand{\ger}[1]{\zeroout\selectlanguage{austrian}\colchunk[2]{#1}}
% Ogden column entry:
\newcommand{\ogd}[1]{\zeroout\selectlanguage{english}\colchunk[3]{#1}}
% Pears-McGuinness column entry:
\newcommand{\pmc}[1]{\zeroout\colchunk[4]{#1}\colplacechunks}
%
% German stuff I may want to change
%
\newcommand{\germph}[1]{\so{#1}} % German emphasis
\newcommand{\gdql}{\quotedblbase} % left double quote
\newcommand{\gdqr}{``} % right double quote
\newcommand{\gsql}{\quotesinglbase} % left single quote
\newcommand{\gsqr}{`} % right single quote
%
% Logical/mathematical symbolism
%
% LW's N operator:
\newcommand{\nop}{\mathop{\mathrm{N}}}
% BR/LW negation:
\newcommand{\rnot}{\mathord{\sim}}
% BR/LW material implication, without and with dots:
\newcommand{\rimplies}{\supset}
\newcommand{\drimpliesd}{\mathrel{\mathord{.}\mathord{\supset}\mathord{.}}}
\newcommand{\drimplies}{\mathrel{\mathord{.}\mathord{\supset}}}
\newcommand{\ddrimpliesdd}{\mathrel{\mathord{:}\mathord{\supset}\mathord{:}}}
\newcommand{\rimpliesd}{\mathrel{\mathord{\supset}\mathord{.}}}
% BR/LW universal quantifier, with dots:
\newcommand{\ralld}[1]{\mathop{(#1)\mathord{.}}}
\newcommand{\ralldd}[1]{\mathop{(#1)\mathord{:}}}
% BR/LW existential quantifier, with and without dots:
\newcommand{\rsome}[1]{\mathop{(\mathord{\exists} #1)}}
\newcommand{\rsomed}[1]{\mathop{(\mathord{\exists} #1)\mathord{.}}}
\newcommand{\rsomedd}[1]{\mathop{(\mathord{\exists} #1)\mathord{:}}}
% Conjunction dot:
\newcommand{\rand}{\mathrel{.}}
% Disjunction with dots: 
\newcommand{\dlord}{\mathrel{\mathord{.}\mathord{\lor}\mathord{.}}}
\newcommand{\ddlordd}{\mathrel{\mathord{:}\mathord{\lor}\mathord{:}}}
% Sheffer stroke, and with dots:
\newcommand{\sheffer}{\mathrel{\vert}}
\newcommand{\dshefferd}{\mathrel{\mathord{.}\mathord{\vert}\mathord{.}}}
% Generic operator sign O'
\newcommand{\Op}{\mathop{\text{O'}}}
% Omega operators
\newcommand{\omop}[1][]{\mathop{\mathord{\Omega}^{#1}\mathord{\text{'}}}}
\newcommand{\omopparen}[2][]{\mathop{(\mathord{\Omega}^{#1})^{#2}\mathord{\text{'}}}}
% Other repeated math
\newcommand{\possibilities}{$\mathrm{K}_n = \displaystyle\sum_{\nu = 0}^{n}  \dbinom{n}{\nu}$} %4.27
\newcommand{\morepossibilities}{$\displaystyle\sum_{\kappa = 0}^{\mathrm{K}_n} \dbinom{\mathrm{K}_n}{\kappa} = \mathrm{L}_n$} %4.43

% Formatting of the index
%
% Sets up indents for index
\setitemize[1]{label={},leftmargin=\parindent,itemindent=-1\parindent,nolistsep}
\setitemize[2]{label={},leftmargin=\parindent,itemindent=-1.5\parindent,nolistsep}
\setitemize[3]{label={},leftmargin=\parindent,itemindent=0pt,nolistsep}
%
% Codes for index entries
\newcommand{\indexentry}[1]{\item #1}
\newcommand{\indexsubentry}[1]{\begin{itemize} \item #1 \end{itemize}}
\newcommand{\indexsubsubentry}[1]{\begin{itemize} \item \begin{itemize} \item #1 \end{itemize} \end{itemize}}
\newcommand{\indexref}[1]{\hyperlink{prop#1}{#1}}
\newcommand{\indexgap}{\bigskip}
%
% My additions
%
\newcommand{\kckaddition}[1]{\textcolor{red}{#1}}
%
%
% Figures and drawings to repeat
%
% The cube from 5.423
\newcommand{\thecube}{\begin{center} \begin{tikzpicture} [label distance=-4pt]
    \path node (aone) at (0,0) [label={[label distance=-9pt]below left:$a$}] {}
          node (atwo) at (0,2) [label={[label distance=-6pt]left:$a$}] {}
          node (athree) at (2,2) [label=above:$a\phantom{.}$] {}
          node (afour) at (2,0) [label=below:$a$] {}
          node (bone) at (1,1) [label=below:$\phantom{.}b$] {}
          node (btwo) at (1,3) [label=above:$b$] {}
          node (bthree) at (3,3) [label=above:$b$] {}
          node (bfour) at (3,1) [label={[label distance=-9pt]below right:$b$}] {};
    \draw [thick] (0,0) rectangle (2,2);
    \draw [thick] (1,1) rectangle (3,3);
    \draw [thick] (0,0) -- (1,1);
    \draw [thick] (0,2) -- (1,3);
    \draw [thick] (2,2) -- (3,3);
    \draw [thick] (2,0) -- (3,1);
 \end{tikzpicture} \end{center}}
%
% Sets German name for the eye
\newcommand{\eyename}{\textsf{Auge}}
%
% The eye from 5.6331
\newcommand{\theeye}{\begin{center}
  \begin{tikzpicture}
      \path node (eye) at (0,0) [shape=circle,draw,thick,inner sep=0.05cm,label={[label distance=-3pt]left:{\eyename \thinspace---}}] {};
      \draw [thick] (-.04,0.06) arc (140:90:3cm) arc (90:-90:1cm) arc (-90:-129:3.64cm);%   
  \end{tikzpicture}   
\end{center}}
%
% Several AB-notation maps from 6.1203
\newcommand{\abfigureonegerman}{\begin{center}
\begin{tikzpicture}[thick,line join=round]
  \node (W1) at (0,0) {\vphantom{F$pq$}W};
  \node (p) at (.5,0) {\vphantom{WF$q$}$p$};
  \node (F1) at (.95,0) {\vphantom{W$qp$}F};
  \node (W2) at (4,0) {\vphantom{F$qp$}W};
  \node (q) at (4.5,0) {\vphantom{WF$p$}$q$};
  \node (F2) at (4.95,0) {\vphantom{W$pq$}F};
  \draw (W1.110) arc (160:90:1) -- ++(2.5,0) arc 
          (-90:0:.2) arc (180:270:.2) -- ++(0.3,0) arc (90:20:1);
  \draw (W1.90) arc (145:90:.7) -- ++(1,0) arc 
          (-90:0:.2) arc (180:270:.2) -- ++ (1.427,0) arc (90:35:.7);
  \draw (F2.270) arc (-20:-90:1) -- ++(-0.3,0) arc (90:180:.2) arc (0:90:.2)
        -- ++(-1.5,0) arc (270:200:1);
  \draw (F1.270) arc (-145:-90:.7) -- ++(1,0) arc (90:0:.2) arc (180:90:.2)
    -- ++(.5,0) arc (-90:-35:.7);
\end{tikzpicture}
\end{center}}
%
\newcommand{\abfigureoneenglish}{\begin{center}
\begin{tikzpicture}[thick,line join=round]
  \node (W1) at (0,0) {\vphantom{F$pq$}T};
  \node (p) at (.5,0) {\vphantom{TF$q$}$p$};
  \node (F1) at (.95,0) {\vphantom{T$qp$}F};
  \node (W2) at (4,0) {\vphantom{F$qp$}T};
  \node (q) at (4.5,0) {\vphantom{TF$p$}$q$};
  \node (F2) at (4.95,0) {\vphantom{T$pq$}F};
  \draw (W1.110) arc (160:90:1) -- ++(2.5,0) arc 
          (-90:0:.2) arc (180:270:.2) -- ++(0.3,0) arc (90:20:1);
  \draw (W1.90) arc (145:90:.7) -- ++(1,0) arc 
          (-90:0:.2) arc (180:270:.2) -- ++ (1.427,0) arc (90:35:.7);
  \draw (F2.270) arc (-20:-90:1) -- ++(-0.3,0) arc (90:180:.2) arc (0:90:.2)
        -- ++(-1.5,0) arc (270:200:1);
  \draw (F1.270) arc (-145:-90:.7) -- ++(1,0) arc (90:0:.2) arc (180:90:.2)
    -- ++(.5,0) arc (-90:-35:.7);
\end{tikzpicture}
\end{center}}
%
%
\newcommand{\abfigureoneenglishpmc}{\begin{center}
\begin{tikzpicture}[thick,line join=round]
  \node (W1) at (0,0) {\vphantom{F$pq$}T};
  \node (p) at (.5,0) {\vphantom{TF$q$}$p$};
  \node (F1) at (.95,0) {\vphantom{T$qp$}F};
  \node (W2) at (4,0) {\vphantom{F$qp$}T};
  \node (q) at (4.5,0) {\vphantom{TF$p$}$q$};
  \node (F2) at (4.95,0) {\vphantom{T$pq$}F,};
  \draw (W1.110) arc (160:90:1) -- ++(2.5,0) arc 
          (-90:0:.2) arc (180:270:.2) -- ++(0.3,0) arc (90:20:1);
  \draw (W1.90) arc (145:90:.7) -- ++(1,0) arc 
          (-90:0:.2) arc (180:270:.2) -- ++ (1.427,0) arc (90:35:.7);
  \draw (F2.270) arc (-20:-90:1) -- ++(-0.3,0) arc (90:180:.2) arc (0:90:.2)
        -- ++(-1.5,0) arc (270:200:1);
  \draw (F1.270) arc (-145:-90:.7) -- ++(1,0) arc (90:0:.2) arc (180:90:.2)
    -- ++(.5,0) arc (-90:-35:.7);
\end{tikzpicture}
\end{center}}
%
\newcommand{\abfiguretwogerman}{\begin{center}
\begin{tikzpicture}[thick,line join=round]
  \node (W1) at (0,0) {\vphantom{F$pq$}W};
  \node (p) at (.5,0) {\vphantom{WF$q$}$p$};
  \node (F1) at (.95,0) {\vphantom{W$qp$}F};
  \node (W2) at (4,0) {\vphantom{F$qp$}W};
  \node (q) at (4.5,0) {\vphantom{WF$p$}$q$};
  \node (F2) at (4.95,0) {\vphantom{W$pq$}F};
  \draw (W1.110) arc (160:90:1) -- ++(1.2,0) arc (-90:0:.2) 
    -- ++(.3,.5) node [coordinate,label={[label distance=-2pt]90:~~F}] {} ++(-.3,-.5) 
    arc (180:270:.2) -- ++(1.6,0) arc (90:20:1);
  \draw (W1.90) arc (145:90:.7) -- ++(.8,0) arc (-90:0:.2) 
    -- ++(1.2,-2.5) ++(-1.2,2.5)
    arc (180:270:.2) -- ++ (1.627,0) arc (90:35:.7);
  \draw (F2.270) arc (-20:-90:1) -- ++(-0.4,0) arc (90:180:.2) 
      -- ++(-.2,-.45) ++(.2,.45)
      arc (0:90:.2)
        -- ++(-1.4,0) arc (270:200:1);
  \draw (F1.270) arc (-145:-90:.7) -- ++(0.9,0) arc (90:0:.2) 
      -- ++(.4,-.8) node [coordinate,label={[label distance=-1pt]-90:W}] {} ++(-.4,.8)   
      arc (180:90:.2)
      -- ++(.6,0) arc (-90:-35:.7);
\end{tikzpicture}
\end{center}}
%
\newcommand{\abfiguretwoenglish}{\begin{center}
\begin{tikzpicture}[thick,line join=round]
  \node (W1) at (0,0) {\vphantom{F$pq$}T};
  \node (p) at (.5,0) {\vphantom{TF$q$}$p$};
  \node (F1) at (.95,0) {\vphantom{T$qp$}F};
  \node (W2) at (4,0) {\vphantom{F$qp$}T};
  \node (q) at (4.5,0) {\vphantom{TF$p$}$q$};
  \node (F2) at (4.95,0) {\vphantom{T$pq$}F};
  \draw (W1.110) arc (160:90:1) -- ++(1.2,0) arc (-90:0:.2) 
    -- ++(.3,.5) node [coordinate,label={[label distance=-2pt]90:~~F}] {} ++(-.3,-.5) 
    arc (180:270:.2) -- ++(1.6,0) arc (90:20:1);
  \draw (W1.90) arc (145:90:.7) -- ++(.8,0) arc (-90:0:.2) 
    -- ++(1.25,-2.5) ++(-1.25,2.5)
    arc (180:270:.2) -- ++ (1.627,0) arc (90:35:.7);
  \draw (F2.270) arc (-20:-90:1) -- ++(-0.4,0) arc (90:180:.2) 
      -- ++(-.25,-.45) ++(.25,.45)
      arc (0:90:.2)
        -- ++(-1.4,0) arc (270:200:1);
  \draw (F1.270) arc (-145:-90:.7) -- ++(0.9,0) arc (90:0:.2) 
      -- ++(.4,-.8) node [coordinate,label={[label distance=-1pt]-90:T}] {} ++(-.4,.8)   
      arc (180:90:.2)
      -- ++(.6,0) arc (-90:-35:.7);
\end{tikzpicture}
\end{center}}
%
\newcommand{\abfiguretwoenglishpmc}{\begin{center}
\begin{tikzpicture}[thick,line join=round]
  \node (W1) at (0,0) {\vphantom{F$pq$}T};
  \node (p) at (.5,0) {\vphantom{TF$q$}$p$};
  \node (F1) at (.95,0) {\vphantom{T$qp$}F};
  \node (W2) at (4,0) {\vphantom{F$qp$}T};
  \node (q) at (4.5,0) {\vphantom{TF$p$}$q$};
  \node (F2) at (4.95,0) {\vphantom{T$pq$}F.};
  \draw (W1.110) arc (160:90:1) -- ++(1.2,0) arc (-90:0:.2) 
    -- ++(.3,.5) node [coordinate,label={[label distance=-2pt]90:~~F}] {} ++(-.3,-.5) 
    arc (180:270:.2) -- ++(1.6,0) arc (90:20:1);
  \draw (W1.90) arc (145:90:.7) -- ++(.8,0) arc (-90:0:.2) 
    -- ++(1.25,-2.5) ++(-1.25,2.5)
    arc (180:270:.2) -- ++ (1.627,0) arc (90:35:.7);
  \draw (F2.270) arc (-20:-90:1) -- ++(-0.4,0) arc (90:180:.2) 
      -- ++(-.25,-.45) ++(.25,.45)
      arc (0:90:.2)
        -- ++(-1.4,0) arc (270:200:1);
  \draw (F1.270) arc (-145:-90:.7) -- ++(0.9,0) arc (90:0:.2) 
      -- ++(.4,-.8) node [coordinate,label={[label distance=-1pt]-90:T}] {} ++(-.4,.8)   
      arc (180:90:.2)
      -- ++(.6,0) arc (-90:-35:.7);
\end{tikzpicture}
\end{center}}
%
\newcommand{\abfigurethreegerman}{\begin{center}
\begin{tikzpicture}[thick,line join=round]
  \node (W1) at (0,0) {\vphantom{F$\xi$}\gdql W};
  \node (xi) at (.5,0) {\vphantom{WF}$\xi$};
  \node (F1) at (.95,0) {\vphantom{W$\xi$}F{}\gdqr};
  \draw (F1.110) -- ++(-.7,.5) node [coordinate,label={[label distance=-2pt]above:W}] {};
  \draw (W1.-70) -- ++(.7,-.5) node [coordinate,label={[label distance=-2pt]below:F}] {};
\end{tikzpicture}
\end{center}}
%
\newcommand{\abfigurethreeenglish}{\begin{center}
\begin{tikzpicture}[thick,line join=round]
  \node (W1) at (0,0) {\vphantom{F$\xi$}``T};
  \node (xi) at (.5,0) {\vphantom{TF}$\xi$};
  \node (F1) at (.95,0) {\vphantom{T$\xi$}F''};
  \draw (F1.110) -- ++(-.7,.5) node [coordinate,label={[label distance=-2pt]above:T}] {};
  \draw (W1.-70) -- ++(.7,-.5) node [coordinate,label={[label distance=-2pt]below:F}] {};
\end{tikzpicture}
\end{center}}
%
\newcommand{\abfigurethreeenglishpmc}{\begin{center}
\begin{tikzpicture}[thick,line join=round]
  \node (W1) at (0,0) {\vphantom{F$\xi$}`T};
  \node (xi) at (.5,0) {\vphantom{TF}$\xi$};
  \node (F1) at (.95,0) {\vphantom{T$\xi$}F',};
  \draw (F1.110) -- ++(-.7,.5) node [coordinate,label={[label distance=-2pt]above:T}] {};
  \draw (W1.-70) -- ++(.7,-.5) node [coordinate,label={[label distance=-2pt]below:F}] {};
\end{tikzpicture}
\end{center}}
%
\newcommand{\abfigurefourgerman}{\begin{center}
\begin{tikzpicture}[thick,line join=round]
  \node (W1) at (0,0) {\vphantom{F$\xi\eta$}W};
  \node (xi) at (.5,0) {\vphantom{WF$\eta$}$\xi$};
  \node (F1) at (.95,0) {\vphantom{W$\xi\eta$}F};
  \node (W2) at (4,0) {\vphantom{F$\xi\eta$}W};
  \node (q) at (4.5,0) {\vphantom{WF$\xi\eta$}$\eta$};
  \node (F2) at (4.95,0) {\vphantom{W$pq$}F};
   \draw (W1.-110) arc (-160:-90:1) -- ++(2,0) arc (90:0:.2) 
     -- ++(0,-.5) node [coordinate,label={[label distance=-2pt]-90:~F}] {} ++(0,.5) 
     arc (180:90:.2) -- ++(.8,0) arc (-90:-20:1);
\draw (W1.-90) arc (-145:-90:.7) -- ++(1,0) arc (90:0:.2) 
     -- ++(.2,2.2) node [coordinate,label={[label distance=-2pt]90:W}] {} ++(-.2,-2.2) 
     arc (180:90:.2) -- ++ (1.427,0) arc (-90:-35:.7);
   \draw (F2.90) arc (20:90:1) -- ++(-0.4,0) arc (-90:-180:.2) 
       -- ++(-.13,-3) ++(.13,3)
       arc (0:-90:.2)  -- ++(-1.4,0) arc (90:160:1);
   \draw (F1.90) arc (145:90:.7) -- ++(0.7,0) arc (-90:0:.2) 
       -- ++(.43,-2.7) ++(-.43,2.7)   
       arc (-180:-90:.2) -- ++(.8,0) arc (90:35:.7);
\end{tikzpicture}
\end{center}}
%
\newcommand{\abfigurefourenglish}{\begin{center}
\begin{tikzpicture}[thick,line join=round]
  \node (W1) at (0,0) {\vphantom{F$\xi\eta$}T};
  \node (xi) at (.5,0) {\vphantom{TF$\eta$}$\xi$};
  \node (F1) at (.95,0) {\vphantom{T$\xi\eta$}F};
  \node (W2) at (4,0) {\vphantom{F$\xi\eta$}T};
  \node (q) at (4.5,0) {\vphantom{TF$\xi\eta$}$\eta$};
  \node (F2) at (4.95,0) {\vphantom{T$pq$}F};
   \draw (W1.-110) arc (-160:-90:1) -- ++(2,0) arc (90:0:.2) 
     -- ++(0,-.5) node [coordinate,label={[label distance=-2pt]-90:~F}] {} ++(0,.5) 
     arc (180:90:.2) -- ++(.8,0) arc (-90:-20:1);
\draw (W1.-90) arc (-145:-90:.7) -- ++(1,0) arc (90:0:.2) 
     -- ++(.2,2.2) node [coordinate,label={[label distance=-2pt]90:T}] {} ++(-.2,-2.2) 
     arc (180:90:.2) -- ++ (1.427,0) arc (-90:-35:.7);
   \draw (F2.90) arc (20:90:1) -- ++(-0.4,0) arc (-90:-180:.2) 
       -- ++(-.13,-3) ++(.13,3)
       arc (0:-90:.2)  -- ++(-1.4,0) arc (90:160:1);
   \draw (F1.90) arc (145:90:.7) -- ++(0.7,0) arc (-90:0:.2) 
       -- ++(.43,-2.7) ++(-.43,2.7)   
       arc (-180:-90:.2) -- ++(.8,0) arc (90:35:.7);
\end{tikzpicture}
\end{center}}
%
\newcommand{\abfigurefourenglishpmc}{\begin{center}
\begin{tikzpicture}[thick,line join=round]
  \node (W1) at (0,0) {\vphantom{F$\xi\eta$}T};
  \node (xi) at (.5,0) {\vphantom{TF$\eta$}$\xi$};
  \node (F1) at (.95,0) {\vphantom{T$\xi\eta$}F};
  \node (W2) at (4,0) {\vphantom{F$\xi\eta$}T};
  \node (q) at (4.5,0) {\vphantom{TF$\xi\eta$}$\eta$};
  \node (F2) at (4.95,0) {\vphantom{T$pq$}F.};
   \draw (W1.-110) arc (-160:-90:1) -- ++(2,0) arc (90:0:.2) 
     -- ++(0,-.5) node [coordinate,label={[label distance=-2pt]-90:~F}] {} ++(0,.5) 
     arc (180:90:.2) -- ++(.8,0) arc (-90:-20:1);
\draw (W1.-90) arc (-145:-90:.7) -- ++(1,0) arc (90:0:.2) 
     -- ++(.2,2.2) node [coordinate,label={[label distance=-2pt]90:T}] {} ++(-.2,-2.2) 
     arc (180:90:.2) -- ++ (1.427,0) arc (-90:-35:.7);
   \draw (F2.90) arc (20:90:1) -- ++(-0.4,0) arc (-90:-180:.2) 
       -- ++(-.13,-3) ++(.13,3)
       arc (0:-90:.2)  -- ++(-1.4,0) arc (90:160:1);
   \draw (F1.90) arc (145:90:.7) -- ++(0.7,0) arc (-90:0:.2) 
       -- ++(.43,-2.7) ++(-.43,2.7)   
       arc (-180:-90:.2) -- ++(.8,0) arc (90:35:.7);
\end{tikzpicture}
\end{center}}
%
\newcommand{\abfigurefivegerman}{
\begin{center}
\begin{tikzpicture}[thick,line join=round]
  \node (W1) at (0,0) {\vphantom{F$pq$}W};
  \node (q) at (.5,0) {\vphantom{WF$pq$}$q$};
  \node (F1) at (.95,0) {\vphantom{W$pq$}F};
  \node (W2) at (3,0) {\vphantom{F$pq$}W};
  \node (p) at (3.5,0) {\vphantom{WF$pq$}$p$};
  \node (F2) at (3.95,0) {\vphantom{W$pq$}F};
  \draw (F1.110) -- ++(-.7,.5) node [coordinate,label={[label distance=-2pt]above:W}] {};
  \draw (W1.-70) -- ++(.7,-.5) node [coordinate,label={[label distance=-2pt]below:F}] {};
  \draw (.1,1.3) arc (127:72:0.7) -- ++(-18:0.6) arc (-109:-18:0.2) 
     -- ++(.5,.5) node (top) [coordinate,label={[label distance=-5pt]45:W}] {} ++(-.5,-.5) 
    arc (161:251:0.2) -- ++(-18:0.85) arc (72:17:0.7);
  \draw (0,1.4) arc (142:72:1) --+(-18:1) arc (-109:-18:0.2) 
    node (botone) [coordinate] {}
    arc (161:251:0.2) --+(-18:.9) arc (72:16:1);
  \draw (.9,-1.3) arc (-120:-65:.7)-- ++(25:0.2) arc (115:25:0.2) 
    node (bottwo) [coordinate] {}
    arc (-155:-245:0.2) -- ++(25:0.7) arc (-55:-3:0.7);
  \draw (.8,-1.4) arc (-135:-65:1) -- ++(25:1.2) arc (115:25:0.2) 
    -- ++(0.1,-0.5) node (bot) [coordinate,label={[label distance=-2pt]-90:F}] {} ++(-0.1,0.5)
    arc (-155:-245:0.2) -- ++(25:0.2) arc (-55:-15:1);
  \draw (top.45) ++(0.4,0.4) -- ++(0.3,0.3) node [coordinate,label={[label distance=-4pt]45:F}] {};
  \draw (botone) arc (150:240:2.4);
  \draw (bottwo) arc (-160:-75:0.95);
  \draw (bot.-45) ++ (0.1,-0.4) -- ++(0.3,-0.3) node [coordinate,label={[label distance=-2pt]-90:~~W}] {};
\end{tikzpicture}
\end{center}}
%
\newcommand{\abfigurefiveenglish}{\begin{center}
\begin{tikzpicture}[thick,line join=round]
  \node (W1) at (0,0) {\vphantom{F$pq$}T};
  \node (q) at (.5,0) {\vphantom{TF$pq$}$q$};
  \node (F1) at (.95,0) {\vphantom{T$pq$}F};
  \node (W2) at (3,0) {\vphantom{F$pq$}T};
  \node (p) at (3.5,0) {\vphantom{TF$pq$}$p$};
  \node (F2) at (3.95,0) {\vphantom{W$pq$}F};
  \draw (F1.110) -- ++(-.7,.5) node [coordinate,label={[label distance=-2pt]above:T}] {};
  \draw (W1.-70) -- ++(.7,-.5) node [coordinate,label={[label distance=-2pt]below:F}] {};
  \draw (.1,1.3) arc (127:72:0.7) -- ++(-18:0.6) arc (-109:-18:0.2) 
     -- ++(.5,.5) node (top) [coordinate,label={[label distance=-5pt]45:T}] {} ++(-.5,-.5) 
    arc (161:251:0.2) -- ++(-18:0.85) arc (72:17:0.7);
  \draw (0,1.4) arc (142:72:1) --+(-18:1) arc (-109:-18:0.2) 
    node (botone) [coordinate] {}
    arc (161:251:0.2) --+(-18:.9) arc (72:16:1);
  \draw (.9,-1.3) arc (-120:-65:.7)-- ++(25:0.2) arc (115:25:0.2) 
    node (bottwo) [coordinate] {}
    arc (-155:-245:0.2) -- ++(25:0.7) arc (-55:-3:0.7);
  \draw (.8,-1.4) arc (-135:-65:1) -- ++(25:1.2) arc (115:25:0.2) 
    -- ++(0.1,-0.5) node (bot) [coordinate,label={[label distance=-2pt]-90:F}] {} ++(-0.1,0.5)
    arc (-155:-245:0.2) -- ++(25:0.2) arc (-55:-15:1);
  \draw (top.45) ++(0.4,0.4) -- ++(0.3,0.3) node [coordinate,label={[label distance=-4pt]45:F}] {};
  \draw (botone) arc (150:240:2.4);
  \draw (bottwo) arc (-160:-75:0.95);
  \draw (bot.-45) ++ (0.1,-0.4) -- ++(0.3,-0.3) node [coordinate,label={[label distance=-2pt]-90:~~T}] {};
\end{tikzpicture}
\end{center}}
%
\newcommand{\abfigurefiveenglishpmc}{\begin{center}
\begin{tikzpicture}[thick,line join=round]
  \node (W1) at (0,0) {\vphantom{F$pq$}T};
  \node (q) at (.5,0) {\vphantom{TF$pq$}$q$};
  \node (F1) at (.95,0) {\vphantom{T$pq$}F};
  \node (W2) at (3,0) {\vphantom{F$pq$}T};
  \node (p) at (3.5,0) {\vphantom{TF$pq$}$p$};
  \node (F2) at (3.95,0) {\vphantom{W$pq$}F.};
  \draw (F1.110) -- ++(-.7,.5) node [coordinate,label={[label distance=-2pt]above:T}] {};
  \draw (W1.-70) -- ++(.7,-.5) node [coordinate,label={[label distance=-2pt]below:F}] {};
  \draw (.1,1.3) arc (127:72:0.7) -- ++(-18:0.6) arc (-109:-18:0.2) 
     -- ++(.5,.5) node (top) [coordinate,label={[label distance=-5pt]45:T}] {} ++(-.5,-.5) 
    arc (161:251:0.2) -- ++(-18:0.85) arc (72:17:0.7);
  \draw (0,1.4) arc (142:72:1) --+(-18:1) arc (-109:-18:0.2) 
    node (botone) [coordinate] {}
    arc (161:251:0.2) --+(-18:.9) arc (72:16:1);
  \draw (.9,-1.3) arc (-120:-65:.7)-- ++(25:0.2) arc (115:25:0.2) 
    node (bottwo) [coordinate] {}
    arc (-155:-245:0.2) -- ++(25:0.7) arc (-55:-3:0.7);
  \draw (.8,-1.4) arc (-135:-65:1) -- ++(25:1.2) arc (115:25:0.2) 
    -- ++(0.1,-0.5) node (bot) [coordinate,label={[label distance=-2pt]-90:F}] {} ++(-0.1,0.5)
    arc (-155:-245:0.2) -- ++(25:0.2) arc (-55:-15:1);
  \draw (top.45) ++(0.4,0.4) -- ++(0.3,0.3) node [coordinate,label={[label distance=-4pt]45:F}] {};
  \draw (botone) arc (150:240:2.4);
  \draw (bottwo) arc (-160:-75:0.95);
  \draw (bot.-45) ++ (0.1,-0.4) -- ++(0.3,-0.3) node [coordinate,label={[label distance=-2pt]-90:~~T}] {};
\end{tikzpicture}
\end{center}}
%
\newcommand{\theline}{%
\begin{center}%
   \begin{tikzpicture}
      \draw[thick,dashed] (0.4,0) to (1,0);
      \draw[thick, o-, shorten >=3pt] (1,0) to node [below] {\scriptsize $~a$} (2,0) node {$\times$};
      \draw[thick,dashed, shorten <=3pt, shorten >=3pt] (2,0) to (2.55,0) node {$\times$};
      \draw[thick, shorten <=3pt,-o] (2.55,0) to node [below] {\scriptsize $b$} (3.55,0);
      \draw[thick,dashed] (3.65,0) to (4.25,0);
   \end{tikzpicture}%
\end{center}}
%
\newcommand{\fourthreeonetablegerman}{%
\begin{tabular}{c|c|c}
$p$ & $q$ & $r$ \\ \hhline{=|=|=}
W & W & W \\ \hline%\hhline{-|-|-~-|-~-}
F & W & W \\ \hline%\hhline{-|-|-~-|-~-}
W & F & W \\ \hline%\hhline{-|-|-~-|-~~}
W & W & F \\ \hline%\hhline{-|-|-~-|-~~}
F & F & W \\ \hline%\hhline{-|-|-~~~~~}
F & W & F \\ \hline%\hhline{-|-|-~~~~~}
W & F & F \\ \hline%\hhline{-|-|-~~~~~}
F & F & F \\ \hline%\hhline{-|-|-~~~~~}
\end{tabular}\qquad
\begin{tabular}{c|c}
$p$ & $q$ \\ \hhline{=|=}
W & W \\ \hline
F & W \\ \hline
W & F \\ \hline
F & F \\ \hline   
\end{tabular}\qquad
\begin{tabular}{c}
   $p$ \\ \hhline{=}
W \\ \hline
F \\ \hline
\end{tabular}%
}
%
\newcommand{\fourthreeonetableenglish}{%
\begin{tabular}{c|c|c}
$p$ & $q$ & $r$ \\ \hhline{=|=|=}
T & T & T \\ \hline%\hhline{-|-|-~-|-~-}
F & T & T \\ \hline%\hhline{-|-|-~-|-~-}
T & F & T \\ \hline%\hhline{-|-|-~-|-~~}
T & T & F \\ \hline%\hhline{-|-|-~-|-~~}
F & F & T \\ \hline%\hhline{-|-|-~~~~~}
F & T & F \\ \hline%\hhline{-|-|-~~~~~}
T & F & F \\ \hline%\hhline{-|-|-~~~~~}
F & F & F \\ \hline%\hhline{-|-|-~~~~~}
\end{tabular}\qquad
\begin{tabular}{c|c}
$p$ & $q$ \\ \hhline{=|=}
T & T \\ \hline
F & T \\ \hline
T & F \\ \hline
F & F \\ \hline   
\end{tabular}\qquad
\begin{tabular}{c}
   $p$ \\ \hhline{=}
T \\ \hline
F \\ \hline
\end{tabular}}
%
%
\newcommand{\fourfourfourtwotablegerman}{%
{\hfil \begin{tabular}{r c|c|c l}
&$p$ & $q$ & & \gdqr \\ \hhline{~=|=|=~}
& W  &  W  &  W &\\ \hhline{~---~}
& F  &  W  &  W &\\ \hhline{~---~}
& W  &  F  &  & \\ \hhline{~---~}
\gdql & F  &  F  &  W \\ \hhline{~---~}
\end{tabular} \hfil}}
%
\newcommand{\fourfourfourtwotableogden}{%
{\hfil\begin{tabular}{r c|c|c l}
`` &$p$ & $q$ & & \\ \hhline{~=|=|=~}
& T  &  T  &  T &\\ \hhline{~---~}
& F  &  T  &  T &\\ \hhline{~---~}
& T  &  F  &  & \\ \hhline{~---~}
& F  &  F  &  T & ''\\ \hhline{~---~}
\end{tabular}\hfil}
}
%
\newcommand{\fourfourfourtwotablepmc}{%
\pbk\pbk{\hfil\begin{tabular}{r c|c|c l}
` &$p$ & $q$ & & ' \\ \hhline{~=|=|=~}
& T  &  T  &  T &\\ \hhline{~---~}
& F  &  T  &  T &\\ \hhline{~---~}
& T  &  F  &  & \\ \hhline{~---~}
& F  &  F  &  T & \\ \hhline{~---~}
\end{tabular}\hfil}
}
%
%
\newcommand{\fiveonezeroonetablegerman}{\begin{scriptsize}\begin{longtable}{@{}l@{} @{}c@{} @{}c@{} @{}r@{} @{\thinspace}l@{~} @{~}l@{} @{~}l@{}}%
\vspace*{0pt}(W&W&W&W) & $(p,q)$ &Tautologie                  &(Wenn $p$, so $p$; und wenn $q$, so $q$.) \\
\vspace*{0pt}  & & &   &         &                             &\qquad $(p \rimplies p \rand q \rimplies q)$\\
\vspace*{0pt}(F&W&W&W) & $(p,q)$ &in Worten:                   &Nicht beides $p$ und $q$. \quad$(\rnot (p \rand q))$\\
\vspace*{0pt}(W&F&W&W) & $(p,q)$ & \phantom{i}'' \quad\quad'' &Wenn $q$, so $p$. \quad$(q \rimplies p)$\\
\vspace*{0pt}(W&W&F&W) & $(p,q)$ & \phantom{i}'' \quad\quad'' &Wenn $p$, so $q$. \quad$(p \rimplies q)$\\
\vspace*{0pt}(W&W&W&F) & $(p,q)$ & \phantom{i}'' \quad\quad'' &$p$ oder $q$. \quad$(p \lor q)$\\
\vspace*{0pt}(F&F&W&W) & $(p,q)$ & \phantom{i}'' \quad\quad'' &Nicht $q$. \quad$(\rnot q)$\\
\vspace*{0pt}(F&W&F&W) & $(p,q)$ & \phantom{i}'' \quad\quad'' &Nicht $p$. \quad$(\rnot p)$\\
\vspace*{0pt}(F&W&W&F) & $(p,q)$ & \phantom{i}'' \quad\quad'' &$p$, oder $q$, aber nicht beide. \\
\vspace*{0pt}  & & &   &         &                            &  \qquad$(p \rand \rnot q \ddlordd q \rand \rnot p)$\\
\vspace*{0pt}(W&F&F&W) & $(p,q)$ & \phantom{i}'' \quad\quad'' & Wenn $p$, so $q$; und wenn $q$, so $p$. \\
\vspace*{0pt}  & & &   &         &                            & \qquad$(p \equiv q)$\\
\vspace*{0pt}(W&F&W&F) & $(p,q)$ & \phantom{i}'' \quad\quad'' &$p$ \\
\vspace*{0pt}(W&W&F&F) & $(p,q)$ & \phantom{i}'' \quad\quad'' &$q$ \\
\vspace*{0pt}(F&F&F&W) & $(p,q)$ & \phantom{i}'' \quad\quad'' &Weder $p$ noch $q$. \\
\vspace*{0pt}  & & &   &         &                            & \qquad$(\rnot p \rand \rnot q)$ oder $(p \sheffer q )$\\
\vspace*{0pt}(F&F&W&F) & $(p,q)$ & \phantom{i}'' \quad\quad'' &$p$ und nicht $q$. \quad$(p \rand \rnot q)$ \\
\vspace*{0pt}(F&W&F&F) & $(p,q)$ & \phantom{i}'' \quad\quad'' &$q$ und nicht $p$. \quad$(q \rand \rnot p)$ \\
\vspace*{0pt}(W&F&F&F) & $(p,q)$ & \phantom{i}'' \quad\quad'' &$q$ und $p$. \quad$(q \rand p)$ \\
\vspace*{0pt}(F&F&F&F) & $(p,q)$ &\multicolumn{2}{l}{\hspace*{-7pt}Kontradiktion ($p$ und nicht $p$; und} \\
\vspace*{0pt}  & & &   &         &                            & $q$ und nicht $q$.) \quad$(p \rand \rnot p \rand q \rand \rnot q)$ \\
\end{longtable}\end{scriptsize}}
%
%
\newcommand{\fiveonezeroonetableogden}{\begin{scriptsize}\begin{longtable}{@{}l@{} @{}c@{} @{}c@{} @{}r@{} @{\thinspace}l@{~} @{~}l@{} @{~}l@{}}%
\vspace*{0pt}(T&T&T&T) & $(p,q)$ &Tautology                  &(if $p$ then $p$; and if $q$ then $q$) \\
\vspace*{0pt}  & & &   &         &                             &\qquad $[p \rimplies p \rand q \rimplies q]$\\
\vspace*{0pt}(F&T&T&T) & $(p,q)$ &in words:                   &Not both $p$ and $q$. \quad$[\rnot (p \rand q)]$\\
\vspace*{0pt}(T&F&T&T) & $(p,q)$ & \phantom{i}'' \quad~~'' &If $q$ then $p$. \quad$[q \rimplies p]$\\
\vspace*{0pt}(T&T&F&T) & $(p,q)$ & \phantom{i}'' \quad~~'' &If $p$ then $q$. \quad$[p \rimplies q]$\\
\vspace*{0pt}(T&T&T&F) & $(p,q)$ & \phantom{i}'' \quad~~'' &$p$ or $q$. \quad$[p \lor q]$\\
\vspace*{0pt}(F&F&T&T) & $(p,q)$ & \phantom{i}'' \quad~~'' &Not $q$. \quad$[\rnot q]$\\
\vspace*{0pt}(F&T&F&T) & $(p,q)$ & \phantom{i}'' \quad~~'' &Not $p$. \quad$[\rnot p]$\\
\vspace*{0pt}(F&T&T&F) & $(p,q)$ & \phantom{i}'' \quad~~'' &$p$ or $q$, but not both. \\
\vspace*{0pt}  & & &   &         &                            &  \qquad$[p \rand \rnot q \ddlordd q \rand \rnot p]$\\
\vspace*{0pt}(T&F&F&T) & $(p,q)$ & \phantom{i}'' \quad~~'' & If $p$, then $q$; and if $q$, then $p$. \\
\vspace*{0pt}  & & &   &         &                            & \qquad$[p \equiv q]$\\
\vspace*{0pt}(T&F&T&F) & $(p,q)$ & \phantom{i}'' \quad~~'' &$p$ \\
\vspace*{0pt}(T&T&F&F) & $(p,q)$ & \phantom{i}'' \quad~~'' &$q$ \\
\vspace*{0pt}(F&F&F&T) & $(p,q)$ & \phantom{i}'' \quad~~'' &Neither $p$ nor $q$. \\
\vspace*{0pt}  & & &   &         &                            & \qquad$[\rnot p \rand \rnot q$ or $p \sheffer q]$\\
\vspace*{0pt}(F&F&T&F) & $(p,q)$ & \phantom{i}'' \quad~~'' &$p$ and not $q$. \quad$[p \rand \rnot q]$ \\
\vspace*{0pt}(F&T&F&F) & $(p,q)$ & \phantom{i}'' \quad~~'' &$q$ and not $p$. \quad$[q \rand \rnot p]$ \\
\vspace*{0pt}(T&F&F&F) & $(p,q)$ & \phantom{i}'' \quad~~'' &$p$ and $q$. \quad$[p \rand q]$ \\
\vspace*{0pt}(F&F&F&F) & $(p,q)$ &\multicolumn{2}{l}{\hspace*{-7pt}Contradiction ($p$ and not $p$; and} \\
\vspace*{0pt}  & & &   &         &                            & $q$ and not $q$.) \quad$[p \rand \rnot p \rand q \rand \rnot q]$ \\
\end{longtable}\end{scriptsize}}
%
%
\newcommand{\fiveonezeroonetablepmc}{\begin{scriptsize}\begin{longtable}{@{}l@{} @{}c@{} @{}c@{} @{}r@{} @{\thinspace}l@{~} @{~}l@{} @{~}l@{}}%
\vspace*{0pt}(T&T&T&T) & $(p,q)$ &Tautology                  &(If $p$ then $p$; and if $q$ then $q$.) \\
\vspace*{0pt}  & & &   &         &                             &\qquad $(p \rimplies p \rand q \rimplies q)$\\
\vspace*{0pt}(F&T&T&T) & $(p,q)$ &In words:                   &Not both $p$ and $q$. \quad$(\rnot (p \rand q))$\\
\vspace*{0pt}(T&F&T&T) & $(p,q)$ & \phantom{i}'' \quad~~''\quad: &If $q$ then $p$. \quad$(q \rimplies p)$\\
\vspace*{0pt}(T&T&F&T) & $(p,q)$ & \phantom{i}'' \quad~~''\quad: &If $p$ then $q$. \quad$(p \rimplies q)$\\
\vspace*{0pt}(T&T&T&F) & $(p,q)$ & \phantom{i}'' \quad~~''\quad: &$p$ or $q$. \quad$(p \lor q)$\\
\vspace*{0pt}(F&F&T&T) & $(p,q)$ & \phantom{i}'' \quad~~''\quad: &Not $q$. \quad$(\rnot q)$\\
\vspace*{0pt}(F&T&F&T) & $(p,q)$ & \phantom{i}'' \quad~~''\quad: &Not $p$. \quad$(\rnot p)$\\
\vspace*{0pt}(F&T&T&F) & $(p,q)$ & \phantom{i}'' \quad~~''\quad: &$p$ or $q$, but not both. \\
\vspace*{0pt}  & & &   &         &                            &  \qquad$(p \rand \rnot q \ddlordd q \rand \rnot p)$\\
\vspace*{0pt}(T&F&F&T) & $(p,q)$ & \phantom{i}'' \quad~~''\quad: & If $p$ then $q$, and if $q$ then $p$. \\
\vspace*{0pt}  & & &   &         &                            & \qquad$(p \equiv q)$\\
\vspace*{0pt}(T&F&T&F) & $(p,q)$ & \phantom{i}'' \quad~~''\quad: &$p$ \\
\vspace*{0pt}(T&T&F&F) & $(p,q)$ & \phantom{i}'' \quad~~'' &$q$ \\
\vspace*{0pt}(F&F&F&T) & $(p,q)$ & \phantom{i}'' \quad~~''\quad: &Neither $p$ nor $q$. \\
\vspace*{0pt}  & & &   &         &                            & \qquad$(\rnot p \rand \rnot q$ or $p \sheffer q)$\\
\vspace*{0pt}(F&F&T&F) & $(p,q)$ & \phantom{i}'' \quad~~''\quad: &$p$ and not $q$. \quad$(p \rand \rnot q)$ \\
\vspace*{0pt}(F&T&F&F) & $(p,q)$ & \phantom{i}'' \quad~~''\quad: &$q$ and not $p$. \quad$(q \rand \rnot p)$ \\
\vspace*{0pt}(T&F&F&F) & $(p,q)$ & \phantom{i}'' \quad~~''\quad: &$q$ and $p$. \quad$(q \rand p)$ \\
\vspace*{0pt}(F&F&F&F) & $(p,q)$ &\multicolumn{2}{l}{\hspace*{-7pt}Contradiction ($p$ and not $p$, and} \\
\vspace*{0pt}  & & &   &         &                            & $q$ and not $q$.) \quad$(p \rand \rnot p \rand q \rand \rnot q)$ \\
\end{longtable}\end{scriptsize}}
%
% Gets rid of extra spacing around mathematical displays
\newcommand{\zeroout}{
    \setlength{\abovedisplayskip}{0pt}
    \setlength{\abovedisplayshortskip}{0pt}
    \setlength{\belowdisplayskip}{0pt}      
    \setlength{\belowdisplayshortskip}{0pt}
}
%
%
\newcommand{\sixzerotwostackonegerman}{\[ \begin{aligned}     &x = \omop[0] x \text{ Def.\ und} \\
     &\omop \omop[\nu] x = \omop[\nu + 1] x \text{ Def.} \end{aligned} \]}
\newcommand{\sixzerotwostackoneogden}{\[ \begin{aligned}     &x = \omop[0] x \text{ Def.\ and} \\
     &\omop \omop[\nu] x = \omop[\nu + 1] x \text{ Def.} \end{aligned} \]}
\newcommand{\sixzerotwostackonepmc}{\[ \begin{aligned}     &x = \omop[0] x \text{ Def.} \\
     &\omop \omop[\nu] x = \omop[\nu + 1] x \text{ Def.} \end{aligned} \]}
%
%
\newcommand{\sixzerotwostacktwogerman}{\[\begin{aligned} &0+1=1 \text{ Def.}\\
                   &0+1+1=2 \text{ Def.}\\
                   &0+1+1+1=3 \text{ Def.}\\
                   &\text{(u.\ s.\ f.)} 
    \end{aligned}\]}
\newcommand{\sixzerotwostacktwoogden}{\[ \begin{aligned} &0+1=1 \text{ Def.}\\
                   &0+1+1=2 \text{ Def.}\\
                   &0+1+1+1=3 \text{ Def.}\\
                   &\text{and so on.} 
    \end{aligned}\]}
\newcommand{\sixzerotwostacktwopmc}{\[ \begin{aligned} 0+1&=1 \text{ Def.,}\\
                   0+1+1&=2 \text{ Def.,}\\
                   0+1+1+1&=3 \text{ Def.,}\\
                   \text{(and so}&\text{ on).} 
    \end{aligned}\]}
%
%
\begin{document}
% sloppy mode reduces badboxes in tight quarters
\sloppy
%$$$$$$$$$$$$$$$$$$$$$
% Title page
%$$$$$$$$$$$$$$$$$$$$$
\begin{titlepage}
\begin{center}
\vspace*{2in}
{\Huge Tractatus Logico-Philosophicus}\\[20pt]
{\LARGE Logisch-philosophische Abhandlung}\\[30pt]
{\Large \textit{By Ludwig Wittgenstein}}\\[90pt]
{First published by Kegan Paul (London), 1922.}\\[20pt]
{\textsc{Side-by-side-by-side edition, version \version\ (\today),}}\\
{containing the original German, alongside both the
Ogden/Ramsey, and Pears/McGuinness English translations.}
\end{center}
\end{titlepage}
%=============================================
% RUSSELL'S INTRODUCTION
%=============================================
\setlength{\parskip}{0pt}
\clearpage\phantomsection\addcontentsline{toc}{chapter}{Introduction (by Bertrand Russell)}%
\begin{multicols}{2}[\section*{Introduction}By Bertrand Russell, F.\ R.\ S.]
\selectlanguage{english}\noindent% STARTINTROCONVERT
\textsc{Mr.\ Wittgenstein's} \emph{Tractatus Logico-Phil\-o\-soph\-i\-cus}, whether or not it prove to give the ultimate truth on the matters with which it deals, certainly deserves, by its breadth and scope and profundity, to be considered an important event in the philosophical world. Starting from the principles of Symbolism and the relations which are necessary between words and things in any language, it applies the result of this inquiry to various departments of traditional philosophy, showing in each case how traditional philosophy and traditional solutions arise out of ignorance of the principles of Symbolism and out of misuse of language.

The logical structure of propositions and the nature of logical inference are first dealt with. Thence we pass successively to Theory of Knowledge, Principles of Physics, Ethics, and finally the Mystical (\emph{das Mystische}).

In order to understand Mr.\ Wittgenstein's book, it is necessary to realize what is the problem with which he is concerned. In the part of his theory which deals with Symbolism he is concerned with the conditions which would have to be fulfilled by a logically perfect language. There are various problems as regards language. First, there is the problem what actually occurs in our minds when we use language with the intention of meaning something by it; this problem belongs to psychology. Secondly, there is the problem as to what is the relation subsisting between thoughts, words, or sentences, and that which they refer to or mean; this problem belongs to epistemology. Thirdly, there is the problem of using sentences so as to convey truth rather that falsehood; this belongs to the special sciences dealing with the subject-matter of the sentences in question. Fourthly, there is the question: what relation must one fact (such as a sentence) have to another in order to be \emph{capable} of being a symbol for that other? This last is a logical question, and is the one with which Mr.\ Wittgenstein is concerned. He is concerned with the conditions for \emph{accurate} Symbolism, i.e.\ for Symbolism in which a sentence ``means'' something quite definite. In practice, language is always more or less vague, so that what we assert is never quite precise. Thus, logic has two problems to deal with in regard to Symbolism: (1) the conditions for sense rather than nonsense in combinations of symbols; (2) the conditions for uniqueness of meaning or reference in symbols or combinations of symbols. A logically perfect language has rules of syntax which prevent nonsense, and has single symbols which always have a definite and unique meaning. Mr.\ Wittgenstein is concerned with the conditions for a logically perfect language---not that any language is logically perfect, or that we believe ourselves capable, here and now, of constructing a logically perfect language, but that the whole function of language is to have meaning, and it only fulfills this function in proportion as it approaches to the ideal language which we postulate.

The essential business of language is to assert or deny facts. Given the syntax of language, the meaning of a sentence is determined as soon as the meaning of the component words is known. In order that a certain sentence should assert a certain fact there must, however the language may be constructed, be something in common between the structure of the sentence and the structure of the fact. This is perhaps the most fundamental thesis of Mr.\ Wittgenstein's theory. That which has to be in common between the sentence and the fact cannot, he contends, be itself in turn \emph{said} in language. It can, in his phraseology, only be \emph{shown}, not said, for whatever we may say will still need to have the same structure.

The first requisite of an ideal language would be that there should be one name for every simple, and never the same name for two different simples. A name is a simple symbol in the sense that it has no parts which are themselves symbols. In a logically perfect language nothing that is not simple will have a simple symbol. The symbol for the whole will be a ``complex'', containing the symbols for the parts. (In speaking of a ``complex'' we are, as will appear later, sinning against the rules of philosophical grammar, but this is unavoidable at the outset. ``Most propositions and questions that have been written about philosophical matters are not false but senseless. We cannot, therefore, answer questions of this kind at all, but only state their senselessness. Most questions and propositions of the philosophers result from the fact that we do not understand the logic of our language. They are of the same kind as the question whether the Good is more or less identical than the Beautiful'' (4.003).) What is complex in the world is a fact. Facts which are not compounded of other facts are what Mr.\ Wittgenstein calls \emph{Sachverhalte}, whereas a fact which may consist of two or more facts is a \emph{Tatsache}: thus, for example ``Socrates is wise'' is a \emph{Sachverhalt}, as well as a \emph{Tatsache}, whereas ``Socrates is wise and Plato is his pupil'' is a \emph{Tatsache} but not a \emph{Sachverhalt}.

He compares linguistic expression to projection in geometry. A geometrical figure may be projected in many ways: each of these ways corresponds to a different language, but the projective properties of the original figure remain unchanged whichever of these ways may be adopted. These projective properties correspond to that which in his theory the proposition and the fact must have in common, if the proposition is to assert the fact.

In certain elementary ways this is, of course, obvious. It is impossible, for example, to make a statement about two men (assuming for the moment that the men may be treated as simples), without employing two names, and if you are going to assert a relation between the two men it will be necessary that the sentence in which you make the assertion shall establish a relation between the two names. If we say ``Plato loves Socrates'', the word ``loves'' which occurs between the word ``Plato'' and the word ``Socrates'' establishes a certain relation between these two words, and it is owing to this fact that our sentence is able to assert a relation between the persons named by the words ``Plato'' and ``Socrates''. ``We must not say, the complex sign `$aRb$' says that `$a$ stands in a certain relation $R$ to $b$'; but we must say, that `$a$' stands in a certain relation to `$b$' says \emph{that} $aRb$'' (3.1432).

Mr.\ Wittgenstein begins his theory of Symbolism with the statement (2.1): ``We make to ourselves pictures of facts.'' A picture, he says, is a model of the reality, and to the objects in the reality correspond the elements of the picture: the picture itself is a fact. The fact that things have a certain relation to each other is represented by the fact that in the picture its elements have a certain relation to one another. ``In the picture and the pictured there must be something identical in order that the one can be a picture of the other at all. What the picture must have in common with reality in order to be able to represent it after its manner---rightly or falsely---is its form of representation'' (2.161, 2.17).

We speak of a logical picture of a reality when we wish to imply only so much resemblance as is essential to its being a picture in any sense, that is to say, when we wish to imply no more than identity of logical form. The logical picture of a fact, he says, is a \emph{Gedanke}. A picture can correspond or not correspond with the fact and be accordingly true or false, but in both cases it shares the logical form with the fact. The sense in which he speaks of pictures is illustrated by his statement: ``The gramophone record, the musical thought, the score, the waves of sound, all stand to one another in that pictorial internal relation which holds between language and the world. To all of them the logical structure is common. (Like the two youths, their two horses and their lilies in the story. They are all in a certain sense one)'' (4.014). The possibility of a proposition representing a fact rests upon the fact that in it objects are represented by signs. The so-called logical ``constants'' are not represented by signs, but are themselves present in the proposition as in the fact. The proposition and the fact must exhibit the same logical ``manifold'', and this cannot be itself represented since it has to be in common between the fact and the picture. Mr.\ Wittgenstein maintains that everything properly philosophical belongs to what can only be shown, or to what is in common between a fact and its logical picture. It results from this view that nothing correct can be said in philosophy. Every philosophical proposition is bad grammar, and the best that we can hope to achieve by philosophical discussion is to lead people to see that philosophical discussion is a mistake. ``Philosophy is not one of the natural sciences. (The word `philosophy' must mean something which stands above or below, but not beside the natural sciences.) The object of philosophy is the logical clarification of thoughts. Philosophy is not a theory but an activity. A philosophical work consists essentially of elucidations. The result of philosophy is not a number of `philosophical propositions', but to make propositions clear. Philosophy should make clear and delimit sharply the thoughts which otherwise are, as it were, opaque and blurred'' (4.111 and 4.112). In accordance with this principle the things that have to be said in leading the reader to understand Mr.\ Wittgenstein's theory are all of them things which that theory itself condemns as meaningless. With this proviso we will endeavour to convey the picture of the world which seems to underlie his system.

The world consists of facts: facts cannot strictly speaking be defined, but we can explain what we mean by saying that facts are what makes propositions true, or false. Facts may contain parts which are facts or may contain no such parts; for example: ``Socrates was a wise Athenian'', consists of the two facts, ``Socrates was wise'', and ``Socrates was an Athenian.'' A fact which has no parts that are facts is called by Mr.\ Wittgenstein a \emph{Sachverhalt}. This is the same thing that he calls an atomic fact. An atomic fact, although it contains no parts that are facts, nevertheless does contain parts. If we may regard ``Socrates is wise'' as an atomic fact we perceive that it contains the constituents ``Socrates'' and ``wise''. If an atomic fact is analyzed as fully as possible (theoretical, not practical possibility is meant) the constituents finally reached may be called ``simples'' or ``objects''. It is a logical necessity demanded by theory, like an electron. His ground for maintaining that there must be simples is that every complex presupposes a fact. It is not necessarily assumed that the complexity of facts is finite; even if every fact consisted of an infinite number of atomic facts and if every atomic fact consisted of an infinite number of objects there would still be objects and atomic facts (4.2211). The assertion that there is a certain complex reduces to the assertion that its constituents are related in a certain way, which is the assertion of a \emph{fact}: thus if we give a name to the complex the name only has meaning in virtue of the truth of a certain proposition, namely the proposition asserting the relatedness of the constituents of the complex. Thus the naming of complexes presupposes propositions, while propositions presuppose the naming of simples. In this way the naming of simples is shown to be what is logically first in logic.

The world is fully described if all atomic facts are known, together with the fact that these are all of them. The world is not described by merely naming all the objects in it; it is necessary also to know the atomic facts of which these objects are constituents. Given this totality of atomic facts, every true proposition, however complex, can theoretically be inferred. A proposition (true or false) asserting an atomic fact is called an atomic proposition. All atomic propositions are logically independent of each other. No atomic proposition implies any other or is inconsistent with any other. Thus the whole business of logical inference is concerned with propositions which are not atomic. Such propositions may be called molecular.

Wittgenstein's theory of molecular propositions turns upon his theory of the construction of truth-functions.

A truth-function of a proposition $p$ is a proposition containing $p$ and such that its truth or falsehood depends only upon the truth or falsehood of $p$, and similarly a truth-function of several propositions $p,\thickspace q,\thickspace r,\thickspace \ldots$ is one containing $p,\thickspace q,\thickspace r,\thickspace \ldots$ and such that its truth or falsehood depends only upon the truth or falsehood of $p,\thickspace q,\thickspace r,\thickspace \ldots$ It might seem at first sight as though there were other functions of propositions besides truth-functions; such, for example, would be ``A believes $p$'', for in general A will believe some true propositions and some false ones: unless he is an exceptionally gifted individual, we cannot infer that $p$ is true from the fact that he believes it or that $p$ is false from the fact that he does not believe it. Other apparent exceptions would be such as ``$p$ is a very complex proposition'' or ``$p$ is a proposition about Socrates''. Mr.\ Wittgenstein maintains, however, for reasons which will appear presently, that such exceptions are only apparent, and that every function of a proposition is really a truth-function. It follows that if we can define truth-functions generally, we can obtain a general definition of all propositions in terms of the original set of atomic propositions. This Wittgenstein proceeds to do.

It has been shown by Dr.\ Sheffer (\emph{Trans.\ Am.\ Math.\ Soc.}, Vol.\ XIV. pp.\ 481--488) that all truth-functions of a given set of propositions can be constructed out of either of the two functions ``not-$p$ or not-$q$'' or ``not-$p$ and not-$q$''. Wittgenstein makes use of the latter, assuming a knowledge of Dr.\ Sheffer's work. The manner in which other truth-functions are constructed out of ``not-$p$ and not-$q$'' is easy to see. ``Not-$p$ and not-$p$'' is equivalent to ``not-$p$'', hence we obtain a definition of negation in terms of our primitive function: hence we can define ``$p$ or $q$'', since this is the negation of ``not-$p$ and not-$q$'', i.e.\ of our primitive function. The development of other truth-functions out of ``not-$p$'' and ``$p$ or $q$'' is given in detail at the beginning of \emph{Principia Mathematica}. This gives all that is wanted when the propositions which are arguments to our truth-function are given by enumeration. Wittgenstein, however, by a very interesting analysis succeeds in extending the process to general propositions, i.e.\ to cases where the propositions which are arguments to our truth-function are not given by enumeration but are given as all those satisfying some condition. For example, let $f\negthinspace x$ be a propositional function (i.e.\ a function whose values are propositions), such as ``$x$ is human''---then the various values of $f\negthinspace x$ form a set of propositions. We may extend the idea ``not-$p$ and not-$q$'' so as to apply to the simultaneous denial of all the propositions which are values of $f\negthinspace x$. In this way we arrive at the proposition which is ordinarily represented in mathematical logic by the words ``$f\negthinspace x$ is false for all values of $x$''. The negation of this would be the proposition ``there is at least one $x$ for which $f\negthinspace x$ is true'' which is represented by ``$\rsomed{x} f\negthinspace x$''. If we had started with not-$f\negthinspace x$ instead of $f\negthinspace x$ we should have arrived at the proposition ``$f\negthinspace x$ is true for all values of $x$'' which is represented by ``$\ralld{x} f\negthinspace x$''. Wittgenstein's method of dealing with general propositions [i.e.\ ``$\ralld{x} f\negthinspace x$'' and ``$\rsomed{x} f\negthinspace x$''] differs from previous methods by the fact that the generality comes only in specifying the set of propositions concerned, and when this has been done the building up of truth-functions proceeds exactly as it would in the case of a finite number of enumerated arguments $p,\thickspace q,\thickspace r,\thickspace \ldots$

Mr.\ Wittgenstein's explanation of his symbolism at this point is not quite fully given in the text. The symbol he uses is $[\overline{p},\thickspace \overline{\xi},\thickspace \nop(\overline{\xi})].$ The following is the explanation of this symbol:
%
\begin{description}[noitemsep,labelindent=1em,leftmargin=3em,rightmargin=1em]
  \item $\overline{p}$ stands for all atomic propositions.
  \item $\overline{\xi}$ stands for any set of propositions.
  \item $\nop(\overline{\xi})$ stands for the negation of all the propositions making up $\overline{\xi}$.
 \end{description}

The whole symbol $[\overline{p},\thickspace \overline{\xi},\thickspace \nop(\overline{\xi})]$ means whatever can be obtained by taking any selection of atomic propositions, negating them all, then taking any selection of the set of propositions now obtained, together with any of the originals---and so on indefinitely. This is, he says, the general truth-function and also the general form of proposition. What is meant is somewhat less complicated than it sounds. The symbol is intended to describe a process by the help of which, given the atomic propositions, all others can be manufactured. The process depends upon:

(a). Sheffer's proof that all truth-functions can be obtained out of simultaneous negation, i.e.\ out of ``not-$p$ and not-$q$'';

(b). Mr.\ Wittgenstein's theory of the derivation of general propositions from conjunctions and disjunctions;

(c). The assertion that a proposition can only occur in another proposition as argument to a truth-function. Given these three foundations, it follows that all propositions which are not atomic can be derived from such as are, buy a uniform process, and it is this process which is indicated by Mr.\ Wittgenstein's symbol.

From this uniform method of construction we arrive at an amazing simplification of the theory of inference, as well as a definition of the sort of propositions that belong to logic. The method of generation which has just been described, enables Wittgenstein to say that all propositions can be constructed in the above manner from atomic propositions, and in this way the totality of propositions is defined. (The apparent exceptions which we mentioned above are dealt with in a manner which we shall consider later.) Wittgenstein is enabled to assert that propositions are all that follows from the totality of atomic propositions (together with the fact that it is the totality of them); that a proposition is always a truth-function of atomic propositions; and that if $p$ follows from $q$ the meaning of $p$ is contained in the meaning of $q$, from which of course it results that nothing can be deduced from an atomic proposition. All the propositions of logic, he maintains, are tautologies, such, for example, as ``$p$ or not $p$''.

The fact that nothing can be deduced from an atomic proposition has interesting applications, for example, to causality. There cannot, in Wittgenstein's logic, be any such thing as a causal nexus. ``The events of the future'', he says, ``\emph{cannot} be inferred from those of the present. Superstition is the belief in the causal nexus.'' That the sun will rise to-morrow is a hypothesis. We do not in fact know whether it will rise, since there is no compulsion according to which one thing must happen because another happens.

Let us now take up another subject---that of names. In Wittgenstein's theoretical logical language, names are only given to simples. We do not give two names to one thing, or one name to two things. There is no way whatever, according to him, by which we can describe the totality of things that can be named, in other words, the totality of what there is in the world. In order to be able to do this we should have to know of some property which must belong to every thing by a logical necessity. It has been sought to find such a property in self-identity, but the conception of identity is subjected by Wittgenstein to a destructive criticism from which there seems no escape. The definition of identity by means of the identity of indiscernibles is rejected, because the identity of indiscernibles appears to be not a logically necessary principle. According to this principle $x$ is identical with $y$ if every property of $x$ is a property of $y$, but it would, after all be logically possible for two things to have exactly the same properties. If this does not in fact happen that is an accidental characteristic of the world, not a logically necessary characteristic, and accidental characteristics of the world must, of course, not be admitted into the structure of logic. Mr.\ Wittgenstein accordingly banishes identity and adopts the convention that different letters are to mean different things. In practice, identity is needed as between a name and a description or between two descriptions. It is needed for such propositions as ``Socrates is the philosopher who drank the hemlock'', or ``The even prime is the next number after 1.'' For such uses of identity it is easy to provide on Wittgenstein's system.

The rejection of identity removes one method of speaking of the totality of things, and it will be found that any other method that may be suggested is equally fallacious: so, at least, Wittgenstein contends and, I think, rightly. This amounts to saying that ``object'' is a pseudo-concept. To say ``$x$ is an object'' is to say nothing. It follows from this that we cannot make such statements as ``there are more than three objects in the world'', or ``there are an infinite number of objects in the world''. Objects can only be mentioned in connexion with some definite property. We can say ``there are more than three objects which are human'', or ``there are more than three objects which are red'', for in these statements the word object can be replaced by a variable in the language of logic, the variable being one which satisfies in the first case the function ``$x$ is human''; in the second the function ``$x$ is red''. But when we attempt to say ``there are more than three objects'', this substitution of the variable for the word ``object'' becomes impossible, and the proposition is therefore seen to be meaningless.

We here touch one instance of Wittgenstein's fundamental thesis, that it is impossible to say anything about the world as a whole, and that whatever can be said has to be about bounded portions of the world. This view may have been originally suggested by notation, and if so, that is much in its favor, for a good notation has a subtlety and suggestiveness which at times make it seem almost like a live teacher. Notational irregularities are often the first sign of philosophical errors, and a perfect notation would be a substitute for thought. But although notation may have first suggested to Mr.\ Wittgenstein the limitation of logic to things within the world as opposed to the world as a whole, yet the view, once suggested, is seen to have much else to recommend it. Whether it is ultimately true I do not, for my part, profess to know. In this Introduction I am concerned to expound it, not to pronounce upon it. According to this view we could only say things about the world as a whole if we could get outside the world, if, that is to say, it ceased to be for us the whole world. Our world may be bounded for some superior being who can survey it from above, but for us, however finite it may be, it cannot have a boundary, since it has nothing outside it. Wittgenstein uses, as an analogy, the field of vision. Our field of vision does not, for us, have a visual boundary, just because there is nothing outside it, and in like manner our logical world has no logical boundary because our logic knows of nothing outside it. These considerations lead him to a somewhat curious discussion of Solipsism. Logic, he says, fills the world. The boundaries of the world are also its boundaries. In logic, therefore, we cannot say, there is this and this in the world, but not that, for to say so would apparently presuppose that we exclude certain possibilities, and this cannot be the case, since it would require that logic should go beyond the boundaries of the world as if it could contemplate these boundaries from the other side also. What we cannot think we cannot think, therefore we also cannot say what we cannot think.

This, he says, gives the key to solipsism. What Solipsism intends is quite correct, but this cannot be said, it can only be shown. That the world is \emph{my} world appears in the fact that the boundaries of language (the only language I understand) indicate the boundaries of my world. The metaphysical subject does not belong to the world but is a boundary of the world.

We must take up next the question of molecular propositions which are at first sight not truth-functions, of the propositions that they contain, such, for example, as ``A believes $p$.''

Wittgenstein introduces this subject in the statement of his position, namely, that all molecular functions are truth-functions. He says (5.54): ``In the general propositional form, propositions occur in a proposition only as bases of truth-operations.'' At first sight, he goes on to explain, it seems as if a propositions could also occur in other ways, e.g.\ ``A believes $p$.'' Here it seems superficially as if the proposition $p$ stood in a sort of relation to the object A. ``But it is clear that `A believes that $p$,' `A thinks $p$,' `A says $p$' are of the form ``{}`$p$' says $p$''; and here we have no co-ordination of a fact and an object, but a co-ordination of facts by means of a co-ordination of their objects'' (5.542).

What Mr.\ Wittgenstein says here is said so shortly that its point is not likely to be clear to those who have not in mind the controversies with which he is concerned. The theory which which he is disagreeing will be found in my articles on the nature of truth and falsehood in \emph{Philosophical Essays} and \emph{Proceedings of the Aristotelian Society}, 1906--7. The problem at issue is the problem of the logical form of belief, i.e.\ what is the schema representing what occurs when a man believes. Of course, the problem applies not only to belief, but also to a host of other mental phenomena which may be called propositional attitudes: doubting, considering, desiring, etc. In all these cases it seems natural to express the phenomenon in the form ``A doubts $p$'', ``A considers $p$'', ``A desires $p$'', etc., which makes it appear as though we were dealing with a relation between a person and a proposition. This cannot, of course, be the ultimate analysis, since persons are fictions and so are propositions, except in the sense in which they are facts on their own account. A proposition, considered as a fact on its own account, may be a set of words which a man says over to himself, or a complex image, or train of images passing through his mind, or a set of incipient bodily movements. It may be any one of innumerable different things. The proposition as a fact on its own account, for example, the actual set of words the man pronounces to himself, is not relevant to logic. What is relevant to logic is that common element among all these facts, which enables him, as we say, to \emph{mean} the fact which the proposition asserts. To psychology, of course, more is relevant; for a symbol does not mean what it symbolizes in virtue of a logical relation alone, but in virtue also of a psychological relation of intention, or association, or what-not. The psychological part of meaning, however, does not concern the logician. What does concern him in this problem of belief is the logical schema. It is clear that, when a person believes a proposition, the person, considered as a metaphysical subject, does not have to be assumed in order to explain what is happening. What has to be explained is the relation between the set of words which is the proposition considered as a fact on its own account, and the ``objective'' fact which makes the proposition true or false. This reduces ultimately to the question of the meaning of propositions, that is to say, the meaning of propositions is the only non-psychological portion of the problem involved in the analysis of belief. This problem is simply one of a relation of two facts, namely, the relation between the series of words used by the believer and the fact which makes these words true or false. The series of words is a fact just as much as what makes it true or false is a fact. The relation between these two facts is not unanalyzable, since the meaning of a proposition results from the meaning of its constituent words. The meaning of the series of words which is a proposition is a function of the meaning of the separate words. Accordingly, the proposition as a whole does not really enter into what has to be explained in explaining the meaning of a propositions. It would perhaps help to suggest the point of view which I am trying to indicate, to say that in the cases which have been considering the proposition occurs as a fact, not as a proposition. Such a statement, however, must not be taken too literally. The real point is that in believing, desiring, etc., what is logically fundamental is the relation of a proposition \emph{considered as a fact}, to the fact which makes it true or false, and that this relation of two facts is reducible to a relation of their constituents. Thus the proposition does not occur at all in the same sense in which it occurs in a truth-function.

There are some respects, in which, as it seems to me, Mr.\ Wittgenstein's theory stands in need of greater technical development. This applies in particular to his theory of number (6.02ff.) which, as it stands, is only capable of dealing with finite numbers. No logic can be considered adequate until it has been shown to be capable of dealing with transfinite numbers. I do not think there is anything in Mr.\ Wittgenstein's system to make it impossible for him to fill this lacuna.

More interesting than such questions of comparative detail is Mr.\ Wittgenstein's attitude towards the mystical. His attitude upon this grows naturally out of his doctrine in pure logic, according to which the logical proposition is a picture (true or false) of the fact, and has in common with the fact a certain structure. It is this common structure which makes it capable of being a picture of the fact, but the structure cannot itself be put into words, since it is a structure \emph{of} words, as well as of the fact to which they refer. Everything, therefore, which is involved in the very idea of the expressiveness of language must remain incapable of being expressed in language, and is, therefore, inexpressible in a perfectly precise sense. This inexpressible contains, according to Mr.\ Wittgenstein, the whole of logic and philosophy. The right method of teaching philosophy, he says, would be to confine oneself to propositions of the sciences, stated with all possible clearness and exactness, leaving philosophical assertions to the learner, and proving to him, whenever he made them, that they are meaningless. It is true that the fate of Socrates might befall a man who attempted this method of teaching, but we are not to be deterred by that fear, if it is the only right method. It is not this that causes some hesitation in accepting Mr.\ Wittgenstein's position, in spite of the very powerful arguments which he brings to its support. What causes hesitation is the fact that, after all, Mr.\ Wittgenstein manages to say a good deal about what cannot be said, thus suggesting to the sceptical reader that possibly there may be some loophole through a hierarchy of languages, or by some other exit. The whole subject of ethics, for example, is placed by Mr.\ Wittgenstein in the mystical, inexpressible region. Nevertheless he is capable of conveying his ethical opinions. His defence would be that what he calls the mystical can be shown, although it cannot be said. It may be that this defence is adequate, but, for my part, I confess that it leaves me with a certain sense of intellectual discomfort.

There is one purely logical problem in regard to which these difficulties are peculiarly acute. I mean the problem of generality. In the theory of generality it is necessary to consider all propositions of the form $f\negthinspace x$ where $f\negthinspace x$ is a given propositional function. This belongs to the part of logic which can be expressed, according to Mr.\ Wittgenstein's system. But the totality of possible values of $x$ which might seem to be involved in the totality of propositions of the form $f\negthinspace x$ is not admitted by Mr.\ Wittgenstein among the things that can be spoken of, for this is no other than the totality of things in the world, and thus involves the attempt to conceive the world as a whole; ``the feeling of the world as a bounded whole is the mystical''; hence the totality of the values of $x$ is mystical (6.45). This is expressly argued when Mr.\ Wittgenstein denies that we can make propositions as to how may things there are in the world, as for example, that there are more than three.

These difficulties suggest to my mind some such possibility as this: that every language has, as Mr.\ Wittgenstein says, a structure concerning which \emph{in the language}, nothing can be said, but that there may be another language dealing with the structure of the first language, and having itself a new structure, and that to this hierarchy of languages there may be no limit. Mr.\ Wittgenstein would of course reply that his whole theory is applicable unchanged to the totality of such languages. The only retort would be to deny that there is any such totality. The totalities concerning which Mr.\ Wittgenstein holds that it is impossible to speak logically are nevertheless thought by him to exist, and are the subject-matter of his mysticism. The totality resulting from our hierarchy would be not merely logically inexpressible, but a fiction, a mere delusion, and in this way the supposed sphere of the mystical would be abolished. Such a hypothesis is very difficult, and I can see objections to it which at the moment I do not know how to answer. Yet I do not see how any easier hypothesis can escape from Mr.\ Wittgenstein's conclusions. Even if this very difficult hypothesis should prove tenable, it would leave untouched a very large part of Mr.\ Wittgenstein's theory, though possibly not the part upon which he himself would wish to lay most stress. As one with a long experience of the difficulties of logic and of the deceptiveness of theories which seem irrefutable, I find myself unable to be sure of the rightness of a theory, merely on the ground that I cannot see any point on which it is wrong. But to have constructed a theory of logic which is not at any point obviously wrong is to have achieved a work of extraordinary difficulty and importance. This merit, in my opinion, belongs to Mr.\ Wittgenstein's book, and makes it one which no serious philosopher can afford to neglect.

\hfill\textsc{Bertrand Russell.}\phantom{xxx} %ENDINTROCONVERT


\textit{May} 1922.
\end{multicols}
% Set up for the tripartitle table for Preface
\clearpage\phantomsection\addcontentsline{toc}{chapter}{Dedication page}\thispagestyle{empty}
\begin{center}
\vspace*{50pt}{\Huge Tractatus Logico-Philosophicus}\\[150pt]
{\textsc{Dedicated}}\\
{\textsc{to the Memory of My Friend}}\\[12pt]
{\Large\textsc{David H. Pinsent}}\\[150pt]
{\germph{Motto}: \ldots und alles, was man weiss, nicht bloss rauschen and brausen geh{\"o}rt hat, l{\"a}sst sich in drei Worten sagen.}
{\quad\textsc{--K{\"u}rnberger.}}
\end{center}
\clearpage\phantomsection\addcontentsline{toc}{chapter}{Preface}\section*{Vorwort (Preface)}
%=============================================
% PREFACE
%=============================================
%
\begin{parcolumns}[sloppy,%
                    rulebetween,
                    colwidths={1={0.05in},2={3.15in},3={3.15in},4={3.15in}}%
                    ]{4}
%
\pnskip
\ger{\negpbk\textbf{German}\\~}
\ogd{\negpbk\textbf{Ogden}\\~}
\pmc{\negpbk\textbf{Pears/McGuinness}\\~}

\pnskip %STARTPREFCONVERT
\ger{\hypertarget{pref1}{Dieses} Buch wird vielleicht nur der verstehen, der die Gedanken, die darin ausgedr{\"u}ckt sind---oder doch {\"a}hnliche Gedanken---schon selbst einmal gedacht hat.---Es ist also kein Lehrbuch.---Sein Zweck w{\"a}re erreicht, wenn es einem, der es mit Verst{\"a}ndnis liest, Vergn{\"u}gen bereitete.}
\ogd{This book will perhaps only be understood by those who have themselves already thought the thoughts which are expressed in it---or similar thoughts. It is therefore not a text-book. Its object would be attained if there were one person who read it with understanding and to whom it afforded pleasure.}
\pmc{Perhaps this book will be understood only by someone who has himself already had the thoughts that are expressed in it---or at least similar thoughts.---So it is not a textbook.---Its purpose would be achieved if it gave pleasure to one person who read and understood it.}

\pnskip
\ger{\hypertarget{pref2}{Das} Buch behandelt die philosophischen Probleme und zeigt---wie ich glaube---dass die Fragestellung dieser Probleme auf dem Missverst{\"a}ndnis der Logik unserer Sprache beruht. Man k{\"o}nnte den ganzen Sinn des Buches etwa in die Worte fassen: Was sich {\"u}berhaupt sagen l{\"a}sst, l{\"a}sst sich klar sagen; und wovon man nicht reden kann, dar{\"u}ber muss man schweigen.}
\ogd{The book deals with the problems of philosophy and shows, as I believe, that the method of formulating these problems rests on the misunderstanding of the logic of our language. Its whole meaning could be summed up somewhat as follows: What can be said at all can be said clearly; and whereof one cannot speak thereof one must be silent.}
\pmc{The book deals with the problems of philosophy, and shows, I believe, that the reason why these problems are posed is that the logic of our language is misunderstood. The whole sense of the book might be summed up in the following words: what can be said at all can be said clearly, and what we cannot talk about we must pass over in silence.}

\pnskip
\ger{\hypertarget{pref3}{Das} Buch will also dem Denken eine Grenze ziehen, oder vielmehr---nicht dem Denken, sondern dem Ausdruck der Gedanken: Denn um dem Denken eine Grenze zu ziehen, m{\"u}ssten wir beide Seiten dieser Grenze denken k{\"o}nnen (wir m{\"u}ssten als denken k{\"o}nnen, was sich nicht denken l{\"a}sst).}
\ogd{The book will, therefore, draw a limit to thinking, or rather---not to thinking, but to the expression of thoughts; for, in order to draw a limit to thinking we should have to be able to think both sides of this limit (we should therefore have to be able to think what cannot be thought).}
\pmc{Thus the aim of the book is to draw a limit to thought, or rather---not to thought, but to the expression of thoughts: for in order to be able to draw a limit to thought, we should have to find both sides of the limit thinkable (i.e.\ we should have to be able to think what cannot be thought).}

\pnskip
\ger{\hypertarget{pref4}{Die} Grenze wird also nur in der Sprache gezogen werden k{\"o}nnen und was jenseits der Grenze liegt, wird einfach Unsinn sein.}
\ogd{The limit can, therefore, only be drawn in language and what lies on the other side of the limit will be simply nonsense.}
\pmc{It will therefore only be in language that the limit can be drawn, and what lies on the other side of the limit will simply be nonsense.}

\pnskip
\ger{\hypertarget{pref5}{Wieweit} meine Bestrebungen mit denen anderer Philosophen zusammenfallen, will ich nicht beurteilen. Ja, was ich hier geschrieben habe, macht im Einzelnen {\"u}berhaupt nicht den Anspruch auf Neuheit; und darum gebe ich auch keine Quellen an, weil es mir gleichg{\"u}ltig ist, ob das was ich gedacht habe, vor mir schon ein anderer gedacht hat.}
\ogd{How far my efforts agree with those of other philosophers I will not decide. Indeed what I have here written makes no claim to novelty in points of detail; and therefore I give no sources, because it is indifferent to me whether what I have thought has already been thought before me by another.}
\pmc{I do not wish to judge how far my efforts coincide with those of other philosophers. Indeed, what I have written here makes no claim to novelty in detail, and the reason why I give no sources is that it is a matter of indifference to me whether the thoughts that I have had have been anticipated by someone else.}

\pnskip
\ger{\hypertarget{pref6}{Nur} das will ich erw{\"a}hnen, dass ich den gro{\ss}artigen Werken Freges und den Arbeiten meines Freundes Herrn Bertrand Russell einen gro{\ss}en Teil der Anregung zu meinen Gedanken schulde.}
\ogd{I will only mention that to the great works of Frege and the writings of my friend Bertrand Russell I owe in large measure the stimulation of my thoughts.}
\pmc{I will only mention that I am indebted to Frege's great works and of the writings of my friend Mr.\ Bertrand Russell for much of the stimulation of my thoughts.}

\pnskip
\ger{\hypertarget{pref7}{Wenn} diese Arbeit einen Wert hat, so besteht er in Zweierlei. Erstens darin, dass in ihr Gedanken ausgedr{\"u}ckt sind, und dieser Wert wird umso gr{\"o}{\ss}er sein, je besser die Gedanken ausgedr{\"u}ckt sind. Je mehr der Nagel auf den Kopf getroffen ist.---Hier bin ich mir bewusst, weit hinter dem M{\"o}glichen zur{\"u}ckgeblieben zu sein. Einfach darum, weil meine Kraft zur Bew{\"a}ltigung der Aufgabe zu gering ist.---M{\"o}gen andere kommen und es besser machen.}
\ogd{If this work has a value it consists in two things. First that in it thoughts are expressed, and this value will be the greater the better the thoughts are expressed. The more the nail has been hit on the head.---Here I am conscious that I have fallen far short of the possible. Simply because my powers are insufficient to cope with the task.---May others come and do it better.}
\pmc{If this work has any value, it consists in two things: the first is that thoughts are expressed in it, and on this score the better the thoughts are expressed---the more the nail has been hit on the head---the greater will be its value.---Here I am conscious of having fallen a long way short of what is possible. Simply because my powers are too slight for the accomplishment of the task.---May others come and do it better.}

\pnskip
\ger{\hypertarget{pref8}{Dagegen} scheint mir die \germph{Wahrheit} der hier mitgeteilten Gedanken unantastbar und definitiv. Ich bin also der Meinung, die Probleme im Wesentlichen endg{\"u}ltig gel{\"o}st zu haben. Und wenn ich mich hierin nicht irre, so besteht nun der Wert dieser Arbeit zweitens darin, dass sie zeigt, wie wenig damit getan ist, dass diese Probleme gel{\"o}st sind.}
\ogd{On the other hand the \emph{truth} of the thoughts communicated here seems to me unassailable and definitive. I am, therefore, of the opinion that the problems have in essentials been finally solved. And if I am not mistaken in this, then the value of this work secondly consists in the fact that it shows how little has been done when these problems have been solved.}
\pmc{On the other hand the \emph{truth} of the thoughts that are here communicated seems to me unassailable and definitive. I therefore believe myself to have found, on all essential points, the final solution of the problems. And if I am not mistaken in this belief, then the second thing in which the of this work consists is that it shows how little is achieved when these problems are solved.}

\pnskip %ENDPREFCONVERT
\ger{\phantom{a} \hfill L.\ W.\\ \textit{Wien, 1918}}
\ogd{\phantom{a} \hfill L.\ W.\\ \textit{Vienna, 1918}}
\pmc{\phantom{a} \hfill L.\ W.\\ \textit{Vienna, 1918}}
\end{parcolumns}\clearpage\phantomsection\addcontentsline{toc}{chapter}{Tractatus Logico-Philosophicus}\section*{Tractatus Logico-Philosophicus}
%
%=============================================
% MAIN BOOK
%=============================================
%
% The footnote to sec. 1
\renewcommand{\thefootnote}{}
\footnotetext{* \kckaddition{[German]} Die Decimalzahlen als Nummern der einzelnen S\"a{}tze deuten das logische Gewicht der S\"a{}tze an, den Nachdruck, der auf ihnen in meiner Darstellung liegt, Die S\"a{}tze $n.1,\thickspace n.2,\thickspace n.3,$ etc., sind Bemerkungen zum S\"a{}tze No.\ $n$; die S\"a{}tze $n.m1,\thickspace n.m2,$ etc.\ Bemerkungen zum Satze No.\ $n.m$; und so weiter.\ / \kckaddition{[Ogden]} The decimal figures as numbers of the separate propositions indicate the logical importance of the propositions, the emphasis laid upon them in my exposition. The propositions $n.1,\thickspace n.2,\thickspace n.3,$ etc., are comments on proposition No.\ $n$; the propositions $n.m1,\thickspace n.m2,$ etc., are comments on the proposition No.\ $n.m$; and so on.\ / \kckaddition{[Pears \& McGuinness]} The decimal numbers assigned to the individual propositions indicate the logical importance of the propositions, the stress laid on them in my exposition. The propositions $n.1,\thickspace n.2,\thickspace n.3,$ etc.\ are comments on proposition no. $n$; the propositions $n.m1,\thickspace n.m2,$ etc.\ are comments on proposition no. $n.m$; and so on.}
% Basic set up for the tripartitle table
\begin{parcolumns}[sloppy,% 
                    rulebetween,
                    colwidths={1={.8in},2={2.9in},3={2.9in},4={2.9in}}%
                    ]{4}
\pnskip
\ger{\negpbk\textbf{German}\\~}
\ogd{\negpbk\textbf{Ogden}\\~}
\pmc{\negpbk\textbf{Pears/McGuinness}\\~}

% STARTBODYCONVERT
\pn{1*}
\ger{Die Welt ist alles, was der Fall ist.}
\ogd{The world is everything that is the case.}
\pmc{The world is all that is the case.}

\pn{1.1}
\ger{Die Welt ist die Gesamtheit der Tatsachen, nicht der Dinge.}
\ogd{The world is the totality of facts, not of things.}
\pmc{The world is the totality of facts, not of things.}

\pn{1.11}
\ger{Die Welt ist durch die Tatsachen bestimmt und dadurch, dass es \germph{alle} Tatsachen sind.}
\ogd{The world is determined by the facts, and by these being \emph{all} the facts.}
\pmc{The world is determined by the facts, and by their being \emph{all} the facts.}

\pn{1.12} 
\ger{Denn, die Gesamtheit der Tatsachen bestimmt, was der Fall ist und auch, was alles nicht der Fall ist.}
\ogd{For the totality of facts determines both what is the case, and also all that is not the case.}
\pmc{For the totality of facts determines what is the case, and also whatever is not the case.}

\pn{1.13}
\ger{Die Tatsachen im logischen Raum sind die Welt.}
\ogd{The facts in logical space are the world.}
\pmc{The facts in logical space are the world.}

\pn{1.2}
\ger{Die Welt zerf{\"a}llt in Tatsachen.}
\ogd{The world divides into facts.}
\pmc{The world divides into facts.}

\pn{1.21}
\ger{Eines kann der Fall sein oder nicht der Fall sein und alles {\"u}brige gleich bleiben.}
\ogd{Any one can either be the case or not be the case, and everything else remain the same.}
\pmc{Each item can be the case or not the case while everything else remains the same.}

\pn{2}
\ger{Was der Fall ist, die Tatsache, ist das Bestehen von Sachverhalten.}
\ogd{What is the case, the fact, is the existence of atomic facts.}
\pmc{What is the case---a fact---is the existence of states of affairs.}

\pn{2.01}
\ger{Der Sachverhalt ist eine Verbindung von Gegenst{\"a}nden. (Sachen, Dingen.)}
\ogd{An atomic fact is a combination of objects (entities, things).}
\pmc{A state of affairs (a state of things) is a combination of objects (things).}

\pn{2.011}
\ger{Es ist dem Ding wesentlich, der Bestandteil eines Sachverhaltes sein zu k{\"o}nnen.}
\ogd{It is essential to a thing that it can be a constituent part of an atomic fact.}
\pmc{It is essential to things that they should be possible constituents of states of affairs.}

\pn{2.012}
\ger{In der Logik ist nichts zuf{\"a}llig: Wenn das Ding im Sachverhalt vorkommen \germph{kann}, so muss die M{\"o}glichkeit des Sachverhaltes im Ding bereits pr{\"a}judiziert sein.}
\ogd{In logic nothing is accidental: if a thing \emph{can} occur in an atomic fact the possibility of that atomic fact must already be prejudged in the thing.}
\pmc{In logic nothing is accidental: if a thing \emph{can} occur in a state of affairs, the possibility of the state of affairs must be written into the thing itself.}

\pn{2.0121}
\ger{Es erschiene gleichsam als Zufall, wenn dem Ding, das allein f{\"u}r sich bestehen k{\"o}nnte, nachtr{\"a}glich eine Sachlage passen w{\"u}rde.}
\ogd{It would, so to speak, appear as an accident, when to a thing that could exist alone on its own account, subsequently a state of affairs could be made to fit.}
\pmc{It would seem to be a sort of accident, if it turned out that a situation would fit a thing that could already exist entirely on its own.}
%
\pnskip%
\ger{Wenn die Dinge in Sachverhalten vorkommen k{\"o}nnen, so muss dies schon in ihnen liegen.}
\ogd{If things can occur in atomic facts, this possibility must already lie in them.}
\pmc{If things can occur in states of affairs, this possibility must
be in them from the beginning. }

\pnskip
\ger{(Etwas Logisches kann nicht nur-m{\"o}glich sein. Die Logik handelt von jeder M{\"o}glichkeit und alle M{\"o}glichkeiten sind ihre Tatsachen.)}
\ogd{(A logical entity cannot be merely possible. Logic treats of every possibility, and all possibilities are its facts.)}
\pmc{(Nothing in the province of logic can be merely possible. Logic deals with every possibility and all possibilities are its facts.)}

\pnskip
\ger{Wie wir uns r{\"a}umliche Gegenst{\"a}nde {\"u}berhaupt nicht au{\ss}erhalb des Raumes, zeitliche nicht au{\ss}erhalb der Zeit denken k{\"o}nnen, so k{\"o}nnen wir uns \germph{keinen} Gegenstand au{\ss}erhalb der M{\"o}glichkeit seiner Verbindung mit anderen denken.}
\ogd{Just as we cannot think of spatial objects at all apart from space, or temporal objects apart from time, so we cannot think of \emph{any} object apart from the possibility of its connexion with other things.}
\pmc{Just as we are quite unable to imagine spatial objects outside space or temporal objects outside time, so too there is \emph{no} object that we can imagine excluded from the possibility of combining with others.}

\pnskip
\ger{Wenn ich mir den Gegenstand im Verbande des Sachverhalts denken kann, so kann ich ihn nicht au{\ss}erhalb der \germph{M{\"o}glichkeit} dieses Verbandes denken.}
\ogd{If I can think of an object in the context of an atomic fact, I cannot think of it apart from the \emph{possibility} of this context.}
\pmc{If I can imagine objects combined in states of affairs, I cannot imagine them excluded from the possibility of such combinations.}

\pn{2.0122}
\ger{Das Ding ist selbst{\"a}ndig, insofern es in allen \germph{m{\"o}glichen} Sachlagen vorkommen kann, aber diese Form der Selbst{\"a}ndigkeit ist eine Form des Zusammenhangs mit dem Sachverhalt, eine Form der Unselbst{\"a}ndigkeit. (Es ist unm{\"o}glich, dass Worte in zwei verschiedenen Weisen auftreten, allein und im Satz.)}
\ogd{The thing is independent, in so far as it can occur in all \emph{possible} circumstances, but this form of independence is a form of connexion with the atomic fact, a form of dependence. (It is impossible for words to occur in two different ways, alone and in the proposition.)}
\pmc{Things are independent in so far as they can occur in all \emph{possible} situations, but this form of independence is a form of connexion with states of affairs, a form of dependence. (It is impossible for words to appear in two different roles: by themselves, and in propositions.)}

\pn{2.0123}
\ger{Wenn ich den Gegenstand kenne, so kenne ich auch s{\"a}mtliche M{\"o}glichkeiten seines Vorkommens in Sachverhalten.}
\ogd{If I know an object, then I also know all the possibilities of its occurrence in atomic facts.}
\pmc{If I know an object I also know all its possible occurrences in states of affairs.}

\pnskip
\ger{(Jede solche M{\"o}glichkeit muss in der Natur des Gegenstandes liegen.)}
\ogd{(Every such possibility must lie in the nature of the object.)}
\pmc{(Every one of these possibilities must be part of the nature of the object.)}

\pnskip
\ger{Es kann nicht nachtr{\"a}glich eine neue M{\"o}glichkeit gefunden werden.}
\ogd{A new possibility cannot subsequently be found.}
\pmc{A new possibility cannot be discovered later.}

\pn{2.01231}
\ger{Um einen Gegenstand zu kennen, muss ich zwar nicht seine externen---aber ich muss alle seine internen Eigenschaften kennen.}
\ogd{In order to know an object, I must know not its external but all its internal qualities.}
\pmc{If I am to know an object, though I need not know its external properties, I must know all its internal properties.}

\pn{2.0124}
\ger{Sind alle Gegenst{\"a}nde gegeben, so sind damit auch alle \germph{m{\"o}glichen} Sachverhalte gegeben.}
\ogd{If all objects are given, then thereby are all \emph{possible} atomic facts also given.}
\pmc{If all objects are given, then at the same time all \emph{possible} states of affairs are also given.}

\pn{2.013}
\ger{Jedes Ding ist, gleichsam, in einem Raume m{\"o}glicher Sachverhalte. Diesen Raum kann ich mir leer denken, nicht aber das Ding ohne den Raum.}
\ogd{Every thing is, as it were, in a space of possible atomic facts. I can think of this space as empty, but not of the thing without the space.}
\pmc{Each thing is, as it were, in a space of possible states of affairs. This space I can imagine empty, but I cannot imagine the thing without the space.}

\pn{2.0131}
\ger{Der r{\"a}umliche Gegenstand muss im unendlichen Raume liegen. (Der Raumpunkt ist eine Argumentstelle.)}
\ogd{A spatial object must lie in infinite space. (A point in space is an argument place.)}
\pmc{A spatial object must be situated in infinite space. (A spatial point is an argument-place.) }

\pnskip
\ger{Der Fleck im Gesichtsfeld muss zwar nicht rot sein, aber eine Farbe muss er haben: er hat sozusagen den Farbenraum um sich. Der Ton muss \germph{eine} H{\"o}he haben, der Gegenstand des Tastsinnes \germph{eine} H{\"a}rte, usw.}
\ogd{A speck in a visual field need not be red, but it must have a colour; it has, so to speak, a colour space round it. A tone must have \emph{a} pitch, the object of the sense of touch \emph{a} hardness, etc.}
\pmc{A speck in the visual field, thought it need not be red, must have some colour: it is, so to speak, surrounded by colour-space. Notes must have \emph{some} pitch, objects of the sense of touch \emph{some} degree of hardness, and so on.}

\pn{2.014}
\ger{Die Gegenst{\"a}nde enthalten die M{\"o}glichkeit aller Sachlagen.}
\ogd{Objects contain the possibility of all states of affairs.}
\pmc{Objects contain the possibility of all situations.}

\pn{2.0141}
\ger{Die M{\"o}glichkeit seines Vorkommens in Sachverhalten, ist die Form des Gegenstandes.}
\ogd{The possibility of its occurrence in atomic facts is the form of the object.}
\pmc{The possibility of its occurring in states of affairs is the form of an object.}

\pn{2.02}
\ger{Der Gegenstand ist einfach.}
\ogd{The object is simple.}
\pmc{Objects are simple.}

\pn{2.0201}
\ger{Jede Aussage {\"u}ber Komplexe l{\"a}sst sich in eine Aussage {\"u}ber deren Bestandteile und in diejenigen S{\"a}tze zerlegen, welche die Komplexe vollst{\"a}ndig beschreiben.}
\ogd{Every statement about complexes can be analysed into a statement about their constituent parts, and into those propositions which completely describe the complexes.}
\pmc{Every statement about complexes can be resolved into a statement about their constituents and into the propositions that describe the complexes completely.}

\pn{2.021}
\ger{Die Gegenst{\"a}nde bilden die Substanz der Welt. Darum k{\"o}nnen sie nicht zusammengesetzt sein.}
\ogd{Objects form the substance of the world. Therefore they cannot be compound.}
\pmc{Objects make up the substance of the world. That is why they cannot be composite.}

\pn{2.0211}
\ger{H{\"a}tte die Welt keine Substanz, so w{\"u}rde, ob ein Satz Sinn hat, davon abh{\"a}ngen, ob ein anderer Satz wahr ist.}
\ogd{If the world had no substance, then whether a proposition had sense would depend on whether another proposition was true.}
\pmc{If the world had no substance, then whether a proposition had sense would depend on whether another proposition was true.}

\pn{2.0212}
\ger{Es w{\"a}re dann unm{\"o}glich, ein Bild der Welt (wahr oder falsch) zu entwerfen.}
\ogd{It would then be impossible to form a picture of the world (true or false).}
\pmc{In that case we could not sketch any picture of the world (true or false).}

\pn{2.022}
\ger{Es ist offenbar, dass auch eine von der wirklichen noch so verschieden gedachte Welt Etwas---eine Form---mit der wirklichen gemein haben muss.}
\ogd{It is clear that however different from the real one an imagined world may be, it must have something---a form---in common with the real world.}
\pmc{It is obvious that an imagined world, however different it may be from the real one, must have \emph{something}---a form---in common with it.}%???

\pn{2.023}
\ger{Diese feste Form besteht eben aus den Gegenst{\"a}nden.}
\ogd{This fixed form consists of the objects.}
\pmc{Objects are just what constitute this unalterable form.}

\pn{2.0231}
\ger{Die Substanz der Welt \germph{kann} nur eine Form und keine materiellen Eigenschaften bestimmen. Denn diese werden erst durch die S{\"a}tze dargestellt---erst durch die Konfiguration der Gegenst{\"a}nde gebildet.}
\ogd{The substance of the world \emph{can} only determine a form and not any material properties. For these are first presented by the propositions---first formed by the configuration of the objects.}
\pmc{The substance of the world \emph{can} only determine a form, and not any material properties. For it is only by means of propositions that material properties are represented---only by the configuration of objects that they are produced.}

\pn{2.0232}
\ger{Beil{\"a}ufig gesprochen: Die Gegenst{\"a}nde sind farblos.}
\ogd{Roughly speaking: objects are colourless.}
\pmc{In a manner of speaking, objects are colourless.}

\pn{2.0233}
\ger{Zwei Gegenst{\"a}nde von der gleichen logischen Form sind---abgesehen von ihren externen Eigenschaften---von einander nur dadurch unterschieden, dass sie verschieden sind.}
\ogd{Two objects of the same logical form are---apart from their external properties---only differentiated from one another in that they are different.}
\pmc{If two objects have the same logical form, the only distinction between them, apart from their external properties, is that they are different.}

\pn{2.02331}
\ger{Entweder ein Ding hat Eigenschaften, die kein anderes hat, dann kann man es ohneweiteres durch eine Beschreibung aus den anderen herausheben, und darauf hinweisen; oder aber, es gibt mehrere Dinge, die ihre s{\"a}mtlichen Eigenschaften gemeinsam haben, dann ist es {\"u}berhaupt unm{\"o}glich auf eines von ihnen zu zeigen.}
\ogd{Either a thing has properties which no other has, and then one can distinguish it straight away from  the others by a description and refer to it; or, on the other hand, there are several things which have the totality of their properties in common, and then it is quite impossible to point to any one of them.}
\pmc{Either a thing has properties that nothing else has, in which
case we can immediately use a description to distinguish it from the
others and refer to it; or, on the other hand, there are several things
that have the whole set of their properties in common, in which case it
is quite impossible to indicate one of them.}

\pnskip
\ger{Denn, ist das Ding durch nichts hervorgehoben, so kann ich es nicht hervorheben, denn sonst ist es eben hervorgehoben.}
\ogd{For it a thing is not distinguished by anything, I cannot distinguish it---for otherwise it would be distinguished.}
\pmc{For it there is nothing to distinguish a thing, I cannot distinguish it, since otherwise it would be distinguished after all.}

\pn{2.024}
\ger{Die Substanz ist das, was unabh{\"a}ngig von dem was der Fall ist, besteht.}
\ogd{Substance is what exists independently of what is the case.}
\pmc{The substance is what subsists independently of what is the case.}

\pn{2.025}
\ger{Sie ist Form und Inhalt.}
\ogd{It is form and content.}
\pmc{It is form and content.}

\pn{2.0251}
\ger{Raum, Zeit und Farbe (F{\"a}rbigkeit) sind Formen der Gegenst{\"a}nde.}
\ogd{Space, time and colour (colouredness) are forms of objects.}
\pmc{Space, time, colour (being coloured) are forms of objects.}

\pn{2.026}
\ger{Nur wenn es Gegenst{\"a}nde gibt, kann es eine feste Form der Welt geben.}
\ogd{Only if there are objects can there be a fixed form of the world.}
\pmc{There must be objects, if the world is to have unalterable form.}

\pn{2.027}
\ger{Das Feste, das Bestehende und der Gegenstand sind Eins.}
\ogd{The fixed, the existent and the object are one.}
\pmc{Objects, the unalterable, and the subsistent are one and the same.}

\pn{2.0271}
\ger{Der Gegenstand ist das Feste, Bestehende; die Konfiguration ist das Wechselnde, Unbest{\"a}ndige.}
\ogd{The object is the fixed, the existent; the configuration is the changing, the variable.}
\pmc{Objects are what is unalterable and subsistent; their configuration is what is changing and unstable.}

\pn{2.0272}
\ger{Die Konfiguration der Gegenst{\"a}nde bildet den Sachverhalt.}
\ogd{The configuration of the objects forms the atomic fact.}
\pmc{The configuration of objects produces states of affairs.}

\pn{2.03}
\ger{Im Sachverhalt h{\"a}ngen die Gegenst{\"a}nde ineinander, wie die Glieder einer Kette.}
\ogd{In the atomic fact objects hang one in another, like the links of a chain.}
\pmc{In a state of affairs objects fit into one another like the links of a chain.}

\pn{2.031}
\ger{Im Sachverhalt verhalten sich die Gegenst{\"a}nde in bestimmter Art und Weise zueinander.}
\ogd{In the atomic fact the objects are combined in a definite way.}
\pmc{In a state of affairs objects stand in a determinate relation to one another.}

\pn{2.032}
\ger{Die Art und Weise, wie die Gegenst{\"a}nde im Sachverhalt zusammenh{\"a}ngen, ist die Struktur des Sachverhaltes.}
\ogd{The way in which objects hang together in the atomic fact is the structure of the atomic fact.}
\pmc{The determinate way in which objects are connected in a state of affairs is the structure of the state of affairs.}

\pn{2.033}
\ger{Die Form ist die M{\"o}glichkeit der Struktur.}
\ogd{The form is the possibility of the structure.}
\pmc{Form is the possibility of structure.}

\pn{2.034}
\ger{Die Struktur der Tatsache besteht aus den Strukturen der Sachverhalte.}
\ogd{The structure of the fact consists of the structures of the atomic facts.}
\pmc{The structure of a fact consists of the structures of states of affairs.}

\pn{2.04}
\ger{Die Gesamtheit der bestehenden Sachverhalte ist die Welt.}
\ogd{The totality of existent atomic facts is the world.}
\pmc{The totality of existing states of affairs is the world.}

\pn{2.05}
\ger{Die Gesamtheit der bestehenden Sachverhalte bestimmt auch, welche Sachverhalte nicht bestehen.}
\ogd{The totality of existent atomic facts also determines which atomic facts do not exist.}
\pmc{The totality of existing states of affairs also determines which states of affairs do not exist.}

\pn{2.06}
\ger{Das Bestehen und Nichtbestehen von Sachverhalten ist die Wirklichkeit.}
\ogd{The existence and non-existence of atomic facts is the reality.}
\pmc{The existence and non-existence of states of affairs is reality.}

\pnskip
\ger{(Das Bestehen von Sachverhalten nennen wir auch eine positive, das Nichtbestehen eine negative Tatsache.)}
\ogd{(The existence of atomic facts we also call a positive fact, their non-existence a negative fact.)}
\pmc{(We call the existence of states of affairs a positive fact, and their non-existence a negative fact.)}

\pn{2.061}
\ger{Die Sachverhalte sind von einander unabh{\"a}ngig.}
\ogd{Atomic facts are independent of one another.}
\pmc{States of affairs are independent of one another.}

\pn{2.062}
\ger{Aus dem Bestehen oder Nichtbestehen eines Sachverhaltes kann nicht auf das Bestehen oder Nichtbestehen eines anderen geschlossen werden.}
\ogd{From the existence of non-existence of an atomic fact we cannot infer the existence of non-existence of another.}
\pmc{From the existence or non-existence of one state of affairs it is impossible to infer the existence or non-existence of another.}

\pn{2.063}
\ger{Die gesamte Wirklichkeit ist die Welt.}
\ogd{The total reality is the world.}
\pmc{The sum-total of reality is the world.}

\pn{2.1}
\ger{Wir machen uns Bilder der Tatsachen.}
\ogd{We make to ourselves pictures of facts.}
\pmc{We picture facts to ourselves.}

\pn{2.11}
\ger{Das Bild stellt die Sachlage im logischen Raume, das Bestehen und Nichtbestehen von Sachverhalten vor.}
\ogd{The picture presents the facts in logical space, the existence and non-existence of atomic facts.}
\pmc{A picture presents a situation in logical space, the existence and non-existence of states of affairs.}

\pn{2.12}
\ger{Das Bild ist ein Modell der Wirklichkeit.}
\ogd{The picture is a model of reality.}
\pmc{A picture is a model of reality.}

\pn{2.13}
\ger{Den Gegenst{\"a}nden entsprechen im Bilde die Elemente des Bildes.}
\ogd{To the objects correspond in the picture the elements of the picture.}
\pmc{In a picture objects have the elements of the picture corresponding to them.}

\pn{2.131}
\ger{Die Elemente des Bildes vertreten im Bild die Gegenst{\"a}nde.}
\ogd{The elements of the picture stand, in the picture, for the objects.}
\pmc{In a picture the elements of the picture are the representatives of objects.}

\pn{2.14}
\ger{Das Bild besteht darin, dass sich seine Elemente in bestimmter Art und Weise zu einander verhalten.}
\ogd{The picture consists in the fact that its elements are combined with one another in a definite way.}
\pmc{What constitutes a picture is that its elements are related to one another in a determinate way.}

\pn{2.141}
\ger{Das Bild ist eine Tatsache.}
\ogd{The picture is a fact.}
\pmc{A picture is a fact.}

\pn{2.15}
\ger{Dass sich die Elemente des Bildes in bestimmter Art und Weise zu einander verhalten, stellt vor, dass sich die Sachen so zu einander verhalten.}
\ogd{That the elements of the picture are combined with one another in a definite way, represents that the things are so combined with one another.}
\pmc{The fact that the elements of a picture are related to one another in a determinate way represents that things are related to one another in the same way.}

\pnskip
\ger{Dieser Zusammenhang der Elemente des Bildes hei{\ss}e seine Struktur und ihre M{\"o}glichkeit seine Form der Abbildung.}
\ogd{This connexion of the elements of the picture is called its structure, and the possibility of this structure is called the form of representation of the picture.}
\pmc{Let us call this connexion of its elements the structure of the picture, and let us call the possibility of this structure the pictorial form of the picture.}

\pn{2.151}
\ger{Die Form der Abbildung ist die M{\"o}glichkeit, dass sich die Dinge so zu einander verhalten, wie die Elemente des Bildes.}
\ogd{The form of representation is the possibility that the things are combined with one another as are the elements of the picture.}
\pmc{Pictorial form is the possibility that things are related to one another in the same way as the elements of the picture.}

\pn{2.1511}
\ger{Das Bild ist \germph{so} mit der Wirklichkeit verkn{\"u}pft---es reicht bis zu ihr.}
\ogd{\emph{Thus} the picture is linked with reality; it reaches up to it.}
\pmc{\emph{That} is how a picture is attached to reality; it reaches right out to it.}

\pn{2.1512}
\ger{Es ist wie ein Ma{\ss}stab an die Wirklichkeit angelegt.}
\ogd{It is like a scale applied to reality.}
\pmc{It is laid against reality like a measure.}

\pn{2.15121}
\ger{Nur die {\"a}u{\ss}ersten Punkte der Teilstriche \germph{ber{\"u}hren} den zu messenden Gegenstand.}
\ogd{Only the outermost points of the dividing lines \emph{touch} the object to be measured.}
\pmc{Only the end-points of the graduating lines actually \emph{touch} the object that is to be measured.}

\pn{2.1513}
\ger{Nach dieser Auffassung geh{\"o}rt also zum Bilde auch noch die abbildende Beziehung, die es zum Bild macht.}
\ogd{According to this view the representing relation which makes it a picture, also belongs to the picture.}
\pmc{So a picture, conceived in this way, also includes the pictorial relationship, which makes it into a picture.}

\pn{2.1514}
\ger{Die abbildende Beziehung besteht aus den Zuordnungen der Elemente des Bildes und der Sachen.}
\ogd{The representing relation consists of the co-ordinations of the elements of the picture and the things.}
\pmc{The pictorial relationship consists of the correlations of the picture's elements with things.}%???

\pn{2.1515}
\ger{Diese Zuordnungen sind gleichsam die F{\"u}hler der Bildelemente, mit denen das Bild die Wirklichkeit ber{\"u}hrt.}
\ogd{These co-ordinations are as it were the feelers of its elements with which the picture touches reality.}
\pmc{These correlations are, as it were, the feelers of the picture's elements, with which the picture touches reality.}

\pn{2.16}
\ger{Die Tatsache muss, um Bild zu sein, etwas mit dem Abgebildeten gemeinsam haben.}
\ogd{In order to be a picture a fact must have something in common with what it pictures.}
\pmc{If a fact is to be a picture, it must have something in common with what it depicts.}

\pn{2.161}
\ger{In Bild und Abgebildetem muss etwas identisch sein, damit das eine {\"u}berhaupt ein Bild des anderen sein kann.}
\ogd{In the picture and the pictured there must be something identical in order that the one can be a picture of the other at all.}
\pmc{There must be something identical in a picture and what it depicts, to enable the one to be a picture of the other at all.}

\pn{2.17}
\ger{Was das Bild mit der Wirklichkeit gemein haben muss, um sie auf seine Art und Weise---richtig oder falsch---abbilden zu k{\"o}nnen, ist seine Form der Abbildung.}
\ogd{What the picture must have in common with reality in order to be able to represent it after is manner---rightly or falsely---is its form of representation.}
\pmc{What a picture must have in common with reality, in order to be able to depict it---correctly or incorrectly---in the way that it does, is its pictorial form.}

\pn{2.171}
\ger{Das Bild kann jede Wirklichkeit abbilden, deren Form es hat.}
\ogd{The picture can represent every reality whose form it has.}
\pmc{A picture can depict any reality whose form it has.}

\pnskip
\ger{Das r{\"a}umliche Bild alles R{\"a}umliche, das farbige alles Farbige, etc.}
\ogd{The spatial picture, everything spatial, the coloured, everything coloured, etc.}
\pmc{A spatial picture can depict anything spatial, a coloured one anything coloured, etc.}

\pn{2.172}
\ger{Seine Form der Abbildung aber, kann das Bild nicht abbilden; es weist sie auf.}
\ogd{The picture, however, cannot represent its form of representation; it shows it forth.}
\pmc{A picture cannot, however, depict its pictorial form: it displays it.}

\pn{2.173}
\ger{Das Bild stellt sein Objekt von au{\ss}erhalb dar (sein Standpunkt ist seine Form der Darstellung), darum stellt das Bild sein Objekt richtig oder falsch dar.}
\ogd{The picture represents its object from without (its standpoint is its form of representation), therefore the picture represents its object rightly or falsely.}
\pmc{A picture represents its subject from a position outside it. (Its standpoint is its representational form.) That is why a picture represents its subject correctly or incorrectly.}

\pn{2.174}
\ger{Das Bild kann sich aber nicht au{\ss}erhalb seiner Form der Darstellung stellen.}
\ogd{But the picture cannot place itself outside of its form of representation.}
\pmc{A picture cannot, however, place itself outside its representational form.}

\pn{2.18}
\ger{Was jedes Bild, welcher Form immer, mit der Wirklichkeit gemein haben muss, um sie {\"u}berhaupt---richtig oder falsch---abbilden zu k{\"o}nnen, ist die logische Form, das ist, die Form der Wirklichkeit.}
\ogd{What every picture, of whatever form, must have in common with reality in order to be able to represent it at all---rightly or falsely---is the logical form, that is, the form of reality.}
\pmc{What any picture, of whatever form, must have in common with reality, in order to be able to depict it---correctly or incorrectly---in any way at all, is logical form, i.e.\ the form of reality.}

\pn{2.181}
\ger{Ist die Form der Abbildung die logische Form, so hei{\ss}t das Bild das logische Bild.}
\ogd{If the form of representation is the logical form, then the picture is called a logical picture.}
\pmc{A picture whose pictorial form is logical form is called a logical picture.}

\pn{2.182}
\ger{Jedes Bild ist \germph{auch} ein logisches. (Dagegen ist z.\ B.\ nicht jedes Bild ein r{\"a}umliches.)}
\ogd{Every picture is \emph{also} a logical picture. (On the other hand, for example, not every picture is spatial.)}
\pmc{Every picture is \emph{at the same time} a logical one. (On the other hand, not every picture is, for example, a spatial one.)}

\pn{2.19}
\ger{Das logische Bild kann die Welt abbilden.}
\ogd{The logical picture can depict the world.}
\pmc{Logical pictures can depict the world.}

\pn{2.2}
\ger{Das Bild hat mit dem Abgebildeten die logische Form der Abbildung gemein.}
\ogd{The picture has the logical form of representation in common with what it pictures.}
\pmc{A picture has logico-pictorial form in common with what it depicts.}

\pn{2.201}
\ger{Das Bild bildet die Wirklichkeit ab, indem es eine M{\"o}glichkeit des Bestehens und Nichtbestehens von Sachverhalten darstellt.}
\ogd{The picture depicts reality by representing a possibility of the existence and non-existence of atomic facts.}
\pmc{A picture depicts reality by representing a possibility of existence and non-existence of states of affairs.}

\pn{2.202}
\ger{Das Bild stellt eine m{\"o}gliche Sachlage im logischen Raume dar.}
\ogd{The picture represents a possible state of affairs in logical space.}
\pmc{A picture represents a possible situation in logical 
space.}%???

\pn{2.203}
\ger{Das Bild enth{\"a}lt die M{\"o}glichkeit der Sachlage, die es darstellt.}
\ogd{The picture contains the possibility of the state of affairs which it represents.}
\pmc{A picture contains the possibility of the situation that it represents.}

\pn{2.21}
\ger{Das Bild stimmt mit der Wirklichkeit {\"u}berein oder nicht; es ist richtig oder unrichtig, wahr oder falsch.}
\ogd{The picture agrees with reality or not; it is right or wrong, true or false.}
\pmc{A picture agrees with reality or fails to agree; it is correct or incorrect, true or false.}

\pn{2.22}
\ger{Das Bild stellt dar, was es darstellt, unabh{\"a}ngig von seiner Wahr- oder Falschheit, durch die Form der Abbildung.}
\ogd{The picture represents what it represents, independently of its truth or falsehood, through the form of representation.}
\pmc{What a picture represents it represents independently of its truth or falsity, by means of its pictorial form.}

\pn{2.221}
\ger{Was das Bild darstellt, ist sein Sinn.}
\ogd{What the picture represents is its sense.}
\pmc{What a picture represents is its sense.}

\pn{2.222}
\ger{In der {\"U}bereinstimmung oder Nicht{\"u}bereinstimmung seines Sinnes mit der Wirklichkeit, besteht seine Wahrheit oder Falschheit.}
\ogd{In the agreement or disagreement of its sense with reality, its truth or falsity consists.}
\pmc{The agreement or disagreement or its sense with reality constitutes its truth or falsity.}

\pn{2.223}
\ger{Um zu erkennen, ob das Bild wahr oder falsch ist, m{\"u}ssen wir es mit der Wirklichkeit vergleichen.}
\ogd{In order to discover whether the picture is true or false we must compare it with reality.}
\pmc{In order to tell whether a picture is true or false we must compare it with reality.}

\pn{2.224}
\ger{Aus dem Bild allein ist nicht zu erkennen, ob es wahr oder falsch ist.}
\ogd{It cannot be discovered from the picture alone whether it is true or false.}
\pmc{It is impossible to tell from the picture alone whether it is true or false.}

\pn{2.225}
\ger{Ein a priori wahres Bild gibt es nicht.}
\ogd{There is no picture which is a priori true.}
\pmc{There are no pictures that are true \emph{a priori}.}

\pn{3}
\ger{Das logische Bild der Tatsachen ist der Gedanke.}
\ogd{The logical picture of the facts is the thought.}
\pmc{A logical picture of facts is a thought.}

\pn{3.001}
\ger{\gdql Ein Sachverhalt ist denkbar\gdqr{} hei{\ss}t: Wir k{\"o}nnen uns ein Bild von ihm machen.}
\ogd{``An atomic fact is thinkable''---means: we can imagine it.}
\pmc{`A state of affairs is thinkable': what this means is that we can
picture it to ourselves.}

\pn{3.01}
\ger{Die Gesamtheit der wahren Gedanken sind ein Bild der Welt.}
\ogd{The totality of true thoughts is a picture of the world.}
\pmc{The totality of true thoughts is a picture of the world.}

\pn{3.02}
\ger{Der Gedanke enth{\"a}lt die M{\"o}glichkeit der Sachlage, die er denkt. Was denkbar ist, ist auch m{\"o}glich.}
\ogd{The thought contains the possibility of the state of affairs which it thinks. What is thinkable is also possible.}
\pmc{A thought contains the possibility of the situation of which it is the thought. What is thinkable is possible too.}

\pn{3.03}
\ger{Wir k{\"o}nnen nichts Unlogisches denken, weil wir sonst unlogisch denken m{\"u}ssten.}
\ogd{We cannot think anything unlogical, for otherwise we should have to think unlogically.}
\pmc{Thought can never be of anything illogical, since, if it were, we should have to think illogically.}

\pn{3.031}
\ger{Man sagte einmal, dass Gott alles schaffen k{\"o}nne, nur nichts, was den logischen Gesetzen zuwider w{\"a}re.---Wir k{\"o}nnen n{\"a}mlich von einer \gdql unlogischen\gdqr{} Welt nicht \germph{sagen}, wie sie auss{\"a}he.}
\ogd{It used to be said that God could create everything, except what was contrary to the laws of logic. The truth is, we could not \emph{say} of an ``unlogical'' world how it would look.}
\pmc{It used to be said that God could create anything except what would be contrary to the laws of logic. The truth is that we could not \emph{say} what an `illogical' world would look like.}

\pn{3.032}
\ger{Etwas \gdql der Logik widersprechendes\gdqr{} in der Sprache darstellen, kann man ebensowenig, wie in der Geometrie eine den Gesetzen des Raumes widersprechende Figur durch ihre Koordinaten darstellen; oder die Koordinaten eines Punktes angeben, welcher nicht existiert.}
\ogd{To present in language anything which ``contradicts logic'' is as impossible as in geometry to present by its co-ordinates a figure which contradicts the laws of space; or to give the co-ordinates of a point which does not exist.}
\pmc{It is as impossible to represent in language anything that `contradicts logic' as it is in geometry to represent by its co-ordinates a figure that contradicts the laws of space, or to give the co-ordinates of a point that does not exist.}

\pn{3.0321}
\ger{Wohl k{\"o}nnen wir einen Sachverhalt r{\"a}umlich darstellen, welcher den Gesetzen der Physik, aber keinen, der den Gesetzen der Geometrie zuwiderliefe.}
\ogd{We could present spatially an atomic fact which contradicted the laws of physics, but not one which contradicted the laws of geometry.}
\pmc{Though a state of affairs that would contravene the laws of physics can be represented by us spatially, one that would contravene the laws of geometry cannot.}

\pn{3.04}
\ger{Ein a priori richtiger Gedanke w{\"a}re ein solcher, dessen M{\"o}glichkeit seine Wahrheit bedingte.}
\ogd{An a priori true thought would be one whose possibility guaranteed its truth.}
\pmc{If a thought were correct \emph{a priori}, it would be a thought whose possibility ensured its truth.}

\pn{3.05}
\ger{Nur so k{\"o}nnten wir a priori wissen, dass ein Gedanke wahr ist, wenn aus dem Gedanken selbst (ohne Vergleichsobjekt) seine Wahrheit zu erkennen w{\"a}re.}
\ogd{Only if we could know a priori that a thought is true if its truth was to be recognized from the thought itself (without an object of comparison).}
\pmc{\emph{A priori} knowledge that a thought was true would be possible only if its truth were recognizable from the thought itself (without anything to compare it with).}

\pn{3.1}
\ger{Im Satz dr{\"u}ckt sich der Gedanke sinnlich wahrnehmbar aus.}
\ogd{In the proposition the thought is expressed perceptibly through the senses.}
\pmc{In a proposition a thought finds an expression that can be perceived by the senses.}

\pn{3.11}
\ger{Wir ben{\"u}tzen das sinnlich wahrnehmbare Zeichen (Laut- oder Schriftzeichen etc.)\ des Satzes als Projektion der m{\"o}glichen Sachlage.}
\ogd{We use the sensibly perceptible sign (sound or
written sign, etc.)\ of the proposition as a projection of the possible
state of affairs.}
\pmc{We use the perceptible sign of a proposition (spoken or written,
etc.)\ as a projection of a possible situation.}

\pnskip
\ger{Die Projektionsmethode ist das Denken des Satz-Sinnes.}
\ogd{The method of projection is the thinking of the sense of the proposition.}
\pmc{The method of projection is to think of the sense of the proposition.}

\pn{3.12}
\ger{Das Zeichen, durch welches wir den Gedanken ausdr{\"u}cken, nenne ich das Satzzeichen. Und der Satz ist das Satzzeichen in seiner projektiven Beziehung zur Welt.}
\ogd{The sign through which we express the though I call the propositional sign. And the proposition is the propositional sign in its projective relation to the world.}
\pmc{I call the sign with which we express a thought a propositional sign.---And a proposition is a propositional sign in its projective relation to the world.}

\pn{3.13}
\ger{Zum Satz geh{\"o}rt alles, was zur Projektion geh{\"o}rt; aber nicht das Projizierte.}
\ogd{To the proposition belongs everything which belongs to the projection; but not what is projected.}
\pmc{A proposition includes all that the projection includes, but not what is projected.}%???

\pnskip
\ger{Also die M{\"o}glichkeit des Projizierten, aber nicht dieses selbst.}
\ogd{Therefore the possibility of what is projected but not this itself.}
\pmc{Therefore, though what is projected is not itself included, its possibility is.}%???

\pnskip
\ger{Im Satz ist also sein Sinn noch nicht enthalten, wohl aber die M{\"o}glichkeit, ihn auszud{\"u}cken.}
\ogd{In the proposition, therefore, its sense is not yet contained, but the possibility of expressing it.}
\pmc{A proposition, therefore, does not actually contain its sense, but does contain the possibility of expressing it.}

\pnskip
\ger{(\gdql Der Inhalt des Satzes\gdqr{} hei{\ss}t der Inhalt des sinnvollen Satzes.)}
\ogd{(``The content of the proposition'' means the content of the significant proposition.)}
\pmc{(`The content of a proposition' means the content of a proposition that has sense.) }

\pnskip
\ger{Im Satz ist die Form seines Sinnes enthalten, aber nicht dessen Inhalt.}
\ogd{In the proposition the form of its sense is contained, but not its content.}
\pmc{A proposition contains the form, but not the content, of its sense.}

\pn{3.14}
\ger{Das Satzzeichen besteht darin, dass sich seine Elemente, die W{\"o}rter, in ihm auf bestimmte Art und Weise zu einander verhalten.}
\ogd{The propositional sign consists in the fact that its elements, the words, are combined in it in a definite way.}
\pmc{What constitutes a propositional sign is that in its elements (the words) stand in a determinate relation to one another.}

\pnskip
\ger{Das Satzzeichen ist eine Tatsache.}
\ogd{The propositional sign is a fact.}
\pmc{A propositional sign is a fact.}

\pn{3.141}
\ger{Der Satz ist kein W{\"o}rtergemisch.---(Wie das musikalische Thema kein Gemisch von T{\"o}nen.)}
\ogd{The proposition is not a mixture of words (just as the musical theme is not a mixture of tones).}
\pmc{A proposition is not a blend of words.---(Just as a theme in music is not a blend of notes.)}

\pnskip
\ger{Der Satz ist artikuliert.}
\ogd{The proposition is articulate.}
\pmc{A proposition is articulate.}

\pn{3.142}
\ger{Nur Tatsachen k{\"o}nnen einen Sinn ausdr{\"u}cken, eine Klasse von Namen kann es nicht.}
\ogd{Only facts can express a sense, a class of names cannot.}
\pmc{Only facts can express a sense, a set of names cannot.}

\pn{3.143}
\ger{Dass das Satzzeichen eine Tatsache ist, wird durch die gew{\"o}hnliche Ausdrucksform der Schrift oder des Druckes verschleiert.}
\ogd{That the propositional sign is a fact is concealed by the ordinary form of expression, written or printed.}
\pmc{Although a propositional sign is a fact, this is obscured by the usual form of expression in writing or print.}

\pnskip
\ger{Denn im gedruckten Satz z.\ B.\ sieht das Satzzeichen nicht wesentlich verschieden aus vom Wort.}
\ogd{For in the printed proposition, for example, the sign of a proposition does not appear essentially different from a word.}
\pmc{For in a printed proposition, for example, no essential difference is apparent between a propositional sign and a word.}

\pnskip
\ger{(So war es m{\"o}glich, dass Frege den Satz einen zusammengesetzten Namen nannte.)}
\ogd{(Thus it was possible for Frege to call the proposition a compounded name.)}
\pmc{(That is what made it possible for Frege to call a proposition a composite name.)}

\pn{3.1431}
\ger{Sehr klar wird das Wesen des Satzzeichens, wenn wir es uns, statt aus Schriftzeichen, aus r{\"a}umlichen Gegenst{\"a}nden (etwa Tischen, St{\"u}hlen, B{\"u}chern) zusammengesetzt denken.}
\ogd{The essential nature of the propositional sign becomes very clear when we imagine it made up of spatial objects (such as tables, chairs, books) instead of written signs.}
\pmc{The essence of a propositional sign is very clearly seen if we imagine one composed of spatial objects (such as tables, chairs, and books) instead of written signs.}

\pnskip
\ger{Die gegenseitige r{\"a}umliche Lage dieser Dinge dr{\"u}ckt dann den Sinn des Satzes aus.}
\ogd{The mutual spatial position of these things then expresses the sense of the proposition.}
\pmc{Then the spatial arrangement of these things will express the sense of the proposition.}%???

\pn{3.1432}
\ger{Nicht: \gdql Das komplexe Zeichen {\gsql}$aRb${\gsqr} sagt, dass $a$ in der Beziehung $R$ zu $b$ steht\gdqr{}, sondern: \germph{Dass} \gdql $a$\gdqr{} in einer gewissen Beziehung zu \gdql $b$\gdqr{} steht, sagt, \germph{dass} $aRb$.}
\ogd{We must not say, ``The complex sign `$aRb$' says `$a$ stands in relation $R$ to $b$'{}''; but we must say, ``\emph{That} `$a$' stands in a certain relation to `$b$' says \emph{that} $aRb$''.}
\pmc{Instead of, `The complex sign ``$aRb$'' says that $a$ stands to $b$ in
the relation $R$', we ought to put, `\emph{That} ``$a$'' stands to ``$b$'' in a certain
relation says \emph{that} $aRb$.'}

\pn{3.144}
\ger{Sachlagen kann man beschreiben, nicht \germph{benennen}.}
\ogd{States of affairs can be described but not \emph{named}.}
\pmc{Situations can be described but not \emph{given names}.}

\pnskip
\ger{(Namen gleichen Punkten, S{\"a}tze Pfeilen, sie haben Sinn.)}
\ogd{(Names resemble points; propositions resemble arrows, they have sense.)}
\pmc{(Names are like points; propositions like arrows---they have sense.)}%???

\pn{3.2}
\ger{Im Satze kann der Gedanke so ausgedr{\"u}ckt sein, dass den Gegenst{\"a}nden des Gedankens Elemente des Satzzeichens entsprechen.}
\ogd{In propositions thoughts can be so expressed that to the objects of the thoughts correspond the elements of the propositional sign.}
\pmc{In a proposition a thought can be expressed in such a way that elements of the propositional sign correspond to the objects of the thought.}

\pn{3.201}
\ger{Diese Elemente nenne ich \gdql einfache Zeichen\gdqr{} und den Satz \gdql vollst{\"a}ndig analysiert\gdqr{}.}
\ogd{These elements I call ``simple signs'' and the proposition ``completely analysed''.}
\pmc{I call such elements `simple signs', and such a proposition `complete analysed'.}

\pn{3.202}
\ger{Die im Satze angewandten einfachen Zeichen hei{\ss}en Namen.}
\ogd{The simple signs employed in propositions are called names.}
\pmc{The simple signs employed in propositions are called names.}

\pn{3.203}
\ger{Der Name bedeutet den Gegenstand. Der Gegenstand ist seine Bedeutung. (\gdql $A$\gdqr{} ist dasselbe Zeichen wie \gdql $A$\gdqr{}.)}
\ogd{The name means the object. The object is its meaning. (``$A$'' is the same sign as ``$A$''.)}
\pmc{A name means an object. The object is its meaning. (`$A$' is the same sign as `$A$'.)}

\pn{3.21}
\ger{Der Konfiguration der einfachen Zeichen im Satzzeichen entspricht die Konfiguration der Gegenst{\"a}nde in der Sachlage.}
\ogd{To the configuration of the simple signs in the propositional sign corresponds the configuration of the objects in the state of affairs.}
\pmc{The configuration of objects in a situation corresponds to the configuration of simple signs in the propositional sign.}

\pn{3.22}
\ger{Der Name vertritt im Satz den Gegenstand.}
\ogd{In the proposition the name represents the object.}
\pmc{In a proposition a name is the representative of an object.}%???

\pn{3.221}
\ger{Die Gegenst{\"a}nde kann ich nur \germph{nennen}. Zeichen vertreten sie. Ich kann nur \germph{von} ihnen sprechen, \germph{sie aussprechen} kann ich nicht. Ein Satz kann nur sagen, \germph{wie} ein Ding ist, nicht \germph{was} es ist.}
\ogd{Objects I can only \emph{name}. Signs represent them. I can only speak \emph{of} them. I cannot \emph{assert them}. A proposition can only say \emph{how} a thing is, not \emph{what} it is.}
\pmc{Objects can only be \emph{named}. Signs are their representatives. I can only speak \emph{about} them: I cannot \emph{put them into words}. Propositions can only say \emph{how} things are, not \emph{what} they are.}

\pn{3.23}
\ger{Die Forderung der M{\"o}glichkeit der einfachen Zeichen ist die Forderung der Bestimmtheit des Sinnes.}
\ogd{The postulate of the possibility of the simple signs is the postulate of the determinateness of the sense.}
\pmc{The requirement that simple signs be possible is the requirement that sense be determinate.}

\pn{3.24}
\ger{Der Satz, welcher vom Komplex handelt, steht in interner Beziehung zum Satze, der von dessen Bestandteil handelt.}
\ogd{A proposition about a complex stands in internal relation to the proposition about its constituent part.}
\pmc{A proposition about a complex stands in an internal relation to a proposition about a constituent of the complex.}

\pnskip
\ger{Der Komplex kann nur durch seine Beschreibung gegeben sein, und diese wird stimmen oder nicht stimmen. Der Satz, in welchem von einem Komplex die Rede ist, wird, wenn dieser nicht existiert, nicht unsinnig, sondern einfach falsch sein.}
\ogd{A
complex can only be given by its description, and this will either be
right or wrong. The proposition in which there is mention of a complex,
if this does not exist, becomes not nonsense but simply false.}
\pmc{A complex can be given
only by its description, which will be right or wrong. A proposition
that mentions a complex will not be nonsensical, if the complex does
not exist, but simply false.}

\pnskip
\ger{Dass ein Satzelement einen Komplex bezeichnet, kann man aus einer Unbestimmtheit in den S{\"a}tzen sehen, worin es vorkommt. Wir \germph{wissen}, durch diesen Satz ist noch nicht alles bestimmt. (Die Allgemeinheitsbezeichnung \germph{enth{\"a}lt} ja ein Urbild.)}
\ogd{That a propositional element signifies a complex can be seen from an indeterminateness in the propositions in which it occurs. We \emph{know} that everything is not yet determined by this proposition. (The notation for generality \emph{contains} a prototype.)}
\pmc{When a propositional element signifies a complex, this can be seen from an indeterminateness in the propositions in which it occurs. In such cases we \emph{know} that the proposition leaves something undetermined. (In fact the notation for generality \emph{contains} a prototype.)}

\pnskip
\ger{Die Zusammenfassung des Symbols eines Komplexes in ein einfaches Symbol kann durch eine Definition ausgedr{\"u}ckt werden.}
\ogd{The combination of the symbols of a complex in a simple symbol can be expressed by a definition.}
\pmc{The contraction of a symbol for a complex into a simple symbol can be expressed in a definition.}

\pn{3.25}
\ger{Es gibt eine und nur eine vollst{\"a}ndige Analyse des Satzes.}
\ogd{There is one and only one complete analysis of the proposition.}
\pmc{A proposition has one and only one complete analysis.}%???

\pn{3.251}
\ger{Der Satz dr{\"u}ckt auf bestimmte, klar angebbare Weise aus, was er ausdr{\"u}ckt: Der Satz ist artikuliert.}
\ogd{The proposition expresses what it expresses in a definite and clearly specifiable way: the proposition is articulate.}
\pmc{What a proposition expresses it expresses in a determinate manner, which can be set out clearly: a proposition is articulated.}%???

\pn{3.26}
\ger{Der Name ist durch keine Definition weiter zu zergliedern: er ist ein Urzeichen.}
\ogd{The name cannot be analysed further by any definition. It is a primitive sign.}
\pmc{A name cannot be dissected any further by means of a definition: it is a primitive sign.}

\pn{3.261}
\ger{Jedes difinierte Zeichen bezeichnet \germph{{\"u}ber} jene Zeichen, durch welche es definier wurde; und die Definitionen weisen den Weg.}
\ogd{Every defined sign signifies \emph{via} those signs by which it is defined, and the definitions show the way.}
\pmc{Every sign that has a definition signifies \emph{via} the signs that serve to define it; and the definitions point the way.}

\pnskip
\ger{Zwei Zeichen, ein Urzeichen, und ein durch Urzeichen definiertes, k{\"o}nnen nicht auf dieselbe Art und Weise bezeichnen. Namen \germph{kann} man nicht durch Definitionen auseinanderlegen. (Kein Zeichen, welches allein, selbst{\"a}ndig eine Bedeutung hat.)}
\ogd{Two signs, one a primitive sign, and one defined by primitive signs, cannot signify in the same way. Names \emph{cannot} be taken to pieces by definition (nor any sign which alone and independently has a meaning).}
\pmc{Two signs cannot signify in the same manner if one is primitive and the other is defined by means of primitive signs. Names \emph{cannot} be anatomized by means of definitions. (Nor can any sign that has a meaning independently and on its own.)}%***

\pn{3.262}
\ger{Was in den Zeichen nicht zum Ausdruck kommt, das zeigt ihre Anwendung. Was die Zeichen verschlucken, das spricht ihre Anwendung aus.}
\ogd{What does not get expressed in the sign is shown by its application. What the signs conceal, their application declares.}
\pmc{What signs fail to express, their application shows. What signs slur over, their application says clearly.}

\pn{3.263}
\ger{Die Bedeutung von Urzeichen k{\"o}nnen durch Erl{\"a}uterungen erkl{\"a}rt werden. Erl{\"a}uterungen sind S{\"a}tze, welche die Urzeichen enthalten. Sie k{\"o}nnen also nur verstanden werden, wenn die Bedeutungen dieser Zeichen bereits bekannt sind.}
\ogd{The meanings of primitive signs can be explained by elucidations. Elucidations are propositions which contain the primitive signs. They can, therefore, only be understood when the meanings of these signs are already known.}
\pmc{The meanings of primitive signs can be explained by means of elucidations. Elucidations are propositions that contain the primitive signs. So they can only be understood if the meanings of those signs are already known.}%???

\pn{3.3}
\ger{Nur der Satz hat Sinn; nur im Zusammenhang des Satzes hat ein Name Bedeutung.}
\ogd{Only the proposition has sense; only in the context of a proposition has a name meaning.}
\pmc{Only propositions have sense; only in the nexus of a proposition does a name have meaning.}

\pn{3.31}
\ger{Jeden Teil des Satzes, der seinen Sinn charakterisiert, nenne ich einen Ausdruck (ein Symbol).}
\ogd{Every part of a proposition which characterizes its sense I call an expression (a symbol).}
\pmc{I call any part of a proposition that characterizes its sense an expression (or a symbol).}

\pnskip
\ger{(Der Satz selbst ist ein Ausdruck.)}
\ogd{(The proposition itself is an expression.)}
\pmc{(A proposition is itself an expression.)}

\pnskip
\ger{Ausdruck ist alles, f{\"u}r den Sinn des Satzes wesentliche, was S{\"a}tze miteinander gemein haben k{\"o}nnen.}
\ogd{Expressions are ev\-ery\-thing---es\-sen\-tial for the sense of the prop\-o\-sit\-ion---that prop\-o\-sit\-ions can have in com\-mon with one a\-noth\-er.}
\pmc{Everything essential to their sense that propositions can have in common with one another is an expression.}

\pnskip
\ger{Der Ausdruck kennzeichnet eine Form und einen Inhalt.}
\ogd{An expression characterizes a form and a content.}
\pmc{An expression is the mark of a form and a content.}

\pn{3.311}
\ger{Der Ausdruck setzt die Formen aller S{\"a}tze voraus, in welchem er vorkommen kann. Er ist das gemeinsame charakteristische Merkmal einer Klasse von S{\"a}tzen.}
\ogd{An expression presupposes the forms of all propositions in which it can occur. It is the common characteristic mark of a class of propositions.}
\pmc{An expression presupposes the forms of all the propositions in which it can occur. It is the common characteristic mark of a class of propositions.}

\pn{3.312}
\ger{Er wird also dargestellt durch die allgemeine Form der S{\"a}tze, die er charakterisiert.}
\ogd{It is therefore represented by the general form of the propositions which it characterizes.}
\pmc{It is therefore presented by means of the general form of the propositions that it characterizes.}

\pnskip
\ger{Und zwar wird in dieser Form der Ausdruck \germph{konstant} und alles {\"u}brige \germph{variabel} sein.}
\ogd{And in this form the expression is \emph{constant} and everything else \emph{variable}.}
\pmc{In fact, in this form the expression will be \emph{constant} and everything else \emph{variable}.}

\pn{3.313}
\ger{Der Ausdruck wird also durch eine Variable dargestellt, deren Werte die S{\"a}tze sind, die den Ausdruck enthalten.}
\ogd{An expression is thus presented by a variable, whose values are the propositions which contain the expression.}
\pmc{Thus an expression is presented by means of a variable whose values are the propositions that contain the expression.}

\pnskip
\ger{(Im Grenzfall wird die Variable zur Konstanten, der Ausdruck zum Satz.)}
\ogd{(In the limiting case the variable becomes constant, the expression a proposition.)}
\pmc{(In the limiting case the variable becomes a constant, the expression becomes a proposition.)}

\pnskip
\ger{Ich nenne eine solche Variable \gdql Satzvariable\gdqr{}.}
\ogd{I call such a variable a ``propositional variable''.}
\pmc{I call such a variable a `propositional variable'.}

\pn{3.314}
\ger{Der Ausdruck hat nur im Satz Bedeutung. Jede Variable l{\"a}sst sich als Satzvariable auffassen.}
\ogd{An expression has meaning only in a proposition. Every variable can be conceived as a propositional variable.}
\pmc{An expression has meaning only in a proposition. All variables can be construed as propositional variables.}

\pnskip
\ger{(Auch der variable Name.)}
\ogd{(Including the variable name.)}
\pmc{(Even variable names.)}

\pn{3.315}
\ger{Verwandeln wir einen Bestandteil eines Satzes in eine Variable, so gibt es eine Klasse von S{\"a}tzen, welche s{\"a}mtlich Werte des so entstandenen variablen Satzes sind. Diese Klasse h{\"a}ngt im allgemeinen noch davon ab, was wir, nach willk{\"u}rlicher {\"U}bereinkunft, mit Teilen jenes Satzes meinen. Verwandeln wir aber alle jene Zeichen, deren Bedeutung willk{\"u}rlich bestimmt wurde, in Variable, so gibt es nun noch immer eine solche Klasse. Diese aber ist nun von keiner {\"U}bereinkunft abh{\"a}ngig, sondern nur noch von der Natur des Satzes. Sie entspricht einer logischen Form---einem logischen Urbild.}
\ogd{If we change a constituent part of a proposition into a variable, there is a class of propositions which are all the values of the resulting variable proposition. This class in general still depends on what, by arbitrary agreement, we mean by parts of that proposition. But if we change all those signs, whose meaning was arbitrarily determined, into variables, there always remains such a class. But this is now no longer dependent on any agreement; it depends only on the nature of the proposition. It corresponds to a logical form, to a logical prototype.}
\pmc{If we turn a constituent of a proposition into a variable, there is a class of propositions all of which are values of the resulting variable proposition. In general, this class too will be dependent on the meaning that our arbitrary conventions have given to parts of the original proposition. But if all the signs in it that have arbitrarily determined meanings are turned into variables, we shall still get a class of this kind. This one, however, is not dependent on any convention, but solely on the nature of the proposition. It corresponds to a logical form---a logical prototype.}

\pn{3.316}
\ger{Welche Werte die Satzvariable annehmen darf, wird festgesetzt.}
\ogd{What values the propositional variable can assume is determined.}
\pmc{What values a propositional variable may take is something that is stipulated.}

\pnskip
\ger{Die Festsetzung der Werte \germph{ist} die Variable.}
\ogd{The determination of the values \emph{is} the variable.}
\pmc{The stipulation of values \emph{is} the variable.}

\pn{3.317}
\ger{Die Festsetzung der Werte der Satzvariablen ist die \germph{Angabe der S{\"a}tze}, deren gemeinsames Merkmal die Variable ist.}
\ogd{The determination of the values of the propositional variable is done by \emph{indicating the propositions} whose common mark the variable is.}
\pmc{To stipulate values for a propositional variable is to \emph{give the propositions} whose common characteristic the variable is.}

\pnskip
\ger{Die Festsetzung ist eine Beschreibung dieser S{\"a}tze.}
\ogd{The determination is a description of these propositions.}
\pmc{The stipulation is a description of those propositions.}

\pnskip
\ger{Die Festsetzung wird also nur von Symbolen, nicht von deren Bedeutung handeln.}
\ogd{The determination will therefore deal only with symbols not with their meaning.}
\pmc{The stipulation will therefore be concerned only with symbols, not with their meaning.}

\pnskip
\ger{Und \germph{nur} dies ist der Festsetzung wesentlich, \germph{dass sie nur eine Beschreibung von Symbolen ist und nicht {\"u}ber das Bezeichnete aussagt}.}
\ogd{And \emph{only} this is essential to the determination, \emph{that it is only a description of symbols and asserts nothing about what is symbolized}.}
\pmc{And the \emph{only} thing essential to the stipulation is \emph{that it is merely a description of symbols and states nothing about what is signified}.}

\pnskip
\ger{Wie die Beschreibung der S{\"a}tze geschieht, ist unwesentlich.}
\ogd{The way in which we describe the propositions is not essential.}
\pmc{How the description of the propositions is produced is not essential.}

\pn{3.318}
\ger{Den Satz fasse ich---wie Frege und Russell---als Funktion der in ihm enthaltenen Ausdr{\"u}cke auf.}
\ogd{I conceive the proposition---like Frege and Russell---as a function of the expressions contained in it.}
\pmc{Like Frege and Russell I construe a proposition as a function of the expressions contained in it.}

\pn{3.32}
\ger{Das Zeichen ist das sinnlich Wahrnehmbare am Symbol.}
\ogd{The sign is the part of the symbol perceptible by the senses.}
\pmc{A sign is what can be perceived of a symbol.}

\pn{3.321}
\ger{Zwei verschiedene Symbole k{\"o}nnen also das Zeichen (Schriftzeichen oder Lautzeichen etc.) miteinander gemein haben---sie bezeichnen dann auf verschiedene Art und Weise.}
\ogd{Two different symbols can therefore have the sign (the written sign or the sound sign) in common---they then signify in different ways.}
\pmc{So one and the same sign (written or spoken, etc.)\ can be common to two different symbols---in which case they will signify in different ways.}

\pn{3.322}
\ger{Es kann nie das gemeinsame Merkmal zweier Gegenst{\"a}nde anzeigen, dass wir sie mit demselben Zeichen, aber durch zwei verschiedene \germph{Bezeichnungsweisen} bezeichnen. Denn das Zeichen ist ja willk{\"u}rlich. Man k{\"o}nnte also auch zwei verschiedene Zeichen w{\"a}hlen, und wo bliebe dann das Gemeinsame in der Bezeichnung?}
\ogd{It can never indicate the common characteristic of two objects that we symbolize them with the same signs but by different \emph{methods of symbolizing}. For the sign is arbitrary. We could therefore equally well choose two different signs and where then would be what was common in the symbolization?}
\pmc{Our use of the same sign to signify two different objects can never indicate a common characteristic of the two, if we use it with two different \emph{modes of signification}. For the sign, of course, is arbitrary. So we could choose two different signs instead, and then what would be left in common on the signifying side?}

\pn{3.323}
\ger{In der Umgangssprache kommt es ungemein h{\"a}ufig vor, dass dasselbe Wort auf verschiedene Art und Weise bezeichnet---also verschiedene Symbolen angeh{\"o}rt---, oder, dass zwei W{\"o}rter, die auf verschiedene Art und Weise bezeichnen, {\"a}u{\ss}erlich in der gleichen Weise im Satz angewandt werden.}
\ogd{In the language of everyday life it very often happens that the same word signifies in two different ways---and therefore belongs to two different symbols---or that two words, which signify in different ways, are apparently applied in the same way in the proposition.}
\pmc{In everyday language it very frequently happens that the same word has different modes of signification---and so belongs to different symbols---or that two words that have different modes of signification are employed in propositions in what is superficially the same way.}

\pnskip
\ger{So erscheint das Wort \gdql ist\gdqr{} als Kopula, als Gleichheitszeichen und als Ausdruck der Existenz; \gdql existieren\gdqr{} als intransitives Zeitwort wie \gdql gehen\gdqr{}; \gdql identisch\gdqr{} als Eigenschaftswort; wir reden von \germph{Etwas}, aber auch davon, dass \germph{etwas} geschieht.}
\ogd{Thus the word ``is'' appears as the copula, as the sign of equality, and as the expression of existence; ``to exist'' as an intransitive verb like ``to go''; ``identical'' as an adjective; we speak of \emph{something} but also of the fact of \emph{something} happening.}
\pmc{Thus the word `is' figures as the copula, as a sign for identity, and as an expression for existence; `exist' figures as an intransitive verb like `go', and `identical' as an adjective; we speak of \emph{something}, but also of \emph{something's} happening.}

\pnskip
\ger{(Im Satze \gdql Gr{\"u}n ist gr{\"u}n\gdqr{}---wo das erste Wort ein Personenname, das letzte ein Eigenschaftswort ist---haben diese Worte nicht einfach verschiedene Bedeutung, sondern es sind \germph{verschiedene Symbole}.)}
\ogd{(In the proposition ``Green is green''---where the first word is a proper name as the last an adjective---these words have not merely different meanings but they are \emph{different symbols}.)}
\pmc{(In the proposition, `Green is green'---where the first word is the proper name of a person and the last an adjective---these words do not merely have different meanings: they are \emph{different symbols}.)}

\pn{3.324}
\ger{So entstehen leicht die fundamentalsten Verwechselungen (deren die ganze Philosophie voll ist).}
\ogd{Thus there easily arise the most fundamental confusions (of which the whole of philosophy is full).}
\pmc{In this way the most fundamental confusions are easily produced (the whole of philosophy is full of them).}

\pn{3.325}
\ger{Um diesen Irrt{\"u}mern zu entgehen, m{\"u}ssen wir eine Zeichensprache verwenden, welche sie ausschlie{\ss}t, indem sie nicht das gleiche Zeichen in verschiednen Symbolen, und Zeichen, welche auf verschiedene Art bezeichnen, nicht {\"a}u{\ss}erlich auf die gleiche Art verwendet. Eine Zeichensprache also, die der \germph{logischen} Grammatik---der logischen Syntax---gehorcht.}
\ogd{In order to avoid these errors, we must employ a symbolism which excludes them, by not applying the same sign in different symbols and by not applying signs in the same way which signify in different ways. A symbolism, that is to say, which obeys the rules of \emph{logical} grammar---of logical syntax.}
\pmc{In order to avoid such errors we must make use of a sign-language that excludes them by not using the same sign for different symbols and by not using in a superficially similar way signs that have different modes of signification: that is to say, a sign-language that is governed by \emph{logical} grammar---by logical syntax.}

\pnskip
\ger{(Die Begriffsschrift Freges und Russells ist eine solche Sprache, die allerdings noch nicht alle Fehler ausschlie{\ss}t.)}
\ogd{(The logical symbolism of Frege and Russell is such a language, which, however, does still not exclude all errors.)}
\pmc{(The conceptual notation of Frege and Russell is such a language, though, it is true, it fails to exclude all mistakes.)}

\pn{3.326}
\ger{Um das Symbol am Zeichen zu erkennen, muss man auf den sinnvollen Gebrauch achten.}
\ogd{In order to recognize the symbol in the sign we must consider the significant use.}
\pmc{In order to recognize a symbol by its sign we must observe how it is used with a sense.}

\pn{3.327}
\ger{Das Zeichen bestimmt erst mit seiner logisch-syntaktischen Verwendung zusammen eine logische Form.}
\ogd{The sign determines a logical form only together with its logical syntactic application.}
\pmc{A sign does not determine a logical form unless it is taken together with its logico-syntactical employment.}

\pn{3.328}
\ger{Wird ein Zeichen \germph{nicht gebraucht}, so ist es bedeutungslos. Das ist der Sinn der Devise Occams.}
\ogd{If a sign is \emph{not necessary} then it is meaningless. That is the meaning of Occam's razor.}
\pmc{If a sign is \emph{useless}, it is meaningless. That is the point of Occam's maxim.}

\pnskip
\ger{(Wenn sich alles so verh{\"a}lt als h{\"a}tte ein Zeichen Bedeutung, dann hat es auch Bedeutung.)}
\ogd{(If everything in the symbolism works as though a sign had meaning, then it has meaning.)}
\pmc{(If everything behaves as if a sign had meaning, then it does have meaning.)}

\pn{3.33}
\ger{In der logischen Syntax darf nie die Bedeutung eines Zeichens eine Rolle spielen; sie muss sich aufstellen lassen, ohne dass dabei von der \germph{Bedeutung} eines Zeichens die Rede w{\"a}re, sie darf \germph{nur} die Beschreibung der Ausdr{\"u}cke voraussetzen.}
\ogd{In logical syntax the meaning of a sign ought never to play a r{\^o}le; it must admit of being established without mention being thereby made of the \emph{meaning} of a sign; it ought to presuppose \emph{only} the description of the expressions.}
\pmc{In logical syntax the meaning of a sign should never play a role. It must be possible to establish logical syntax without mentioning the \emph{meaning} of a sign: \emph{only} the description of expressions may be presupposed.}

\pn{3.331}
\ger{Von dieser Bemerkung sehen wir in Russells \gdql Theory of types\gdqr{} hin{\"u}ber: Der Irrtum Russells zeigt sich darin, dass er bei der Aufstellung der Zeichenregeln von der Bedeutung der Zeichen reden musste.}
\ogd{From this observation we get a further view---into Russell's \emph{Theory of Types}. Russell's error is shown by the fact that in drawing up his symbolic rules he has to speak about the things his signs mean.}
\pmc{From this observation we turn to Russell's `theory of types'. It can be seen that Russell must be wrong, because he had to mention the meaning of signs when establishing the rules for them.}

\pn{3.332}
\ger{Kein Satz kann etwas {\"u}ber sich selbst aussagen, weil das Satzzeichen nicht in sich selbst enthalten sein kann (das ist die ganze \gdql Theory of types\gdqr{}).}
\ogd{No proposition can say anything about itself, because the propositional sign cannot be contained in itself (that is the ``whole theory of types'').}
\pmc{No proposition can make a statement about itself, because a propositional sign cannot be contained in itself (that is the whole of the `theory of types').}

\pn{3.333}
\ger{Eine Funktion kann darum nicht ihr eigenes Argument sein, weil das Funktionszeichen bereits das Urbild seines Arguments enth{\"a}lt und es sich nicht selbst enthalten kann.}
\ogd{A function cannot be its own argument, because the functional sign already contains the prototype of its own argument and it cannot contain itself.}
\pmc{The reason why a function cannot be its own argument is that the sign for a function already contains the prototype of its argument, and it cannot contain itself.}

\pnskip
\ger{Nehmen wir n{\"a}mlich an, die Funktion $F(f\negthinspace x)$ k{\"o}nnte ihr eigenes Argument sein; dann g{\"a}be es also einen Satz: \gdql $F(F(f\negthinspace x))$\gdqr{} und in diesem m{\"u}ssen die {\"a}u{\ss}ere Funktion $F$ und die innere Funtion $F$ verschiedene Bedeutungen haben, denn die innere hat die Form $\phi(f\negthinspace x)$, die {\"a}u{\ss}ere die Form $\psi(\phi(f\negthinspace x))$. Gemeinsam ist den beiden Funktionen nur der Buchstabe \gdql $F$\gdqr{}, der aber allein nichts bezeichnet.}
\ogd{If, for example, we suppose that the function $F(f\negthinspace x)$ could be its own argument, then there would be a proposition ``$F(F(f\negthinspace x))$'', and in this the outer function $F$ and the inner function $F$ must have different meanings; for the inner has the form $\phi(f\negthinspace x)$, the outer the form $\psi(\phi(f\negthinspace x))$. Common to both functions is only the letter ``$F$'', which by itself signifies nothing.}
\pmc{For let us suppose that the function $F(f\negthinspace x)$ could be its own argument: in that case there would be a proposition `$F(F(f\negthinspace x))$', in which the outer function $F$ and the inner function $F$ must have different meanings, since the inner one has the form $\phi(f\negthinspace x)$ and the outer one has the form $\psi(\phi(f\negthinspace x))$. Only the letter `$F$' is common to the two functions, but the letter by itself signifies nothing.}

\pnskip
\ger{Dies wird sofort klar, wenn wir statt \gdql $F(Fu)$\gdqr{} schreiben \gdql $\rsomedd{\phi} F(\phi u) \rand \phi u=Fu$\gdqr{}.}
\ogd{This is at once clear, if instead of ``$F(F\negthinspace u)$'' we write ``$\rsomedd{\phi} F(\phi u) \rand \phi u=F\negthinspace u$''.}
\pmc{This immediately becomes clear if instead of `$F(F\negthinspace u)$' we write `$\rsomedd{\phi} F(\phi u) \rand \phi u=F\negthinspace u$'.}

\pnskip
\ger{Hiermit erledigt sich Russells Paradox.}
\ogd{Herewith Russell's paradox vanishes.}
\pmc{That disposes of Russell's paradox.}

\pn{3.334}
\ger{Die Regeln der logischen Syntax m{\"u}ssen sich von selbst verstehen, wenn man nur wei{\ss}, wie ein jedes Zeichen bezeichnet.}
\ogd{The rules of logical syntax must follow of themselves, if we only know how every single sign signifies.}
\pmc{The rules of logical syntax must go without saying, once we know how each individual sign signifies.}

\pn{3.34}
\ger{Der Satz besitzt wesentliche und zuf{\"a}llige Z{\"u}ge.}
\ogd{A proposition possesses essential and accidental features.}
\pmc{A proposition possesses essential and accidental features.}

\pnskip
\ger{Zuf{\"a}llig sind die Z{\"u}ge, die von der besonderen Art der Hervorbringung des Satzzeichens herr{\"u}hren. Wesentlich diejenigen, welche allein den Satz bef{\"a}higen, seinen Sinn auszudr{\"u}cken.}
\ogd{Accidental are the features which are due to a particular way of producing the propositional sign. Essential are those which alone enable the proposition to express its sense.}
\pmc{Accidental features are those that result from the particular way in which the propositional sign is produced. Essential features are those without which the proposition could not express its sense.}

\pn{3.341}
\ger{Das Wesentliche am Satz ist also das, was allen S{\"a}tzen, welche den gleichen Sinn ausdr{\"u}cken k{\"o}nnen, gemeinsam ist.}
\ogd{The essential in a proposition is therefore that which is common to all propositions which can express the same sense.}
\pmc{So what is essential in a proposition is what all propositions that can express the same sense have in common.}

\pnskip
\ger{Und ebenso ist allgemein das Wesentliche am Symbol das, was alle Symbole, die denselben Zweck erf{\"u}llen k{\"o}nnen, gemeinsam haben.}
\ogd{And in the same way in general the essential in a symbol is that which all symbols which can fulfill the same purpose have in common.}
\pmc{And similarly, in general, what is essential in a symbol is what all symbols that can serve the same purpose have in common.}

\pn{3.3411}
\ger{Man k{\"o}nnte also sagen: Der eigentliche Name ist das, was alle Symbole, die den Gegenstand bezeichnen, gemeinsam haben. Es w{\"u}rde sich so successive ergeben, dass keinerlei Zusammensetzung f{\"u}r den Namen wesentlich ist.}
\ogd{One could therefore say the real name is that which all symbols, which signify an object, have in common. It would then follow, step by step, that no sort of composition was essential for a name.}
\pmc{So one could say that the real name of an object was what all symbols that signified it had in common. Thus, one by one, all kinds of composition would prove to be unessential to a name.}

\pn{3.342}
\ger{An unseren Notationen ist zwar etwas willk{\"u}rlich, aber \germph{das} ist nicht willk{\"u}rlich: Dass, \germph{wenn} wir etwas willk{\"u}rlich bestimmt haben, dann etwas anderes der Fall sein muss. (Dies h{\"a}ngt von dem \germph{Wesen} der Notation ab.)}
\ogd{In our notations there is indeed something arbitrary, but \emph{this} is not arbitrary, namely that \emph{if} we have determined anything arbitrarily, then something else \emph{must} be the case. (This results from the \emph{essence} of the notation.)}
\pmc{Although there is something arbitrary in our notations, \emph{this much} is not arbitrary---that \emph{when} we have determined one thing arbitrarily, something else is necessarily the case. (This derives from the \emph{essence} of notation.)}

\pn{3.3421}
\ger{Eine besondere Bezeichnungsweise mag unwichtig sein, aber wichtig ist es immer, dass diese eine \germph{m{\"o}gliche} Bezeichnungsweise ist. Und so verh{\"a}lt es sich in der Philosophie {\"u}berhaupt: Das Einzelne erweist sich immer wieder als unwichtig, aber die M{\"o}glichkeit jedes Einzelnen gibt uns einen Aufschluss {\"u}ber das Wesen der Welt.}
\ogd{A particular method of symbolizing may be unimportant, but it is always important that this is a \emph{possible} method of symbolizing. And this happens as a rule in philosophy: The single thing proves over and over again to be unimportant, but the possibility of every single thing reveals something about the nature of the world.}
\pmc{A particular mode of signifying may be unimportant but it is always important that it is a \emph{possible} mode of signifying. And that is generally so in philosophy: again and again the individual case turns out to be unimportant, but the possibility of each individual case discloses something about the essence of the world.}

\pn{3.343}
\ger{Definitionen sind Regeln der {\"U}bersetzung von einer Sprache in eine andere. Jede richtige Zeichensprache muss sich in jede andere nach solchen Regeln {\"u}bersetzen lassen: \germph{Dies} ist, was sie alle gemeinsam haben.}
\ogd{Definitions are rules for the translation of one language into another. Every correct symbolism must be translatable into every other according to such rules. It is \emph{this} which all have in common.}
\pmc{Definitions are rules for translating from one language into another. Any correct sign-language must be translatable into any other in accordance with such rules: it is \emph{this} that they all have in common.}

\pn{3.344}
\ger{Das, was am Symbol bezeichnet, ist das Gemeinsame aller jener Symbole, durch die das erste den Regeln der logischen Syntax zufolge ersetzt werden kann.}
\ogd{What signifies in the symbol is what is common to all those symbols by which it can be replaced according to the rules of logical syntax.}
\pmc{What signifies in a symbol is what is common to all the symbols that the rules of logical syntax allow us to substitute for it.}

\pn{3.3441}
\ger{Man kann z.\ B.\ das Gemeinsame aller Notationen f{\"u}r die Wahrheitsfunktionen so ausdr{\"u}cken: Es ist ihnen gemeinsam, dass sich alle---z.\ B.---durch die Notation von \gdql $\rnot p$\gdqr{} (\gdql nicht $p$\gdqr{}) und \gdql $p \lor q$\gdqr{} (\gdql $p$ oder $q$\gdqr{}) \germph{ersetzen lassen}.}
\ogd{We can, for example, express what is common to all notations for the truth-functions as follows: It is common to them that they all, for example, \emph{can be replaced} by the notations of ``$\rnot p$'' (``not $p$'') and ``$p \lor q$'' (``$p$ or $q$'').}
\pmc{For instance, we can express what is common to all notations for truth-functions in the following way: they have in common that, for example, the notation that uses `$\rnot p$' (`not $p$') and `$p \lor q$' (`$p$ or $q$') \emph{can be substituted} for any of them.}

\pnskip
\ger{(Hiermit ist die Art und Weise gekennzeichnet, wie eine spezielle m{\"o}gliche Notation uns allgemeine Aufschl{\"u}sse geben kann.)}
\ogd{(Herewith is indicated the way in which a special possible notation can give us general information.)}
\pmc{(This serves to characterize the way in which something general can be disclosed by the possibility of a specific notation.)}

\pn{3.3442}
\ger{Das Zeichen des Komplexes l{\"o}st sich auch bei der Analyse nicht willk{\"u}rlich auf, so dass etwa seine Aufl{\"o}sung in jedem Satzgef{\"u}ge eine andere w{\"a}re.}
\ogd{The sign of the complex is not arbitrarily resolved in the analysis, in such a way that its resolution would be different in every propositional structure.}
\pmc{Nor does analysis resolve the sign for a complex in an arbitrary way, so that it would have a different resolution every time that it was incorporated in a different proposition.}

\pn{3.4}
\ger{Der Satz bestimmt einen Ort im logischen Raum. Die Existenz dieses logischen Ortes ist durch die Existenz der Bestandteile allein verb{\"u}rgt, durch die Existenz des sinnvollen Satzes.}
\ogd{The proposition determines a place in logical space: the existence of this logical place is guaranteed by the existence of the constituent parts alone, by the existence of the significant proposition.}
\pmc{A proposition determines a place in logical space. The existence of this logical place is guaranteed by the mere existence of the constituents---by the existence of the proposition with a sense.}

\pn{3.41}
\ger{Das Satzzeichen und die logischen Koordinaten: Das ist der logische Ort.}
\ogd{The propositional sign and the logical co-ordinates: that is the logical place.}
\pmc{The propositional sign with logical co-ordinates---that is the logical place.}

\pn{3.411}
\ger{Der geometrische und der logische Ort stimmen darin {\"u}berein, dass beide die M{\"o}glichkeit einer Existenz sind.}
\ogd{The geometrical and the logical place agree in that each is the possibility of an existence.}
\pmc{In geometry and logic alike a place is a possibility: something can exist in it.}

\pn{3.42}
\ger{Obwohl der Satz nur einen Ort des logischen Raumes bestimmen darf, so muss doch durch ihn schon der ganze logische Raum gegeben sein.}
\ogd{Although a proposition may only determine one place in logical space, the whole logical space must already be given by it.}
\pmc{A proposition can determine only one place in logical space: nevertheless the whole of logical space must already be given by it.}

\pnskip
\ger{(Sonst w{\"u}rden durch die Verneinung, die logische Summe, das logische Produkt, etc. immer neue Elemente---in Koordinaten---eingef{\"u}hrt.)}
\ogd{(Otherwise denial, the logical sum, the logical product, etc., would always introduce new elements---in co-ordination.)}
\pmc{(Otherwise negation, logical sum, logical product, etc., would introduce more and more new elements---in co-ordination.)}

\pnskip
\ger{(Das logische Ger{\"u}st um das Bild herum bestimmt den logischen Raum. Der Satz durchgreift den ganzen logischen Raum.)}
\ogd{(The logical scaffolding round the picture determines the logical space. The proposition reaches through the whole logical space.)}
\pmc{(The logical scaffolding surrounding a picture determines logical space. The force of a proposition reaches through the whole of logical space.)}

\pn{3.5}
\ger{Das angewandte, gedachte Satzeichen ist der Gedanke.}
\ogd{The applied, thought, propositional sign, is the thought.}
\pmc{A propositional sign, applied and thought out, is a thought.}

\pn{4}
\ger{Der Gedanke ist der sinnvolle Satz.}
\ogd{The thought is the significant proposition.}
\pmc{A thought is a proposition with a sense.}

\pn{4.001}
\ger{Die Gesamtheit der S{\"a}tze ist die Sprache.}
\ogd{The totality of propositions is the language.}
\pmc{The totality of propositions is language.}

\pn{4.002}
\ger{Der Mensch besitzt die F{\"a}higkeit Sprachen zu bauen, womit sich jeder Sinn ausdr{\"u}cken l{\"a}sst, ohne eine Ahnung davon zu haben, wie und was jedes Wort bedeutet.---Wie man auch spricht, ohne zu wissen, wie die einzelnen Laute hervorgebracht werden.}
\ogd{Man possesses the capacity of constructing languages, in which every sense can be expressed, without having an idea how and what each word means---just as one speaks without knowing how the single sounds are produced.}
\pmc{Man possesses the ability to construct languages capable of expressing every sense, without having any idea how each word has meaning or what its meaning is---just as people speak without knowing how the individual sounds are produced.}

\pnskip
\ger{Die Umgangssprache ist ein Teil des menschlichen Organismus und nicht weniger kompliziert als dieser.}
\ogd{Colloquial language is a part of the human organism and is not less complicated than it.}
\pmc{Everyday language is a part of the human organism and is no less complicated than it.}

\pnskip
\ger{Es ist menschenunm{\"o}glich, die Sprach\-log\-ik aus ihr unmittelbar zu entnehmen.}
\ogd{From it it is humanly impossible to gather immediately the logic of language.}
\pmc{It is not humanly possible to gather immediately from it what the logic of language is.}

\pnskip
\ger{Die Sprache verkleidet den Gedanken. Und zwar so, dass man nach der {\"a}u{\ss}eren Form des Kleides, nicht auf deie Form des bekleideten Gedankens schlie{\ss}en kann; weil die {\"a}u{\ss}ere Form des Kleides nach ganz anderen Zwecken gebildet ist als danach, die Form des K{\"o}rpers erkennen zu lassen.}
\ogd{Language disguises the thought; so that from the external form of the clothes one cannot infer the form of the thought they clothe, because the external form of the clothes is constructed with quite another object than to let the form of the body be recognized.}
\pmc{Language disguises thought. So much so, that from the outward form of the clothing it is impossible to infer the form of the thought beneath it, because the outward form of the clothing is not designed to reveal the form of the body, but for entirely different purposes.}

\pnskip
\ger{Die stillschweigenden Abmachungen zum Verst{\"a}ndnis der Umgangssprache sind enorm kompliziert.}
\ogd{The silent adjustments to understand colloquial language are enormously complicated.}
\pmc{The tacit conventions on which the understanding of everyday language depends are enormously complicated.}

\pn{4.003}
\ger{Die meisten S{\"a}tze und Fragen, welche {\"u}ber philosophische Dinge geschrieben worden sind, sind nicht falsch, sondern unsinnig. Wir k{\"o}nnen daher Fragen dieser Art {\"u}berhaupt nicht beantworten, sondern nur ihre Unsinnigkeit feststellen. Die meisten Fragen und S{\"a}tze der Philosophen beruhen darauf, dass wir unsere Sprachlogik nicht verstehen.}
\ogd{Most propositions and questions, that have been written about philosophical matters, are not false, but senseless. We cannot, therefore, answer questions of this kind at all, but only state their senselessness. Most questions and propositions of the philosophers result from the fact that we do not understand the logic of our language.}
\pmc{Most of the propositions and questions to be found in philosophical works are not false but nonsensical. Consequently we cannot give any answer to questions of this kind, but can only point out that they are nonsensical. Most of the propositions and questions of philosophers arise from our failure to understand the logic of our language.}

\pnskip
\ger{(Sie sind von der Art der Frage, ob das Gute mehr oder weniger identisch sei als das Sch{\"o}ne.)}
\ogd{(They are of the same kind as the question whether the Good is more or less identical than the Beautiful.)}
\pmc{(They belong to the same class as the question whether the good is more or less identical than the beautiful.)}

\pnskip
\ger{Und es ist nicht verwunderlich, dass die tiefsten Probleme eigentlich \germph{keine} Probleme sind.}
\ogd{And so it is not to be wondered at that the deepest problems are really \emph{no} problems.}
\pmc{And it is not surprising that the deepest problems are in fact \emph{not} problems at all.}

\pn{4.0031}
\ger{Alle Philosophie ist \gdql Sprachkritik\gdqr{}. (Allerdings nicht im Sinne Mauthners.) Russells Verdienst ist es, gezeigt zu haben, dass die scheinbar logische Form des Satzes nicht seine wirkliche sein muss.}
\ogd{All philosophy is ``Critique of language'' (but not at all in Mauthner's sense). Russell's merit is to have shown that the apparent logical form of the proposition need not be its real form.}
\pmc{All philosophy is a `critique of language' (though not in Mauthner's sense). It was Russell who performed the service of showing that the apparent logical form of a proposition need not be its real one.}

\pn{4.01}
\ger{Der Satz ist ein Bild der Wirklichkeit.}
\ogd{The proposition is a picture of reality.}
\pmc{A proposition is a picture of reality.}

\pnskip
\ger{Der Satz ist ein Modell der Wirklichkeit, so wie wir sie uns denken.}
\ogd{The proposition is a model of the reality as we think it is.}
\pmc{A proposition is a model of reality as we imagine it.}

\pn{4.011}
\ger{Auf den ersten Blick scheint der Satz---wie er etwa auf dem Papier gedruckt steht---kein Bild der Wirklichkeit zu sein, von der er handelt. Aber auch die Notenschrift scheint auf den ersten Blick kein Bild der Musik zu sein, und unsere Lautzeichen-(Buchstaben-)Schrift kein Bild unserer Lautsprache.}
\ogd{At the first glance the proposition---say as it stands printed on paper---does not seem to be a picture of the reality of which it treats. But nor does the musical score appear at first sight to be a picture of a musical piece; nor does our phonetic spelling (letters) seem to be a picture of our spoken language.}
\pmc{At first sight a proposition---one set out on the printed page, for example---does not seem to be a picture of the reality with which it is concerned. But neither do written notes seem at first sight to be a picture of a piece of music, nor our phonetic notation (the alphabet) to be a picture of our speech.}

\pnskip
\ger{Und doch erweisen sich diese Zeichensprachen auch im gew{\"o}hnlichen Sinne als Bilder dessen, was sie darstellen.}
\ogd{And yet these symbolisms prove to be pictures---even in the ordinary sense of the word---of what they represent.}
\pmc{And yet these sign-languages prove to be pictures, even in the ordinary sense, of what they represent.}

\pn{4.012}
\ger{Offenbar ist, dass wir einen Satz von der Form \gdql $aRb$\gdqr{} als Bild empfinden. Hier ist das Zeichen offenbar ein Gleichnis des Bezeichneten.}
\ogd{It is obvious that we perceive a proposition of the form $aRb$ as a picture. Here the sign is obviously a likeness of the signified.}
\pmc{It is obvious that a proposition of the form `$aRb$' strikes us as a picture. In this case the sign is obviously a likeness of what is signified.}

\pn{4.013}
\ger{Und wenn wir in das Wesentliche dieser Bildhaftigkeit eindringen, so sehen wir, dass dieselbe durch \germph{scheinbare Unregelm{\"a}{\ss}igkeiten} (wie die Verwendung von $\sharp$ und $\flat$ in der Notenschrift) \germph{nicht} gest{\"o}rt wird.}
\ogd{And if we penetrate to the essence of this pictorial nature we see that this is not disturbed by \emph{apparent irregularities} (like the use of $\sharp$ and $\flat$  in the score).}
\pmc{And if we penetrate to the essence of this pictorial character, we see that it is \emph{not} impaired by \emph{apparent irregularities} (such as the use of $\sharp$ and $\flat$ in musical notation).}

\pnskip
\ger{Denn auch diese Unregelm{\"a}{\ss}igkeiten bilden das ab, was sie ausdr{\"u}cken sollen; nur auf eine andere Art und Weise.}
\ogd{For these irregularities also picture what they are to express; only in another way.}
\pmc{For even these irregularities depict what they are intended to express; only they do it in a different way.}

\pn{4.014}
\ger{Die Grammophonplatte, der musikalische Gedanke, die Notenschrift, die Schallwellen, stehen alle in jener abbildenden internen Beziehung zu einander, die zwischen Sprache und Welt besteht.}
\ogd{The gramophone record, the musical thought, the score, the waves of sound, all stand to one another in that pictorial internal relation, which holds between language and the world.}
\pmc{A gramophone record, the musical idea, the written notes, and the sound-waves, all stand to one another in the same internal relation of depicting that holds between language and the world. }

\pnskip
\ger{Ihnen allen ist der logische Bau gemeinsam.}
\ogd{To all of them the logical structure is common.}
\pmc{They are all constructed according to a common logical pattern.}

\pnskip
\ger{(Wie im M{\"a}rchen die zwei J{\"u}nglinge, ihre zwei Pferde und ihre Lilien. Sie sind alle in gewissem Sinne Eins.)}
\ogd{(Like the two youths, their two horses and their lilies in the story. They are all in a certain sense one.)}
\pmc{(Like the two youths in the fairy-tale, their two horses, and their lilies. They are all in a certain sense one.)}

\pn{4.0141}
\ger{Dass es eine allgemeine Regel gibt, durch die der Musiker aus der Partitur die Symphonie entnehmen kann, durch welche man aus der Linie auf der Grammophonplatte die Symphonie und nach der ersten Regel wieder die Partitur ableiten kann, darin besteht eben die innere {\"A}hnlichkeit dieser scheinbar so ganz verschiedenen Gebilde. Und jene Regel ist das Gesetz der Projektion, welches die Symphonie in die Notensprache projiziert. Sie ist die Regel der {\"U}bersetzung der Notensprache in die Sprache der Grammophonplatte.}
\ogd{In the fact that there is a general rule by which the musician is able to read the symphony out of the score, and that there is a rule by which one could reconstruct the symphony from the line on a gramophone record and from this again---by means of the first rule---construct the score, herein lies the internal similarity between these things which at first sight seem to be entirely different. And the rule is the law of projection which projects the symphony into the language of the musical score. It is the rule of translation of this language into the language of the gramophone record.}
\pmc{There is a general rule by means of which the musician can obtain the symphony from the score, and which makes it possible to derive the symphony from the groove on the gramophone record, and, using the first rule, to derive the score again. That is what constitutes the inner similarity between these things which seem to be constructed in such entirely different ways. And that rule is the law of projection which projects the symphony into the language of musical notation. It is the rule for translating this language into the language of gramophone records.}

\pn{4.015}
\ger{Die M{\"o}glichkeit aller Gleichnisse, der ganzen Bildhaftigkeit unserer Ausdrucksweise, ruht in der Logik der Abbildung.}
\ogd{The possibility of all similes, of all the images of our language, rests on the logic of representation.}
\pmc{The possibility of all imagery, of all our pictorial modes of expression, is contained in the logic of depiction.}

\pn{4.016}
\ger{Um das Wesen des Satzes zu verstehen, denken wir an die Hieroglyphenschrift, welche die Tatsachen die sie beschreibt abbildet.}
\ogd{In order to understand the essence of the proposition, consider hieroglyphic writing, which pictures the facts it describes.}
\pmc{In order to understand the essential nature of a proposition, we should consider hieroglyphic script, which depicts the facts that it describes.}

\pnskip
\ger{Und aus ihr wurde die Buchstabenschrift, ohne das Wesentliche der Abbildung zu verlieren.}
\ogd{And from it came the alphabet without the essence of the representation being lost.}
\pmc{And alphabetic script developed out of it without losing what was essential to depiction.}

\pn{4.02}
\ger{Dies sehen wir daraus, dass wir den Sinn des Satzzeichens verstehen, ohne dass er uns erkl{\"a}rt wurde.}
\ogd{This we see from the fact that we understand the sense of the propositional sign, without having had it explained to us.}
\pmc{We can see this from the fact that we understand the sense of a propositional sign without its having been explained to us.}

\pn{4.021}
\ger{Der Satz ist ein Bild der Wirklichkeit: Denn ich kenne die von ihm dargestelle Sachlage, wenn ich den Satz verstehe. Und den Satz verstehe ich, ohne dass mir sein Sinn erkl{\"a}rt wurde.}
\ogd{The proposition is a picture of reality, for I know the state of affairs presented by it, if I understand the proposition. And I understand the proposition, without its sense having been explained to me.}
\pmc{A proposition is a picture of reality: for if I understand a proposition, I know the situation that it represents. And I understand the proposition without having had its sense explained to me.}

\pn{4.022}
\ger{Der Satz \germph{zeigt} seinen Sinn.}
\ogd{The proposition \emph{shows} its sense.}
\pmc{A proposition \emph{shows} its sense.}

\pnskip
\ger{Der Satz \germph{zeigt}, wie es sich verh{\"a}lt, \germph{wenn} er wahr ist. Und er \germph{sagt}, \germph{dass} es sich so verh{\"a}lt.}
\ogd{The proposition \emph{shows} how things stand, \emph{if} it is true. And it \emph{says}, that they do so stand.}
\pmc{A proposition \emph{shows} how things stand \emph{if} it is true. And it \emph{says that} they do so stand.}

\pn{4.023}
\ger{Die Wirklichkeit muss durch den Satz auf ja oder nein fixiert sein.}
\ogd{The proposition determines reality to this extent, that one only needs to say ``Yes'' or ``No'' to it to make it agree with reality.}
\pmc{A proposition must restrict reality to two alternatives: yes or no.}

\pnskip
\ger{Dazu muss sie durch ihn vollst{\"a}ndig beschrieben werden.}
\ogd{Reality must therefore be completely described by the proposition.}
\pmc{In order to do that, it must describe reality completely.}

\pnskip
\ger{Der Satz ist die Beschreibung eines Sachverhaltes.}
\ogd{A proposition is the description of a fact.}
\pmc{A proposition is a description of a state of affairs.}

\pnskip
\ger{Wie die Beschreibung einen Gegenstand nach seinen externen Eigenschaften, so beschreibt der Satz die Wirklichkeit nach ihren internen Eigenschaften.}
\ogd{As the description of an object describes it by its external properties so propositions describe reality by its internal properties.}
\pmc{Just as a description of an object describes it by giving its external properties, so a proposition describes reality by its internal properties.}

\pnskip
\ger{Der Satz konstruiert eine Welt mit Hilfe eines logischen Ger{\"u}stes und darum kann man am Satz auch sehen, wie sich alles Logische verh{\"a}lt, \germph{wenn} er wahr ist. Man kann aus einem falschen Satz \germph{Schl{\"u}sse ziehen}.}
\ogd{The proposition constructs a world with the help of a logical scaffolding, and therefore one can actually see in the proposition all the logical features possessed by reality \emph{if} it is true. One can \emph{draw conclusions} from a false proposition.}
\pmc{A proposition constructs a world with the help of a logical scaffolding, so that one can actually see from the proposition how everything stands logically \emph{if} it is true. One can \emph{draw inferences} from a false proposition.}

\pn{4.024}
\ger{Einen Satz verstehen, hei{\ss}t, wissen was der Fall ist, wenn er wahr ist.}
\ogd{To understand a proposition means to know what is the case, if it is true.}
\pmc{To understand a proposition means to know what is the case if it is true.}

\pnskip
\ger{(Man kann ihn also verstehen, ohne zu wissen, ob er wahr ist.)}
\ogd{(One can therefore understand it without knowing whether it is true or not.)}
\pmc{(One can understand it, therefore, without knowing whether it is true.)}

\pnskip
\ger{Man versteht ihn, wenn man seine Bestandteile versteht.}
\ogd{One understands it if one understands it constituent parts.}
\pmc{It is understood by anyone who understands its constituents.}

\pn{4.025}
\ger{Die {\"U}bersetzung einer Sprache in eine andere geht nicht so vor sich, dass man jeden \germph{Satz} der einen in einen \germph{Satz} der anderen {\"u}bersetzt, sondern nur die Satzbestandteile werden {\"u}bersetzt.}
\ogd{The translation of one language into another is not a process of translating each proposition of the one into a proposition of the other, but only the constituent parts of propositions are translated.}
\pmc{When translating one language into another, we do not proceed by translating each \emph{proposition} of the one into a \emph{proposition} of the other, but merely by translating the constituents of propositions.}

\pnskip
\ger{(Und das W{\"o}rterbuch {\"u}bersetzt nicht nur Substantiva, sondern auch Zeit-, Eigenschafts- und Bindew{\"o}rter etc.; und es behandelt sie alle gleich.)}
\ogd{(And the dictionary does not only translate substantives but also adverbs and conjunctions, etc., and it treats them all alike.)}
\pmc{(And the dictionary translates not only substantives, but also verbs, adjectives, and conjunctions, etc.; and it treats them all in the same way.)}

\pn{4.026}
\ger{Die Bedeutung der einfachen Zeichen (der W{\"o}rter) m{\"u}ssen uns erkl{\"a}rt werden, dass wir sie verstehen.}
\ogd{The meanings of the simple signs (the words) must be explained to us, if we are to understand them.}
\pmc{The meanings of simple signs (words) must be explained to us if we are to understand them.}

\pnskip
\ger{Mit den S{\"a}tzen aber verst{\"a}ndigen wir uns.}
\ogd{By means of propositions we explain ourselves.}
\pmc{With propositions, however, we make ourselves understood.}

\pn{4.027}
\ger{Es liegt im Wesen des Satzes, dass er uns einen \germph{neuen} Sinn mitteilen kann.}
\ogd{It is essential to propositions, that they can communicate a \emph{new} sense to us.}
\pmc{It belongs to the essence of a proposition that it should be able to communicate a \emph{new} sense to us.}

\pn{4.03}
\ger{Ein Satz muss mit alten Ausdr{\"u}cken einen neuen Sinn mitteilen.}
\ogd{A proposition must communicate a new sense with old words.}
\pmc{A proposition must use old expressions to communicate a new sense.}

\pnskip
\ger{Der Satz teilt uns eine Sachlage mit, also muss er \germph{wesentlich} mit der Sachlage zusammenh{\"a}ngen.}
\ogd{The proposition communicates to us a state of affairs, therefore it must be \emph{essentially} connected with the state of affairs.}
\pmc{A proposition communicates a situation to us, and so it must be \emph{essentially} connected with the situation.}

\pnskip
\ger{Und der Zusammenhang ist eben, dass er ihr logisches Bild ist.}
\ogd{And the connexion is, in fact, that it is its logical picture.}
\pmc{And the connexion is precisely that it is its logical picture.}

\pnskip
\ger{Der Satz sagt nur insoweit etwas aus, als er ein Bild ist.}
\ogd{The proposition only asserts something, in so far as it is a picture.}
\pmc{A proposition states something only in so far as it is a picture.}

\pn{4.031}
\ger{Im Satz wird gleichsam eine Sachlage probeweise zusammengestellt.}
\ogd{In the proposition a state of affairs is, as it were, put together for the sake of experiment.}
\pmc{In a proposition a situation is, as it were, constructed by way of experiment.}

\pnskip
\ger{Man kann geradezu sagen: statt, dieser Satz hat diesen und diesen Sinn; dieser Satz stellt diese und diese Sachlage dar.}
\ogd{One can say, instead of, This proposition has such and such a sense, This proposition represents such and such a state of affairs.}
\pmc{Instead of, `This proposition has such and such a sense', we can simply say, `This proposition represents such and such a situation'.}

\pn{4.0311}
\ger{Ein Name steht f{\"u}r ein Ding, ein anderer f{\"u}r ein anderes Ding und untereinander sind sie verbunden, so stellt das Ganze---wie ein lebendes Bild---den Sachverhalt vor.}
\ogd{One name stands for one thing, and another for another thing, and they are connected together. And so the whole, like a living picture, presents the atomic fact.}
\pmc{One name stands for one thing, another for another thing, and they are combined with one another. In this way the whole group---like a \emph{tableau vivant}---presents a state of affairs.}

\pn{4.0312}
\ger{Die M{\"o}glichkeit des Satzes beruht auf dem Prinzip der Vertretung von Gegenst{\"a}nden durch Zeichen.}
\ogd{The possibility of propositions is based upon the principle of the representation of objects by signs.}
\pmc{The possibility of propositions is based on the principle that objects have signs as their representatives.}

\pnskip
\ger{Mein Grundgedanke ist, dass die \gdql logischen Konstanten\gdqr{} nicht vertreten. Dass sich die \germph{Logik} der Tatsachen nicht vertreten l{\"a}sst.}
\ogd{My fundamental thought is that the ``logical constants'' do not represent. That the \emph{logic} of the facts cannot be represented.}
\pmc{My fundamental idea is that the `logical constants' are not representatives; that there can be no representatives of the \emph{logic} of facts.}

\pn{4.032}
\ger{Nur insoweit ist der Satz ein Bild der Sachlage, als er logisch gegliedert ist.}
\ogd{The proposition is a picture of its state of affairs, only in so far as it is logically articulated.}
\pmc{It is only in so far as a proposition is logically articulated that it is a picture of a situation.}

\pnskip
\ger{(Auch der Satz: \gdql ambulo\gdqr{}, ist zusammengesetzt, denn sein Stamm ergibt mit einer anderen Endung, und seine Endung mit einem anderen Stamm, einen anderen Sinn.)}
\ogd{(Even the proposition ``ambulo'' is composite, for its stem gives a different sense with another termination, or its termination with another stem.)}
\pmc{(Even the proposition, \emph{Ambulo}, is composite: for its stem with a different ending yields a different sense, and so does its ending with a different stem.)}

\pn{4.04}
\ger{Am Satz muss gerade soviel zu unterscheiden sein, als an der Sachlage, die er darstellt.}
\ogd{In the proposition there must be exactly as many thing distinguishable as there are in the state of affairs, which it represents.}
\pmc{In a proposition there must be exactly as many distinguishable parts as in the situation that it represents.}

\pnskip
\ger{Die beiden m{\"u}ssen die gleiche logische (mathematische) Mannigfaltigkeit besitzen. (Vergleiche Hertz's \gdql Mechanik\gdqr{}, {\"u}ber dynamische Modelle.)}
\ogd{They must both possess the same logical (mathematical) multiplicity (cf. Hertz's Mechanics, on Dynamic Models).}
\pmc{The two must possess the same logical (mathematical) multiplicity. (Compare Hertz's \textit{Mechanics} on dynamical models.)}

\pn{4.041}
\ger{Diese mathematische Mannigfaltigkeit kann man nat{\"u}rlich nicht selbst wieder abbilden. Aus ihr kann man beim Abbilden nicht heraus.}
\ogd{This mathematical multiplicity naturally cannot in its turn be represented. One cannot get outside it in the representation.}
\pmc{This mathematical multiplicity, of course, cannot itself be the subject of depiction. One cannot get away from it when depicting.}

\pn{4.0411}
\ger{Wollten wir z.\ B.\ das, was wir durch \gdql $\ralld{x} f\negthinspace x$\gdqr{} ausdr{\"u}cken, durch Vorsetzen eines Indexes von \gdql $f\negthinspace x$\gdqr{} ausdr{\"u}cken---etwa so: \gdql $\mathop{\mathrm{Alg.}}f\negthinspace x$\gdqr{}---es w{\"u}rde nicht gen{\"u}gen---wir w{\"u}ssten nicht, was verallgemeinert wurde. Wollten wir es durch einen Index \gdql $_a$\gdqr{} anzeigen---etwa so: \gdql $f(x_a)$\gdqr{}---es w{\"u}rde auch nicht gen{\"u}gen---wir w{\"u}ssten nicht den Bereich der Allgemeinheitsbezeichnung.}
\ogd{If we tried, for example, to express what is expressed by ``$\ralld{x} f\negthinspace x$'' by putting an index before $f\negthinspace x$, like: ``$\mathop{\mathrm{Gen.}} f\negthinspace x$'', it would not do, we should not know what was generalized. If we tried to show it by an index $g$, like: ``$f(x_g)$'' it would not do---we should not know the scope of the generalization.}
\pmc{If, for example, we wanted to express what we now write as `$\ralld{x} f\negthinspace x$' by putting an affix in front of `$f\negthinspace x$'---for instance by writing `$\mathop{\mathrm{Gen.}} f\negthinspace x$'---it would not be adequate: we should not know what was being generalized. If we wanted to signalize it with an affix `$g$'---for instance by writing `$f(x_g)$'---that would not be adequate either: we should not know the scope of the generality-sign.}

\pnskip
\ger{Wollten wir es durch Einf{\"u}hrung einer Marke in die Argumentstellen versuchen---etwa so: \gdql $\mathop{(A, A)\mathord{.}} F(A, A)$\gdqr{}---es w{\"u}rde nicht gen{\"u}gen---wir k{\"o}nnten die Identit{\"a}t der Variablen nicht feststellen. U.s.w.}
\ogd{If we were to try it by introducing a mark in the argument places, like ``$\mathop{(G, G)\mathord{.}} F(G, G)$'', it would not do---we could not determine the identity of the variables, etc.}
\pmc{If we were to try to do it by introducing a mark into the argument-places---for instance by writing `$\mathop{(G, G)\mathord{.}} F(G, G)$' ---it would not be adequate: we should not be able to establish the identity of the variables. And so on.}

\pnskip
\ger{Alle diese Bezeichnungsweisen gen{\"u}gen nicht, weil sie nicht die notwendige mathematische Mannigfaltigkeit haben.}
\ogd{All these ways of symbolizing are inadequate because they have not the necessary mathematical multiplicity.}
\pmc{All these modes of signifying are inadequate because they lack the necessary mathematical multiplicity.}

\pn{4.0412}
\ger{Aus demselben Grunde gen{\"u}gt die idealistische Erkl{\"a}rung des Sehens der r{\"a}umlichen Beziehung durch die \gdql Raumbrille\gdqr{} nicht, weil sie nicht die Mannigfaltigkeit dieser Beziehungen erkl{\"a}ren kann.}
\ogd{For the same reason the idealist explanation of the seeing of spatial relations through ``spatial spectacles'' does not do, because it cannot explain the multiplicity of these relations.}
\pmc{For the same reason the idealist's appeal to `spatial spectacles' is inadequate to explain the seeing of spatial relations, because it cannot explain the multiplicity of these relations.}

\pn{4.05}
\ger{Die Wirklichkeit wird mit dem Satz verglichen.}
\ogd{Reality is compared with the proposition.}
\pmc{Reality is compared with propositions.}

\pn{4.06}
\ger{Nur dadurch kann der Satz wahr oder falsch sein, indem er ein Bild der Wirklichkeit ist.}
\ogd{Propositions can be true or false only by being pictures of the reality.}
\pmc{A proposition can be true or false only in virtue of being a picture of reality.}

\pn{4.061}
\ger{Beachtet man nicht, dass der Satz einen von den Tatsachen unabh{\"a}ngigen Sinn hat, so kann man leicht glauben, dass wahr und falsch gleichberechtigte Beziehungen von Zeichen und Bezeichnetem sind.}
\ogd{If one does not observe that propositions have a sense independent of the facts, one can easily believe that true and false are two relations between signs and things signified with equal rights.}
\pmc{It must not be overlooked that a proposition has a sense that is independent of the facts: otherwise one can easily suppose that true and false are relations of equal status between signs and what they signify.}

\pnskip
\ger{Man k{\"o}nnte dann z.\ B.\ sagen, dass \gdql $p$\gdqr{} auch die wahre Art bezeichnet, was \gdql $\rnot p$\gdqr{} auf die falsche Art, etc.}
\ogd{One could, then, for example, say that ``$p$'' signifies in the true way what ``$\rnot p$'' signifies in the false way, etc.}
\pmc{In that case one could say, for example, that `$p$' signified in the true way what `$\rnot p$' signified in the false way, etc.}

\pn{4.062}
\ger{Kann man sich nicht mit falschen S{\"a}tzen, wie bisher mit wahren, verst{\"a}ndigen? Solange man nur wei{\ss}, dass sie falsch gemeint sind. Nein! Denn, wahr ist ein Satz, wenn es sich so verh{\"a}lt, wie wir es durch ihn sagen; und wenn wir mit \gdql $p$\gdqr{} $\rnot p$ meinen, und es sich so verh{\"a}lt wie wir es meinen, so ist \gdql $p$\gdqr{} in der neuen Auffassung wahr und nicht falsch.}
\ogd{Can we not make ourselves understood by means of false propositions as hitherto with true ones, so long as we know that they are meant to be false? No! For a proposition is true, if what we assert by means of it is the case; and if by ``$p$'' we mean $\rnot p$, and what we mean is the case, then ``$p$'' in the new conception is true and not false.}
\pmc{Can we not make ourselves understood with false propositions just as we have done up till now with true ones?---So long as it is known that they are meant to be false.---No! For a proposition is true if we use it to say that things stand in a certain way, and they do; and if by `$p$' we mean $\rnot p$ and things stand as we mean that they do, then, construed in the new way, `$p$' is true and not false.}

\pn{4.0621}
\ger{Dass aber die Zeichen \gdql $p$\gdqr{} und \gdql $\rnot p$\gdqr{} das gleiche sagen \germph{k{\"o}nnen}, ist wichtig. Denn es zeigt, dass dem Zeichen \gdql $\rnot$\gdqr{} in der Wirklichkeit nichts entspricht.}
\ogd{That, however, the signs ``$p$'' and ``$\rnot p$'' \emph{can} say the same thing is important, for it shows that the sign ``$\rnot$'' corresponds to nothing in reality.}
\pmc{But it is important that the signs `$p$' and `$\rnot p$' can say the same thing. For it shows that nothing in reality corresponds to the sign `$\rnot$'.}

\pnskip
\ger{Dass in einem Satz die Verneinung vorkommt, ist noch kein Merkmal seines Sinnes ($\rnot\rnot p=p$).}
\ogd{That negation occurs in a proposition, is no characteristic of its sense ($\rnot\rnot p=p$).}
\pmc{The occurrence of negation in a proposition is not enough to characterize its sense ($\rnot\rnot p = p$).}

\pnskip
\ger{Die S{\"a}tze \gdql $p$\gdqr{} und \gdql $\rnot p$\gdqr{} haben entgegengesetzten Sinn, aber es entspricht ihnen eine und dieselbe Wirklichkeit.}
\ogd{The propositions ``$p$'' and ``$\rnot p$'' have opposite senses, but to them corresponds one and the same reality.}
\pmc{The propositions `$p$' and `$\rnot p$' have opposite sense, but there corresponds to them one and the same reality.}

\pn{4.063}
\ger{Ein Bild zur Erkl{\"a}rung des Wahrheitsbegriffes: Schwarzer Fleck auf wei{\ss}em Papier; die Form des Fleckes kann man beschreiben, indem man f{\"u}r jeden Punkt der Fl{\"a}che angibt, ob er wei{\ss} oder schwarz ist. Der Tatsache, dass ein Punkt schwarz ist, entspricht eine positive---der, dass ein Punkt wei{\ss} (nicht schwarz) ist, eine negative Tatsache. Bezeichne ich einen Punkt der Fl{\"a}che (einen Fregeschen Wahrheitswert), so entspricht dies der Annahme, die zur Beurteilung aufgestellt wird, etc.\ etc.}
\ogd{An illustration to explain the concept of truth. A black spot on white paper; the form of the spot can be described by saying of each point of the plane whether it is white or black. To the fact that a point is black corresponds a positive fact; to the fact that a point is white (not black), a negative fact. If I indicate a point of the plane (a truth-value in Frege's terminology), this corresponds to the assumption proposed for judgment, etc.\ etc.}
\pmc{An analogy to illustrate the concept of truth: imagine a black spot on white paper: you can describe the shape of the spot by saying, for each point on the sheet, whether it is black or white. To the fact that a point is black there corresponds a positive fact, and to the fact that a point is white (not black), a negative fact. If I designate a point on the sheet (a truth-value according to Frege), then this corresponds to the supposition that is put forward for judgement, etc.\ etc.}

\pnskip
\ger{Um aber sagen zu k{\"o}nnen, ein Punkt sei schwarz oder wei{\ss}, muss ich vorerst wissen, wann man einen Punkt schwarz und wann man ihn wei{\ss} nennt; um sagen zu k{\"o}nnen: \gdql $p$\gdqr{} ist wahr (oder falsch), muss ich bestimmt haben, unter welchen Umst{\"a}nden ich \gdql $p$\gdqr{} wahr nenne, und damit bestimme ich den Sinn des Satzes.}
\ogd{But to be able to say that a point is black or white, I must first know under what conditions a point is called white or black; in order to be able to say ``$p$'' is true (or false) I must have determined under what conditions I call ``$p$'' true, and thereby I determine the sense of the proposition.}
\pmc{But in order to be able to say that a point is black or white, I must first know when a point is called black, and when white: in order to be able to say, `{}``$p$'' is true (or false)', I must have determined in what circumstances I call `$p$' true, and in so doing I determine the sense of the proposition.}

\pnskip
\ger{Der Punkt, an dem das Gleichnis hinkt ist nun der: Wir k{\"o}nnen auf einen Punkt des Papiers zeigen, auch ohne zu wissen, was wei{\ss} und schwarz ist; einem Satz ohne Sinn aber entspricht gar nichts, denn er bezeichnet kein Ding (Wahrheitswert) dessen Eigenschaften etwa \gdql falsch\gdqr{} oder \gdql wahr\gdqr{} hie{\ss}en; das Verbum eines Satzes ist nicht \gdql ist wahr\gdqr{} oder \gdql ist falsch\gdqr{}---wie Frege glaubte---, sondern das, was \gdql wahr ist\gdqr{}, muss das Verbum schon enthalten.}
\ogd{The point at which the simile breaks down is this: we can indicate a point on the paper, without knowing what white and black are; but to a proposition without a sense corresponds nothing at all, for it signifies no thing (truth-value) whose properties are called ``false'' or ``true''; the verb of the proposition is not ``is true'' or ``is false''---as Frege thought---but that which ``is true'' must already contain the verb.}
\pmc{Now the point where the simile breaks down is this: we can indicate a point on the paper even if we do not know what black and white are, but if a proposition has no sense, nothing corresponds to it, since it does not designate a thing (a truth-value) which might have properties called `false' or `true'. The verb of a proposition is not `is true' or `is false', as Frege thought: rather, that which `is true' must already contain the verb.}

\pn{4.064}
\ger{Jeder Satz muss \germph{schon} einen Sinn haben; die Bejahung kann ihn ihm nicht geben, denn sie bejaht ja gerade den Sinn. Und dasselbe gilt von der Verneinung, etc.}
\ogd{Every proposition must \emph{already} have a sense; assertion cannot give it a sense, for what it asserts is the sense itself. And the same holds of denial, etc.}
\pmc{Every proposition must \emph{already} have a sense: it cannot be given a sense by affirmation. Indeed its sense is just what is affirmed. And the same applies to negation, etc.}

\pn{4.0641}
\ger{Man k{\"o}nnte sagen: Die Verneinung bezieht sich schon auf den logischen Ort, den der verneinte Satz bestimmt.}
\ogd{One could say, the denial is already related to the logical place determined by the proposition that is denied.}
\pmc{One could say that negation must be related to the logical place determined by the negated proposition.}

\pnskip
\ger{Der verneinende Satz bestimmt einen \germph{anderen} logischen Ort als der verneinte.}
\ogd{The denying proposition determines a logical place \emph{other} than does the proposition denied.}
\pmc{The negating proposition determines a logical place \emph{different} from that of the negated proposition.}

\pnskip
\ger{Der verneinende Satz bestimmt einen logischen Ort mit Hilfe des logischen Ortes des verneinten Satzes, indem er jenen als au{\ss}erhalb diesem liegend beschreibt.}
\ogd{The denying proposition determines a logical place, with the help of the logical place of the proposition denied, by saying that it lies outside the latter place.}
\pmc{The negating proposition determines a logical place with the help of the logical place of the negated proposition. For it describes it as lying outside the latter's logical place.}

\pnskip
\ger{Dass man den verneinten Satz wieder verneinen kann, zeigt schon, dass das, was verneint wird, schon ein Satz und nicht erst die Vorbereitung zu einem Satze ist.}
\ogd{That one can deny again the denied proposition, shows that what is denied is already a proposition and not merely the preliminary to a proposition.}
\pmc{The negated proposition can be negated again, and this in itself shows that what is negated is already a proposition, and not merely something that is preliminary to a proposition.}

\pn{4.1}
\ger{Der Satz stellt das Bestehen und Nichtbestehen der Sachverhalte dar.}
\ogd{A proposition presents the existence and non-existence of atomic facts.}
\pmc{Propositions represent the existence and non-existence of states of affairs.}

\pn{4.11}
\ger{Die Gesamtheit der wahren S{\"a}tze ist die gesamte Naturwissenschaft (oder die Gesamtheit der Naturwissenschaften).}
\ogd{The totality of true propositions is the total natural science (or the totality of the natural sciences).}
\pmc{The totality of true propositions is the whole of natural science (or the whole corpus of the natural sciences).}

\pn{4.111}
\ger{Die Philosophie ist keine der Naturwissenschaften.}
\ogd{Philosophy is not one of the natural sciences.}
\pmc{Philosophy is not one of the natural sciences.}

\pnskip
\ger{(Das Wort \gdql Philosophie\gdqr{} muss etwas bedeuten, was {\"u}ber oder unter, aber nicht neben den Naturwissenschaften steht.)}
\ogd{(The word ``philosophy'' must mean something which stands above or below, but not beside the natural sciences.)}
\pmc{(The word `philosophy' must mean something whose place is above or below the natural sciences, not beside them.)}

\pn{4.112}
\ger{Der Zweck der Philosophie ist die logische Kl{\"a}rung der Gedanken.}
\ogd{The object of philosophy is the logical clarification of thoughts.}
\pmc{Philosophy aims at the logical clarification of thoughts.}

\pnskip
\ger{Die Philosophie ist keine Lehre, sondern eine T{\"a}tigkeit.}
\ogd{Philosophy is not a theory but an activity.}
\pmc{Philosophy is not a body of doctrine but an activity.}

\pnskip
\ger{Ein philosophisches Werk besteht wesentlich aus Erl{\"a}uterungen.}
\ogd{A philosophical work consists essentially of elucidations.}
\pmc{A philosophical work consists essentially of elucidations.}

\pnskip
\ger{Das Resultat der Philosophie sind nicht \gdql philosophische S{\"a}tze\gdqr{}, sondern das Klarwerden von S{\"a}tzen.}
\ogd{The result of philosophy is not a number of ``philosophical propositions'', but to make propositions clear.}
\pmc{Philosophy does not result in `philosophical propositions', but rather in the clarification of propositions.}

\pnskip
\ger{Die Philosophie soll die Gedanken, die sonst, gleichsam, tr{\"u}be und verschwommen sind, klar machen und scharf abgrenzen.}
\ogd{Philosophy should make clear and delimit sharply the thoughts which otherwise are, as it were, opaque and blurred.}
\pmc{Without philosophy thoughts are, as it were, cloudy and indistinct: its task is to make them clear and to give them sharp boundaries.}

\pn{4.1121}
\ger{Die Psychologie ist der Philosophie nicht verwandter als irgend eine andere Naturwissenschaft.}
\ogd{Psychology is no nearer related to philosophy, than is any other natural science.}
\pmc{Psychology is no more closely related to philosophy than any other natural science.}

\pnskip
\ger{Erkenntnistheorie ist die Philosophie der Psychologie.}
\ogd{The theory of knowledge is the philosophy of psychology.}
\pmc{Theory of knowledge is the philosophy of psychology.}

\pnskip
\ger{Entspricht nicht mein Studium der Zeichensprache dem Studium der Denkprozesse, welches die Philosophen f{\"u}r die Philosophie der Logik f{\"u}r so wesentlich hielten? Nur verwickelten sie sich meistens in unwesentliche psychologische Untersuchungen und eine analoge Gefahr gibt es auch bei meiner Methode.}
\ogd{Does not my study of sign-language correspond to the study of thought processes which philosophers held to be so essential to the philosophy of logic? Only they got entangled for the most part in unessential psychological investigations, and there is an analogous danger for my method.}
\pmc{Does not my study of sign-language correspond to the study of thought-processes, which philosophers used to consider so essential to the philosophy of logic? Only in most cases they got entangled in unessential psychological investigations, and with my method too there is an analogous risk.}

\pn{4.1122}
\ger{Die Darwinsche Theorie hat mit der Philosophie nicht mehr zu schaffen als irgendeine andere Hypothese der Naturwissenschaft.}
\ogd{The Darwinian theory has no more to do with philosophy than has any other hypothesis of natural science.}
\pmc{Darwin's theory has no more to do with philosophy than any other hypothesis in natural science.}

\pn{4.113}
\ger{Die Philosophie begrenzt das bestreitbare Gebiet der Naturwissenschaft.}
\ogd{Philosophy limits the disputable sphere of natural science.}
\pmc{Philosophy sets limits to the much disputed sphere of natural science.}

\pn{4.114}
\ger{Sie soll das Denkbare abgrenzen und damit das Undenkbare.}
\ogd{It should limit the thinkable and thereby the unthinkable.}
\pmc{It must set limits to what can be thought; and, in doing so, to what cannot be thought.}

\pnskip
\ger{Sie soll das Undenkbare von innen durch das Denkbare begrenzen.}
\ogd{It should limit the unthinkable from within through the thinkable.}
\pmc{It must set limits to what cannot be thought by working outwards through what can be thought.}

\pn{4.115}
\ger{Sie wird das Unsagbare bedeuten, indem sie das Sagbare klar darstellt.}
\ogd{It will mean the unspeakable by clearly displaying the speakable.}
\pmc{It will signify what cannot be said, by presenting clearly what can be said.}

\pn{4.116}
\ger{Alles was {\"u}berhaupt gedacht werden kann, kann klar gedacht werden. Alles, was sich aussprechen l{\"a}{\ss}t, l{\"a}{\ss}t sich klar aussprechen.}
\ogd{Everything that can be thought at all can be thought clearly. Everything that can be said can be said clearly.}
\pmc{Everything that can be thought at all can be thought clearly. Everything that can be put into words can be put clearly.}

\pn{4.12}
\ger{Der Satz kann die gesamte Wirklichkeit darstellen, aber er kann nicht das darstellen, was er mit der Wirklichkeit gemein haben muss, um sie darstellen zu k{\"o}nnen---die logische Form.}
\ogd{Propositions can represent the whole reality, but they cannot represent what they must have in common with reality in order to be able to represent it---the logical form.}
\pmc{Propositions can represent the whole of reality, but they cannot represent what they must have in common with reality in order to be able to represent it---logical form.}

\pnskip
\ger{Um die logische Form darstellen zu k{\"o}nnen, m{\"u}ssten wir uns mit dem Satze au{\ss}erhalb der Logik aufstellen k{\"o}nnen, das hei{\ss}t au{\ss}erhalb der Welt.}
\ogd{To be able to represent the logical form, we should have to be able to put ourselves with the propositions outside logic, that is outside the world.}
\pmc{In order to be able to represent logical form, we should have to be able to station ourselves with propositions somewhere outside logic, that is to say outside the world.}

\pn{4.121}
\ger{Der Satz kann die logische Form nicht darstellen, sie spiegelt sich in ihm.}
\ogd{Propositions cannot represent the logical form: this mirrors itself in the propositions.}
\pmc{Propositions cannot represent logical form: it is mirrored in them.}

\pnskip
\ger{Was sich in der Sprache spiegelt, kann sie nicht darstellen.}
\ogd{That which mirrors itself in language, language cannot represent.}
\pmc{What finds its reflection in language, language cannot represent.}

\pnskip
\ger{Was \germph{sich} in der Sprache ausdr{\"u}ckt, k{\"o}nnen \germph{wir} nicht durch sie ausdr{\"u}cken.}
\ogd{That which expresses \emph{itself} in language, \emph{we} cannot express by language.}
\pmc{What expresses \emph{itself} in language, \emph{we} cannot express by means of language.}

\pnskip
\ger{Der Satz \germph{zeigt} die logische Form der Wirklichkeit.}
\ogd{The propositions \emph{show} the logical form of reality.}
\pmc{Propositions \emph{show} the logical form of reality.}

\pnskip
\ger{Er weist sie auf.}
\ogd{They exhibit it.}
\pmc{They display it.}

\pn{4.1211}
\ger{So zeigt ein Satz \gdql $f\negthinspace a$\gdqr{}, dass in seinem Sinn der Gegenstand $a$ vorkommt, zwei S{\"a}tze \gdql $f\negthinspace a$\gdqr{} und \gdql $ga$\gdqr{}, dass in ihnen beiden von demselben Gegenstand die Rede ist.}
\ogd{Thus a proposition ``$f\negthinspace a$'' shows that in its sense the object $a$ occurs, two propositions ``$f\negthinspace a$'' and ``$ga$'' that they are both about the same object.}
\pmc{Thus one proposition `$f\negthinspace a$' shows that the object $a$ occurs in its sense, two propositions `$f\negthinspace a$' and `$ga$' show that the same object is mentioned in both of them.}

\pnskip
\ger{Wenn zwei S{\"a}tze einander widersprechen. So zeigt dies ihre Struktur; ebenso, wenn einer aus dem anderen folgt. U.s.w.}
\ogd{If two propositions contradict one another, this is shown by their structure; similarly if one follows from another, etc.}
\pmc{If two propositions contradict one another, then their structure shows it; the same is true if one of them follows from the other. And so on.}

\pn{4.1212}
\ger{Was gezeigt werden \germph{kann}, \germph{kann} nicht gesagt werden.}
\ogd{What \emph{can} be shown \emph{cannot} be said.}
\pmc{What \emph{can} be shown, \emph{cannot} be said.}

\pn{4.1213}
\ger{Jetzt verstehen wir auch unser Gef{\"u}hl: dass wir im Besitze einer richtigen logischen Auffassung seien, wenn nur einmal alles in unserer Zeichensprache stimmt.}
\ogd{Now we understand our feeling that we are in possession of the right logical conception, if only all is right in our symbolism.}
\pmc{Now, too, we understand our feeling that once we have a sign-language in which everything is all right, we already have a correct logical point of view.}

\pn{4.122}
\ger{Wir k{\"o}nnen in gewissem Sinne von formalen Eigenschaften der Gegenst{\"a}nde und Sachverhalte bezw.\ von Eigenschaften der Struktur der Tatsachen reden, und in demselben Sinne von formalen Relationen und Relationen von Strukturen.}
\ogd{We can speak in a certain sense of formal properties of objects and atomic facts, or of properties of the structure of facts, and in the same sense of formal relations and relations of structures.}
\pmc{In a certain sense we can talk about formal properties of objects and states of affairs, or, in the case of facts, about structural properties: and in the same sense about formal relations and structural relations.}

\pnskip
\ger{(Statt Eigenschaft der Struktur sage ich auch \gdql interne Eigenschaft\gdqr{}; statt Relation der Strukturen \gdql interne Relation\gdqr{}.}
\ogd{(Instead of property of the structure I also say ``internal property''; instead of relation of structures ``internal relation''.}
\pmc{(Instead of `structural property' I also say `internal property'; instead of `structural relation', `internal relation'.}

\pnskip
\ger{Ich f{\"u}hre diese Ausdr{\"u}cke ein, um den Grund der bei den Philosophen sehr verbreiteten Verwechslung zwischen den internen Relationen und den eigentlichen (externen) Relationen zu zeigen.)}
\ogd{I introduce these expressions in order to show the reason for the confusion, very widespread among philosophers, between internal relations and proper (external) relations.)}
\pmc{I introduce these expressions in order to indicate the source of the confusion between internal relations and relations proper (external relations), which is very widespread among philosophers.)}

\pnskip
\ger{Das Bestehen solcher interner Eigenschaften und Relationen kann aber nicht durch S{\"a}tze behauptet werden, sondern es zeigt sich in den S{\"a}tzen, welche jene Sachverhalte darstellen und von jenen Gegenst{\"a}nden handeln.}
\ogd{The holding of such internal properties and relations cannot, however, be asserted by propositions, but it shows itself in the propositions, which present the facts and treat of the objects in question.}
\pmc{It is impossible, however, to assert by means of propositions that such internal properties and relations obtain: rather, this makes itself manifest in the propositions that represent the relevant states of affairs and are concerned with the relevant objects.}

\pn{4.1221}
\ger{Eine interne Eigenschaft einer Tatsache k{\"o}nnen wir auch einen Zug dieser Tatsache nennen. (In dem Sinn, in welchem wir etwa von Gesichtsz{\"u}gen sprechen.)}
\ogd{An internal property of a fact we also call a feature of this fact. (In the sense in which we speak of facial features.)}
\pmc{An internal property of a fact can also be called a feature of that fact (in the sense in which we speak of facial features, for example).}

\pn{4.123}
\ger{Eine Eigenschaft ist intern, wenn es undenkbar ist, dass ihr Gegenstand sie nicht besitzt.}
\ogd{A property is internal if it is unthinkable that its object does not possess it.}
\pmc{A property is internal if it is unthinkable that its object should not possess it.}

\pnskip
\ger{(Diese blaue Farbe und jene stehen in der internen Relation von heller und dunkler eo ipso. Es ist undenkbar, dass \germph{diese} beiden Gegenst{\"a}nde nicht in dieser Relation st{\"u}nden.)}
\ogd{(This bright blue colour and that stand in the internal relation of bright and darker eo ipso. It is unthinkable that \emph{these} two objects should not stand in this relation.)}
\pmc{(This shade of blue and that one stand, \emph{eo ipso}, in the internal relation of lighter to darker. It is unthinkable that \emph{these} two objects should not stand in this relation.)}

\pnskip
\ger{(Hier entspricht dem schwankenden Gebrauch der Worte \gdql Eigenschaft\gdqr{} und \gdql Relation\gdqr{} der schwankende Gebrauch des Wortes \gdql Gegenstand\gdqr{}.)}
\ogd{(Here to the shifting use of the words ``property'' and ``relation'' there corresponds the shifting use of the word ``object''.)}
\pmc{(Here the shifting use of the word `object' corresponds to the shifting use of the words `property' and `relation'.)}

\pn{4.124}
\ger{Das Bestehen einer internen Eigenschaft einer m{\"o}glichen Sachlage wird nicht durch einen Satz ausgedr{\"u}ckt, sondern es dr{\"u}ckt sich in dem sie darstellenden Satz durch eine interne Eigenschaft dieses Satzes aus.}
\ogd{The existence of an internal property of a possible state of affairs is not expressed by a proposition, but it expresses itself in the proposition which presents that state of affairs, by an internal property of this proposition.}
\pmc{The existence of an internal property of a possible situation is not expressed by means of a proposition: rather, it expresses itself in the proposition representing the situation, by means of an internal property of that proposition.}

\pnskip
\ger{Es w{\"a}re ebenso unsinnig, dem Satze eine formale Eigenschaft zuzusprechen, als sie ihm abzusprechen.}
\ogd{It would be as senseless to ascribe a formal property to a proposition as to deny it the formal property.}
\pmc{It would be just as nonsensical to assert that a proposition had a formal property as to deny it.}

\pn{4.1241}
\ger{Formen kann man nicht dadurch voneinander unterscheiden, dass man sagt, die eine habe diese, die andere aber jene Eigenschaft; denn dies setzt voraus, dass es einen Sinn habe, beide Eigenschaften von beiden Formen auszusagen.}
\ogd{One cannot distinguish forms from one another by saying that one has this property, the other that: for this assumes that there is a sense in asserting either property of either form.}
\pmc{It is impossible to distinguish forms from one another by saying that one has this property and another that property: for this presupposes that it makes sense to ascribe either property to either form.}

\pn{4.125}
\ger{Das Bestehen einer internen Relation zwischen m{\"o}glichen Sachlagen dr{\"u}ckt sich sprachlich durch eine interne Relation zwischen den sie darstellenden S{\"a}tzen aus.}
\ogd{The existence of an internal relation between possible states of affairs expresses itself in language by an internal relation between the propositions presenting them.}
\pmc{The existence of an internal relation between possible situations expresses itself in language by means of an internal relation between the propositions representing them.}

\pn{4.1251}
\ger{Hier erledigt sich nun die Streitfrage, \gdql ob alle Relationen intern oder extern seien\gdqr{}.}
\ogd{Now this settles the disputed question ``whether all relations are internal or external''.}
\pmc{Here we have the answer to the vexed question `whether all relations are internal or external'.}

\pn{4.1252}
\ger{Reihen, welche durch \germph{interne} Relationen geordnet sind, nenne ich Formenreihen.}
\ogd{Series which are ordered by \emph{internal} relations I call formal series.}
\pmc{I call a series that is ordered by an internal relation a series of forms.}

\pnskip
\ger{Die Zahlenreihe ist nicht nach einer externen, sondern nach einer internen Relation geordnet.}
\ogd{The series of numbers is ordered not by an external, but by an internal relation.}
\pmc{The order of the number-series is not governed by an external relation but by an internal relation.}

\pnskip
\ger{Ebenso die Reihe der S{\"a}tze \gdql $aRb$\gdqr{},}
\ogd{Similarly the series of propositions ``$aRb$'',}
\pmc{The same is true of the series of propositions `$aRb$',}

\pnskip
\ger{\gdql $\rsomedd{x} aRx \rand xRb$\gdqr{},}
\ogd{``$\rsomedd{x} aRx \rand xRb$'',}
\pmc{`$\rsomedd{x} aRx \rand xRb$',}

\pnskip
\ger{\gdql $\rsomedd{x,y} aRx\rand xRy \rand yRb$\gdqr{}, u.\ s.\ f.}
\ogd{``$\rsomedd{x,y} aRx\rand xRy \rand yRb$'', etc.}
\pmc{`$\rsomedd{x,y} aRx\rand xRy \rand yRb$', and so forth.}

\pnskip
\ger{(Steht $b$ in einer dieser Beziehungen zu $a$, so nenne ich $b$ einen Nachfolder von $a$.)}
\ogd{(If $b$ stands in one of these relations to $a$, I call $b$ a successor of $a$.)}
\pmc{(If $b$ stands in one of these relations to $a$, I call $b$ a successor of $a$.)}

\pn{4.126}
\ger{In dem Sinne, in welchem wir von formalen Eigenschaften sprechen, k{\"o}nnen wir nun auch von formalen Begriffen reden.}
\ogd{In the sense in which we speak of formal properties we can now speak also of formal concepts.}
\pmc{We can now talk about formal concepts, in the same sense that we speak of formal properties.}

\pnskip
\ger{(Ich f{\"u}hre diesen Ausdruck ein, um den Grund der Verwechslung der formalen Begriffe mit den eigentlichen Begriffen, welche die ganze alte Logik durchzieht, klar zu machen.)}
\ogd{(I introduce this expression in order to make clear the confusion of formal concepts with proper concepts which runs through the whole of the old logic.)}
\pmc{(I introduce this expression in order to exhibit the source of the confusion between formal concepts and concepts proper, which pervades the whole of traditional logic.)}

\pnskip
\ger{Dass etwas unter einen formalen Begriff als dessen Gegenstand f{\"a}llt, kann nicht durch einen Satz ausgedr{\"u}ckt werden. Sondern es zeigt sich an dem Zeichen dieses Gegenstandes selbst. (Der Name zeigt, dass er einen Gegenstand bezeichnet, das Zahlenzeichen, dass es eine Zahl bezeichnet etc.)}
\ogd{That anything falls under a formal concept as an object belonging to it, cannot be expressed by a proposition. But it is shown in the symbol for the object itself. (The name shows that it signifies an object, the numerical sign that it signifies a number, etc.)}
\pmc{When something falls under a formal concept as one of its objects, this cannot be expressed by means of a proposition. Instead it is shown in the very sign for this object. (A name shows that it signifies an object, a sign for a number that it signifies a number, etc.)}

\pnskip
\ger{Die formalen Begriffe k{\"o}nnen ja nicht, wie die eigentlichen Begriffe, durch eine Funktion dargestellt werden.}
\ogd{Formal concepts, cannot, like proper concepts, be presented by a function.}
\pmc{Formal concepts cannot, in fact, be represented by means of a function, as concepts proper can.}

\pnskip
\ger{Denn ihre Merkmale, die formalen Eigenschaften, werden nicht durch Funktionen ausgedr{\"u}ckt.}
\ogd{For their characteristics, the formal properties, are not expressed by the functions.}
\pmc{For their characteristics, formal properties, are not expressed by means of functions.}

\pnskip
\ger{Der Ausdruck der formalen Eigenschaft ist ein Zug gewisser Symbole.}
\ogd{The expression of a formal property is a feature of certain symbols.}
\pmc{The expression for a formal property is a feature of certain symbols.}

\pnskip
\ger{Das Zeichen der Merkmale eines formalen Begriffes ist also ein charakteristischer Zug aller Symbole, deren Bedeutungen unter den Begriff fallen.}
\ogd{The sign that signifies the characteristics of a formal concept is, therefore, a characteristic feature of all symbols, whose meanings fall under the concept.}
\pmc{So the sign for the characteristics of a formal concept is a distinctive feature of all symbols whose meanings fall under the concept.}

\pnskip
\ger{Der Ausdruck des formalen Begriffes, also, eine Satzvariable, in welcher nur dieser charakteristische Zug konstant ist.}
\ogd{The expression of the formal concept is therefore a propositional variable in which only this characteristic feature is constant.}
\pmc{So the expression for a formal concept is a propositional variable in which this distinctive feature alone is constant.}

\pn{4.127}
\ger{Die Satzvariable bezeichnet den formalen Begriff und ihre Werte die Gegenst{\"a}nde, welche unter diesen Begriff fallen.}
\ogd{The propositional variable signifies the formal concept, and its values signify the objects which fall under this concept.}
\pmc{The propositional variable signifies the formal concept, and its values signify the objects that fall under the concept.}

\pn{4.1271}
\ger{Jede Variable ist das Zeichen eines formalen Begriffes.}
\ogd{Every variable is the sign of a formal concept.}
\pmc{Every variable is the sign for a formal concept.}

\pnskip
\ger{Denn jede Variable stellt eine konstante Form dar, welche alle ihre Werte besitzen, und die als formale Eigenschaft dieser Werte aufgefasst werden kann.}
\ogd{For every variable presents a constant form, which all its values possess, and which can be conceived as a formal property of these values.}
\pmc{For every variable represents a constant form that all its values possess, and this can be regarded as a formal property of those values.}

\pn{4.1272}
\ger{So ist der variable Name \gdql $x$\gdqr{} das eigentliche Zeichen des Scheinbegriffes \germph{Gegenstand}.}
\ogd{So the variable name ``$x$'' is the proper sign of the pseudo-concept \emph{object}.}
\pmc{Thus the variable name `$x$' is the proper sign for the pseudo-concept \emph{object}.}

\pnskip
\ger{Wo immer das Wort \gdql Gegenstand\gdqr{} (\gdql Ding\gdqr{}, \gdql Sache\gdqr{}, etc.) richtig gebraucht wird, wird es in der Begriffsschrift durch den variablen Namen ausgedr{\"u}ckt.}
\ogd{Wherever the word ``object'' (``thing'', ``entity'', etc.) is rightly used, it is expressed in logical symbolism by the variable name.}
\pmc{Wherever the word `object' (`thing', etc.) is correctly used, it is expressed in conceptual notation by a variable name.}

\pnskip
\ger{Zum Beispiel in dem Satz \gdql es gibt 2 Gegenst{\"a}nde, welche \ldots\gdqr{} durch \gdql $\rsome{x,y} \ldots$\gdqr{}.}
\ogd{For example in the proposition ``there are two objects which \ldots'', by ``$\rsome{x,y} \ldots$''.}
\pmc{For example, in the proposition, `There are 2 objects which \ldots', it is expressed by `$\rsome{x,y} \ldots$'.}

\pnskip
\ger{Wo immer es anders, also als eigentliches Begriffswort gebraucht wird, entstehen unsinnige Scheins{\"a}tze.}
\ogd{Wherever it is used otherwise, \emph{i.e.}\ as a proper concept word, there arise senseless pseudo-propositions.}
\pmc{Wherever it is used in a different way, that is as a proper concept-word, nonsensical pseudo-propositions are the result.}

\pnskip
\ger{So kann man z.\ B.\ nicht sagen \gdql Es gibt Gegenst{\"a}nde\gdqr{}, wie man etwa sagt: \gdql Es gibt B{\"u}cher\gdqr{}. Und ebenso wenig: \gdql Es gibt 100 Gegenst{\"a}nde\gdqr{}, oder \gdql Es gibt $\aleph_0$ Gegenst{\"a}nde\gdqr{}.}
\ogd{So one cannot, \emph{e.g.}\ say ``There are objects'' as one says ``There are books''. Nor ``There are 100 objects'' or ``There are $\aleph_0$ objects''.}
\pmc{So one cannot say, for example, `There are objects', as one might say, `There are books'. And it is just as impossible to say, `There are 100 objects', or, `There are $\aleph_0$ objects'.}

\pnskip
\ger{Und es ist unsinnig, von der \germph{Anzahl aller Gegenst{\"a}nde} zu sprechen.}
\ogd{And it is senseless to speak of the \emph{number of all objects}.}
\pmc{And it is nonsensical to speak of the \emph{total number of objects}.}

\pnskip
\ger{Dasselbe gilt von den Worten \gdql Komplex\gdqr{}, \gdql Tatsache\gdqr{}, \gdql Funktion\gdqr{}, \gdql Zahl\gdqr{}, etc.}
\ogd{The same holds of the words ``Complex'', ``Fact'', ``Function'', ``Number'', etc.}
\pmc{The same applies to the words `complex', `fact', `function', `number', etc.}

\pnskip
\ger{Sie alle bezeichnen formale Begriffe und werden in der Begriffsschrift durch Variable, nicht durch Funktionen oder Klassen dargestellt. (Wie Frege und Russell glaubten.)}
\ogd{They all signify formal concepts and are presented in logical symbolism by variables, not by functions or classes (as Frege and Russell thought).}
\pmc{They all signify formal concepts, and are represented in conceptual notation by variables, not by functions or classes (as Frege and Russell believed).}

\pnskip
\ger{Ausdr{\"u}cke wie \gdql 1 ist eine Zahl\gdqr{}, \gdql Es gibt nur Eine Null\gdqr{} und alle {\"a}hnlichen sind unsinnig.}
\ogd{Expressions like ``1 is a number'', ``there is only one number nought'', and all like them are senseless.}
\pmc{`1 is a number', `There is only one zero', and all similar expressions are nonsensical.}

\pnskip
\ger{(Es ist ebenso unsinnig zu sagen: \gdql Es gibt nur Eine 1\gdqr{}, als es unsinnig w{\"a}re, zu sagen: \gdql $2+2$ ist um 3 Uhr gleich 4\gdqr{}.)}
\ogd{(It is as senseless to say, ``there is only one 1'' as it would be to say: $2+2$ is at 3 o'clock equal to 4.)}
\pmc{(It is just as nonsensical to say, `There is only one 1', as it would be to say, `$2+2$ at 3 o'clock equals 4'.)}

\pn{4.12721}
\ger{Der formale Begriff ist mit einem Gegenstand, der unter ihn f{\"a}llt, bereits gegeben. Man kann also nicht Gegenst{\"a}nde eines formalen Begriffes \germph{und} den formalen Begriff selbst als Grundbegriffe einf{\"u}hren. Man kann also z.\ B.\ nicht den Begriff der Funktion, und auch spezielle Funktionen (wie Russell) als Grundbegriffe einf{\"u}hren; oder den Begriff der Zahl und bestimmte Zahlen.}
\ogd{The formal concept is already given with an object, which falls under it. One cannot, therefore, introduce both, the objects which fall under a formal concept \emph{and} the formal concept itself, as primitive ideas. One cannot, therefore, \emph{e.g.}\ introduce (as Russell does) the concept of function and also special functions as primitive ideas; or the concept of number and definite numbers.}
\pmc{A formal concept is given immediately any object falling under it is given. It is not possible, therefore, to introduce as primitive ideas objects belonging to a formal concept \emph{and} the formal concept itself. So it is impossible, for example, to introduce as primitive ideas both the concept of a function and specific functions, as Russell does; or the concept of a number and particular numbers.}

\pn{4.1273}
\ger{Wollen wir den allgemeinen Satz: \gdql $b$ ist ein Nachfolger von $a$\gdqr{} in der Begriffsschrift ausdr{\"u}cken, so brauchen wir hierzu einen Ausdruck f{\"u}r das allgemeine Glied der Formenreihe: }
\ogd{If we want to express in logical symbolism the general proposition ``$b$ is a successor of $a$'' we need for this an expression for the general term of the formal series: }
\pmc{If we want to express in conceptual notation the general proposition, `$b$ is a successor of $a$', then we require an expression for the general term of the series of forms}

\pnskip
\ger{\begin{center}$aRb$, \\ $\rsomedd{x} aRx \rand xRb$,\\ $\rsomedd{x,y} aRx \rand xRy \rand yRb$,\\ \ldots~ .\end{center} }
\ogd{\begin{center}$aRb$,\\ $\rsomedd{x} aRx \rand xRb$,\\ $\rsomedd{x,y} aRx \rand xRy \rand yRb$,\\ \ldots~ .\end{center}}
\pmc{\begin{center}$aRb$,\\ $\rsomedd{x} aRx \rand xRb$,\\ $\rsomedd{x,y} aRx \rand xRy \rand yRb$,\\ \ldots~ .\end{center}}

\pnskip
\ger{\negpbk Das allgemeine Glied einer Formenreihe kann man nur durch eine Variable ausdr{\"u}cken, denn der Begriff: Glied dieser Formenreihe, ist ein \germph{formaler} Begriff. (Dies haben Frege und Russell {\"u}bersehen; die Art und Weise, wie sie allgemeine S{\"a}tze wie den obigen ausdr{\"u}cken wollen, ist daher falsch; sie enth{\"a}lt einen circulus vitiosus.)}
\ogd{\negpbk The general term of a formal series can only be expressed by a variable, for the concept symbolized by ``term of this formal series'' is a \emph{formal} concept. (This Frege and Russell overlooked; the way in which they express general propositions like the above is, therefore, false; it contains a vicious circle.)}
\pmc{\negpbk In order to express the general term of a series of forms, we must use a variable, because the concept `term of that series of forms' is a \emph{formal} concept. (This is what Frege and Russell overlooked: consequently the way in which they want to express general propositions like the one above is incorrect; it contains a vicious circle.)}

\pnskip
\ger{Wir k{\"o}nnen das allgemeine Glied der Formenreihe bestimmen, indem wir ihr erstes Glied angeben und die allgemeine Form der Operation, welche das folgende Glied aus dem vorhergehenden Satz erzeugt.}
\ogd{We can determine the general term of the formal series by giving its first term and the general form of the operation, which generates the following term out of the preceding proposition.}
\pmc{We can determine the general term of a series of forms by giving its first term and the general form of the operation that produces the next term out of the proposition that precedes it.}

\pn{4.1274}
\ger{Die Frage nach der Existenz eines formalen Begriffes ist unsinnig. Denn kein Satz kann eine solche Frage beantworten.}
\ogd{The question about the existence of a formal concept is senseless. For no proposition can answer such a question.}
\pmc{To ask whether a formal concept exists is nonsensical. For no proposition can be the answer to such a question.}

\pnskip
\ger{(Man kann also z.\ B.\ nicht fragen: \gdql Gibt es unanalysierbare Subjekt-Pr{\"a}dikats{\"a}tze?\gdqr{})}
\ogd{(For example, one cannot ask: ``Are there unanalysable subject-predicate propositions?'')}
\pmc{(So, for example, the question, `Are there unanalysable subject-predicate propositions?' cannot be asked.)}

\pn{4.128}
\ger{Die logischen Formen sind zah\germph{llos}.}
\ogd{The logical forms are \emph{anumerical}.}
\pmc{Logical forms are \emph{without} number.}

\pnskip
\ger{Darum gibt es in der Logik keine ausgezeichneten Zahlen und darum gibt es keinen philosophischen Monismus oder Dualismus, etc.}
\ogd{Therefore there are in logic no pre-eminent numbers, and therefore there is no philosophical monism or dualism, etc.}
\pmc{Hence there are no pre-eminent numbers in logic, and hence there is no possibility of philosophical monism or dualism, etc.}

\pn{4.2}
\ger{Der Sinn des Satzes ist seine {\"U}bereinstimmung und Nicht{\"u}bereinstimmung mit den M{\"o}glichkeiten des Bestehens und Nichtbestehens der Sachverhalte.}
\ogd{The sense of a proposition is its agreement and disagreement with the possibilities of the existence and non-existence of the atomic facts.}
\pmc{The sense of a proposition is its agreement and disagreement with possibilities of existence and non-existence of states of affairs.}

\pn{4.21}
\ger{Der einfachste Satz, der Elementarsatz, behauptet das Bestehen eines Sachverhaltes.}
\ogd{The simplest proposition, the elementary proposition, asserts the existence of an atomic fact.}
\pmc{The simplest kind of proposition, an elementary proposition, asserts the existence of a state of affairs.}

\pn{4.211}
\ger{Ein Zeichen des Elementarsatzes ist es, dass kein Elementarsatz mit ihm in Widerspruch stehen kann.}
\ogd{It is a sign of an elementary proposition, that no elementary proposition can contradict it.}
\pmc{It is a sign of a proposition's being elementary that there can be no elementary proposition contradicting it.}

\pn{4.22}
\ger{Der Elementarsatz besteht aus Namen. Er ist ein Zusammenhang, eine Verkettung, von Namen.}
\ogd{The elementary proposition consists of names. It is a connexion, a concatenation, of names.}
\pmc{An elementary proposition consists of names. It is a nexus, a concatenation, of names.}

\pn{4.221}
\ger{Es ist offenbar, dass wir bei der Analyse der S{\"a}tze auf Elementars{\"a}tze kommen m{\"u}ssen, die aus Namen in unmittelbarer Verbindung bestehen.}
\ogd{It is obvious that in the analysis of propositions we must come to elementary propositions, which consist of names in immediate combination.}
\pmc{It is obvious that the analysis of propositions must bring us to elementary propositions which consist of names in immediate combination.}

\pnskip
\ger{Es fr{\"a}gt sich hier, wie kommt der Satzverband zustande.}
\ogd{The question arises here, how the propositional connexion comes to be.}
\pmc{This raises the question how such combination into propositions comes about.}

\pn{4.2211}
\ger{Auch wenn die Welt unendlich komplex ist, so dass jede Tatsache aus unendlich vielen Sachverhalten besteht und jeder Sachverhalt aus unendlich vielen Gegenst{\"a}nden zusammengesetzt ist, auch dann m{\"u}sste es Gegenst{\"a}nde und Sachverhalte geben.}
\ogd{Even if the world is infinitely complex, so that every fact consists of an infinite number of atomic facts and every atomic fact is composed of an infinite number of objects, even then there must be objects and atomic facts.}
\pmc{Even if the world is infinitely complex, so that every fact consists of infinitely many states of affairs and every state of affairs is composed of infinitely many objects, there would still have to be objects and states of affairs.}

\pn{4.23}
\ger{Der Name kommt im Satz nur im Zusammenhange des Elementarsatzes vor.}
\ogd{The name occurs in the proposition only in the context of the elementary proposition.}
\pmc{It is only in the nexus of an elementary proposition that a name occurs in a proposition.}

\pn{4.24}
\ger{Die Namen sind die einfachen Symbole, ich deute sie durch einzelne Buchstaben (\gdql $x$\gdqr{}, \gdql $y$\gdqr{}, \gdql $z$\gdqr{}) an.}
\ogd{The names are the simple symbols, I indicate them by single letters ($x$, $y$, $z$).}
\pmc{Names are the simple symbols: I indicate them by single letters (`$x$', `$y$', `$z$').}

\pnskip
\ger{Den Elementarsatz schreibe ich als Funktion der Namen in der Form: \gdql $f\negthinspace x$\gdqr{}, \gdql $\phi (x,y)$\gdqr{}, etc.}
\ogd{The elementary proposition I write as function of the names, in the form ``$f\negthinspace x$'', ``$\phi (x,y)$'', etc.}
\pmc{I write elementary propositions as functions of names, so that they have the form `$f\negthinspace x$', `$\phi (x,y)$', etc.}

\pnskip
\ger{Oder ich deute ihn durch die Buchstaben $p$, $q$, $r$ an.}
\ogd{Or I indicate it by the letters $p$, $q$, $r$.}
\pmc{Or I indicate them by the letters `$p$', `$q$', `$r$'.}

\pn{4.241}
\ger{Gebrauche ich zwei Zeichen in ein und derselben Bedeutung, so dr{\"u}cke ich dies aus, indem ich zwischen beide das Zeichen \gdql =\gdqr{} setze.}
\ogd{If I use two signs with one and the same meaning, I express this by putting between them the sign ``=''.}
\pmc{When I use two signs with one and the same meaning, I express this by putting the sign `=' between them.}

\pnskip
\ger{\gdql $a=b$\gdqr{} hei{\ss}t also: das Zeichen \gdql $a$\gdqr{} ist durch das Zeichen \gdql $b$\gdqr{} ersetzbar.}
\ogd{``$a=b$'' means then, that the sign ``$a$'' is replaceable by the sign ``$b$''.}
\pmc{So `$a = b$' means that the sign `$b$' can be substituted for the sign `$a$'.}

\pnskip
\ger{(F{\"u}hre ich durch eine Gleichung ein neues Zeichen \gdql $b$\gdqr{} ein, indem ich bestimme, es solle ein bereits bekanntes Zeichen \gdql $a$\gdqr{} ersetzen, so schreibe ich die Gleichung---Definition---(wie Russell) in der Form \gdql $a=b$ Def.\gdqr{}. Die Definition ist eine Zeichenregel.)}
\ogd{(If I introduce by an equation a new sign ``$b$'', by determining that it shall replace a previously known sign ``$a$'', I write the equation---definition---(like Russell) in the form ``$a = b$ Def.''. A definition is a symbolic rule.)}
\pmc{(If I use an equation to introduce a new sign `$b$', laying down that it shall serve as a substitute for a sign `$a$' that is already known, then, like Russell, I write the equation---definition---in the form `$a = b$ Def.' A definition is a rule dealing with signs.)}

\pn{4.242}
\ger{Ausdr{\"u}cke von der Form \gdql $a = b$\gdqr{} sind also nur Behelfe der Darstellung; sie sagen nichts {\"u}ber die Bedeutung der Zeichen \gdql $a$\gdqr{}, \gdql $b$\gdqr{} aus.}
\ogd{Expressions of the form ``$a = b$'' are therefore only expedients in presentation: They assert nothing about the meaning of the signs ``$a$'' and ``$b$''.}
\pmc{Expressions of the form `$a = b$' are, therefore, mere representational devices. They state nothing about the meaning of the signs `$a$' and `$b$'.}

\pn{4.243}
\ger{K{\"o}nnen wir zwei Namen verstehen, ohne zu wissen, ob sie dasselbe Ding oder zwei verschiedene Dinge bezeichnen?---K{\"o}nnen wir einen Satz, worin zwei Namen vorkommen, verstehen, ohne zu wissen, ob sie Dasselbe oder Verschiedenes bedeuten?}
\ogd{Can we understand two names without knowing whether they signify the same thing or two different things? Can we understand a proposition in which two names occur, without knowing if they mean the same or different things?}
\pmc{Can we understand two names without knowing whether they signify the same thing or two different things?---Can we understand a proposition in which two names occur without knowing whether their meaning is the same or different?}

\pnskip
\ger{Kenne ich etwa die Bedeutung eines englischen und eines gleichbedeutenden deutschen Wortes, so ist es unm{\"o}glich, dass ich nicht wei{\ss}, dass die beiden gleichbedeutend sind; es ist unm{\"o}glich, dass ich sie nicht ineinander {\"u}bersetzen kann.}
\ogd{If I know the meaning of an English and a synonymous German word, it is impossible for me not to know that they are synonymous, it is impossible for me not to be able to translate them into one another.}
\pmc{Suppose I know the meaning of an English word and of a German word that means the same: then it is impossible for me to be unaware that they do mean the same; I must be capable of translating each into the other.}

\pnskip
\ger{Ausdr{\"u}cke wie \gdql $a = a$\gdqr{}, oder von diesen abgeleitete, sind weder Elementars{\"a}tze, noch sonst sinnvolle Zeichen. (Dies wird sich sp{\"a}ter zeigen.)}
\ogd{Expressions like ``$a = a$'', or expressions deduced from these are neither elementary propositions nor otherwise significant signs. (This will be shown later.)}
\pmc{Expressions like `$a = a$', and those derived from them, are neither elementary propositions nor is there any other way in which they have sense. (This will become evident later.)}

\pn{4.25}
\ger{Ist der Elementarsatz wahr, so besteht der Sachverhalt; ist der Elementarsatz falsch, so besteht der Sachverhalt nicht.}
\ogd{If the elementary proposition is true, the atomic fact exists; if it is false the atomic fact does not exist.}
\pmc{If an elementary proposition is true, the state of affairs exists: if an elementary proposition is false, the state of affairs does not exist.}

\pn{4.26}
\ger{Die Angabe aller wahren Elementars{\"a}tze beschreibt die Welt vollst{\"a}ndig. Die Welt ist vollst{\"a}ndig beschrieben durch die Angaben aller Elementars{\"a}tze plus der Angabe, welche von ihnen wahr und welche falsch sind.}
\ogd{The specification of all true elementary propositions describes the world completely. The world is completely described by the specification of all elementary propositions plus the specification, which of them are true and which false.}
\pmc{If all true elementary propositions are given, the result is a complete description of the world. The world is completely described by giving all elementary propositions, and adding which of them are true and which false.}

\pn{4.27}
\ger{Bez{\"u}glich des Bestehens und Nicht\-be\-steh\-ens von $n$ Sachverhalten gibt es \possibilities\ M{\"o}glichkeiten.}
\ogd{With regard to the existence of $n$ atomic facts there are \possibilities\  possibilities.}
\pmc{For $n$ states of affairs, there are \possibilities\ possibilities of existence and non-existence.}%???

\pnskip
\ger{Es k{\"o}nnen alle Kombinationen der Sachverhalte bestehen, die andern nicht bestehen.}
\ogd{It is possible for all combinations of atomic facts to exist, and the others not to exist.}
\pmc{Of these states of affairs any combination can exist and the remainder not exist.}

\pn{4.28}
\ger{Diesen Kombinationen ent\-sprech\-en ebenso viele M{\"o}g\-lich\-keit\-en der Wahr\-heit---und Falsch\-heit---von $n$ El\-em\-en\-tar\-s{\"a}tz\-en.}
\ogd{To these combinations correspond the same number of possibilities of the truth---and falsehood---of $n$ elementary propositions.}
\pmc{There correspond to these combinations the same number of possibilities of truth---and falsity---for $n$ elementary propositions.}

\pn{4.3}
\ger{Die Wahrheitsm{\"o}glichkeiten der Elementars{\"a}tze bedeuten die M{\"o}glichkeiten des Bestehens und Nichtbestehens der Sachverhalte.}
\ogd{The truth-possibilities of the elementary propositions mean the possibilities of the existence and non-existence of the atomic facts.}
\pmc{Truth-possibilities of el\-em\-en\-tary prop\-o\-sit\-ions mean pos\-si\-bil\-i\-ties of ex\-is\-tence and non-ex\-is\-tence of sta\-tes of af\-fairs.}

\pn{4.31}
\ger{Die Wahrheitsm{\"o}glichkeiten k{\"o}nnen wir durch Schemata folgender Art darstellen (\gdql W\gdqr{} bedeutet \gdql wahr\gdqr{}, \gdql F\gdqr{}, \gdql falsch\gdqr{}. Die Reihen der \gdql W\gdqr{} und \gdql F\gdqr{} unter der Reihe der Elementars{\"a}tze bedeuten in leichtverst{\"a}ndlicher Symbolik deren Wahrheitsm{\"o}glichkeiten):}
\ogd{The truth-possibilities can be presented by schemata of the following kind (``T'' means ``true'', ``F'' ``false''. The rows of T's and F's under the row of the elementary propositions mean their truth-possibilities in an easily intelligible symbolism).}
\pmc{We can represent truth-possibilities by schemata of the following kind (`T' means `true', `F' means `false'; the rows of `T's' and `F's' under the row of elementary propositions symbolize their truth-possibilities in a way that can easily be understood):}

\pnskip
\ger{\fourthreeonetablegerman}
\ogd{\fourthreeonetableenglish}
\pmc{\fourthreeonetableenglish}

\pn{4.4}
\ger{Der Satz ist der Ausdruck der {\"U}bereinstimmung und Nicht{\"u}bereinstimmung mit den Wahrheitsm{\"o}glichkeiten der Elementars{\"a}tze.}
\ogd{A proposition is the expression of agreement and disagreement with the truth-possibilities of the elementary propositions.}
\pmc{A proposition is an expression of agreement and disagreement with truth-possibilities of elementary propositions.}

\pn{4.41}
\ger{Die Wahrheitsm{\"o}glichkeiten der Elementars{\"a}tze sind die Bedingungen der Wahrheit und Falschheit der S{\"a}tze.}
\ogd{The truth-possibilities of the elementary propositions are the conditions of the truth and falsehood of the propositions.}
\pmc{Truth-possibilities of el\-em\-en\-tar\-y prop\-o\-si\-tions are the con\-dit\-ions of the truth and fal\-sity of prop\-o\-sit\-ions.}

\pn{4.411}
\ger{Es ist von vornherein wahrscheinlich, dass die Einf{\"u}hrung der Elementars{\"a}tze f{\"u}r das Verst{\"a}ndnis aller anderen Satzarten grundlegend ist. Ja, das Verst{\"a}ndnis der allgemeinen S{\"a}tze h{\"a}ngt \germph{f{\"u}hlbar} von dem der Elementars{\"a}tze ab.}
\ogd{It seems probable even at first sight that the introduction of the elementary propositions is fundamental for the comprehension of the other kinds of propositions. Indeed the comprehension of the general propositions depends \emph{palpably} on that of the elementary propositions.}
\pmc{It immediately strikes one as probable that the introduction of elementary propositions provides the basis for understanding all other kinds of proposition. Indeed the understanding of general propositions \emph{palpably} depends on the understanding of elementary propositions.}

\pn{4.42}
\ger{Bez{\"u}glich der {\"U}bereinstimmung und Nicht{\"u}berein stimmung eines Satzes mit den Wahrheitsm{\"o}glichkeiten von $n$ Elementars{\"a}tzen gibt es \morepossibilities\ M{\"o}glichkeiten.}
\ogd{With regard to the agreement and disagreement of a proposition with the truth-possibilities of $n$ elementary propositions there are \morepossibilities\  possibilities.}
\pmc{For $n$ elementary propositions there are \morepossibilities\  ways in which a proposition can agree and disagree with their truth possibilities.}%???

\pn{4.43}
\ger{Die {\"U}bereinstimmung mit den Wahr\-heits\-m{\"o}g\-lich\-kei\-ten k{\"o}nnen wir dadurch ausdr{\"u}cken, indem wir ihnen im Schema etwa das Abzeichen \gdql W\gdqr{} (wahr) zuordnen.}
\ogd{Agreement with the truth-pos\-si\-bil\-i\-ties can be expressed by co-ordinating with them in the schema the mark ``T'' (true).}
\pmc{We can express agreement with truth-possibilities by correlating the mark `T' (true) with them in the schema.}

\pnskip
\ger{Das Fehlen dieses Abzeichens bedeutet die Nicht{\"u}bereinstimmung.}
\ogd{Absence of this mark means disagreement.}
\pmc{The absence of this mark means disagreement.}

\pn{4.431}
\ger{Der Ausdruck der {\"U}bereinstimmung und Nicht{\"u}bereinstimmung mit den Wahrheitsm{\"o}glichkeiten der Elementars{\"a}tze dr{\"u}ckt die Wahrheitsbedingungen des Satzes aus.}
\ogd{The expression of the agreement and disagreement with the truth-possibilities of the elementary propositions expresses the truth-conditions of the proposition.}
\pmc{The expression of agreement and disagreement with the truth possibilities of elementary propositions expresses the truth-conditions of a proposition.}

\pnskip
\ger{Der Satz ist der Ausdruck seiner Wahrheitsbedingungen.}
\ogd{The proposition is the expression of its truth-conditions.}
\pmc{A proposition is the expression of its truth-conditions.}

\pnskip
\ger{(Frege hat sie daher ganz richtig als Erkl{\"a}rung der Zeichen seiner Begriffsschrift vorausgeschickt. Nur ist die Erkl{\"a}rung des Wahrheitsbegriffes bei Frege falsch: W{\"a}ren \gdql das Wahre\gdqr{} und \gdql das Falsche\gdqr{} wirklich Gegenst{\"a}nde und die Argumente in $\rnot p$ etc.\ dann w{\"a}re nach Freges Bestimmung der Sinn von \gdql $\rnot p$\gdqr{} keineswegs bestimmt.)}
\ogd{(Frege has therefore quite rightly put them at the beginning, as explaining the signs of his logical symbolism. Only Frege's explanation of the truth-concept is false: if ``the true'' and ``the false'' were real objects and the arguments in $\rnot p$, etc., then the sense of $\rnot p$ would by no means be determined by Frege's determination.)}
\pmc{(Thus Frege was quite right to use them as a starting point when he explained the signs of his conceptual notation. But the explanation of the concept of truth that Frege gives is mistaken: if `the true' and `the false' were really objects, and were the arguments in $\rnot p$ etc., then Frege's method of determining the sense of `$\rnot p$' would leave it absolutely undetermined.)}

\pn{4.44}
\ger{Das Zeichen, welches durch die Zuordnung jener Abzeichen \gdql W\gdqr{} und der Wahrheitsm{\"o}glichkeiten entsteht, ist ein Satzzeichen.}
\ogd{The sign which arises from the co-ordination of that mark ``T'' with the truth-possibilities is a propositional sign.}
\pmc{The sign that results from correlating the mark `T' with truth-possibilities is a propositional sign.}%???

\pn{4.441}
\ger{Es ist klar, dass dem Komplex der Zeichen \gdql F\gdqr{} und \gdql W\gdqr{} kein Gegenstand (oder Komplex von Gegenst{\"a}nden) entspricht; so wenig, wie den horizontalen und vertikalen Strichen oder den Klammern.---\gdql Logische Gegenst{\"a}nde\gdqr{} gibt es nicht.}
\ogd{It is clear that to the complex of the signs ``F'' and ``T'' no object (or complex of objects) corresponds; any more than to horizontal and vertical lines or to brackets. There are no ``logical objects''.}
\pmc{It is clear that a complex of the signs `F' and `T' has no object (or complex of objects) corresponding to it, just as there is none corresponding to the horizontal and vertical lines or to the brackets.---There are no `logical objects'.}

\pnskip
\ger{Analoges gilt nat{\"u}rlich f{\"u}r alle Zeichen, die dasselbe ausdr{\"u}cken wie die Schemata der \gdql W\gdqr{} und \gdql F\gdqr{}.}
\ogd{Something analogous holds of course for all signs, which express the same as the schemata of ``T'' and ``F''.}
\pmc{Of course the same applies to all signs that express what the schemata of `T's' and `F's' express.}

\pn{4.442}
\ger{Es ist z.\ B.:}
\ogd{Thus \emph{e.g.}}
\pmc{For example, the following is a propositional sign:}

\pnskip
\ger{\negpbk\fourfourfourtwotablegerman%\linebreak
\flushright ein Satzzeichen.%
}
\ogd{\negpbk\fourfourfourtwotableogden%\linebreak
\flushright is a propositional sign.%
}
\pmc{\negpbk\fourfourfourtwotablepmc}

\pnskip
\ger{(Frege's \gdql Urtelistrich\gdqr{} \gdql$\vdash$\gdqr{} ist logisch ganz bedeutunglos; er zeigt bei Frege (und Russell) nur an, dass diese Autoren die so bezeichneten S{\"a}tze f{\"u}r wahr halten. 
\gdql$\vdash$\gdqr{} geh{\"o}rt daher ebenso wenig zum Satzgef{\"u}ge, wie etwa die Nummer des Satzes. Ein Satz kann unm{\"o}glich von sich selbst aussagen, dass er wahr ist.)}
\ogd{(Frege's assertion sign ``$\vdash$'' is logically altogether meaningless; in Frege (and Russell) it only shows that these authors hold as true the propositions marked in this way. ``$\vdash$'' belongs therefore to the propositions no more than does the number of the proposition. A proposition cannot possibly assert of itself that it is true.}
\pmc{(Frege's `judgement stroke' `$\vdash$' is logically quite meaningless: in the works of Frege (and Russell) it simply indicates that these authors hold the propositions marked with this sign to be true. Thus `$\vdash$' is no more a component part of a proposition than is, for instance, the proposition's number. It is quite impossible for a proposition to state that it itself is true.)}

\pnskip
\ger{Ist die Rei\-hen\-fol\-ge der Wahr\-heits\-m{\"o}g\-lich\-kei\-ten im Schema durch eine Kombinationsregel ein f{\"u}r allemal festgesetzt, dann ist die letzte Kolonne allein schon ein Ausdruck der Wahrheitsbedingungen. Schreiben wir diese Kolonne als Reihe hin, so wird das Satzzeichen zu}
\ogd{If the sequence of the truth-pos\-si\-bil\-i\-ties in the schema is once for all determined by a rule of combination, then the last column is by itself an expression of the truth-conditions. If we write this column as a row the propositional sign becomes: }
\pmc{If the order or the truth-possibilities in a schema is fixed once and for all by a combinatory rule, then the last column by itself will be an expression of the truth-conditions. If we now write this column as a row, the propositional sign will become}

\pnskip
\ger{\[ \text{\gdql} \mathop{(\mathrm{WW-W})}~ (p, q)\text{\gdqr} \] oder deutlicher \[ \text{\gdql} \mathop{(\mathrm{WWFW})}~ (p, q)\text{\gdqr}. \]}
\ogd{\[ ``\mathop{(\mathrm{TT-T})}~ (p, q)\text{''}, \] or more plainly: \[``\mathop{(\mathrm{TTFT})}~ (p, q)\text{''.}\]}
\pmc{\[ ``\mathop{(\mathrm{TT-T})}~ (p, q)\text{''}, \] or more explicitly \[``\mathop{(\mathrm{TTFT})}~ (p, q)\text{''.}\]}

\pnskip
\ger{(Die Anzahl der Stellen in der linken Klammer ist durch die Anzahl der Glieder in der rechten bestimmt.)}
\ogd{(The number of places in the left-hand bracket is determined by the number of terms in the right-hand bracket.)}
\pmc{(The number of places in the left-hand pair of brackets is determined by the number of terms in the right-hand pair.)}

\pn{4.45}
\ger{F{\"u}r $n$ Elementars{\"a}tze gibt es $\mathrm{L}_n$ m{\"o}gliche Gruppen von Wahrheitsbedingungen.}
\ogd{For $n$ \mbox{elementary} prop\-o\-si\-tions there are $\mathrm{L}_n$ possible groups of truth-con\-di\-tions.}
\pmc{For $n$ \mbox{elementary} prop\-o\-si\-tions there are $\mathrm{L}_n$ pos\-sible groups of truth-con\-di\-tions.}

\pnskip
\ger{Die Gruppen von Wahr\-heits\-be\-ding\-un\-gen, welche zu den Wahr\-heits\-m{\"o}g\-lich\-keit\-en einer Anzahl von El\-em\-en\-tar\-s{\"a}tz\-en geh{\"o}ren, lassen sich in eine Reihe ordnen.}
\ogd{The groups of truth-conditions which belong to the truth-possibilities of a number of elementary propositions can be ordered in a series.}
\pmc{The groups of truth-conditions that are obtainable from the truth-possibilities of a given number of elementary propositions can be arranged in a series.}

\pn{4.46}
\ger{Unter den m{\"o}glichen Gruppen von Wahrheitsbedingungen gibt es zwei extreme F{\"a}lle.}
\ogd{Among the possible groups of truth-conditions there are two extreme cases.}
\pmc{Among the possible groups of truth-conditions there are two extreme cases.}

\pnskip
\ger{In dem einen Fall ist der Satz f{\"u}r s{\"a}mtliche Wahrheitsm{\"o}glichkeiten der Elementars{\"a}tze wahr. Wir sagen, die Wahrheitsbedingungen sind \germph{tautologisch}.}
\ogd{In the one case the proposition is true for all the truth-possibilities of the elementary propositions. We say that the truth-conditions are \emph{tautological}.}
\pmc{In one of these cases the proposition is true for all the truth-possibilities of the elementary propositions. We say that the truth-conditions are \emph{tautological}.}

\pnskip
\ger{Im zweiten Fall ist der Satz f{\"u}r s{\"a}mtliche Wahrheitsm{\"o}glichkeiten falsch: Die Wahrheitsbedingungen sind \germph{kontradiktorisch}.}
\ogd{In the second case the proposition is false for all the truth-possibilities. The truth-conditions are \emph{self-contradictory}.}
\pmc{In the second case the proposition is false for all the truth-possibilities: the truth-conditions are \emph{contradictory}.}

\pnskip
\ger{Im ersten Fall nennen wir den Satz eine Tautologie, im zweiten Fall eine Kontradiktion.}
\ogd{In the first case we call the proposition a tautology, in the second case a contradiction.}
\pmc{In the first case we call the proposition a tautology; in the second, a contradiction.}

\pn{4.461}
\ger{Der Satz zeigt was er sagt, die Tautologie und die Kontradiktion, dass sie nichts sagen.}
\ogd{The proposition shows what it says, the tautology and the contradiction that they say nothing.}
\pmc{Propositions show what they say: tautologies and contradictions show that they say nothing.}

\pnskip
\ger{Die Tautologie hat keine Wahrheitsbedingungen, denn sie ist bedingungslos wahr; und die Kontradiktion ist unter keiner Bedingung wahr.}
\ogd{The tautology has no truth-conditions, for it is unconditionally true; and the contradiction is on no condition true.}
\pmc{A tautology has no truth-conditions, since it is unconditionally true: and a contradiction is true on no condition.}

\pnskip
\ger{Tautologie und Kontradiktion sind sinnlos.}
\ogd{Tautology and contradiction are without sense.}
\pmc{Tautologies and contradictions lack sense.}

\pnskip
\ger{(Wie der Punkt, von dem zwei Pfeile in entgegengesetzter Richtung auseinandergehen.)}
\ogd{(Like the point from which two arrows go out in opposite directions.)}
\pmc{(Like a point from which two arrows go out in opposite directions to one another.)}

\pnskip
\ger{(Ich wei{\ss} z.\ B.\ nichts {\"u}ber das Wetter, wenn ich wei{\ss}, dass es regnet oder nicht regnet.)}
\ogd{(I know, \emph{e.g.}\ nothing about the weather, when I know that it rains or does not rain.)}
\pmc{(For example, I know nothing about the weather when I know that it is either raining or not raining.)}

\pn{4.4611}
\ger{Tautologie und Kontradiktion sind aber nicht unsinnig; sie geh{\"o}ren zum Symbolismus, und zwar {\"a}hnlich wie die \gdql 0\gdqr{} zum Symbolismus der Arithmetik.}
\ogd{Tautology and contradiction are, however, not senseless; they are part of the symbolism, in the same way that ``0'' is part of the symbolism of Arithmetic.}
\pmc{Tautologies and contradictions are not, however, nonsensical. They are part of the symbolism, much as `0' is part of the symbolism of arithmetic.}

\pn{4.462}
\ger{Tautologie und Kontradiktion sind nicht Bilder der Wirklichkeit. Sie stellen keine m{\"o}gliche Sachlage dar. Denn jene l{\"a}sst \germph{jede} m{\"o}gliche Sachlage zu, diese \germph{keine}.}
\ogd{Tautology and contradiction are not pictures of the reality. They present no possible state of affairs. For the one allows \emph{every} possible state of affairs, the other \emph{none}.}
\pmc{Tautologies and contradictions are not pictures of reality. They do not represent any possible situations. For the former admit \emph{all} possible situations, and latter \emph{none}.}

\pnskip
\ger{In der Tautologie heben die Bedingungen der {\"U}bereinstimmung mit der Welt---die darstellenden Beziehungen---einander auf, so dass sie in keiner darstellenden Beziehung zur Wirklichkeit steht.}
\ogd{In the tautology the conditions of agreement with the world---the presenting relations---cancel one another, so that it stands in no presenting relation to reality.}
\pmc{In a tautology the conditions of agreement with the world---the representational relations---cancel one another, so that it does not stand in any representational relation to reality.}

\pn{4.463}
\ger{Die Wahrheitsbedingungen bestimmen den Spielraum, der den Tatsachen durch den Satz gelassen wird.}
\ogd{The truth-conditions determine the range, which is left to the facts by the proposition.}
\pmc{The truth-conditions of a proposition determine the range that it leaves open to the facts.}

\pnskip
\ger{(Der Satz, das Bild, das Modell, sind im negativen Sinne wie ein fester K{\"o}rper, der die Bewegungsfreiheit der anderen beschr{\"a}nkt; im positiven Sinne, wie der von fester Substanz begrenzte Raum, worin ein K{\"o}rper Platz hat.)}
\ogd{(The proposition, the picture, the model, are in a negative sense like a solid body, which restricts the free movement of another: in a positive sense, like the space limited by solid substance, in which a body may be placed.)}
\pmc{(A proposition, a picture, or a model is, in the negative sense, like a solid body that restricts the freedom of movement of others, and, in the positive sense, like a space bounded by solid substance in which there is room for a body.)}

\pnskip
\ger{Die Tautologie l{\"a}sst der Wirk\-lich\-keit den ganz\-en---un\-end\-lich\-en---log\-isch\-en Raum; die Kon\-tra\-dik\-tion er\-f{\"u}llt den ganz\-en log\-isch\-en Raum und l{\"a}sst der Wirk\-lich\-keit kei\-nen Punkt. Kei\-ne von bei\-den kann da\-her die Wirk\-lich\-keit ir\-gend\-wie be\-stim\-men.}
\ogd{Tautology leaves to reality the whole infinite logical space; contradiction fills the whole logical space and leaves no point to reality. Neither of them, therefore, can in any way determine reality.}
\pmc{A tautology leaves open to reality the whole---the infinite whole---of logical space: a contradiction fills the whole of logical space leaving no point of it for reality. Thus neither of them can determine reality in any way.}

\pn{4.464}
\ger{Die Wahrheit der Tautologie ist gewiss, des Satzes m{\"o}glich, der Kontradiktion unm{\"o}glich.}
\ogd{The truth of tautology is certain, of propositions possible, of contradiction impossible.}
\pmc{A tautology's truth is certain, a proposition's possible, a contradiction's impossible.}

\pnskip
\ger{(Gewiss, m{\"o}glich, unm{\"o}glich: Hier haben wir das Anzeichen jener Gradation, die wir in der Wahrscheinlichkeitslehre brauchen.)}
\ogd{(Certain, possible, impossible: here we have an indication of that gradation which we need in the theory of probability.)}
\pmc{(Certain, possible, impossible: here we have the first indication of the scale that we need in the theory of probability.)}

\pn{4.465}
\ger{Das logische Produkt einer Tautologie und eines Satzes sagt dasselbe, wie der Satz. Also ist jenes Produkt identisch mit dem Satz. Denn man kann das Wesentliche des Symbols nicht {\"a}ndern, ohne seinen Sinn zu {\"a}ndern.}
\ogd{The logical product of a tautology and a proposition says the same as the proposition. Therefore that product is identical with the proposition. For the essence of the symbol cannot be altered without altering its sense.}
\pmc{The logical product of a tautology and a proposition says the same thing as the proposition. This product, therefore, is identical with the proposition. For it is impossible to alter what is essential to a symbol without altering its sense.}

\pn{4.466}
\ger{Einer be\-stim\-mten lo\-gisch\-en Ver\-bin\-dung von Zei\-chen ent\-spricht eine be\-stimm\-te lo\-gische Ver\-bind\-ung ihrer Be\-deut\-un\-gen; \germph{jede beliebige} Ver\-bin\-dung ent\-spricht nur den un\-ver\-bun\-den\-en Zei\-chen.}
\ogd{To a definite logical combination of signs corresponds a definite logical combination of their meanings; \emph{every arbitrary} combination only corresponds to the unconnected signs.}
\pmc{What corresponds to a determinate logical combination of signs is a determinate logical combination of their meanings. It is only to the uncombined signs that \emph{absolutely any} combination corresponds.}

\pnskip
\ger{Das hei{\ss}t, S{\"a}tze, die f{\"u}r jede Sachlage wahr sind, k{\"o}nnen {\"u}berhaupt keine Zeichenverbindungen sein, denn sonst k{\"o}nnten ihnen nur bestimmte Verbindungen von Gegenst{\"a}nden entsprechen.}
\ogd{That is, propositions which are true for every state of affairs cannot be combinations of signs at all, for otherwise there could only correspond to them definite combinations of objects.}
\pmc{In other words, propositions that are true for every situation cannot be combinations of signs at all, since, if they were, only determinate combinations of objects could correspond to them.}



\pnskip
\ger{(Und keiner logischen Verbindung entspricht \germph{keine} Verbindung der Gegenst{\"a}nde.)}
\ogd{(And to no logical combination corresponds \emph{no} combination of the objects.)}
\pmc{(And what is not a logical combination has \emph{no} combination of objects corresponding to it.)} 

\pnskip
\ger{Tautologie und Kontradiktion sind die Grenzf{\"a}lle der Zeichenverbindung, n{\"a}mlich ihre Aufl{\"o}sung.}
\ogd{Tautology and contradiction are the limiting cases of the combination of symbols, namely their dissolution.}
\pmc{Tau\-to\-log\-y and con\-tra\-dic\-tion are the lim\-it\-ing cases---in\-deed the dis\-in\-te\-gra\-tion---of the com\-bi\-na\-tion of signs.}

\pn{4.4661}
\ger{Freilich sind auch in der Tautologie und Kontradiktion die Zeichen noch mit einander verbunden, d.\ h.\ sie stehen in Beziehungen zu einander, aber diese Beziehungen sind bedeutungslos, dem \germph{Symbol} unwesentlich.}
\ogd{Of course the signs are also combined with one another in the tautology and contradiction, \emph{i.e.}\ they stand in relations to one another, but these relations are meaningless, unessential to the \emph{symbol}.}
\pmc{Admittedly the signs are still combined with one another even in tautologies and contradictions---i.e. they stand in certain relations to one another: but these relations have no meaning, they are not essential to the \emph{symbol}.}

\pn{4.5}
\ger{Nun scheint es m{\"o}glich zu sein, die allgemeinste Satzform anzugeben: das hei{\ss}t, eine Beschreibung der S{\"a}tze \germph{irgend einer} Zeichensprache zu geben, so dass jeder m{\"o}gliche Sinn durch ein Symbol, auf welches die Beschreibung passt, ausgedr{\"u}ckt werden kann, und dass jedes Symbol, worauf die Beschreibung passt, einen Sinn ausdr{\"u}cken kann, wenn die Bedeutungen der Namen entsprechend gew{\"a}hlt werden.}
\ogd{Now it appears to be possible to give the most general form of proposition; \emph{i.e.}\ to give a description of the propositions of some one sign language, so that every possible sense can be expressed by a symbol, which falls under the description, and so that every symbol which falls under the description can express a sense, if the meanings of the names are chosen accordingly.}
\pmc{It now seems possible to give the most general propositional form: that is, to give a description of the propositions of \emph{any} sign-language \emph{whatsoever} in such a way that every possible sense can be expressed by a symbol satisfying the description, and every symbol satisfying the description can express a sense, provided that the meanings of the names are suitably chosen.}

\pnskip
\ger{Es ist klar, dass bei der Beschreibung der allgemeinsten Satzform \germph{nur} ihr Wesentliches beschrieben werden darf,---sonst w{\"a}re sie n{\"a}mlich nicht die allgemeinste.}
\ogd{It is clear that in the description of the most general form of proposition \emph{only} what is essential to it may be described---otherwise it would not be the most general form.}
\pmc{It is clear that \emph{only} what is essential to the most general propositional form may be included in its description---for otherwise it would not be the most general form.}

\pnskip
\ger{Dass es eine allgemeine Satzform gibt, wird dadurch bewiesen, dass es keinen Satz geben darf, dessen Form man nicht h{\"a}tte voraussehen (d.\ h.\ konstruieren) k{\"o}nnen. Die allgemeine Form des Satzes ist: Es verh{\"a}lt sich so und so.}
\ogd{That there is a general form is proved by the fact that there cannot be a proposition whose form could not have been foreseen (\emph{i.e.}\ constructed). The general form of proposition is: Such and such is the case.}
\pmc{The existence of a general propositional form is proved by the fact that there cannot be a proposition whose form could not have been foreseen (i.e.\ constructed). The general form of a proposition is: This is how things stand.}

\pn{4.51}
\ger{Angenommen, mir w{\"a}ren \germph{alle} Elementars{\"a}tze gegeben: Dann l{\"a}sst sich einfach fragen: Welche S{\"a}tze kann ich aus ihnen bilden? Und das sind \germph{alle} S{\"a}tze und \germph{so} sind sie begrenzt.}
\ogd{Suppose \emph{all} elementary propositions were given me: then we can simply ask: what propositions I can build out of them. And these are \emph{all} propositions and \emph{so} are they limited.}
\pmc{Suppose that I am given \emph{all} elementary propositions: then I can simply ask what propositions I can construct out of them. And there I have \emph{all} propositions, and \emph{that} fixes their limits.}

\pn{4.52}
\ger{Die S{\"a}tze sind alles, was aus der Gesamtheit aller Elementars{\"a}tze folgt (nat{\"u}rlich auch daraus, dass es die \germph{Gesamtheit aller} ist). (So k{\"o}nnte man in gewissem Sinne sagen, dass \germph{alle} S{\"a}tze Verallgemeinerungen der Elementars{\"a}tze sind.)}
\ogd{The propositions are everything which follows from the totality of all elementary propositions (of course also from the fact that it is the \emph{totality of them all}). (So, in some sense, one could say, that \emph{all} propositions are generalizations of the elementary propositions.)}
\pmc{Propositions comprise all that follows from the totality of all elementary propositions (and, of course, from its being the \emph{totality} of them \emph{all}). (Thus, in a certain sense, it could be said that \emph{all} propositions were generalizations of elementary propositions.)}

\pn{4.53}
\ger{Die allgemeine Satzform ist eine Variable.}
\ogd{The general proposition form is a variable.}
\pmc{The general propositional form is a variable.}

\pn{5}
\ger{Der Satz ist eine Wahrheitsfunktion der Elementars{\"a}tze.}
\ogd{Propositions are truth-functions of elementary propositions.}
\pmc{A proposition is a truth-function of elementary propositions.}

\pnskip
\ger{(Der Elementarsatz ist eine Wahrheitsfunktion seiner selbst.)}
\ogd{(An elementary proposition is a truth-function of itself.)}
\pmc{(An elementary proposition is a truth-function of itself.)}

\pn{5.01}
\ger{Die Elementars{\"a}tze sind die Wahrheitsargumente des Satzes.}
\ogd{The elementary propositions are the truth-arguments of propositions.}
\pmc{Elementary propositions are the truth-arguments of propositions.}

\pn{5.02}
\ger{Es liegt nahe, die Argumente von Funktionen mit den Indices von Namen zu verwechseln. Ich erkenne n{\"a}mlich sowohl am Argument wie am Index die Bedeutung des sie enthaltenden Zeichens.}
\ogd{It is natural to confuse the arguments of functions with the indices of names. For I recognize the meaning of the sign containing it from the argument just as much as from the index.}
\pmc{The arguments of functions are readily confused with the affixes of names. For both arguments and affixes enable me to recognize the meaning of the signs containing them.}

\pnskip
\ger{In Russells \gdql $+_c$\gdqr{} ist z.B.\ \gdql $_c$\gdqr{} ein Index, der darauf hinweist, dass das ganze Zeichen das Additionszeichen f{\"u}r Kardinalzahlen ist. Aber diese Bezeichnung beruht auf willk{\"u}rlicher {\"U}bereinkunft und man k{\"o}nnte statt \gdql $+_c$\gdqr{} auch ein einfaches Zeichen w{\"a}hlen; in \gdql $\rnot p$\gdqr{} aber ist \gdql $p$\gdqr{} kein Index, sondern ein Argument: der Sinn von \gdql $\rnot p$\gdqr{} \germph{kann nicht} verstanden werden, ohne dass vorher der Sinn von \gdql $p$\gdqr{} verstanden worden w{\"a}re. (Im Namen Julius C{\"a}sar ist \gdql Julius\gdqr{} ein Index. Der Index ist immer ein Teil einer Beschreibung des Gegenstandes, dessen Namen wir ihn anh{\"a}ngen. Z.\ B.\ \germph{der} C{\"a}sar aus dem Geschlechte der Julier.)}
\ogd{In Russell's ``$+_c$'', for example, ``$_c$'' is an index which indicates that the whole sign is the addition sign for cardinal numbers. But this way of symbolizing depends on arbitrary agreement, and one could choose a simple sign instead of ``$+_c$'': but in ``$\rnot p$'' ``$p$'' is not an index but an argument; the sense of ``$\rnot p$'' \emph{cannot} be understood, unless the sense of ``$p$'' has previously been understood. (In the name Julius C\ae{}sar, Julius is an index. The index is always part of a description of the object to whose name we attach it, \emph{e.g.} \emph{The} C\ae{}sar of the Julian gens.)}
\pmc{For example, when Russell writes `$+_c$', the `$_c$' is an affix which indicates that the sign as a whole is the addition-sign for cardinal numbers. But the use of this sign is the result of arbitrary convention and it would be quite possible to choose a simple sign instead of `$+_c$'; in `$\rnot p$', however, `$p$' is not an affix but an argument: the sense of `$\rnot p$' \emph{cannot} be understood unless the sense of `$p$' has been understood already. (In the name Julius Caesar `Julius' is an affix. An affix is always part of a description of the object to whose name we attach it: e.g.\ \emph{the} Caesar of the Julian gens.)}

\pnskip
\ger{Die Verwechslung von Argument und Index liegt, wenn ich mich nicht irre, der Theorie Freges von der Bedeutung der S{\"a}tze und Funktionen zugrunde. F{\"u}r Frege waren die S{\"a}tze der Logik Namen, und deren Argumente die Indices dieser Namen.}
\ogd{The confusion of argument and index is, if I am not mistaken, at the root of Frege's theory of the meaning of propositions and functions. For Frege the propositions of logic were names and their arguments the indices of these names.}
\pmc{If I am not mistaken, Frege's theory about the meaning of propositions and functions is based on the confusion between an argument and an affix. Frege regarded the propositions of logic as names, and their arguments as the affixes of those names.}

\pn{5.1}
\ger{Die Wahrheitsfunktionen lassen sich in Reihen ordnen.}
\ogd{The truth-functions can be ordered in series.}
\pmc{Truth-functions can be arranged in series.}

\pnskip
\ger{Das ist die Grundlage der Wahrscheinlichkeitslehre.}
\ogd{That is the foundation of the theory of probability.}
\pmc{That is the foundation of the theory of probability.}

\pn{5.101}
\ger{Die Wahrheitsfunktionen jeder Anzahl von Elementars{\"a}tzen lassen sich in einem Schema folgender Art hinschreiben:}
\ogd{The truth-functions of every number of elementary propositions can be written in a schema of the following kind:}
\pmc{The truth-functions of a given number of elementary propositions can always be set out in a schema of the following kind:}

\pnskip
\ger{\negpbk\fiveonezeroonetablegerman}
\ogd{\negpbk\fiveonezeroonetableogden}
\pmc{\negpbk\fiveonezeroonetablepmc}

\pnskip
\ger{Diejenigen Wahrheitsm{\"o}glichkeiten sei\-ner Wahr\-heits\-arg\-u\-men\-te, wel\-che den Satz be\-wahr\-hei\-ten, will ich sei\-ne \germph{Wahr\-heits\-gr{\"u}n\-de} nen\-nen.}
\ogd{Those truth-possibilities of its truth-arguments, which verify the proposition, I shall call its \emph{truth-grounds}.}
\pmc{I will give the name \emph{truth-grounds} of a proposition to those truth-possibilities of its truth-arguments that make it true.}

\pn{5.11}
\ger{Sind die Wahrheitsgr{\"u}nde, die einer Anzahl von S{\"a}tzen gemeinsam sind, s{\"a}mtlich auch Wahrheitsgr{\"u}nde eines bestimmten Satzes, so sagen wir, die Wahrheit dieses Satzes folge aus der Wahrheit jener S{\"a}tze.}
\ogd{If the truth-grounds which are common to a number of propositions are all also truth-grounds of some one proposition, we say that the truth of this proposition follows from the truth of those propositions.}
\pmc{If all the truth-grounds that are common to a number of propositions are at the same time truth-grounds of a certain proposition, then we say that the truth of that proposition follows from the truth of the others.}

\pn{5.12}
\ger{Insbesondere folgt die Wahrheit eines Satzes \gdql $p$\gdqr{} aus der Wahrheit eines anderen \gdql $q$\gdqr{}, wenn alle Wahrheitsgr{\"u}nde des zweiten Wahrheitsgr{\"u}nde des ersten sind.}
\ogd{In particular the truth of a proposition $p$ follows from that of a proposition $q$, if all the truth-grounds of the second are truth-grounds of the first.}
\pmc{In particular, the truth of a proposition `$p$' follows from the truth of another proposition `$q$' if all the truth-grounds of the latter are truth-grounds of the former.}%???

\pn{5.121}
\ger{Die Wahrheitsgr{\"u}nde des einen sind in denen des anderen enthalten; $p$ folgt aus $q$.}
\ogd{The truth-grounds of $q$ are contained in those of $p$; $p$ follows from $q$.}
\pmc{The truth-grounds of the one are contained in those of the other: $p$ follows from $q$.}

\pn{5.122}
\ger{Folgt $p$ aus $q$, so ist der Sinn von \gdql $p$\gdqr{} im Sinne von \gdql $q$\gdqr{} enthalten.}
\ogd{If $p$ follows from $q$, the sense of ``$p$'' is contained in that of ``$q$''.}
\pmc{If $p$ follows from $q$, the sense of `$p$' is contained in the sense of `$q$'.}

\pn{5.123}
\ger{Wenn ein Gott eine Welt erschafft, worin gewisse S{\"a}tze wahr sind, so schafft er damit auch schon eine Welt, in welcher alle ihre Folges{\"a}tze stimmen. Und {\"a}hnlich k{\"o}nnte er keine Welt schaffen, worin der Satz \gdql $p$\gdqr{} wahr ist, ohne seine s{\"a}mtlichen Gegenst{\"a}nde zu schaffen.}
\ogd{If a god creates a world in which certain propositions are true, he creates thereby also a world in which all propositions consequent on them are true. And similarly he could not create a world in which the proposition ``$p$'' is true without creating all its objects.}
\pmc{If a god creates a world in which certain propositions are true, then by that very act he also creates a world in which all the propositions that follow from them come true. And similarly he could not create a world in which the proposition `$p$' was true without creating all its objects.}

\pn{5.124}
\ger{Der Satz bejaht jeden Satz, der aus ihm folgt.}
\ogd{A proposition asserts every proposition which follows from it.}
\pmc{A proposition affirms every proposition that follows from it.}

\pn{5.1241}
\ger{\gdql $p \rand q$\gdqr{} ist einer der S{\"a}tze, welche \gdql $p$\gdqr{} bejahen, und zugleich einer der S{\"a}tze, welche \gdql $q$\gdqr{} bejahen.}
\ogd{``$p \rand q$'' is one of the propositions which assert ``$p$'' and at the same time one of the propositions which assert ``$q$''.}
\pmc{`$p \rand q$' is one of the propositions that affirm `$p$' and at the same time one of the propositions that affirm `$q$'.}

\pnskip
\ger{Zwei S{\"a}tze sind einander entgegengesetzt, wenn es keinen sinnvollen Satz gibt, der sie beide bejaht.}
\ogd{Two propositions are opposed to one another if there is no significant proposition which asserts them both.}
\pmc{Two propositions are opposed to one another if there is no proposition with a sense, that affirms them both.}

\pnskip
\ger{Jeder Satz der einem anderen widerspricht, verneint ihn.}
\ogd{Every proposition which contradicts another, denies it.}
\pmc{Every proposition that contradicts another negates it.}%???

\pn{5.13}
\ger{Dass die Wahrheit eines Satzes aus der Wahrheit anderer S{\"a}tze folgt, ersehen wir aus der Struktur der S{\"a}tze.}
\ogd{That the truth of one proposition follows from the truth of other propositions, we perceive from the structure of the propositions.}
\pmc{When the truth of one proposition follows from the truth of others, we can see this from the structure of the propositions.}

\pn{5.131}
\ger{Folgt die Wahrheit eines Satzes aus der Wahrheit anderer, so dr{\"u}ckt sich dies durch Beziehungen aus, in welchen die Formen jener S{\"a}tze zu einander stehen; und zwar brauchen wir sie nicht erst in jene Beziehungen zu setzen, indem wir sie in einem Satz miteinander verbinden, sondern diese Beziehungen sind intern und bestehen, sobald, und dadurch dass, jene S{\"a}tze bestehen.}
\ogd{If the truth of one proposition follows from the truth of others, this expresses itself in relations in which the forms of these propositions stand to one another, and we do not need to put them in these relations first by connecting them with one another in a proposition; for these relations are internal, and exist as soon as, and by the very fact that, the propositions exist.}
\pmc{If the truth of one proposition follows from the truth of others, this finds expression in relations in which the forms of the propositions stand to one another: nor is it necessary for us to set up these relations between them, by combining them with one another in a single proposition; on the contrary, the relations are internal, and their existence is an immediate result of the existence of the propositions.}

\pn{5.1311}
\ger{Wenn wir von $p \lor q$ und $\rnot p$ auf $q$ schlie{\ss}en, so ist hier durch die Bezeichnungsweise die Beziehung der Satzformen von \gdql $p \lor q$\gdqr{} und \gdql $\rnot p$\gdqr{} verh{\"u}llt. Schreiben wir aber z.\ B.\ statt \gdql $p \lor q$\gdqr{} \gdql $p \sheffer q \dshefferd p \sheffer q$\gdqr{} und statt \gdql $\rnot p$\gdqr{} \gdql $p \sheffer p$\gdqr{} ($p \sheffer q$ = weder $p$, noch $q$), so wird der innere Zusammenhang offenbar.}
\ogd{When we conclude from $p \lor q$ and $\rnot p$ to $q$ the relation between the forms of the propositions ``$p \lor q$'' and ``$\rnot p$'' is here concealed by the method of symbolizing. But if we write, \emph{e.g.}\ instead of ``$p \lor q$'' ``$p \sheffer q \dshefferd p \sheffer q$'' and instead of ``$\rnot p$'' ``$p \sheffer p$'' ($p \sheffer q$ = neither $p$ nor $q$), then the inner connexion becomes obvious.}
\pmc{When we infer $q$ from $p \lor q$ and $\rnot p$, the relation between the propositional forms of `$p \lor q$' and `$\rnot p$' is masked, in this case, by our mode of signifying. But if instead of `$p \lor q$' we write, for example, `$p \sheffer q \dshefferd p \sheffer q$', and instead of `$\rnot p$', `$p \sheffer p$' ($p \sheffer q$ = neither $p$ nor $q$), then the inner connexion becomes obvious.}

\pnskip
\ger{(Dass man aus $\ralld{x} f\negthinspace x$ auf $f\negthinspace a$ schlie{\ss}en kann, das zeigt, dass die Allgemeinheit auch im Symbol \gdql $\ralld{x} f\negthinspace x$\gdqr{} vorhanden ist.)}
\ogd{(The fact that we can infer $f\negthinspace a$ from $\ralld{x} f\negthinspace x$ shows that generality is present also in the symbol ``$\ralld{x} f\negthinspace x$''.}
\pmc{(The possibility of inference from $\ralld{x} f\negthinspace x$ to $f\negthinspace a$ shows that the symbol $\ralld{x} f\negthinspace x$ itself has generality in it.)}

\pn{5.132}
\ger{Folgt $p$ aus $q$, so kann ich von $q$ auf $p$ schlie{\ss}en; $p$ aus $q$ folgern.}
\ogd{If $p$ follows from $q$, I can conclude from $q$ to $p$; infer $p$ from $q$.}
\pmc{If $p$ follows from $q$, I can make an inference from $q$ to $p$, deduce $p$ from $q$.}

\pnskip
\ger{Die Art des Schlusses ist allein aus den beiden S{\"a}tzen zu entnehmen.}
\ogd{The method of inference is to be understood from the two propositions alone.}
\pmc{The nature of the inference can be gathered only from the two propositions.}

\pnskip
\ger{Nur sie selbst k{\"o}nnen den Schluss rechtfertigen.}
\ogd{Only they themselves can justify the inference.}
\pmc{They themselves are the only possible justification of the inference.}

\pnskip
\ger{\gdql Schlussgesetze\gdqr{}, welche---wie bei Frege und Russell---die Schl{\"u}sse rechtfertigen sollen, sind sinnlos, und w{\"a}ren {\"u}berfl{\"u}ssig.}
\ogd{Laws of inference, which---as in Frege and Russell---are to justify the conclusions, are senseless and would be superfluous.}
\pmc{`Laws of inference', which are supposed to justify inferences, as in the works of Frege and Russell, have no sense, and would be superfluous.}

\pn{5.133}
\ger{Alles Folgern geschieht a priori.}
\ogd{All inference takes place a priori.}
\pmc{All deductions are made \emph{a priori}.}

\pn{5.134}
\ger{Aus einem Elementarsatz l{\"a}sst sich kein anderer folgern.}
\ogd{From an elementary proposition no other can be inferred.}
\pmc{One elementary proposition cannot be deduced form another.}

\pn{5.135}
\ger{Auf keine Weise kann aus dem Bestehen irgend einer Sachlage auf das Bestehen einer von ihr g{\"a}nzlich verschiedenen Sachlage geschlossen werden.}
\ogd{In no way can an inference be made from the existence of one state of affairs to the existence of another entirely different from it.}
\pmc{There is no possible way of making an inference from the existence of one situation to the existence of another, entirely different situation.}

\pn{5.136}
\ger{Einen Kausalnexus, der einen solchen Schluss rechtfertigte, gibt es nicht.}
\ogd{There is no causal nexus which justifies such an inference.}
\pmc{There is no causal nexus to justify such an inference.}

\pn{5.1361}
\ger{Die Ereignisse der Zukunft \germph{k{\"o}nnen} wir nicht aus den gegenw{\"a}rtigen erschlie{\ss}en.}
\ogd{The events of the future \emph{cannot} be inferred from those of the present.}
\pmc{We \emph{cannot} infer the events of the future from those of the present.}

\pnskip
\ger{Der Glaube an den Kausalnexus ist der \germph{Aberglaube}.}
\ogd{Superstition is the belief in the causal nexus.}
\pmc{Belief in the causal nexus is \emph{superstition}.}

\pn{5.1362}
\ger{Die Willensfreiheit besteht darin, dass zuk{\"u}nftige Handlungen jetzt nicht gewusst werden k{\"o}nnen. Nur dann k{\"o}nnten wir sie wissen, wenn die Kausalit{\"a}t eine \germph{innere} Notwendigkeit w{\"a}re, wie die des logischen Schlusses.---Der Zusammenhang von Wissen und Gewusstem ist der der logischen Notwendigkeit.}
\ogd{The freedom of the will consists in the fact that future actions cannot be known now. We could only know them if causality were an \emph{inner} necessity, like that of logical deduction.---The connexion of knowledge and what is known is that of logical necessity.}
\pmc{The freedom of the will consists in the impossibility of knowing actions that still lie in the future. We could know them only if causality were an \emph{inner} necessity like that of logical inference.---The connexion between knowledge and what is known is that of logical necessity.}

\pnskip
\ger{(\gdql A wei{\ss}, dass $p$ der Fall ist\gdqr{} ist sinnlos, wenn $p$ eine Tautologie ist.)}
\ogd{(``A knows that $p$ is the case'' is senseless if $p$ is a tautology.)}
\pmc{(`A knows that $p$ is the case', has no sense if $p$ is a tautology.)}

\pn{5.1363}
\ger{Wenn daraus, dass ein Satz uns einleuchtet, nicht \germph{folgt}, dass er wahr ist, so ist das Einleuchten auch keine Rechtfertigung f{\"u}r unseren Glauben an seine Wahrheit.}
\ogd{If from the fact that a proposition is obvious to us it does not \emph{follow} that it is true, then obviousness is no justification for our belief in its truth.}
\pmc{If the truth of a proposition does not \emph{follow} from the fact that it is self-evident to us, then its self-evidence in no way justifies our belief in its truth.}

\pn{5.14}
\ger{Folgt ein Satz aus einem anderen, so sagt dieser mehr als jener, jener weniger als dieser.}
\ogd{If a proposition follows from another, then the latter says more than the former, the former less than the latter.}
\pmc{If one proposition follows from another, then the latter says more than the former, and the former less than the latter.}

\pn{5.141}
\ger{Folgt $p$ aus $q$ und $q$ aus $p$, so sind sie ein und derselbe Satz.}
\ogd{If $p$ follows from $q$ and $q$ from $p$ then they are one and the same proposition.}
\pmc{If $p$ follows from $q$ and $q$ from $p$, then they are one and the same proposition.}%???

\pn{5.142}
\ger{Die Tautologie folgt aus allen S{\"a}tzen: sie sagt nichts.}
\ogd{A tautology follows from all propositions: it says nothing.}
\pmc{A tautology follows from all propositions: it says nothing.}

\pn{5.143}
\ger{Die Kontradiktion ist das Gemeinsame der S{\"a}tze, was \germph{kein} Satz mit einem anderen gemein hat. Die Tautologie ist das Gemeinsame aller S{\"a}tze, welche nichts miteinander gemein haben.}
\ogd{Contradiction is something shared by propositions, which \emph{no} proposition has in common with another. Tautology is that which is shared by all propositions, which have nothing in common with one another.}
\pmc{Contradiction is that common factor of propositions which \emph{no} proposition has in common with another. Tautology is the common factor of all propositions that have nothing in common with one another. }

\pnskip
\ger{Die Kontradiktion verschwindet sozusagen au{\ss}erhalb, die Tautologie innerhalb aller S{\"a}tze.}
\ogd{Contradiction vanishes so to speak outside, tautology inside all propositions.}
\pmc{Contradiction, one might say, vanishes outside all propositions: tautology vanishes inside them.}

\pnskip
\ger{Die Kontradiktion ist die {\"a}u{\ss}ere Grenze der S{\"a}tze, die Tautologie ihr substanzloser Mittelpunkt.}
\ogd{Contradiction is the external limit of the propositions, tautology their substanceless centre.}
\pmc{Contradiction is the outer limit of propositions: tautology is the unsubstantial point at their centre.}

\pn{5.15}
\ger{Ist $\mathrm{W}_r$ die Anzahl der Wahrheitsgr{\"u}nde des Satzes \gdql $r$\gdqr{}, $\mathrm{W}_{rs}$ die Anzahl derjenigen Wahrheitsgr{\"u}nde des Satzes \gdql $s$\gdqr{}, die zugleich Wahrheitsgr{\"u}nde von \gdql $r$\gdqr{} sind, dann nennen wir das Verh{\"a}ltnis: $\mathrm{W}_{rs} : \mathrm{W}_r$ das Ma{\ss} der \germph{Wahrscheinlichkeit}, welche der Satz \gdql $r$\gdqr{} dem Satz \gdql $s$\gdqr{} gibt.}
\ogd{If $\mathrm{T}_r$ is the number of the truth-grounds of the proposition ``$r$'', $\mathrm{T}_{rs}$ the number of those truth-grounds of the proposition ``$s$'' which are at the same time truth-grounds of ``$r$'', then we call the ratio $\mathrm{T}_{rs} : \mathrm{T}_r$ the measure of the \emph{probability} which the proposition ``$r$'' gives to the proposition ``$s$''.}
\pmc{If $\mathrm{T}_r$ is the number of the truth-grounds of a proposition `$r$', and if $\mathrm{T}_{rs}$ is the number of the truth-grounds of a proposition `$s$' that are at the same time truth-grounds of `$r$', then we call the ratio $\mathrm{T}_{rs} : \mathrm{T}_r$ the degree of probability that the proposition `$r$' gives to the proposition `$s$'.}

\pn{5.151}
\ger{Sei in einem Schema wie dem obigen in No.\ 5.101 $\mathrm{W}_r$ die Anzahl der \gdql W\gdqr{} im Satze $r$; $\mathrm{W}_{rs}$ die Anzahl derjenigen \gdql W\gdqr{} im Satze $s$, die in gleichen Kolonnen mit \gdql W\gdqr{} des Satzes $r$ stehen. Der Satz $r$ gibt dann dem Satze $s$ die Wahrscheinlichkeit: $\mathrm{W}_{rs} : \mathrm{W}_r$.}
\ogd{Suppose in a schema like that above in No.\ 5.101 $\mathrm{T}_r$ is the number of the ``T''\thinspace's in the proposition $r$, $\mathrm{T}_{rs}$ the number of those ``T''\thinspace's in the proposition $s$, which stand in the same columns as ``T''\thinspace's of the proposition $r$; then the proposition $r$ gives to the proposition $s$ the probability $\mathrm{T}_{rs} : \mathrm{T}_r$.}
\pmc{In a schema like the one above in 5.101, let $\mathrm{T}_r$ be the number of `T's' in the proposition $r$, and let $\mathrm{T}_{rs}$, be the number of `T's' in the proposition $s$ that stand in columns in which the proposition $r$ has `T's'. Then the proposition $r$ gives to the proposition $s$ the probability $\mathrm{T}_{rs} : \mathrm{T}_r$.}

\pn{5.1511}
\ger{Es gibt keinen besonderen Gegenstand, der den Wahrscheinlichkeitss{\"a}tzen eigen w{\"a}re.}
\ogd{There is no special object peculiar to probability propositions.}
\pmc{There is no special object peculiar to probability propositions.}

\pn{5.152}
\ger{S{\"a}tze, welche keine Wahrheitsargumente mit einander gemein haben, nennen wir von einander unabh{\"a}ngig.}
\ogd{Propositions which have no truth-arguments in common with one another we call independent.}
\pmc{When propositions have no truth-arguments in common with one another, we call them independent of one another.}

\pnskip
\ger{Zwei Elementars{\"a}tze geben einander die Wahrscheinlichkeit $\tfrac{1}{2}$.}
\ogd{Independent propositions (\emph{e.g.}\ any two elementary propositions) give to one another the probability $\tfrac{1}{2}$.}
\pmc{Two elementary propositions give one another the probability $\tfrac{1}{2}$.}

\pnskip
\ger{Folgt $p$ aus $q$, so gibt der Satz \gdql $q$\gdqr{} dem Satz \gdql $p$\gdqr{} die Wahrscheinlichkeit 1. Die Gewissheit des logischen Schlusses ist ein Grenzfall der Wahrscheinlichkeit.}
\ogd{If $p$ follows from $q$, the proposition $q$ gives to the proposition $p$ the probability 1. The certainty of logical conclusion is a limiting case of probability.}
\pmc{If $p$ follows from $q$, then the proposition `$q$' gives to the proposition `$p$' the probability 1. The certainty of logical inference is a limiting case of probability.}

\pnskip
\ger{(Anwendung auf Tautologie und Kontradiktion.)}
\ogd{(Application to tautology and contradiction.)}
\pmc{(Application of this to tautology and contradiction.)}

\pn{5.153}
\ger{Ein Satz ist an sich weder wahrscheinlich noch unwahrscheinlich. Ein Ereignis trifft ein, oder es trifft nicht ein, ein Mittelding gibt es nicht.}
\ogd{A proposition is in itself neither probable nor improbable. An event occurs or does not occur, there is no middle course.}
\pmc{In itself, a proposition is neither probable nor improbable. Either an event occurs or it does not: there is no middle way.}

\pn{5.154}
\ger{In einer Urne seien gleichviel wei{\ss}e und schwarze Kugeln (und keine anderen). Ich ziehe eine Kugel nach der anderen und lege sie wieder in die Urne zur{\"u}ck. Dann kann ich durch den Versuch feststellen, dass sich die Zahlen der gezogenen schwarzen und wei{\ss}en Kugeln bei fortgesetztem Ziehen einander n{\"a}hern.}
\ogd{In an urn there are equal numbers of white and black balls (and no others). I draw one ball after another and put them back in the urn. Then I can determine by the experiment that the numbers of the black and white balls which are drawn approximate as the drawing continues.}
\pmc{Suppose that an urn contains black and white balls in equal numbers (and none of any other kind). I draw one ball after another, putting them back into the urn. By this experiment I can establish that the number of black balls drawn and the number of white balls drawn approximate to one another as the draw continues.}

\pnskip 
\ger{\germph{Das} ist also kein mathematisches Faktum.} 
\ogd{So \emph{this} is not a mathematical fact.}
\pmc{So \emph{this} is not a mathematical truth.}

\pnskip 
\ger{Wenn ich nun sage: Es ist gleich wahrscheinlich, dass ich eine wei{\ss}e Kugel wie eine schwarze ziehen werde, so hei{\ss}t das: Alle mir bekannten Umst{\"a}nde (die hypothetisch angenommenen Naturgesetze mitinbegriffen) geben dem Eintreffen des einen Ereignisses nicht \germph{mehr} Wahrscheinlichkeit als dem Eintreffen des anderen. Das hei{\ss}t, sie geben---wie aus den obigen Erkl{\"a}rungen leicht zu entnehmen ist---jedem die Wahrscheinlichkeit $\tfrac{1}{2}$.}
\ogd{If then, I say, It is equally probable that I should draw a white and a black ball, this means, All the circumstances known to me (including the natural laws hypothetically assumed) give to the occurrence of the one event no more probability than to the occurrence of the other. That is they give---as can easily be understood from the above explanations---to each the probability $\tfrac{1}{2}$.}
\pmc{Now, if I say, `The probability of my drawing a white ball is equal to the probability of my drawing a black one', this means that all the circumstances that I know of (including the laws of nature assumed as hypotheses) give no \emph{more} probability to the occurrence of the one event than to that of the other. That is to say, they give each the probability $\tfrac{1}{2}$, as can easily be gathered from the above definitions.}

\pnskip
\ger{Was ich durch den Versuch best{\"a}tige ist, dass das Eintreffen der beiden Ereignisse von den Umst{\"a}nden, die ich nicht n{\"a}her kenne, unabh{\"a}ngig ist.}
\ogd{What I can verify by the experiment is that the occurrence of the two events is independent of the circumstances with which I have no closer acquaintance.}
\pmc{What I confirm by the experiment is that the occurrence of the two events is independent of the circumstances of which I have no more detailed knowledge.}

\pn{5.155}
\ger{Die Einheit des Wahrscheinlichkeitssatzes ist: Die Umst{\"a}nde---die ich sonst nicht weiter kenne---geben dem Eintreffen eines bestimmten Ereignisses den und den Grad der Wahrscheinlichkeit.}
\ogd{The unit of the probability proposition is: The circumstances---with which I am not further acquainted---give to the occurrence of a definite event such and such a degree of probability.}
\pmc{The minimal unit for a probability proposition is this: The circumstances---of which I have no further knowledge---give such and such a degree of probability to the occurrence of a particular event.}

\pn{5.156}
\ger{So ist die Wahrscheinlichkeit eine Verallgemeinerung.}
\ogd{Probability is a generalization.}
\pmc{It is in this way that probability is a generalization.}

\pnskip
\ger{Sie involviert eine allgemeine Beschreibung einer Satzform.}
\ogd{It involves a general description of a propositional form.}
\pmc{It involves a general description of a propositional form.}

\pnskip
\ger{Nur in Ermanglung der Gewissheit gebrauchen wir die Wahrscheinlichkeit.---Wenn wir zwar eine Tatsache nicht vollkommen kennen, wohl aber \germph{etwas} {\"u}ber ihre Form wissen.}
\ogd{Only in default of certainty do we need probability. If we are not completely acquainted with a fact, but know \emph{something} about its form.}
\pmc{We use probability only in default of certainty---if our knowledge of a fact is not indeed complete, but we do know \emph{something} about its form.}

\pnskip
\ger{(Ein Satz kann zwar ein unvollst{\"a}ndiges Bild einer gewissen Sachlage sein, aber er ist immer \germph{ein} vollst{\"a}ndiges Bild.)}
\ogd{(A proposition can, indeed, be an incomplete picture of a certain state of affairs, but it is always \emph{a} complete picture.)}
\pmc{(A proposition may well be an incomplete picture of a certain situation, but it is always a complete picture of \emph{something}.)}

\pnskip
\ger{Der Wahr\-schein\-lich\-keits\-satz ist \linebreak{}gleich\-sam ein Aus\-zug aus an\-de\-ren S{\"a}tz\-en.}
\ogd{The probability proposition is, as it were, an extract from other propositions.}
\pmc{A probability proposition is a sort of excerpt from other propositions.}

\pn{5.2}
\ger{Die Strukturen der S{\"a}tze stehen in internen Beziehungen zu einander.}
\ogd{The structures of propositions stand to one another in internal relations.}
\pmc{The structures of propositions stand in internal relations to one another.}

\pn{5.21}
\ger{Wir k{\"o}nnen diese internen Beziehungen dadurch in unserer Ausdrucksweise hervorheben, dass wir einen Satz als Resultat einer Operation darstellen, die ihn aus anderen S{\"a}tzen (den Basen der Operation) hervorbringt.}
\ogd{We can bring out these internal relations in our manner of expression, by presenting a proposition as the result of an operation which produces it from other propositions (the bases of the operation).}
\pmc{In order to give prominence to these internal relations we can adopt the following mode of expression: we can represent a proposition as the result of an operation that produces it out of other propositions (which are the bases of the operation).}

\pn{5.22}
\ger{Die Operation ist der Ausdruck einer Beziehung zwischen den Strukturen ihres Resultats und ihrer Basen.}
\ogd{The operation is the expression of a relation between the structures of its result and its bases.}
\pmc{An operation is the expression of a relation between the structures of its result and of its bases.}

\pn{5.23}
\ger{Die Operation ist das, was mit dem einen Satz geschehen muss, um aus ihm den anderen zu machen.}
\ogd{The operation is that which must happen to a proposition in order to make another out of it.}
\pmc{The operation is what has to be done to the one proposition in order to make the other out of it.}

\pn{5.231}
\ger{Und das wird nat{\"u}rlich von ihren formalen Eigenschaften, von der internen {\"A}hnlichkeit ihrer Formen abh{\"a}ngen.}
\ogd{And that will naturally depend on their formal properties, on the internal similarity of their forms.}
\pmc{And that will, of course, depend on their formal properties, on the internal similarity of their forms.}

\pn{5.232}
\ger{Die interne Relation, die eine Reihe ordnet, ist {\"a}quivalent mit der Operation, durch welche ein Glied aus dem anderen entsteht.}
\ogd{The internal relation which orders a series is equivalent to the operation by which one term arises from another.}
\pmc{The internal relation by which a series is ordered is equivalent to the operation that produces one term from another.}

\pn{5.233}
\ger{Die Operation kann erst dort auftreten, wo ein Satz auf logisch bedeutungsvolle Weise aus einem anderen entsteht. Also dort, wo die logische Konstruktion des Satzes anf{\"a}ngt.}
\ogd{The first place in which an operation can occur is where a proposition arises from another in a logically significant way; \emph{i.e.}\ where the logical construction of the proposition begins.}
\pmc{Operations cannot make their appearance before the point at which one proposition is generated out of another in a logically meaningful way; i.e.\ the point at which the logical construction of propositions begins.}

\pn{5.234}
\ger{Die Wahrheitsfunktionen der Elementars{\"a}tze sind Resultate von Operationen, die die Elementars{\"a}tze als Basen haben. (Ich nenne diese Operationen Wahrheitsoperationen.)}
\ogd{The truth-functions of elementary proposition, are results of operations which have the elementary propositions as bases. (I call these operations, truth-operations.)}
\pmc{Truth-functions of elementary propositions are results of operations with elementary propositions as bases. (These operations I call truth-operations.)}

\pn{5.2341}
\ger{Der Sinn einer Wahrheitsfunktion von $p$ ist eine Funktion des Sinnes von $p$.}
\ogd{The sense of a truth-function of $p$ is a function of the sense of $p$.}
\pmc{The sense of a truth-function of $p$ is a function of the sense of $p$.}

\pnskip
\ger{Verneinung, logische Addition, logische Multiplikation, etc., etc.\ sind Operationen.}
\ogd{Denial, logical addition, logical multiplication, etc., etc., are operations.}
\pmc{Negation, logical addition, logical multiplication, etc.\ etc.\ are operations.}

\pnskip
\ger{(Die Verneinung verkehrt den Sinn des Satzes.)}
\ogd{(Denial reverses the sense of a proposition.)}
\pmc{(Negation reverses the sense of a proposition.)}

\pn{5.24}
\ger{Die Operation zeigt sich in einer Variablen; sie zeigt, wie man von einer Form von S{\"a}tzen zu einer anderen gelangen kann.}
\ogd{An operation shows itself in a variable; it shows how we can proceed from one form of proposition to another.}
\pmc{An operation manifests itself in a variable; it shows how we can get from one form of proposition to another.}

\pnskip
\ger{Sie bringt den Unterschied der Formen zum Ausdruck.}
\ogd{It gives expression to the difference between the forms.}
\pmc{It gives expression to the difference between the forms.}

\pnskip
\ger{(Und das Gemeinsame zwischen den Basen und dem Resultat der Operation sind eben die Basen.)}
\ogd{(And that which is common the the bases, and the result of an operation, is the bases themselves.)}
\pmc{(And what the bases of an operation and its result have in common is just the bases themselves.)}

\pn{5.241}
\ger{Die Operation kennzeichnet keine Form, sondern nur den Unterschied der Formen.}
\ogd{The operation does not characterize a form but only the difference between forms.}
\pmc{An operation is not the mark of a form, but only of a difference between forms.}

\pn{5.242}
\ger{Dieselbe Operation, die \gdql $q$\gdqr{} aus \gdql $p$\gdqr{} macht, macht aus \gdql $q$\gdqr{} \gdql $r$\gdqr{} u.\ s.\ f. Dies kann nur darin ausgedr{\"u}ckt sein, dass \gdql $p$\gdqr{}, \gdql $q$\gdqr{}, \gdql $r$\gdqr{}, etc.\ Variable sind, die gewisse formale Relationen allgemein zum Ausdruck bringen.}
\ogd{The same operation which makes ``$q$'' from ``$p$'', makes ``$r$'' from ``$q$'', and so on. This can only be expressed by the fact that ``$p$'', ``$q$'', ``$r$'', etc., are variables which give general expression to certain formal relations.}
\pmc{The operation that produces `$q$' from `$p$' also produces `$r$' from `$q$', and so on. There is only one way of expressing this: `$p$', `$q$', `$r$', etc.\ have to be variables that give expression in a general way to certain formal relations.}

\pn{5.25}
\ger{Das Vorkommen der Operation charakterisiert den Sinn des Satzes nicht.}
\ogd{The occurrence of an operation does not characterize the sense of a proposition.}
\pmc{The occurrence of an operation does not characterize the sense of a proposition.}

\pnskip
\ger{Die Operation sagt ja nichts aus, nur ihr Resultat, und dies h{\"a}ngt von den Basen der Operation ab.}
\ogd{For an operation does not assert anything; only its result does, and this depends on the bases of the operation.}
\pmc{Indeed, no statement is made by an operation, but only by its result, and this depends on the bases of the operation.}

\pnskip
\ger{(Operation und Funktion d{\"u}rfen nicht miteinander verwechselt werden.)}
\ogd{(Operation and function must not be confused with one another.)}
\pmc{(Operations and functions must not be confused with each other.)}

\pn{5.251}
\ger{Eine Funktion kann nicht ihr eigenes Argument sein, wohl aber kann das Resultat einer Operation ihre eigene Basis werden.}
\ogd{A function cannot be its own argument, but the result of an operation can be its own basis.}
\pmc{A function cannot be its own argument, whereas an operation can take one of its own results as its base.}

\pn{5.252}
\ger{Nur so ist das Fortschreiten von Glied zu Glied in einer Formenreihe (von Type zu Type in den Hierarchien Russells und Whiteheads) m{\"o}glich. (Russell und Whitehead haben die M{\"o}glichkeit dieses Fortschreitens nicht zugegeben, aber immer wieder von ihr Gebrauch gemacht.)}
\ogd{Only in this way is the progress from term to term in a formal series possible (from type to type in the hierarchy of Russell and Whitehead). (Russell and Whitehead have not admitted the possibility of this progress but have made use of it all the same.)}
\pmc{It is only in this way that the step from one term of a series of forms to another is possible (from one type to another in the hierarchies of Russell and Whitehead). (Russell and Whitehead did not admit the possibility of such steps, but repeatedly availed themselves of it.)}

\pn{5.2521}
\ger{Die fortgesetzte Anwendung einer Operation auf ihr eigenes Resultat nenne ich ihre successive Anwendung (\gdql $\Op\Op\Op a$\gdqr{} ist das Resultat der dreimaligen successiven Anwendung von \gdql $\Op \xi$\gdqr{} auf \gdql $a$\gdqr{}).}
\ogd{The repeated application of an operation to its own result I call its successive application (``$\Op\Op\Op a$'' is the result of the threefold successive application of ``$\Op \xi$'' to ``$a$'').}
\pmc{If an operation is applied repeatedly to its own results, I speak of successive applications of it. (`$\Op\Op\Op a$' is the result of three successive applications of the operation `$\Op \xi$' to `$a$'.)}

\pnskip
\ger{In einem {\"a}hnlichen Sinne rede ich von der successiven Anwendung \germph{mehrerer} Operationen auf eine Anzahl von S{\"a}tzen.}
\ogd{In a similar sense I speak of the successive application of \emph{several} operations to a number of propositions.}
\pmc{In a similar sense I speak of successive applications of \emph{more than one} operation to a number of propositions.}

\pn{5.2522}
\ger{Das allgemeine Glied einer Formenreihe $a,\thickspace \Op a,\thickspace \Op \Op a,\thickspace \dotsc$ schreibe ich daher so: \gdql $[a,\thickspace x,\thickspace \Op x]$\gdqr{}. Dieser Klammerausdruck ist eine Variable. Das erste Glied des Klammerausdruckes ist der Anfang der Formenreihe, das zweite die Form eines beliebigen Gliedes $x$ der Reihe und das dritte die Form desjenigen Gliedes der Reihe, welches auf $x$ unmittelbar folgt.}
\ogd{The general term of the formal series $a,\thickspace \Op a,\thickspace \Op \Op a,\thickspace \dotsc$. I write thus: ``$[a,\thickspace x,\thickspace \Op x]$''. This expression in brackets is a variable. The first term of the expression is the beginning of the formal series, the second the form of an arbitrary term $x$ of the series, and the third the form of that term of the series which immediately follows $x$.}
\pmc{Accordingly I use the sign `$[a,\thickspace x,\thickspace \Op x]$' for the general term of the series of forms $a,\thickspace \Op a,\thickspace \Op \Op a,\thickspace \dotsc$. This bracketed expression is a variable: the first term of the bracketed expression is the beginning of the series of forms, the second is the form of a term $x$ arbitrarily selected from the series, and the third is the form of the term that immediately follows $x$ in the series.}

\pn{5.2523}
\ger{Der Begriff der successiven Anwendung der Operation ist {\"a}quivalent mit dem Begriff \gdql und so weiter\gdqr{}.}
\ogd{The concept of the successive application of an operation is equivalent to the concept ``and so on''.}
\pmc{The concept of successive applications of an operation is equivalent to the concept `and so on'.}

\pn{5.253}
\ger{Eine Operation kann die Wirkung einer anderen r{\"u}ckg{\"a}ngig machen. Operationen k{\"o}nnen einander aufheben.}
\ogd{One operation can reverse the effect of another. Operations can cancel one another.}
\pmc{One operation can counteract the effect of another. Operations can cancel one another.}

\pn{5.254}
\ger{Die Operation kann verschwinden (z.\ B.\ die Verneinung in \gdql $\rnot \rnot p$\gdqr{}: $\rnot\rnot p=p$).}
\ogd{Operations can vanish (\emph{e.g.}\ denial in ``$\rnot \rnot p$''. $\rnot \rnot p = p$).}
\pmc{An operation can vanish (e.g.\ negation in `$\rnot \rnot p$': $\rnot \rnot p = p$).}

\pn{5.3}
\ger{Alle S{\"a}tze sind Resultate von Wahrheitsoperationen mit den Elementars{\"a}tzen.}
\ogd{All propositions are results of truth-operations on the elementary propositions.}
\pmc{All propositions are results of truth-operations on elementary propositions.}

\pnskip
\ger{Die Wahrheitsoperation ist die Art und Weise, wie aus den Elementars{\"a}tzen die Wahrheitsfunktion entsteht.}
\ogd{The truth-operation is the way in which a truth-function arises from elementary propositions.}
\pmc{A truth-operation is the way in which a truth-function is produced out of elementary propositions.}

\pnskip
\ger{Nach dem Wesen der Wahrheitsoperation wird auf die gleiche Weise, wie aus den Elementars{\"a}tzen ihre Wahrheitsfunktion, aus Wahrheitsfunktionen eine neue. Jede Wahrheitsoperation erzeugt aus Wahrheitsfunktionen von Elementars{\"a}tzen wieder eine Wahrheitsfunktion von Elementars{\"a}tzen, einen Satz. Das Resultat jeder Wahrheitsoperation mit den Resultaten von Wahrheitsoperationen mit Elementars{\"a}tzen ist wieder das Resultat \germph{Einer} Wahrheitsoperation mit Elementars{\"a}tzen.}
\ogd{According to the nature of truth-operations, in the same way as out of elementary propositions arise their truth-functions, from truth-functions arises a new one. Every truth-operation creates from truth-functions of elementary propositions, another truth-function of elementary propositions \emph{i.e.}\ a proposition. The result of every truth-operation on the results of truth-operations on elementary propositions is also the result of \emph{one} truth-operation on elementary propositions.}
\pmc{It is of the essence of truth-operations that, just as elementary propositions yield a truth-function of themselves, so too in the same way truth-functions yield a further truth-function. When a truth-operation is applied to truth-functions of elementary propositions, it always generates another truth-function of elementary propositions, another proposition. When a truth-operation is applied to the results of truth-operations on elementary propositions, there is always a \emph{single} operation on elementary propositions that has the same result.}

\pnskip
\ger{Jeder Satz ist das Resultat von Wahrheitsoperationen mit Elementars{\"a}tzen.}
\ogd{Every proposition is the result of truth-operations on elementary propositions.}
\pmc{Every proposition is the result of truth-operations on elementary propositions.}

\pn{5.31}
\ger{Die Schemata No. 4.31 haben auch dann eine Bedeutung, wenn \gdql $p$\gdqr{}, \gdql $q$\gdqr{}, \gdql $r$\gdqr{}, etc.\ nicht Elementars{\"a}tze sind.}
\ogd{The Schemata No.\ 4.31 are also significant, if ``$p$'', ``$q$'', ``$r$'', etc.\ are not elementary propositions.}
\pmc{The schemata in 4.31 have a meaning even when `$p$', `$q$', `$r$', etc. are not elementary propositions.}

\pnskip
\ger{Und es ist leicht zu sehen, dass das Satzzeichen in No.\ 4.442, auch wenn \gdql $p$\gdqr{} und \gdql $q$\gdqr{} Wahrheitsfunktionen von Elementars{\"a}tzen sind, Eine Wahrheitsfunktion von Elementars{\"a}tzen ausdr{\"u}ckt.}
\ogd{And it is easy to see that the propositional sign in No.\ 4.442 expresses one truth-function of elementary propositions even when ``$p$'' and ``$q$'' are truth-functions of elementary propositions.}
\pmc{And it is easy to see that the propositional sign in 4.442 expresses a single truth-function of elementary propositions even when `$p$' and `$q$' are truth-functions of elementary propositions.}

\pn{5.32}
\ger{Alle Wahrheitsfunktionen sind Resultate der successiven Anwendung einer endlichen Anzahl von Wahrheitsoperationen auf die Elementars{\"a}tze.}
\ogd{All truth-functions are results of the successive application of a finite number of truth-operations to elementary propositions.}
\pmc{All truth-functions are results of successive applications to elementary propositions of a finite number of truth-operations.}

\pn{5.4}
\ger{Hier zeigt es sich, dass es \gdql logische Gegenst{\"a}nde\gdqr{}, \gdql logische Konstante\gdqr{} (im Sinne Freges und Russells) nicht gibt.}
\ogd{Here it becomes clear that there are no such things as ``logical objects'' or ``logical constants'' (in the sense of Frege and Russell).}
\pmc{At this point it becomes manifest that there are no `logical objects' or `logical constants' (in Frege's and Russell's sense).}

\pn{5.41}
\ger{Denn: Alle Resultate von Wahrheitsoperationen mit Wahrheitsfunktionen sind identisch, welche eine und dieselbe Wahrheitsfunktion von Elementars{\"a}tzen sind.}
\ogd{For all those re\-sults of truth-op\-er\-a\-tions on truth-func\-tions are iden\-ti\-cal, which are one and the same truth-func\-tion of el\-em\-en\-tary prop\-o\-si\-tions.}
\pmc{The reason is that the results of truth-operations on truth-functions are always identical whenever they are one and the same truth-function of elementary propositions.}

\pn{5.42}
\ger{Dass $\lor$, $\rimplies$, etc.\ nicht Beziehungen im Sinne von rechts und links etc.\ sind, leuchtet ein.}
\ogd{That $\lor$, $\rimplies$, etc., are not relations in the sense of right and left, etc., is obvious.}
\pmc{It is self-evident that $\lor$, $\rimplies$, etc.\ are not relations in the sense in which right and left etc.\ are relations.}

\pnskip
\ger{Die M{\"o}glichkeit des kreuzweisen Definierens der logischen \gdql Urzeichen\gdqr{} Freges und Russells zeigt schon, dass diese keine Urzeichen sind, und schon erst recht, dass sie keine Relationen bezeichnen.}
\ogd{The possibility of crosswise definition of the logical ``primitive signs'' of Frege and Russell shows by itself that these are not primitive signs and that they signify no relations.}
\pmc{The interdefinability of Frege's and Russell's `primitive signs' of logic is enough to show that they are not primitive signs, still less signs for relations.}

\pnskip
\ger{Und es ist offenbar, dass das \gdql $\rimplies$\gdqr{}, welches wir durch \gdql $\rnot$\gdqr{} und \gdql $\lor$\gdqr{} definieren, identisch ist mit dem, durch welches wir \gdql $\lor$\gdqr{} mit \gdql $\rnot$\gdqr{} definieren, und dass dieses \gdql $\lor$\gdqr{} mit dem ersten identisch ist. U.\ s.\ w.}
\ogd{And it is obvious that the ``$\rimplies$'' which we define by means of ``$\rnot$'' and ``$\lor$'' is identical with that by which we define ``$\lor$'' with the help of ``$\rnot$'', and that this ``$\lor$'' is the same as the first, and so on.}
\pmc{And it is obvious that the `$\rimplies$' defined by means of `$\rnot$' and `$\lor$' is identical with the one that figures with `$\rnot$' in the definition of `$\lor$'; and that the second `$\lor$' is identical with the first one; and so on.}

\pn{5.43}
\ger{Dass aus einer Tatsache $p$ unendlich viele \germph{andere} folgen sollten, n{\"a}mlich $\rnot \rnot p$, $\rnot \rnot \rnot \rnot p$, etc., ist doch von vornherein kaum zu glauben. Und nicht weniger merkw{\"u}rdig ist, dass die unendliche Anzahl der S{\"a}tze der Logik (der Mathematik) aus einem halben Dutzend \gdql Grundgesetzen\gdqr{} folgen.}
\ogd{That from a fact $p$ an infinite number of \emph{others} should follow, namely, $\rnot \rnot p$, $\rnot \rnot \rnot \rnot p$, etc., is indeed hardly to be believed, and it is no less wonderful that the infinite number of propositions of logic (of mathematics) should follow from half a dozen ``primitive propositions''.}
\pmc{Even at first sight it seems scarcely credible that there should follow from one fact $p$ infinitely many \emph{others}, namely $\rnot \rnot p$, $\rnot \rnot \rnot \rnot p$, etc. And it is no less remarkable that the infinite number of propositions of logic (mathematics) follow from half a dozen `primitive propositions'.}

\pnskip
\ger{Alle S{\"a}tze der Logik sagen aber dasselbe. N{\"a}mlich nichts.}
\ogd{But the propositions of logic say the same thing. That is, nothing.}
\pmc{But in fact all the propositions of logic say the same thing, to wit nothing.}

\pn{5.44}
\ger{Die Wahrheitsfunktionen sind keine materiellen Funktionen.}
\ogd{Truth-functions are not material functions.}
\pmc{Truth-functions are not material functions.}

\pnskip
\ger{Wenn man z.\ B.\ eine Bejahung durch doppelte Verneinung erzeugen kann, ist dann die Verneinung---in irgend einem Sinn---in der Bejahung enthalten? Verneint \gdql $\rnot \rnot p$\gdqr{} $\rnot p$, oder bejaht es $p$; oder beides?}
\ogd{If \emph{e.g.}\ an affirmation can be produced by repeated denial, is the denial---in any sense---contained in the affirmation? Does ``$\rnot \rnot p$'' deny $\rnot p$, or does it affirm $p$; or both?}
\pmc{For example, an affirmation can be produced by double negation: in such a case does it follow that in some sense negation is contained in affirmation? Does `$\rnot \rnot p$' negate $\rnot p$, or does it affirm $p$---or both?}

\pnskip
\ger{Der Satz \gdql $\rnot \rnot p$\gdqr{} handelt nicht von der Verneinung wie von einem Gegenstand; wohl aber ist die M{\"o}glichkeit der Verneinung in der Bejahung bereits pr{\"a}judiziert.}
\ogd{The proposition ``$\rnot \rnot p$'' does not treat of denial as an object, but the possibility of denial is already prejudged in affirmation.}
\pmc{The proposition `$\rnot \rnot p$' is not about negation, as if negation were an object: on the other hand, the possibility of negation is already written into affirmation.}

\pnskip
\ger{Und g{\"a}be es einen Gegenstand, der \gdql $\rnot$\gdqr{} hie{\ss}e, so m{\"u}sste \gdql $\rnot \rnot p$\gdqr{} etwas anderes sagen als \gdql $p$\gdqr{}. Denn der eine Satz w{\"u}rde dann eben von $\rnot$ handeln, der andere nicht.}
\ogd{And if there was an object called ``$\rnot$'', then ``$\rnot \rnot p$'' would have to say something other than ``$p$''. For the one proposition would then treat of $\rnot$, the other would not.}
\pmc{And if there were an object called `$\rnot$', it would follow that `$\rnot \rnot p$' said something different from what `$p$' said, just because the one proposition would then be about $\rnot$ and the other would not.}

\pn{5.441}
\ger{Dieses Verschwinden der scheinbaren logischen Konstanten tritt auch ein, wenn \gdql $\rnot \rsomed{x} \rnot f\negthinspace x$\gdqr{} dasselbe sagt wie \gdql $\ralld{x} f\negthinspace x$\gdqr{}, oder \gdql $\rsomed{x} f\negthinspace x \rand x=a$\gdqr{} dasselbe wie \gdql $f\negthinspace a$\gdqr.}
\ogd{This disappearance of the apparent logical constants also occurs if ``$\rnot \rsomed{x} \rnot f\negthinspace x$'' says the same as ``$\ralld{x} f\negthinspace x$'', or ``$\rsomed{x} f\negthinspace x \rand x=a$'' the same as ``$f\negthinspace a$''.}
\pmc{This vanishing of the apparent logical constants also occurs in the case of `$\rnot \rsomed{x} \rnot f\negthinspace x$', which says the same as `$\ralld{x} f\negthinspace x$', and in the case of `$\rsomed{x} f\negthinspace x \rand x=a$', which says the same as `$f\negthinspace a$'.}

\pn{5.442}
\ger{Wenn uns ein Satz gegeben ist, so sind \germph{mit ihm} auch schon die Resultate aller Wahrheitsoperationen, die ihn zur Basis haben, gegeben.}
\ogd{If a proposition is given to us then the results of all truth-operations which have it as their basis are given \emph{with} it.}
\pmc{If we are given a proposition, then \emph{with it} we are also given the results of all truth-operations that have it as their base.}

\pn{5.45}
\ger{Gibt es logische Urzeichen, so muss eine richtige Logik ihre Stellung zueinander klar machen und ihr Dasein rechtfertigen. Der Bau der Logik \germph{aus} ihren Urzeichen muss klar werden.}
\ogd{If there are logical primitive signs a correct logic must make clear their position relative to one another and justify their existence. The construction of logic \emph{out of} its primitive signs must become clear.}
\pmc{If there are primitive logical signs, then any logic that fails to show clearly how they are placed relatively to one another and to justify their existence will be incorrect. The construction of logic \emph{out of} its primitive signs must be made clear.}

\pn{5.451}
\ger{Hat die Logik Grundbegriffe, so m{\"u}ssen sie von einander unabh{\"a}ngig sein. Ist ein Grundbegriff eingef{\"u}hrt, so muss er in allen Verbindungen eingef{\"u}hrt sein, worin er {\"u}berhaupt vorkommt. Man kann ihn also nicht zuerst f{\"u}r \germph{eine} Verbindung, dann noch einmal f{\"u}r eine andere einf{\"u}hren. Z.\ B.:\ Ist die Verneinung eingef{\"u}hrt, so m{\"u}ssen wir sie jetzt in S{\"a}tzen von der Form \gdql $\rnot p$\gdqr{} ebenso verstehen, wie in S{\"a}tzen wie \gdql $\rnot (p \lor q)$\gdqr{}, \gdql $\rsomed{x} \rnot f\negthinspace x$\gdqr{} u.~a. Wir d{\"u}rfen sie nicht erst f{\"u}r die eine Klasse von F{\"a}llen, dann f{\"u}r die andere einf{\"u}hren, denn es bliebe dann zweifelhaft, ob ihre Bedeutung in beiden F{\"a}llen die gleiche w{\"a}re und es w{\"a}re kein Grund vorhanden, in beiden F{\"a}llen dieselbe Art der Zeichenverbindung zu ben{\"u}tzen.}
\ogd{If logic has primitive ideas these must be independent of one another. If a primitive idea is introduced it must be introduced in all contexts in which it occurs at all. One cannot therefore introduce it for \emph{one} context and then again for another. For example, if denial is introduced, we must understand it in propositions of the form ``$\rnot p$'', just as in propositions like ``$\rnot (p \lor q)$'', ``$\rsomed{x} \rnot f\negthinspace x$'' and others. We may not first introduce it for one class of cases and then for another, for it would then remain doubtful whether its meaning in the two cases was the same, and there would be no reason to use the same way of symbolizing in the two cases.}
\pmc{If logic has primitive ideas, they must be independent of one another. If a primitive idea has been introduced, it must have been introduced in all the combinations in which it ever occurs. It cannot, therefore, be introduced first for \emph{one} combination and later reintroduced for another. For example, once negation has been introduced, we must understand it both in propositions of the form `$\rnot p$' and in propositions like `$\rnot (p \lor q)$', `$\rsomed{x} \rnot f\negthinspace x$', etc. We must not introduce it first for the one class of cases and then for the other, since it would then be left in doubt whether its meaning were the same in both cases, and no reason would have been given for combining the signs in the same way in both cases.}

\pnskip
\ger{(Kurz, f{\"u}r die Einf{\"u}hrung der Urzeichen gilt, mutatis mutandis, dasselbe, was Frege (\gdql Grundgesetze der Arithmetik\gdqr{}) f{\"u}r die Einf{\"u}hrung von Zeichen durch Definitionen gesagt hat.)}
\ogd{(In short, what Frege (``Grundgesetze der Arithmetik'') has said about the introduction of signs by definitions holds, mutatis mutandis, for the introduction of primitive signs also.)}
\pmc{(In short, Frege's remarks about introducing signs by means of definitions (in \textit{The Fundamental Laws of Arithmetic}) also apply, \emph{mutatis mutandis}, to the introduction of primitive signs.)}

\pn{5.452}
\ger{Die Einf{\"u}hrung eines neuen Behelfes in den Symbolismus der Logik muss immer ein folgenschweres Ereignis sein. Kein neuer Behelf darf in die Logik---sozusagen, mit ganz unschuldiger Miene---in Klammern oder unter dem Striche eingef{\"u}hrt werden.}
\ogd{The introduction of a new expedient in the symbolism of logic must always be an event full of consequences. No new symbol may be introduced in logic in brackets or in the margin---with, so to speak, an entirely innocent face.}
\pmc{The introduction of any new device into the symbolism of logic is necessarily a momentous event. In logic a new device should not be introduced in brackets or in a footnote with what one might call a completely innocent air.}

\pnskip
\ger{(So kommen in den \gdql Principia Mathematica\gdqr{} von Russell und Whitehead Definitionen und Grundgesetze in Worten vor. Warum hier pl{\"o}tzlich Worte? Dies bed{\"u}rfte einer Rechtfertigung. Sie fehlt und muss fehlen, da das Vorgehen tats{\"a}chlich unerlaubt ist.)}
\ogd{(Thus in the ``Principia Mathematica'' of Russell and Whitehead there occur definitions and primitive propositions in words. Why suddenly words here? This would need a justification. There was none, and can be none for the process is actually not allowed.)}
\pmc{(Thus in Russell and Whitehead's \textit{Principia Mathematica} there occur definitions and primitive propositions expressed in words. Why this sudden appearance of words? It would require a justification, but none is given, or could be given, since the procedure is in fact illicit.)}

\pnskip
\ger{Hat sich aber die Einf{\"u}hrung eines neuen Behelfes an einer Stelle als n{\"o}tig erwiesen, so muss man sich nun sofort fragen: Wo muss dieser Behelf nun \germph{immer} angewandt werden? Seine Stellung in der Logik muss nun erkl{\"a}rt werden.}
\ogd{But if the introduction of a new expedient has proved necessary in one place, we must immediately ask: Where is this expedient \emph{always} to be used? Its position in logic must be made clear.}
\pmc{But if the introduction of a new device has proved necessary at a certain point, we must immediately ask ourselves, `At what points is the employment of this device now \emph{unavoidable}?' and its place in logic must be made clear.}

\pn{5.453}
\ger{Alle Zahlen der Logik m{\"u}ssen sich rechtfertigen lassen.}
\ogd{All numbers in logic must be capable of justification.}
\pmc{All numbers in logic stand in need of justification.}

\pnskip
\ger{Oder vielmehr: Es muss sich herausstellen, dass es in der Logik keine Zahlen gibt.}
\ogd{Or rather it must become plain that there are no numbers in logic.}
\pmc{Or rather, it must become evident that there are no numbers in logic.}

\pnskip
\ger{Es gibt keine ausgezeichneten Zahlen.}
\ogd{There are no pre-eminent numbers.}
\pmc{There are no privileged numbers.}

\pn{5.454}
\ger{In der Logik gibt es kein Nebeneinander, kann es keine Klassifikation geben.}
\ogd{In logic there is no side by side, there can be no classification.}
\pmc{In logic there is no co-ordinate status, and there can be no classification.}

\pnskip
\ger{In der Logik kann es nicht Allgemeineres und Spezielleres geben.}
\ogd{In logic there cannot be a more general and a more special.}
\pmc{In logic there can be no distinction between the general and the specific.}

\pn{5.4541}
\ger{Die L{\"o}sungen der logischen Probleme m{\"u}ssen einfach sein, denn sie setzen den Standard der Einfachheit.}
\ogd{The solution of logical problems must be simple for they set the standard of simplicity.}
\pmc{The solutions of the problems of logic must be simple, since they set the standard of simplicity.}

\pnskip
\ger{Die Menschen haben immer geahnt, dass es ein Gebiet von Fragen geben m{\"u}sse, deren Antworten---a priori---symmetrisch, und zu einem abgeschlossenen, regelm{\"a}{\ss}igen Gebilde vereint liegen.}
\ogd{Men have always thought that there must be a sphere of questions whose answers---a priori---are symmetrical and united into a closed regular structure.}
\pmc{Men have always had a presentiment that there must be a realm in which the answers to questions are symmetrically combined---\emph{a priori}---to form a self-contained system.}

\pnskip
\ger{Ein Gebiet, in dem der Satz gilt: simplex sigillum veri.}
\ogd{A sphere in which the proposition, simplex sigillum veri, is valid.}
\pmc{A realm subject to the law: \emph{Simplex sigillum veri}.}

\pn{5.46}
\ger{Wenn man die logischen Zeichen richtig einf{\"u}hrte, so h{\"a}tte man damit auch schon den Sinn aller ihrer Kombinationen eingef{\"u}hrt; also nicht nur \gdql $p \lor q$\gdqr{} sondern auch schon \gdql $\rnot (p \lor \rnot q)$\gdqr{} etc.\ etc. Man h{\"a}tte damit auch schon die Wirkung aller nur m{\"o}glichen Kombinationen von Klammern eingef{\"u}hrt. Und damit w{\"a}re es klar geworden, dass die eigentlichen allgemeinen Urzeichen nicht die \gdql $p \lor q $\gdqr{}, \gdql $\rsomed{x} f\negthinspace x$\gdqr{}, etc.\ sind, sondern die allgemeinste Form ihrer Kombinationen.}
\ogd{When we have rightly introduced the logical signs, the sense of all their combinations has been already introduced with them: therefore not only ``$p \lor q$'' but also ``$\rnot (p \lor \rnot q)$'', etc.\ etc. We should then already have introduced the effect of all possible combinations of brackets; and it would then have become clear that the proper general primitive signs are not ``$p \lor q$'', ``$\rsomed{x} f\negthinspace x$'', etc., but the most general form of their combinations.}
\pmc{If we introduced logical signs properly, then we should also have introduced at the same time the sense of all combinations of them; i.e. not only `$p \lor q$' but `$\rnot (p \lor \rnot q)$' as well, etc.\ etc. We should also have introduced at the same time the effect of all possible combinations of brackets. And thus it would have been made clear that the real general primitive signs are not `$p \lor q$', `$\rsomed{x} f\negthinspace x$', etc.\ but the most general form of their combinations.}

\pn{5.461}
\ger{Bedeutungsvoll ist die scheinbar unwichtige Tatsache, dass die logischen Scheinbeziehungen, wie $\lor$ und $\rimplies$, der Klammern bed{\"u}rfen---im Gegensatz zu den wirklichen Beziehungen.}
\ogd{The apparently unimportant fact that the apparent relations like $\lor$ and $\rimplies$ need brackets---unlike real relations---is of great importance.}
\pmc{Though it seems unimportant, it is in fact significant that the pseudo-relations of logic, such as $\lor$ and $\rimplies$, need brackets---unlike real relations.}

\pnskip
\ger{Die Ben{\"u}tzung der Klammern mit jenen scheinbaren Urzeichen deutet ja schon darauf hin, dass diese nicht die wirklichen Urzeichen sind. Und es wird doch wohl niemand glauben, dass die Klammern eine selbst{\"a}ndige Bedeutung haben.}
\ogd{The use of brackets with these apparent primitive signs shows that these are not the real primitive signs; and nobody of course would believe that the brackets have meaning by themselves.}
\pmc{Indeed, the use of brackets with these apparently primitive signs is itself an indication that they are not primitive signs. And surely no one is going to believe brackets have an independent meaning.}

\pn{5.4611}
\ger{Die logischen Operationszeichen sind Interpunktionen.}
\ogd{Logical operation signs are punctuations.}
\pmc{Signs for logical operations are punctuation-marks.}

\pn{5.47}
\ger{Es ist klar, dass alles, was sich {\"u}berhaupt \germph{von vornherein} {\"u}ber die Form aller S{\"a}tze sagen l{\"a}sst, sich \germph{auf einmal} sagen lassen muss.}
\ogd{It is clear that everything which can be said \emph{beforehand} about the form of \emph{all} propositions at all can be said \emph{on one occasion}.}
\pmc{It is clear that whatever we can say \emph{in advance} about the form of all propositions, we must be able to say \emph{all at once}.}

\pnskip
\ger{Sind ja schon im Elementarsatze alle logischen Operationen enthalten. Denn \gdql $f\negthinspace a$\gdqr{} sagt dasselbe wie}
\ogd{For all logical operations are already contained in the elementary proposition. For ``$f\negthinspace a$'' says the same as}
\pmc{An elementary proposition really contains all logical operations in itself. For `$f\negthinspace a$' says the same thing as}

\pnskip
\ger{\[\text{\gdql}\rsomed{x} f\negthinspace x \rand x=a\text{\gdqr}.\]}
\ogd{\[``\rsomed{x} f\negthinspace x \rand x=a\text{''}.\]}
\pmc{\[`\rsomed{x} f\negthinspace x \rand x=a\text{'}.\]}

\pnskip
\ger{Wo Zusammengesetztheit ist, da ist Argument und Funktion, und wo diese sind, sind bereits alle logischen Konstanten.}
\ogd{Where there is composition, there is argument and function, and where these are, all logical constants already are.}
\pmc{Wherever there is compositeness, argument and function are present, and where these are present, we already have all the logical constants.}

\pnskip
\ger{Man k{\"o}nnte sagen: Die Eine logische Konstante ist das, was \germph{alle} S{\"a}tze, ihrer Natur nach, mit einander gemein haben.}
\ogd{One could say: the one logical constant is that which \emph{all} propositions, according to their nature, have in common with one another.}
\pmc{One could say that the sole logical constant was what \emph{all} propositions, by their very nature, had in common with one another.}

\pnskip
\ger{Das aber ist die allgemeine Satzform.}
\ogd{That however is the general form of proposition.}
\pmc{But that is the general propositional form.}

\pn{5.471}
\ger{Die allgemeine Satzform ist das Wesen des Satzes.}
\ogd{The general form of proposition is the essence of proposition.}
\pmc{The general propositional form is the essence of a proposition.}

\pn{5.4711}
\ger{Das Wesen des Satzes angeben, hei{\ss}t, das Wesen aller Beschreibung angeben, also das Wesen der Welt.}
\ogd{To give the essence of proposition means to give the essence of all description, therefore the essence of the world.}
\pmc{To give the essence of a proposition means to give the essence of all description, and thus the essence of the world.}

\pn{5.472}
\ger{Die Beschreibung der allgemeinsten Satzform ist die Beschreibung des einen und einzigen allgemeinen Urzeichens der Logik.}
\ogd{The description of the most general propositional form is the description of the one and only general primitive sign in logic.}
\pmc{The description of the most general propositional form is the description of the one and only general primitive sign in logic.}

\pn{5.473}
\ger{Die Logik muss f{\"u}r sich selber sorgen.}
\ogd{Logic must take care of itself.}
\pmc{Logic must look after itself.}

\pnskip
\ger{Ein \germph{m{\"o}gliches} Zeichen muss auch bezeichnen k{\"o}nnen. Alles was in der Logik m{\"o}glich ist, ist auch erlaubt. (\gdql Sokrates ist identisch\gdqr{} hei{\ss}t darum nichts, weil es keine Eigenschaft gibt, die \gdql identisch\gdqr{} hei{\ss}t. Der Satz ist unsinnig, weil wir eine willk{\"u}rliche Bestimmung nicht getroffen haben, aber nicht darum, weil das Symbol an und f{\"u}r sich unerlaubt w{\"a}re.)}
\ogd{A \emph{possible} sign must also be able to signify. Everything which is possible in logic is also permitted. (``Socrates is identical'' means nothing because there is no property which is called ``identical''. The proposition is senseless because we have not made some arbitrary determination, not because the symbol is in itself unpermissible.)}
\pmc{If a sign is \emph{possible}, then it is also capable of signifying. Whatever is possible in logic is also permitted. (The reason why `Socrates is identical' means nothing is that there is no property called `identical'. The proposition is nonsensical because we have failed to make an arbitrary determination, and not because the symbol, in itself, would be illegitimate.)}

\pnskip
\ger{Wir k{\"o}nnen uns, in gewissem Sinne, nicht in der Logik irren.}
\ogd{In a certain sense we cannot make mistakes in logic.}
\pmc{In a certain sense, we cannot make mistakes in logic.}

\pn{5.4731}
\ger{Das Einleuchten, von dem Russell so viel sprach, kann nur dadurch in der Logik entbehrlich werden, dass die Sprache selbst jeden logischen Fehler verhindert.---Dass die Logik a priori ist, besteht darin, dass nicht unlogisch gedacht werden \germph{kann}.}
\ogd{Self-evidence, of which Russell has said so much, can only be discard in logic by language itself preventing every logical mistake. That logic is a priori consists in the fact that we \emph{cannot} think illogically.}
\pmc{Self-evidence, which Russell talked about so much, can become dispensable in logic, only because language itself prevents every logical mistake.---What makes logic \emph{a priori} is the \emph{impossibility} of illogical thought.}

\pn{5.4732}
\ger{Wir k{\"o}nnen einem Zeichen nicht den unrechten Sinn geben.}
\ogd{We cannot give a sign the wrong sense. }
\pmc{We cannot give a sign the wrong sense.}

\pn{5.47321}
\ger{Occams Devise ist nat{\"u}rlich keine willk{\"u}rliche, oder durch ihren praktischen Erfolg gerechtfertigte Regel: Sie besagt, dass \germph{unn{\"o}tige} Zeicheneinheiten nichts bedeuten.}
\ogd{Occam's razor is, of course, not an arbitrary rule nor one justified by its practical success. It simply says that \emph{unnecessary} elements in a symbolism mean nothing.}
\pmc{Occam's maxim is, of course, not an arbitrary rule, nor one that is justified by its success in practice: its point is that \emph{unnecessary} units in a sign-language mean nothing.}

\pnskip
\ger{Zeichen, die \germph{Einen} Zweck erf{\"u}llen, sind logisch {\"a}quivalent, Zeichen, die \germph{keinen} Zweck erf{\"u}llen, logisch bedeutungslos.}
\ogd{Signs which serve \emph{one} purpose are logically equivalent, signs
which serve \emph{no} purpose are logically meaningless.}
\pmc{Signs that serve \emph{one} purpose are logically equivalent, and signs that serve \emph{none} are logically meaningless.}

\pn{5.4733}
\ger{Frege sagt: Jeder rechtm{\"a}{\ss}ig gebildete Satz muss einen Sinn haben; und ich sage: Jeder m{\"o}gliche Satz ist rechtm{\"a}{\ss}ig gebildet, und wenn er keinen Sinn hat, so kann das nur daran liegen, dass wir einigen seiner Bestandteile keine \germph{Bedeutung} gegeben haben.}
\ogd{Frege says: Every legitimately constructed proposition must have a sense; and I say: Every possible proposition is legitimately constructed, and if it has no sense this can only be because we have given no \emph{meaning} to some of its constituent parts.}
\pmc{Frege says that any legitimately constructed proposition must have a sense. And I say that any possible proposition is legitimately constructed, and, if it has no sense, that can only be because we have failed to give a \emph{meaning} to some of its constituents.}

\pnskip
\ger{(Wenn wir auch glauben, es getan zu haben.)}
\ogd{(Even if we believe that we have done so.)}
\pmc{(Even if we think that we have done so.)}

\pnskip
\ger{So sagt \gdql Sokrates ist identisch\gdqr{} darum nichts, weil wir dem Wort \gdql identisch\gdqr{} als \germph{Eigenschaftswort} \germph{keine} Bedeutung gegeben haben. Denn, wenn es als Gleichheitszeichen auftritt, so symbolisiert es auf ganz andere Art und Weise---die bezeichnende Beziehung ist eine andere,---also ist auch das Symbol in beiden F{\"a}llen ganz verschieden; die beiden Symbole haben nur das Zeichen zuf{\"a}llig miteinander gemein.}
\ogd{Thus ``Socrates is identical'' says nothing, because we have given \emph{no} meaning to the word ``identical'' as \emph{adjective}. For when it occurs as the sign of equality it symbolizes in an entirely different way---the symbolizing relation is another---therefore the symbol is in the two cases entirely different; the two symbols have the sign in common with one another only by accident.}
\pmc{Thus the reason why `Socrates is identical' says nothing is that we have \emph{not} given any \emph{adjectival} meaning to the word `identical'. For when it appears as a sign for identity, it symbolizes in an entirely different way---the signifying relation is a different one---therefore the symbols also are entirely different in the two cases: the two symbols have only the sign in common, and that is an accident.}

\pn{5.474}
\ger{Die Anzahl der n{\"o}tigen Grundoperationen h{\"a}ngt \germph{nur} von unserer Notation ab.}
\ogd{The number of necessary fundamental operations depends \emph{only} on our notation.}
\pmc{The number of fundamental operations that are necessary depends \emph{solely} on our notation.}

\pn{5.475}
\ger{Es kommt nur darauf an, ein Zeichensystem von einer bestimmten Anzahl von Dimensionen---von einer bestimmten mathematischen Mannigfaltigkeit---zu bilden.}
\ogd{It is only a question of constructing a system of signs of a definite number of dimensions---of a definite mathematical multiplicity.}
\pmc{All that is required is that we should construct a system of signs with a particular number of dimensions---with a particular mathematical multiplicity.}

\pn{5.476}
\ger{Es ist klar, dass es sich hier nicht um eine \germph{Anzahl von Grundbegriffen} handelt, die bezeichnet werden m{\"u}ssen, sondern um den Ausdruck einer Regel.}
\ogd{It is clear that we are not concerned here with a \emph{number of primitive ideas} which must be signified but with the expression of a rule.}
\pmc{It is clear that this is not a question of a \emph{number of primitive ideas} that have to be signified, but rather of the expression of a rule.}

\pn{5.5}
\ger{Jede Wahrheitsfunktion ist ein Resultat der successiven Anwendung der Operation $\mathop{(-----\mathrm{W})}\medspace (\xi,~.~.~.~.~.)$ auf Elementars{\"a}tze.}
\ogd{Every truth-function is a result of the successive application of the operation $\mathop{(-----\mathrm{T})}\medspace (\xi,~.~.~.~.~.)$ to elementary propositions.}
\pmc{Every truth-function is a result of successive applications to elementary prop\-o\-si\-tions of the op\-er\-a\-tion `$\mathop{(-----\mathrm{T})}$ $(\xi,~.~.~.~.~.)$'.}

\pnskip
\ger{Diese Operation verneint s{\"a}mtliche S{\"a}tze in der rechten Klammer, und ich nenne sie die Negation dieser S{\"a}tze.}
\ogd{This operation denies all the propositions in the right-hand bracket and I call it the negation of these propositions.}
\pmc{This operation negates all the propositions in the right-hand pair of brackets, and I call it the negation of those propositions.}

\pn{5.501}
\ger{Einen Klammerausdruck, dessen Glieder S{\"a}tze sind, deute ich---wenn die Reihenfolge der Glieder in der Klammer gleichg{\"u}ltig ist---durch ein Zeichen von der Form \gdql $(\overline{\xi})$\gdqr{} an. \gdql $\xi$\gdqr{} ist eine Variable, deren Werte die Glieder des Klammerausdruckes sind; und der Strich {\"u}ber der Variablen deutet an, dass sie ihre s{\"a}mtlichen Werte in der Klammer vertritt.}
\ogd{An expression in brackets whose terms are propositions I indicate---if the order of the terms in the bracket is indifferent---by a sign of the form ``$(\overline{\xi})$''. ``$\xi$'' is a variable whose values are the terms of the expression in brackets, and the line over the variable indicates that it stands for all its values in the bracket.}
\pmc{When a bracketed expression has propositions as its terms---and the order of the terms inside the brackets is indifferent---then I indicate it by a sign of the form `$(\overline{\xi})$'. `$\xi$' is a variable whose values are terms of the bracketed expression and the bar over the variable indicates that it is the representative of all its values in the brackets.}

\pnskip
\ger{(Hat also $\xi$ etwa die 3 Werte P, Q, R, so ist $(\overline{\xi})= ($P, Q, R).)}
\ogd{(Thus if $\xi$ has the 3 values P, Q, R, then $(\overline{\xi})= ($P, Q, R).)}
\pmc{(E.g. if $\xi$ has the three values P, Q, R, then $(\overline{\xi})= ($P, Q, R). )}

\pnskip
\ger{Die Werte der Variablen werden festgesetzt.}
\ogd{The values of the variables must be determined.}
\pmc{What the values of the variable are is something that is stipulated.}

\pnskip
\ger{Die Festsetzung ist die Beschreibung der S{\"a}tze, welche die Variable vertritt.}
\ogd{The determination is the description of the propositions which the variable stands for.}
\pmc{The stipulation is a description of the propositions that have the variable as their representative. }

\pnskip
\ger{Wie die Beschreibung der Glieder des Klammerausdruckes geschieht, ist unwesentlich.}
\ogd{How the description of the terms of the expression in brackets takes place is unessential.}
\pmc{How the description of the terms of the bracketed expression is produced is not essential.}

\pnskip
\ger{Wir \germph{k{\"o}nnen} drei Arten der Beschreibung unterscheiden: 1.\ Die direkte Aufz{\"a}hlung. In diesem Fall k{\"o}nnen wir statt der Variablen einfach ihre konstanten Werte setzen. 2.\ Die Angabe einer Funktion $f\negthinspace x$, deren Werte f{\"u}r alle Werte von $x$ die zu beschreibenden S{\"a}tze sind. 3.\ Die Angabe eines formalen Gesetzes, nach welchem jene S{\"a}tze gebildet sind. In diesem Falle sind die Glieder des Klammerausdrucks s{\"a}mtliche Glieder einer Formenreihe.}
\ogd{We may distinguish 3 kinds of description: 1.\ Direct enumeration. In this case we can place simply its constant values instead of the variable. 2.\ Giving a function $f\negthinspace x$, whose values for all values of $x$ are the propositions to be described. 3.\ Giving a formal law, according to which those propositions are constructed. In this case the terms of the expression in brackets are all the terms of a formal series.}
\pmc{We \emph{can} distinguish three kinds of description: 1.\ direct enumeration, in which case we can simply substitute for the variable the constants that are its values; 2.\ giving a function $f\negthinspace x$ whose values for all values of $x$ are the propositions to be described; 3.\ giving a formal law that governs the construction of the propositions, in which case the bracketed expression has as its members all the terms of a series of forms.}

\pn{5.502}
\ger{Ich schreibe also statt \gdql $\mathop{(-----\mathrm{W})}$ $(\xi,~.~.~.~.~.)$\gdqr{} \gdql $\nop(\overline{\xi})$\gdqr{}.}
\ogd{Therefore I write instead of ``$\mathop{(-----\mathrm{T})}$ $(\xi,~.~.~.~.~.)$'', ``$\nop(\overline{\xi})$''.}
\pmc{So instead of `$\mathop{(-----\mathrm{T})}$ $(\xi,~.~.~.~.~.)$', I write `$\nop(\overline{\xi})$'.}

\pnskip
\ger{$\nop(\overline{\xi})$ ist die Negation s{\"a}mtlicher Werte der Satzvariablen $\xi$.}
\ogd{$\nop(\overline{\xi})$ is the negation of all the values of the propositional variable $\xi$.}
\pmc{$\nop(\overline{\xi})$ is the negation of all the values of the propositional variable $\xi$.}

\pn{5.503}
\ger{Da sich offenbar leicht ausdr{\"u}cken l{\"a}{\ss}t, wie mit dieser Operation S{\"a}tze gebildet werden k{\"o}nnen und wie S{\"a}tze mit ihr nicht zu bilden sind, so muss dies auch einen exakten Ausdruck finden k{\"o}nnen.}
\ogd{As it is obviously easy to express how propositions can be constructed by means of this operation and how propositions are not to be constructed by means of it, this must be capable of exact expression.}
\pmc{It is obvious that we can easily express how propositions may be constructed with this operation, and how they may not be constructed with it; so it must be possible to find an exact expression for this.}

\pn{5.51}
\ger{Hat $\xi$ nur einen Wert, so ist $\nop(\overline{\xi})=\rnot p$ (nicht $p$), hat es zwei Werte, so ist $\nop (\overline{\xi}) = \rnot p \rand \rnot q$ (weder $p$ noch $q$).}
\ogd{If $\xi$ has only one value, then $\nop(\overline{\xi})=\rnot p$ (not $p$), if it has two values then $\nop (\overline{\xi}) = \rnot p \rand \rnot q$ (neither $p$ nor $q$).}
\pmc{If $\xi$ has only one value, then $\nop(\overline{\xi})=\rnot p$ (not $p$); if it has two values, then $\nop (\overline{\xi}) = \rnot p \rand \rnot q$ (neither $p$ nor $q$).}%???

\pn{5.511}
\ger{Wie kann die allumfassende, weltspiegelnde Logik so spezielle Haken und Manipulationen gebrauchen? Nur, indem sich alle diese zu einem unendlich feinen Netzwerk, zu dem gro{\ss}en Spiegel, verkn{\"u}pfen.}
\ogd{How can the all-embracing logic which mirrors the world use such special catches and manipulations? Only because all these are connected into an infinitely fine network, to the great mirror.}
\pmc{How can logic---all-embracing logic, which mirrors the world---use such peculiar crotchets and contrivances? Only because they are all connected with one another in an infinitely fine network, the great mirror.}

\pn{5.512}
\ger{\gdql $\rnot p$\gdqr{} ist wahr, wenn \gdql $p$\gdqr{} falsch ist. Also in dem wahren Satz \gdql $\rnot p$\gdqr{} ist \gdql $p$\gdqr{} ein falscher Satz. Wie kann ihn nun der Strich \gdql $\rnot$\gdqr{} mit der Wirklichkeit zum Stimmen bringen?}
\ogd{``$\rnot p$'' is true if ``$p$'' is false. Therefore in the true proposition ``$\rnot p$'' ``$p$'' is a false proposition. How then can the stroke ``$\rnot$'' bring it into agreement with reality?}
\pmc{`$\rnot p$' is true if `$p$' is false. Therefore, in the proposition `$\rnot p$', when it is true, `$p$' is a false proposition. How then can the stroke `$\rnot$' make it agree with reality?}

\pnskip
\ger{Das, was in \gdql $\rnot p$\gdqr{} verneint, ist aber nicht das \gdql $\rnot$\gdqr{}, sondern dasjenige, was allen Zeichen dieser Notation, welche $p$ verneinen, gemeinsam ist.}
\ogd{That which denies in ``$\rnot p$'' is however not ``$\rnot$'', but that which all signs of this notation, which deny $p$, have in common.}
\pmc{But in `$\rnot p$' it is not `$\rnot$' that negates, it is rather what is common to all the signs of this notation that negate $p$.}

\pnskip
\ger{Also die gemeinsame Regel, nach welcher \gdql $\rnot p$\gdqr{}, \gdql $\rnot \rnot \rnot p$\gdqr{}, \gdql $\rnot p \lor \rnot p$\gdqr{}, \gdql $\rnot p \rand \rnot p$\gdqr{}, etc.\ etc.\ (ad inf.)\ gebildet werden. Und dies Gemeinsame spiegelt die Verneinung wieder.}
\ogd{Hence the common rule according to which ``$\rnot p$'', ``$\rnot \rnot \rnot p$'', ``$\rnot p \lor \rnot p$'', ``$\rnot p \rand \rnot p$'', etc.\ etc.\ (to infinity) are constructed. And this which is common to them all mirrors denial.}
\pmc{That is to say the common rule that governs the construction of `$\rnot p$', `$\rnot \rnot \rnot p$', `$\rnot p \lor \rnot p$', `$\rnot p \rand \rnot p$', etc.\ etc.\ (ad inf.). And this common factor mirrors negation.}

\pn{5.513}
\ger{Man k{\"o}nnte sagen: Das Gemeinsame aller Symbole, die sowohl $p$ als $q$ bejahen, ist der Satz \gdql $p \rand q$\gdqr{}. Das Gemeinsame aller Symbole, die entweder $p$ oder $q$ bejahen, ist der Satz \gdql $p \lor q$\gdqr{}.}
\ogd{We could say: What is common to all symbols, which assert both $p$ and $q$, is the proposition ``$p \rand q$''. What is common to all symbols, which asserts either $p$ or $q$, is the proposition ``$p \lor q$''.}
\pmc{We might say that what is common to all symbols that affirm both $p$ and $q$ is the proposition `$p \rand q$'; and that what is common to all symbols that affirm either $p$ or $q$ is the proposition `$p \lor q$'.}

\pnskip
\ger{Und so kann man sagen: Zwei S{\"a}tze sind einander entgegengesetzt, wenn sie nichts miteinander gemein haben, und: Jeder Satz hat nur ein Negativ, weil es nur einen Satz gibt, der ganz au{\ss}erhalb seiner liegt.}
\ogd{And similarly we can say: Two propositions are opposed to one another when they have nothing in common with one another; and every proposition has only one negative, because there is only one proposition which lies altogether outside it.}
\pmc{And similarly we can say that two propositions are opposed to one another if they have nothing in common with one another, and that every proposition has only one negative, since there is only one proposition that lies completely outside it.}

\pnskip
\ger{Es zeigt sich so auch in Russells Notation, dass \gdql $q \mathrel{:} p \lor \rnot p$\gdqr{} dasselbe sagt wie \gdql $q$\gdqr{}; dass \gdql $p \lor \rnot p$\gdqr{} nichts sagt.}
\ogd{Thus in Russell's notation also it appears evident that ``$q \mathrel{:} p \lor \rnot p$'' says the same thing as ``$q$''; that ``$p \lor \rnot p$'' says nothing.}
\pmc{Thus in Russell's notation too it is manifest that `$q \mathrel{:} p \lor \rnot p$' says the same thing as `$q$', that `$p \lor \rnot p$' says nothing.}

\pn{5.514}
\ger{Ist eine Notation festgelegt, so gibt es in ihr eine Regel, nach der alle $p$ verneinenden S{\"a}tze gebildet werden, eine Regel, nach der alle $p$ bejahenden S{\"a}tze gebildet werden, eine Regel, nach der alle $p$ oder $q$ bejahenden S{\"a}tze gebildet werden, u.~s.~f. Diese Regeln sind den Symbolen {\"a}quivalent und in ihnen spiegelt sich ihr Sinn wieder.}
\ogd{If a notation is fixed, there is in it a rule according to which all the propositions denying $p$ are constructed, a rule according to which all the propositions asserting $p$ are constructed, a rule according to which all the propositions asserting $p$ or $q$ are constructed, and so on. These rules are equivalent to the symbols and in them their sense is mirrored.}
\pmc{Once a notation has been established, there will be in it a rule governing the construction of all propositions that negate $p$, a rule governing the construction of all propositions that affirm $p$, and a rule governing the construction of all propositions that affirm $p$ or $q$; and so on. These rules are equivalent to the symbols; and in them their sense is mirrored.}

\pn{5.515}
\ger{Es muss sich an unseren Symbolen zeigen, dass das, was durch \gdql $\lor$\gdqr{}, \gdql $\rand$\gdqr{}, etc.\ miteinander verbunden ist, S{\"a}tze sein m{\"u}ssen.}
\ogd{It must be recognized in our symbols that what is connected by ``$\lor$'', ``$\rand$'', etc., must be propositions.}
\pmc{It must be manifest in our symbols that it can only be propositions that are combined with one another by `$\lor$', `$\rnot$', etc.}

\pnskip
\ger{Und dies ist auch der Fall, denn das Symbol \gdql $p$\gdqr{} und \gdql $q$\gdqr{} setzt ja selbst das \gdql $\lor$\gdqr{}, \gdql $\rnot$\gdqr{}, etc.\ voraus. Wenn das Zeichen \gdql $p$\gdqr{} in \gdql $p \lor q$\gdqr{} nicht f{\"u}r ein komplexes Zeichen steht, dann kann es allein nicht Sinn haben; dann k{\"o}nnen aber auch die mit \gdql $p$\gdqr{} gleichsinnigen Zeichen \gdql $p \lor p$\gdqr{}, \gdql $p \rand p$\gdqr{}, etc.\ keinen Sinn haben. Wenn aber \gdql $p \lor p$\gdqr{} keinen Sinn hat, dann kann auch \gdql $p \lor q$\gdqr{} keinen Sinn haben.}
\ogd{And this is the case, for the symbols ``$p$'' and ``$q$'' presuppose ``$\lor$'', ``$\rnot$'', etc. If the sign ``$p$'' in ``$p \lor q$'' does not stand for a complex sign, then by itself it cannot have sense; but then also the signs ``$p \lor p$'', ``$p \rand p$'', etc.\ which have the same sense as ``$p$'' have no sense. If, however, ``$p \lor p$'' has no sense, then also ``$p \lor q$'' can have no sense.}
\pmc{And this is indeed the case, since the symbol in `$p$' and `$q$' itself presupposes `$\lor$', `$\rnot$', etc. If the sign `$p$' in `$p \lor q$' does not stand for a complex sign, then it cannot have sense by itself: but in that case the signs `$p \lor p$', `$p \rand p$', etc., which have the same sense as $p$, must also lack sense. But if `$p \lor p$' has no sense, then `$p \lor q$' cannot have a sense either.}

\pn{5.5151}
\ger{Muss das Zeichen des negativen Satzes mit dem Zeichen des positiven gebildet werden? Warum sollte man den negativen Satz nicht durch eine negative Tatsache ausdr{\"u}cken k{\"o}nnen. (Etwa: Wenn \gdql $a$\gdqr{} nicht in einer bestimmten Beziehung zu \gdql $b$\gdqr{} steht, k{\"o}nnte das ausdr{\"u}cken, dass $aRb$ nicht der Fall ist.)}
\ogd{Must the sign of the negative proposition be constructed by means of the sign of the positive? Why should one not be able to express the negative proposition by means of a negative fact? (Like: if ``$a$'' does not stand in a certain relation to ``$b$'', it could express that $aRb$ is not the case.)}
\pmc{Must the sign of a negative proposition be constructed with that of the positive proposition? Why should it not be possible to express a negative proposition by means of a negative fact? (E.g. suppose that `$a$' does not stand in a certain relation to `$b$'; then this might be used to say that $aRb$ was not the case.)}

\pnskip
\ger{Aber auch hier ist ja der negative Satz indirekt durch den positiven gebildet.}
\ogd{But here also the negative proposition is indirectly constructed with the positive.}
\pmc{But really even in this case the negative proposition is constructed by an indirect use of the positive.}

\pnskip
\ger{Der positive \germph{Satz} muss die Existenz des negativen \germph{Satzes} voraussetzen und umgekehrt.}
\ogd{The positive \emph{proposition} must presuppose the existence of the negative \emph{proposition} and conversely.}
\pmc{The positive \emph{proposition} necessarily presupposes the existence of the negative \emph{proposition} and \emph{vice versa}.}

\pn{5.52}
\ger{Sind die Werte von $\xi$ s{\"a}mtliche Werte einer Funktion $f\negthinspace x$ f{\"u}r alle Werte von $x$, so wird $\nop(\overline{\xi})=\rnot \rsomed{x} f\negthinspace x$.}
\ogd{If the values of $\xi$ are the total values of a function $f\negthinspace x$ for all values of $x$, then $\nop(\overline{\xi})=\rnot \rsomed{x} f\negthinspace x$.}
\pmc{If $\xi$ has as its values all the values of a function $f\negthinspace x$ for all values of $x$, then $\nop(\overline{\xi})=\rnot \rsomed{x} f\negthinspace x$.}

\pn{5.521}
\ger{Ich trenne den Begriff \germph{Alle} von der Wahrheitsfunktion.}
\ogd{I separate the concept \emph{all} from the truth-function.}
\pmc{I dissociate the concept \emph{all} from truth-functions.}

\pnskip
\ger{Frege und Russell haben die Allgemeinheit in Verbindung mit dem logischen Produkt oder der logischen Summe eingef{\"u}hrt. So wurde es schwer, die S{\"a}tze \gdql $\rsomed{x} f\negthinspace x$\gdqr{} und \gdql $\ralld{x} f\negthinspace x$\gdqr{}, in welchen beide Ideen beschlossen liegen, zu verstehen.}
\ogd{Frege and Russell have introduced generality in connexion with the logical product or the logical sum. Then it would be difficult to understand the propositions ``$\rsomed{x} f\negthinspace x$'' and ``$\ralld{x} f\negthinspace x$'' in which both ideas lie concealed.}
\pmc{Frege and Russell introduced generality in association with logical product or logical sum. This made it difficult to understand the propositions `$\rsomed{x} f\negthinspace x$' and `$\ralld{x} f\negthinspace x$', in which both ideas are embedded.}

\pn{5.522}
\ger{Das Eigent{\"u}mliche der Allgemeinheitsbezeichnung ist erstens, dass sie auf ein logisches Urbild hinweist, und zweitens, dass sie Konstante hervorhebt.}
\ogd{That which is peculiar to the ``symbolism of generality'' is firstly, that it refers to a logical prototype, and secondly, that it makes constants prominent.}
\pmc{What is peculiar to the generality-sign is first, that it indicates a logical prototype, and secondly, that it gives prominence to constants.}

\pn{5.523}
\ger{Die Allgemeinheitsbezeichnung tritt als Argument auf.}
\ogd{The generality symbol occurs as an argument.}
\pmc{The generality-sign occurs as an argument.}

\pn{5.524}
\ger{Wenn die Gegenst{\"a}nde gegeben sind, so sind uns damit auch schon \germph{alle} Gegenst{\"a}nde gegeben.}
\ogd{If the objects are given, therewith are \emph{all} objects also given.}
\pmc{If objects are given, then at the same time we are given \emph{all} objects.}

\pnskip
\ger{Wenn die Elementars{\"a}tze gegeben sind, so sind damit auch \germph{alle} Elementars{\"a}tze gegeben.}
\ogd{If the elementary propositions are given, then therewith \emph{all} elementary propositions are also given.}
\pmc{If elementary propositions are given, then at the same time \emph{all} elementary propositions are given.}

\pn{5.525}
\ger{Es ist unrichtig, den Satz \gdql $\rsomed{x} f\negthinspace x$\gdqr{}---wie Russell dies tut---in Worten durch \gdql $f\negthinspace x$ ist \germph{m{\"o}glich}\gdqr{} wiederzugeben.}
\ogd{It is not correct to render the proposition ``$\rsomed{x} f\negthinspace x$''---as Russell does---in the words ``$f\negthinspace x$ is \emph{possible}''.}
\pmc{It is incorrect to render the proposition `$\rsomed{x} f\negthinspace x$' in the words, `$f\negthinspace x$ is \emph{possible}' as Russell does.}

\pnskip
\ger{Gewi{\ss}heit, M{\"o}glichkeit oder Unm{\"o}glichkeit einer Sachlage wird nicht durch einen Satz ausgedr{\"u}ckt, sondern dadurch, dass ein Ausdruck eine Tautologie, ein sinnvoller Satz oder eine Kontradiktion ist.}
\ogd{Certainty, possibility or impossibility of a state of affairs are not expressed by a proposition but by the fact that an expression is a tautology, a significant proposition or a contradiction.}
\pmc{The certainty, possibility, or impossibility of a situation is not expressed by a proposition, but by an expression's being a tautology, a proposition with a sense, or a contradiction.}

\pnskip
\ger{Jener Pr{\"a}zedenzfall, auf den man sich immer berufen m{\"o}chte, muss schon im Symbol selber liegen.}
\ogd{That precedent to which one would always appeal, must be present in the symbol itself.}
\pmc{The precedent to which we are constantly inclined to appeal must reside in the symbol itself.}

\pn{5.526}
\ger{Man kann die Welt vollst{\"a}ndig durch vollkommen verallgemeinerte S{\"a}tze beschreiben, das hei{\ss}t also, ohne irgendeinen Namen von vornherein einem bestimmten Gegenstand zuzuordnen.}
\ogd{One can describe the world completely by completely generalized propositions, \emph{i.e.}\ without from the outset co-ordinating any name with a definite object.}
\pmc{We can describe the world completely by means of fully generalized propositions, i.e.\ without first correlating any name with a particular object.}

\pnskip
\ger{Um dann auf die gew{\"o}hnliche Ausdrucksweise zu kommen, muss man einfach nach einem Ausdruck: \gdql Es gibt ein und nur ein $x$, welches \ldots\gdqr{} sagen: Und dies $x$ ist $a$.}
\ogd{In order then to arrive at the customary way of expression we need simply say after an expression ``there is only and only one $x$, which \ldots'': and this $x$ is $a$.}
\pmc{Then, in order to arrive at the customary mode of expression, we simply need to add, after an expression like, `There is one and only one $x$ such that \ldots', the words, `and that $x$ is $a$'.}%???

\pn{5.5261}
\ger{Ein vollkommen verallgemeinerter Satz ist, wie jeder andere Satz, zusammengesetzt. (Dies zeigt sich daran, dass wir in \gdql $\rsomed{x, \phi} \phi x$\gdqr{} \gdql $\phi$\gdqr{} und \gdql $x$\gdqr{} getrennt erw{\"a}hnen m{\"u}ssen. Beide stehen unabh{\"a}ngig in bezeichnenden Beziehungen zur Welt, wie im unverallgemeinerten Satz.)}
\ogd{A completely generalized proposition is like every other proposition composite. (This is shown by the fact that in ``$\rsomed{x, \phi} \phi x$'' we must mention ``$\phi$'' and ``$x$'' separately. Both stand independently in signifying relations to the world as in the ungeneralized proposition.)}
\pmc{A fully generalized proposition, like every other proposition, is composite. (This is shown by the fact that in `$\rsomed{x, \phi} \phi x$' we have to mention `$\phi$' and `$x$' separately. They both, independently, stand in signifying relations to the world, just as is the case in ungeneralized propositions.)}

\pnskip
\ger{Kennzeichen des zusammengesetzten Symbols: Es hat etwas mit \germph{anderen} Symbolen gemeinsam.}
\ogd{A characteristic of a composite symbol: it has something in common with \emph{other} symbols.}
\pmc{It is a mark of a composite symbol that it has something in common with \emph{other} symbols.}

\pn{5.5262}
\ger{Es ver{\"a}ndert ja die Wahr- oder Falschheit \germph{jedes} Satzes etwas am allgemeinen Bau der Welt. Und der Spielraum, welcher ihrem Bau durch die Gesamtheit der Elementars{\"a}tze gelassen wird, ist eben derjenige, welchen die ganz allgemeinen S{\"a}tze begrenzen.}
\ogd{The truth or falsehood of \emph{every} proposition alters something in the general structure of the world. And the range which is allowed to its structure by the totality of elementary propositions is exactly that which the completely general propositions delimit.}
\pmc{The truth or falsity of \emph{every} proposition does make some alteration in the general construction of the world. And the range that the totality of elementary propositions leaves open for its construction is exactly the same as that which is delimited by entirely general propositions.}

\pnskip
\ger{(Wenn ein Elementarsatz wahr ist, so ist damit doch jedenfalls Ein Elementarsatz \germph{mehr} wahr.)}
\ogd{(If an elementary proposition is true, then, at any rate, there is one \emph{more} elementary proposition true.)}
\pmc{(If an elementary proposition is true, that means, at any rate, one \emph{more} true elementary proposition.)}

\pn{5.53}
\ger{Gleichheit des Gegenstandes dr{\"u}cke ich durch Gleichheit des Zeichens aus, und nicht mit Hilfe eines Gleichheitszeichens. Verschiedenheit der Gegenst{\"a}nde durch Verschiedenheit der Zeichen.}
\ogd{Identity of the object I express by identity of the sign and not by means of a sign of identity. Difference of the objects by difference of the signs.}
\pmc{Identity of object I express by identity of sign, and not by using a sign for identity. Difference of objects I express by difference of signs.}

\pn{5.5301}
\ger{Dass die Identit{\"a}t keine Relation zwischen Gegenst{\"a}nden ist, leuchtet ein. Dies wird sehr klar, wenn man z.~B.\ den Satz \gdql $\ralldd{x} f\negthinspace x \drimpliesd x=a$\gdqr{} betrachtet. Was dieser Satz sagt, ist einfach, dass \germph{nur} $a$ der Funktion $f$ gen{\"u}gt, und nicht, dass nur solche Dinge der Funktion $f$ gen{\"u}gen, welche eine gewisse Beziehung zu $a$ haben.}
\ogd{That identity is not a relation between objects is obvious. This becomes very clear if, for example, one considers the proposition ``$\ralldd{x} f\negthinspace x \drimpliesd x=a$''. What this proposition says is simply that \emph{only} $a$ satisfies the function $f$, and not that only such things satisfy the function $f$ which have a certain relation to $a$.}
\pmc{It is self-evident that identity is not a relation between objects. This becomes very clear if one considers, for example, the proposition `$\ralldd{x} f\negthinspace x \drimpliesd x=a$'. What this proposition says is simply that \emph{only} $a$ satisfies the function $f$, and not that only things that have a certain relation to $a$ satisfy the function $f$.}

\pnskip
\ger{Man k{\"o}nnte nun freilich sagen, dass eben \germph{nur} $a$ diese Beziehung zu $a$ habe, aber, um dies auszudr{\"u}cken, brauchten wir das Gleichheitszeichen selber.}
\ogd{One could of course say that in fact \emph{only} $a$ has this relation to $a$, but in order to express this we should need the sign of identity itself.}
\pmc{Of course, it might then be said that \emph{only} $a$ did have this relation to $a$; but in order to express that, we should need the identity-sign itself.}

\pn{5.5302}
\ger{Russells Definition von \gdql =\gdqr{} gen{\"u}gt nicht; weil man nach ihr nicht sagen kann, dass zwei Gegenst{\"a}nde alle Eigenschaften gemeinsam haben. (Selbst wenn dieser Satz nie richtig ist, hat er doch \germph{Sinn}.)}
\ogd{Russell's definition of ``='' won't do; because according to it one cannot say that two objects have all their properties in common. (Even if this proposition is never true, it is nevertheless \emph{significant}.)}
\pmc{Russell's definition of `=' is inadequate, because according to it we cannot say that two objects have all their properties in common. (Even if this proposition is never correct, it still has \emph{sense}.)}

\pn{5.5303}
\ger{Beil{\"a}ufig gesprochen: Von \germph{zwei} Dingen zu sagen, sie seien identisch, ist ein Unsinn, und von \germph{Einem} zu sagen, es sei identisch mit sich selbst, sagt gar nichts.}
\ogd{Roughly speaking: to say of \emph{two} things that they are identical is nonsense, and to say of \emph{one} thing that it is identical with itself is to say nothing.}
\pmc{Roughly speaking, to say of \emph{two} things that they are identical is nonsense, and to say of \emph{one} thing that it is identical with itself is to say nothing at all.}

\pn{5.531}
\ger{Ich schreibe also nicht \gdql $f(a,b)\rand a=b$\gdqr{}, sondern \gdql $f(a,a)$\gdqr{} (oder \gdql $f(b,b)$\gdqr{}). Und nicht \gdql $f(a,b) \rand \rnot a=b$\gdqr{}, sondern \gdql $f(a,b)$\gdqr{}.}
\ogd{I write therefore not ``$f(a,b)\rand a=b$'' but ``$f(a,a)$'' (or ``$f(b,b)$''). And not ``$f(a,b) \rand \rnot a=b$'', but ``$f(a,b)$''.}
\pmc{Thus I do not write `$f(a,b)\rand a=b$', but `$f(a,a)$' (or `$f(b,b)$'); and not `$f(a,b) \rand \rnot a=b$', but `$f(a,b)$'.}

\pn{5.532}
\ger{Und analog: Nicht \gdql $\rsomed{x,y} f(x,y) \rand x=y$\gdqr{}\mbox{, sondern \gdql $\rsomed{x} f(x,x)$\gdqr{}; und nicht} \gdql $\rsomed{x,y}$ $f(x,y) \rand \rnot x=y$\gdqr{}, sondern \gdql $\rsomed{x,y}$ $f(x,y)$\gdqr{}.}
\ogd{And analogously: not ``$\rsomed{x,y} f(x,y) \rand$ \linebreak \mbox{$x=y$'', but ``$\rsomed{x} f(x,x)$''; and not} ``$\rsomed{x,y}$ $f(x,y) \rand \rnot x=y$'', but ``$\rsomed{x,y} f(x,y)$''.}
\pmc{And analogously I do not write `$\rsomed{x,y} f(x,y) \rand x=y$', but `$\rsomed{x} f(x,x)$'; and not `$\rsomed{x,y} f(x,y) \rand \rnot x=y$', but `$\rsomed{x,y} f(x,y)$'.}

\pnskip
\ger{(Also statt des Russellschen \gdql $\rsomed{x,y}$ $f(x,y)$\gdqr: \gdql $\rsomed{x,y} f(x,y) \dlord \rsomed{x}f(x,x)$\gdqr.)}
\ogd{(Therefore instead of Russell's ``$\rsomed{x,y}$ $f(x,y)$'': ``$\rsomed{x,y} f(x,y) \dlord \rsomed{x}f(x,x)$''.)}
\pmc{(So Russell's `$\rsomed{x,y} f(x,y)$' becomes `$\rsomed{x,y} f(x,y) \dlord \rsomed{x}f(x,x)$'.)}%???

\pn{5.5321}
\ger{Statt \gdql $\ralldd{x} f\negthinspace x \rimplies x=a$\gdqr{} schreiben wir also z.~B.\ \gdql $\rsomed{x} f\negthinspace x \drimpliesd f\negthinspace a \mathrel{:} \rnot \rsomed{x,y} f\negthinspace x \rand f\negthinspace y$\gdqr{}.}
\ogd{Instead of ``$\ralldd{x} f\negthinspace x \rimplies x=a$'' we therefore write \emph{e.g.}\ ``$\rsomed{x} f\negthinspace x \drimpliesd f\negthinspace a \mathrel{:} \rnot \rsomed{x,y} f\negthinspace x \rand f\negthinspace y$''.}
\pmc{\mbox{Thus, for example, instead of `$\ralldd{x} f\negthinspace x$} $\rimplies x=a$' we write `$\rsomed{x} f\negthinspace x \drimpliesd f\negthinspace a \mathrel{:}  \rnot \rsomed{x,y}$ $f\negthinspace x \rand f\negthinspace y$'.}

\pnskip
\ger{Und der Satz: \gdql \germph{nur} Ein $x$ befriedigt $f(\phantom{i})$\gdqr{} lautet: \gdql $\rsomed{x} f\negthinspace x \drimpliesd f\negthinspace a \mathrel{:} \rnot \rsomed{x,y} f\negthinspace x \rand f\negthinspace y$\gdqr.}
\ogd{And if the proposition ``\emph{only} one $x$ satisfies $f(\phantom{i})$'' reads: ``$\rsomed{x} f\negthinspace x \drimpliesd f\negthinspace a \mathrel{:} \rnot \rsomed{x,y} f\negthinspace x \rand f\negthinspace y$''.}
\pmc{And the proposition, `\emph{Only one} $x$ satisfies $f(\phantom{i})$', will read `$\rsomed{x} f\negthinspace x \drimpliesd f\negthinspace a \mathrel{:} \rnot \rsomed{x,y} f\negthinspace x \rand f\negthinspace y$'.}

\pn{5.533}
\ger{Das Gleichheitszeichen ist also kein wesentlicher Bestandteil der Begriffsschrift.}
\ogd{The identity sign is therefore not an essential constituent of logical notation.}
\pmc{The identity-sign, therefore, is not an essential constituent of conceptual notation.}

\pn{5.534}
\ger{Und nun sehen wir, dass Scheins{\"a}tze wie: \gdql $a=a$\gdqr{}, \gdql $a=b \rand b=c \drimplies a=c$\gdqr{}, \gdql $\ralld{x} x=x$\gdqr{}, \gdql $\rsomed{x} x=a$\gdqr{}, etc.\ sich in einer richtigen Begriffsschrift gar nicht hinschreiben lassen.}
\ogd{And we see that the apparent propositions like: ``$a=a$'', ``$a=b \rand b=c \drimplies a=c$'', ``$\ralld{x} x=x$''. ``$\rsomed{x} x=a$'', etc.\ cannot be written in a correct logical notation at all.}
\pmc{And now we see that in a correct conceptual notation pseudo-propositions like `$a = a$', `$a = b \rand b = c \drimplies a = c$', `$\ralld{x} x=x$', `$\rsomed{x} x=a$', etc.\ cannot even be written down.}

\pn{5.535}
\ger{Damit erledigen sich auch alle Probleme, die an solche Scheins{\"a}tze gekn{\"u}pft waren.}
\ogd{So all problems disappear which are connected with such pseudo-propositions.}
\pmc{This also disposes of all the problems that were connected with such pseudo-propositions.}

\pnskip
\ger{Alle Probleme, die Russells \gdql Axiom of Infinity\gdqr{} mit sich bringt, sind schon hier zu l{\"o}sen.}
\ogd{This is the place to solve all the problems with arise through Russell's ``Axiom of Infinity''.}
\pmc{All the problems that Russell's `axiom of infinity' brings with it can be solved at this point.}

\pnskip
\ger{Das, was das Axiom of Infinity sagen soll, w{\"u}rde sich in der Sprache dadurch ausdr{\"u}cken, dass es unendlich viele Namen mit verschiedener Bedeutung g{\"a}be.}
\ogd{What the axiom of infinity is meant to say would be expressed in language by the fact that there is an infinite number of names with different meanings.}
\pmc{What the axiom of infinity is intended to say would express itself in language through the existence of infinitely many names with different meanings.}

\pn{5.5351}
\ger{Es gibt gewisse F{\"a}lle, wo man in Versuchung ger{\"a}t, Ausdr{\"u}cke von der Form \gdql $a=a$\gdqr{} oder \gdql $p \rimplies p$\gdqr{} u.\ dgl.\ zu ben{\"u}tzen. Und zwar geschieht dies, wenn man von dem Urbild: Satz, Ding, etc. reden m{\"o}chte. So hat Russell in den \gdql Principles of Mathematics\gdqr{} den Unsinn \gdql $p$ ist ein Satz\gdqr{} in Symbolen durch \gdql $p \rimplies p$\gdqr{} wiedergegeben und als Hypothese vor gewisse S{\"a}tze gestellt, damit deren Argumentstellen nur von S{\"a}tzen besetzt werden k{\"o}nnten.}
\ogd{There are certain cases in which one is tempted to use expressions of the form ``$a=a$'' or ``$p \rimplies p$'' and of that kind. And indeed this takes place when one would speak of the archetype Proposition, Thing, etc. So Russell in the \emph{Principles of Mathematics} has rendered the nonsense ``$p$ is a proposition'' in symbols by ``$p \rimplies p$'' and has put it as hypothesis before certain propositions to show that their places for arguments could only be occupied by propositions.}
\pmc{There are certain cases in which one is tempted to use expressions of the form `$a=a$' or `$p \rimplies p$' and the like. In fact, this happens when one wants to talk about prototypes, e.g.\ about proposition, thing, etc. Thus in Russell's \textit{Principles of Mathematics} `$p$ is a proposition'---which is nonsense---was given the symbolic rendering `$p \rimplies p$' and placed as an hypothesis in front of certain propositions in order to exclude from their argument-places everything but propositions.}

\pnskip
\ger{(Es ist schon darum Unsinn, die Hypothese $p \rimplies p$ vor einen Satz zu stellen, um ihm Argumente der richtigen Form zu sichern, weil die Hypothese f{\"u}r einen Nicht-Satz als Argument nicht falsch, sondern unsinnig wird, und weil der Satz selbst durch die unrichtige Gattung von Argumenten unsinnig wird, also sich selbst ebenso gut, oder so schlecht, vor den unrechten Argumenten bewahrt wie die zu diesem Zweck angeh{\"a}ngte sinnlose Hypothese.)}
\ogd{(It is nonsense to place the hypothesis $p \rimplies p$ before a proposition in order to ensure that its arguments have the right form, because the hypotheses for a non-proposition as argument becomes not false but meaningless, and because the proposition itself becomes senseless for arguments of the wrong kind, and therefore it survives the wrong arguments no better and no worse than the senseless hypothesis attached for this purpose.)}
\pmc{(It is nonsense to place the hypothesis `$p \rimplies p$' in front of a proposition, in order to ensure that its arguments shall have the right form, if only because with a non-proposition as argument the hypothesis becomes not false but nonsensical, and because arguments of the wrong kind make the proposition itself nonsensical, so that it preserves itself from wrong arguments just as well, or as badly, as the hypothesis without sense that was appended for that purpose.)}

\pn{5.5352}
\ger{Ebenso wollte man \gdql Es gibt keine \germph{Dinge}\gdqr{} ausdr{\"u}cken durch \gdql $\rnot \rsomed{x} x=x$\gdqr{}. Aber selbst wenn dies ein Satz w{\"a}re---w{\"a}re er nicht auch wahr, wenn es zwar \gdql Dinge g{\"a}be\gdqr{}, aber diese nicht mit sich selbst identisch w{\"a}ren?}
\ogd{Similarly it was proposed to express ``There are no things'' by ``$\rnot \rsomed{x} x=x$''. But even if this were a proposition---would it not be true if indeed ``There were things'', but these were not identical with themselves?}
\pmc{In the same way people have wanted to express, `There are no \emph{things}', by writing `$\rnot \rsomed{x} x=x$'. But even if this were a proposition, would it not be equally true if in fact `there were things' but they were not identical with themselves?}

\pn{5.54}
\ger{In der allgemeinen Satzform kommt der Satz im Satze nur als Basis der Wahrheitsoperationen vor.}
\ogd{In the general propositional form, propositions occur in a proposition only as bases of the truth-operations.}
\pmc{In the general propositional form propositions occur in other propositions only as bases of truth-operations.}

\pn{5.541}
\ger{Auf den ersten Blick scheint es, als k{\"o}nne ein Satz in einem anderen auch auf andere Weise vorkommen.}
\ogd{At first sight it appears as if there were also a different way in which one proposition could occur in another.}
\pmc{At first sight it looks as if it were also possible for one proposition to occur in another in a different way.}

\pnskip
\ger{Besonders in gewissen Satzformen der Psychologie, wie \gdql A glaubt, dass $p$ der Fall ist\gdqr{}, oder \gdql A denkt $p$\gdqr{}, etc.}
\ogd{Especially in certain propositional forms of psychology, like ``A thinks, that $p$ is the case'', or ``A thinks $p$'', etc.}
\pmc{Particularly with certain forms of proposition in psychology, such as `A believes that $p$ is the case' and A has the thought $p$', etc.}

\pnskip
\ger{Hier scheint es n{\"a}mlich oberfl{\"a}chlich, als st{\"u}nde der Satz $p$ zu einem Gegenstand A in einer Art von Relation.}
\ogd{Here it appears superficially as if the proposition $p$ stood to the object A in a kind of relation.}
\pmc{For if these are considered superficially, it looks as if the proposition $p$ stood in some kind of relation to an object A.}

\pnskip
\ger{(Und in der modernen Erkenntnistheorie (Russell, Moore, etc.)\ sind jene S{\"a}tze auch so aufgefasst worden.)}
\ogd{(And in modern epistemology (Russell, Moore, etc.)\ those propositions have been conceived in this way.)}
\pmc{(And in modern theory of knowledge (Russell, Moore, etc.)\ these propositions have actually been construed in this way.)}

\pn{5.542}
\ger{Es ist aber klar, dass \gdql A glaubt, dass $p$\gdqr{}, \gdql A denkt $p$\gdqr{}, \gdql A sagt $p$\gdqr{} von der Form \gdql {\gsql}$p${\gsqr} sagt $p$\gdqr{} sind: Und hier handelt es sich nicht um eine Zuordnung von einer Tatsache und einem Gegenstand, sondern um die Zuordnung von Tatsachen durch Zuordnung ihrer Gegenst{\"a}nde.}
\ogd{But it is clear that ``A believes that $p$'', ``A thinks $p$'', ``A says $p$'', are of the form ``{}`$p$' says $p$'': and here we have no co-ordination of a fact and an object, but a co-ordination of facts by means of a co-ordination of their objects.}
\pmc{It is clear, however, that `A believes that $p$', `A has the thought $p$', and `A says $p$' are of the form `{}``$p$'' says $p$': and this does not involve a correlation of a fact with an object, but rather the correlation of facts by means of the correlation of their objects.}

\pn{5.5421}
\ger{Dies zeigt auch, dass die Seele---das Subjekt etc.---wie sie in der heutigen oberfl{\"a}chlichen Psychologie aufgefasst wird, ein Unding ist.}
\ogd{This shows that there is no such thing as the soul---the subject, etc.---as it is conceived in superficial psychology.}
\pmc{This shows too that there is no such thing as the soul---the subject, etc.---as it is conceived in the superficial psychology of the present day.}

\pnskip
\ger{Eine zusammengesetzte Seele w{\"a}re n{\"a}mlich keine Seele mehr.}
\ogd{A composite soul would not be a soul any longer.}
\pmc{Indeed a composite soul would no longer be a soul.}

\pn{5.5422}
\ger{Die richtige Erkl{\"a}rung der Form des Satzes \gdql A urteilt $p$\gdqr{} muss zeigen, dass es unm{\"o}glich ist, einen Unsinn zu urteilen. (Russells Theorie gen{\"u}gt dieser Bedingung nicht.)}
\ogd{The correct explanation of the form of the proposition ``A judges $p$'' must show that it is impossible to judge a nonsense. (Russell's theory does not satisfy this condition.)}
\pmc{The correct explanation of the form of the proposition, `A makes the judgement $p$', must show that it is impossible for a judgement to be a piece of nonsense. (Russell's theory does not satisfy this requirement.)}

\pn{5.5423}
\ger{Einen Komplex wahrnehmen hei{\ss}t wahrnehmen, dass sich seine Bestandteile so und so zu einander verhalten.}
\ogd{To perceive a complex means to perceive that its constituents are combined in such and such a way.}
\pmc{To perceive a complex means to perceive that its constituents are related to one another in such and such a way.}

\pnskip
\ger{Dies erkl{\"a}rt wohl auch, dass man die Figur%
%
\vspace*{-8pt}\thecube}
\ogd{This perhaps explains that the figure \phantom{longwordhere}%
%
\vspace*{-8pt}\thecube}
\pmc{This no doubt also explains why there are two possible ways of seeing the figure%
%
\vspace*{-8pt}\thecube}

\pnskip
%
\ger{\negpbk auf zweierlei Art als W{\"u}rfel sehen kann; und alle {\"a}hnlichen Erscheinungen. Denn wir sehen eben wirklich zwei verschiedene Tatsachen.}
%
\ogd{\negpbk can be seen in two ways as a cube; and all similar phenomena. For we really see two different facts.}
%
\pmc{\negpbk as a cube; and all similar phenomena. For we really see two different facts.}

\pnskip
\ger{(Sehe ich erst auf die Ecken $a$ und nur fl{\"u}chtig auf $b$, so erscheint $a$ vorne; und umgekehrt.)}
\ogd{(If I fix my eyes first on the corners $a$ and only glance at $b$, $a$ appears in front and $b$ behind, and vice versa.)}
\pmc{(If I look in the first place at the corners marked $a$ and only glance at the $b$'s, then the $a$'s appear to be in front, and \emph{vice versa}).}

\pn{5.55}
\ger{Wir m{\"u}ssen nun die Frage nach allen m{\"o}glichen Formen der Elementars{\"a}tze a priori beantworten.}
\ogd{We must now answer a priori the question as to all possible forms of the elementary propositions.}
\pmc{We now have to answer \textit{a priori} the question about all the possible forms of elementary propositions.}

\pnskip
\ger{Der Elementarsatz besteht aus Namen. Da wir aber die Anzahl der Namen von verschiedener Bedeutung nicht angeben k{\"o}nnen, so k{\"o}nnen wir auch nicht die Zusammensetzung des Elementarsatzes angeben.}
\ogd{The elementary proposition consists of names. Since we cannot give the number of names with different meanings, we cannot give the composition of the elementary proposition.}
\pmc{Elementary propositions consist of names. Since, however, we are unable to give the number of names with different meanings, we are also unable to give the composition of elementary propositions.}

\pn{5.551}
\ger{Unser Grundsatz ist, dass jede Frage, die sich {\"u}berhaupt durch die Logik entscheiden l{\"a}{\ss}t, sich ohne weiteres entscheiden lassen muss.}
\ogd{Our fundamental principle is that every question which can be decided at all by logic can be decided without further trouble.}
\pmc{Our fundamental principle is that whenever a question can be decided by logic at all it must be possible to decide it without more ado.}

\pnskip
\ger{(Und wenn wir in die Lage kommen, ein solches Problem durch Ansehen der Welt beantworten zu m{\"u}ssen, so zeigt dies, dass wir auf grundfalscher F{\"a}hrte sind.)}
\ogd{(And if we get into a situation where we need to answer such a problem by looking at the world, this shows that we are on a fundamentally wrong track.)}
\pmc{(And if we get into a position where we have to look at the world for an answer to such a problem, that shows that we are on a completely wrong track.)}

\pn{5.552}
\ger{Die \gdql Erfahrung\gdqr{}, die wir zum Verstehen der Logik brauchen, ist nicht die, dass sich etwas so und so verh{\"a}lt, sondern, dass etwas \germph{ist}: aber das ist eben \germph{keine} Erfahrung.}
\ogd{The ``experience'' which we need to understand logic is not that such and such is the case, but that something \emph{is}; but that is \emph{no} experience.}
\pmc{The `experience' that we need in order to understand logic is not that something or other is the state of things, but that something \emph{is}: that, however, is \emph{not} an experience.}

\pnskip
\ger{Die Logik ist \germph{vor} jeder Erfahrung---dass etwas \germph{so} ist.}
\ogd{Logic \emph{precedes} every experience---that something is \emph{so}.}
\pmc{Logic is \emph{prior} to every experience---that something \emph{is so}.}

\pnskip
\ger{Sie ist vor dem Wie, nicht vor dem Was.}
\ogd{It is before the How, not before the What.}
\pmc{It is prior to the question `How?', not prior to the question `What?'}

\pn{5.5521}
\ger{Und wenn dies nicht so w{\"a}re, wie k{\"o}nnten wir die Logik anwenden? Man k{\"o}nnte sagen: Wenn es eine Logik g{\"a}be, auch wenn es keine Welt g{\"a}be, wie k{\"o}nnte es dann eine Logik geben, da es eine Welt gibt?}
\ogd{And if this were not the case, how could we apply logic? We could say: if there were a logic, even if there were no world, how then could there be a logic, since there is a world?}
\pmc{And if this were not so, how could we apply logic? We might put it in this way: if there would be a logic even if there were no world, how then could there be a logic given that there is a world?}

\pn{5.553}
\ger{Russell sagte, es g{\"a}be einfache Relationen zwischen verschiedenen Anzahlen von Dingen (Individuals). Aber zwischen welchen Anzahlen? Und wie soll sich das entscheiden?---Durch die Erfahrung?}
\ogd{Russell said that there were simple relations between different numbers of things (individuals). But between what numbers? And how should this be decided---by experience?}
\pmc{Russell said that there were simple relations between different numbers of things (individuals). But between what numbers? And how is this supposed to be decided?---By experience?}

\pnskip
\ger{(Eine ausgezeichnete Zahl gibt es nicht.)}
\ogd{(There is no pre-eminent number.)}
\pmc{(There is no privileged number.)}

\pn{5.554}
\ger{Die Angabe jeder speziellen Form w{\"a}re vollkommen willk{\"u}rlich.}
\ogd{The enumeration of any special forms would be entirely arbitrary.}
\pmc{It would be completely arbitrary to give any specific form.}

\pn{5.5541}
\ger{Es soll sich a priori angeben lassen, ob ich z.~B.\ in die Lage kommen kann, etwas mit dem Zeichen einer 27-stelligen Relation bezeichnen zu m{\"u}ssen.}
\ogd{How could we decide a priori whether, for example, I can get into a situation in which I need to symbolize with a sign of a 27-termed relation?}
\pmc{It is supposed to be possible to answer \emph{a priori} the question whether I can get into a position in which I need the sign for a 27-termed relation in order to signify something.}

\pn{5.5542}
\ger{D{\"u}rfen wir denn aber {\"u}berhaupt so fragen? K{\"o}nnen wir eine Zeichenform aufstellen und nicht wissen, ob ihr etwas entsprechen k{\"o}nne?}
\ogd{May we then ask this at all? Can we set out a sign form and not know whether anything can correspond to it?}
\pmc{But is it really legitimate even to ask such a question? Can we set up a form of sign without knowing whether anything can correspond to it?}

\pnskip
\ger{Hat die Frage einen Sinn: Was muss \germph{sein}, damit etwas der-Fall-sein kann?}
\ogd{Has the question sense: what must there \emph{be} in order that anything can be the case?}
\pmc{Does it make sense to ask what there must \emph{be} in order that something can be the case?}

\pn{5.555}
\ger{Es ist klar, wir haben vom Elementarsatz einen Begriff, abgesehen von seiner besonderen logischen Form.}
\ogd{It is clear that we have a concept of the elementary proposition apart from its special logical form.}
\pmc{Clearly we have some concept of elementary propositions quite apart from their particular logical forms. }

\pnskip
\ger{Wo man aber Symbole nach einem System bilden kann, dort ist dieses System das logisch wichtige und nicht die einzelnen Symbole.}
\ogd{Where, however, we can build symbols according to a system, there this system is the logically important thing and not the single symbols.}
\pmc{But when there is a system by which we can create symbols, the system is what is important for logic and not the individual symbols.}

\pnskip
\ger{Und wie w{\"a}re es auch m{\"o}glich, dass ich es in der Logik mit Formen zu tun h{\"a}tte, die ich erfinden kann; sondern mit dem muss ich es zu tun haben, was es mir m{\"o}glich macht, sie zu erfinden.}
\ogd{And how would it be possible that I should have to deal with forms in logic which I can invent: but I must have to deal with that which makes it possible for me to invent them.}
\pmc{And anyway, is it really possible that in logic I should have to deal with forms that I can invent? What I have to deal with must be that which makes it possible for me to invent them.}

\pn{5.556}
\ger{Eine Hierarchie der Formen der Elementars{\"a}tze kann es nicht geben. Nur was wir selbst konstruieren, k{\"o}nnen wir voraussehen.}
\ogd{There cannot be a hierarchy of the forms of the elementary propositions. Only that which we ourselves construct can we foresee.}
\pmc{There cannot be a hierarchy of the forms of elementary propositions. We can foresee only what we ourselves construct.}

\pn{5.5561}
\ger{Die empirische Realit{\"a}t ist begrenzt durch die Gesamtheit der Gegenst{\"a}nde. Die Grenze zeigt sich wieder in der Gesamtheit der Elementars{\"a}tze.}
\ogd{Empirical reality is limited by the totality of objects. The boundary appears again in the totality of elementary propositions.}
\pmc{Empirical reality is limited by the totality of objects. The limit also makes itself manifest in the totality of elementary propositions.}

\pnskip
\ger{Die Hierarchien sind, und m{\"u}ssen unabh{\"a}ngig von der Realit{\"a}t sein.}
\ogd{The hierarchies are and must be independent of reality.}
\pmc{Hierarchies are and must be independent of reality.}

\pn{5.5562}
\ger{Wissen wir aus rein logischen Gr{\"u}nden, dass es Elementars{\"a}tze geben muss, dann muss es jeder wissen, der die S{\"a}tze in ihrer unanalysierten Form versteht.}
\ogd{If we know on purely logical grounds, that there must be elementary propositions, then this must be known by everyone who understands propositions in their unanalysed form.}
\pmc{If we know on purely logical grounds that there must be elementary propositions, then everyone who understands propositions in their unanalyzed form must know it.}%???

\pn{5.5563}
\ger{Alle S{\"a}tze unserer Umgangssprache sind tats{\"a}chlich, so wie sie sind, logisch vollkommen geordnet.---Jenes Einfachste, was wir hier angeben sollen, ist nicht ein Gleichnis der Wahrheit, sondern die volle Wahrheit selbst.}
\ogd{All propositions of our colloquial language are actually, just as they are, logically completely in order. That simple thing which we ought to give here is not a model of the truth but the complete truth itself.}
\pmc{In fact, all the propositions of our everyday language, just as they stand, are in perfect logical order.---That utterly simple thing, which we have to formulate here, is not an image of the truth, but the truth itself in its entirety.}

\pnskip
\ger{(Unsere Probleme sind nicht abstrakt, sondern vielleicht die konkretesten, die es gibt.)}
\ogd{(Our problems are not abstract but perhaps the most concrete that there are.)}
\pmc{(Our problems are not abstract, but perhaps the most concrete that there are.)}

\pn{5.557}
\ger{Die \germph{Anwendung} der Logik entscheidet dar{\"u}ber, welche Elementars{\"a}tze es gibt.}
\ogd{The \emph{application} of logic decides what elementary propositions there are.}
\pmc{The \emph{application} of logic decides what elementary propositions there are.}

\pnskip
\ger{Was in der Anwendung liegt, kann die Logik nicht vorausnehmen.}
\ogd{What lies in its application logic cannot anticipate.}
\pmc{What belongs to its application, logic cannot anticipate.}

\pnskip
\ger{Das ist klar: Die Logik darf mit ihrer Anwendung nicht kollidieren.}
\ogd{It is clear that logic may not conflict with its application.}
\pmc{It is clear that logic must not clash with its application.}

\pnskip
\ger{Aber die Logik muss sich mit ihrer Anwendung ber{\"u}hren.}
\ogd{But logic must have contact with its application.}
\pmc{But logic has to be in contact with its application.}

\pnskip
\ger{Also d{\"u}rfen die Logik und ihre Anwendung einander nicht {\"u}bergreifen.}
\ogd{Therefore logic and its application may not overlap one another.}
\pmc{Therefore logic and its application must not overlap.}

\pn{5.5571}
\ger{Wenn ich die Elementars{\"a}tze nicht a priori angeben kann, dann muss es zu offenbarem Unsinn f{\"u}hren, sie angeben zu wollen.}
\ogd{If I cannot give elementary propositions a priori then it must lead to obvious nonsense to try to give them.}
\pmc{If I cannot say \emph{a priori} what elementary propositions there are, then the attempt to do so must lead to obvious nonsense.}

\pn{5.6}
\ger{\germph{Die Grenzen meiner Sprache} bedeuten die Grenzen meiner Welt.}
\ogd{\emph{The limits of my language} mean the limits of my world.}
\pmc{\emph{The limits of my language} mean the limits of my world.}

\pn{5.61}
\ger{Die Logik erf{\"u}llt die Welt; die Grenzen der Welt sind auch ihre Grenzen.}
\ogd{Logic fills the world: the limits of the world are also its limits.}
\pmc{Logic pervades the world: the limits of the world are also its limits.}

\pnskip
\ger{Wir k{\"o}nnen also in der Logik nicht sagen: Das und das gibt es in der Welt, jenes nicht.}
\ogd{We cannot therefore say in logic: This and this there is in the world, that there is not.}
\pmc{So we cannot say in logic, `The world has this in it, and this, but not that.'}

\pnskip
\ger{Das w{\"u}rde n{\"a}mlich scheinbar voraussetzen, dass wir gewisse M{\"o}glichkeiten ausschlie{\ss}en, und dies kann nicht der Fall sein, da sonst die Logik {\"u}ber die Grenzen der Welt hinaus m{\"u}sste; wenn sie n{\"a}mlich diese Grenzen auch von der anderen Seite betrachten k{\"o}nnte.}
\ogd{For that would apparently presuppose that we exclude certain possibilities, and this cannot be the case since otherwise logic must get outside the limits of the world: that is, if it could consider these limits from the other side also.}
\pmc{For that would appear to presuppose that we were excluding certain possibilities, and this cannot be the case, since it would require that logic should go beyond the limits of the world; for only in that way could it view those limits from the other side as well.}

\pnskip
\ger{Was wir nicht denken k{\"o}nnen, das k{\"o}nnen wir nicht denken; wir k{\"o}nnen also auch nicht \germph{sagen}, was wir nicht denken k{\"o}nnen.}
\ogd{What we cannot think, that we cannot think: we cannot therefore \emph{say} what we cannot think.}
\pmc{We cannot think what we cannot think; so what we cannot think we cannot \emph{say} either.}

\pn{5.62}
\ger{Diese Bemerkung gibt den Schl{\"u}ssel zur Entscheidung der Frage, inwieweit der Solipsismus eine Wahrheit ist.}
\ogd{This remark provides a key to the question, to what extent solipsism is a truth.}
\pmc{This remark provides the key to the problem, how much truth there is in solipsism.}

\pnskip
\ger{Was der So\-lip\-sis\-mus n{\"a}m\-lich \linebreak{}\germph{meint}, ist ganz rich\-tig, nur l{\"a}sst es sich nicht \germph{sag\-en}, son\-dern es zeigt sich.}
\ogd{In fact what solipsism \emph{means}, is quite correct, only it cannot be \emph{said}, but it shows itself.}
\pmc{For what the solipsist \emph{means} is quite correct; only it cannot be \emph{said}, but makes itself manifest.}

\pnskip
\ger{Dass die Welt \germph{meine} Welt ist, das zeigt sich darin, dass die Grenzen \germph{der} Sprache (der Sprache, die allein ich verstehe) die Grenzen \germph{meiner} Welt bedeuten.}
\ogd{That the world is \emph{my} world, shows itself in the fact that the limits of the language (the language which only I understand) mean the limits of \emph{my} world.}
\pmc{The world is \emph{my} world: this is manifest in the fact that the limits of \emph{language} (of that language which alone I understand) mean the limits of \emph{my} world.}

\pn{5.621}
\ger{Die Welt und das Leben sind Eins.}
\ogd{The world and life are one.}
\pmc{The world and life are one.}

\pn{5.63}
\ger{Ich bin meine Welt. (Der Mikrokosmos.)}
\ogd{I am my world. (The microcosm.)}
\pmc{I am my world. (The microcosm.)}

\pn{5.631}
\ger{Das denkende, vorstellende, Subjekt gibt es nicht.}
\ogd{The thinking, presenting subject; there is no such thing.}
\pmc{There is no such thing as the subject that thinks or entertains ideas.}

\pnskip
\ger{Wenn ich ein Buch schriebe \gdql Die Welt, wie ich sie vorfand\gdqr{}, so w{\"a}re darin auch {\"u}ber meinen Leib zu berichten und zu sagen, welche Glieder meinem Willen unterstehen und welche nicht, etc., dies ist n{\"a}mlich eine Methode, das Subjekt zu isolieren, oder vielmehr zu zeigen, dass es in einem wichtigen Sinne kein Subjekt gibt: Von ihm allein n{\"a}mlich k{\"o}nnte in diesem Buche \germph{nicht} die Rede sein.---}
\ogd{If I wrote a book ``The world as I found it'', I should also have therein to report on my body and say which members obey my will and which do not, etc. This then would be a method of isolating the subject or rather of showing that in an important sense there is no subject: that is to say, of it alone in this book mention could \emph{not} be made.}
\pmc{If I wrote a book called \emph{The World as I found it}, I should have to include a report on my body, and should have to say which parts were subordinate to my will, and which were not, etc., this being a method of isolating the subject, or rather of showing that in an important sense there is no subject; for it alone could \emph{not} be mentioned in that book.---}

\pn{5.632}
\ger{Das Subjekt geh{\"o}rt nicht zur Welt, sondern es ist eine Grenze der Welt.}
\ogd{The subject does not belong to the world but it is a limit of the world.}
\pmc{The subject does not belong to the world: rather, it is a limit of the world.}

\pn{5.633}
\ger{Wo \germph{in} der Welt ist ein metaphysisches Subjekt zu merken?}
\ogd{\emph{Where in} the world is a metaphysical subject to be noted?}
\pmc{Where \emph{in} the world is a metaphysical subject to be found?}

\pnskip
\ger{Du sagst, es verh{\"a}lt sich hier ganz wie mit Auge und Gesichtsfeld. Aber das Auge siehst du wirklich \germph{nicht}.}
\ogd{You say that this case is altogether like that of the eye and the field of sight. But you do \emph{not} really see the eye.}
\pmc{You will say that this is exactly like the case of the eye and the visual field. But really you do \emph{not} see the eye.}

\pnskip
\ger{Und nichts \germph{am Gesichtsfeld} l{\"a}sst darauf schlie{\ss}en, dass es von einem Auge gesehen wird.}
\ogd{And from nothing \emph{in the field of sight} can it be concluded that it is seen from an eye.}
\pmc{And nothing \emph{in the visual field} allows you to infer that it is seen by an eye.}

\pn{5.6331}
\ger{Das Gesichtsfeld hat n{\"a}mlich nicht etwa eine solche Form:
\theeye}
\ogd{For the field of sight has not a form like this: \renewcommand{\eyename}{\textsf{Eye}}\theeye}
\pmc{For the form of the visual field is surely not like this \renewcommand{\eyename}{\textsf{Eye}}\theeye}

\pn{5.634}
\ger{Das h{\"a}ngt damit zusammen, dass kein Teil unserer Erfahrung auch a priori ist.}
\ogd{This is connected with the fact that no part of our experience is also a priori.}
\pmc{This is connected with the fact that no part of our experience is at the same time \textit{a priori}.}

\pnskip
\ger{Alles, was wir sehen, k{\"o}nnte auch anders sein.}
\ogd{Everything we see could also be otherwise.}
\pmc{Whatever we see could be other than it is.}

\pnskip
\ger{Alles, was wir {\"u}berhaupt beschreiben k{\"o}nnen, k{\"o}nnte auch anders sein.}
\ogd{Everything we describe at all could also be otherwise.}
\pmc{Whatever we can describe at all could be other than it is.}

\pnskip
\ger{Es gibt keine Ordnung der Dinge a priori.}
\ogd{There is no order of things a priori.}
\pmc{There is no \textit{a priori} order of things.}

\pn{5.64}
\ger{Hier sieht man, dass der Solipsismus, streng durchgef{\"u}hrt, mit dem reinen Realismus zusammenf{\"a}llt. Das Ich des Solipsismus schrumpft zum ausdehnungslosen Punkt zusammen, und es bleibt die ihm koordinierte Realit{\"a}t.}
\ogd{Here we see that solipsism strictly carried out coincides with pure realism. The I in solipsism shrinks to an extensionless point and there remains the reality co-ordinated with it.}
\pmc{Here it can be seen that solipsism, when its implications are followed out strictly, coincides with pure realism. The self of solipsism shrinks to a point without extension, and there remains the reality co-ordinated with it.}

\pn{5.641}
\ger{Es gibt also wirklich einen Sinn, in welchem in der Philosophie nichtpsychologisch vom Ich die Rede sein kann.}
\ogd{There is therefore really a sense in which the philosophy we can talk of a non-psychological I.}
\pmc{Thus there really is a sense in which philosophy can talk about the self in a non-psychological way.}

\pnskip
\ger{Das Ich tritt in die Philosophie dadurch ein, dass \gdql die Welt meine Welt ist\gdqr.}
\ogd{The I occurs in philosophy through the fact that the ``world is my world''.}
\pmc{What brings the self into philosophy is the fact that `the world is my world'.}

\pnskip
\ger{Das philosophische Ich ist nicht der Mensch, nicht der menschliche K{\"o}rper, oder die menschliche Seele, von der die Psychologie handelt, sondern das metaphysische Subjekt, die Grenze---nicht ein Teil---der Welt.}
\ogd{The philosophical I is not the man, not the human body or the human soul of which psychology treats, but the metaphysical subject, the limit---not a part of the world.}
\pmc{The philosophical self is not the human being, not the human body, or the human soul, with which psychology deals, but rather the metaphysical subject, the limit of the world---not a part of it.}

\pn{6}
\ger{Die allgemeine Form der Wahrheitsfunktion ist: $[\overline{p},\thickspace \overline{\xi},\thickspace \nop(\overline{\xi})]$.}
\ogd{The general form of truth-function is: $[\overline{p},\thickspace \overline{\xi},\thickspace \nop(\overline{\xi})]$.}
\pmc{The general form of a truth-function is  $[\overline{p},\thickspace \overline{\xi},\thickspace \nop(\overline{\xi})]$.}

\pnskip
\ger{Dies ist die allgemeine Form des Satzes.}
\ogd{This is the general form of proposition.}
\pmc{This is the general form of a proposition.}

\pn{6.001}
\ger{Dies sagt nichts anderes, als dass jeder Satz ein Resultat der successiven Anwendung der Operation $\mathop{\nop\text{'}}(\overline{\xi})$ auf die Elementars{\"a}tze ist.}
\ogd{This says nothing else than that every proposition is the result of successive applications of the operation $\mathop{\nop\text{'}}(\overline{\xi})$ to the elementary propositions.}
\pmc{What this says is just that every proposition is a result of successive applications to elementary propositions of the operation $\nop(\overline{\xi})$.}

\pn{6.002}
\ger{Ist die allgemeine Form gegeben, wie ein Satz gebaut ist, so ist damit auch schon die allgemeine Form davon gegeben, wie aus einem Satz durch eine Operation ein anderer erzeugt werden kann.}
\ogd{If we are given the general form of the way in which a proposition is constructed, then thereby we are also given the general form of the way in which by an operation out of one proposition another can be created.}
\pmc{If we are given the general form according to which propositions are constructed, then with it we are also given the general form according to which one proposition can be generated out of another by means of an operation.}

\pn{6.01}
\ger{Die allgemeine Form der Operation $\omop (\overline{\eta})$ ist also: $\mathop{[\overline{\xi},\thickspace \nop(\overline{\xi})]\text{'}} (\overline{\eta})\ (=[\overline{\eta},\thickspace \overline{\xi},\thickspace \nop(\overline{\xi})])$.}
\ogd{The general form of the operation $\omop (\overline{\eta})$ is therefore: $\mathop{[\overline{\xi},\thickspace \nop(\overline{\xi})]\text{'}} (\overline{\eta})\ (=[\overline{\eta},\thickspace \overline{\xi},\thickspace \nop(\overline{\xi})])$.}
\pmc{Therefore the general form of an operation $\omop (\overline{\eta})$ is $\mathop{[\overline{\xi},\thickspace \nop(\overline{\xi})]\text{'}} (\overline{\eta})\ (=[\overline{\eta},\thickspace \overline{\xi},\thickspace \nop(\overline{\xi})])$.}

\pnskip
\ger{Das ist die allgemeinste Form des {\"U}berganges von einem Satz zum anderen.}
\ogd{This is the most general form of transition from one proposition to another.}
\pmc{This is the most general form of transition from one proposition to another.}

\pn{6.02}
\ger{Und \germph{so} kommen wir zu den Zahlen: Ich definiere} 
\ogd{And thus we come to numbers: I define}
\pmc{And \emph{this} is how we arrive at numbers. I give the following definitions}

\pnskip
\ger{\sixzerotwostackonegerman}
\ogd{\sixzerotwostackoneogden}
\pmc{\sixzerotwostackonepmc}

\pnskip
\ger{Nach diesen Zeichenregeln schreiben wir also die Reihe}
\ogd{According, then, to these symbolic rules we write the series}
\pmc{So, in accordance with these rules, which deal with signs, we write the series}

\pnskip
\ger{\[x,\thickspace \omop x,\thickspace \omop \omop x,\thickspace \omop \omop \omop x,\thickspace \dots\]}
\ogd{\[x,\thickspace \omop x,\thickspace \omop \omop x,\thickspace \omop \omop \omop x,\thickspace \dots\]}
\pmc{\[x,\thickspace \omop x,\thickspace \omop \omop x,\thickspace \omop \omop \omop x,\thickspace \dots\]}

\pnskip
\ger{so:}
\ogd{as:}
\pmc{in the following way}

\pnskip
\ger{\[\omop[0] x,\thickspace \omop[0+1] x,\thickspace \omop[0+1+1] x,\thickspace \omop[0+1+1+1] x, \thickspace \dots \]}
\ogd{\[\omop[0] x,\thickspace \omop[0+1] x,\thickspace \omop[0+1+1] x,\thickspace \omop[0+1+1+1] x, \thickspace \dots \]}
\pmc{\[\omop[0] x,\thickspace \omop[0+1] x,\thickspace \omop[0+1+1] x,\thickspace \omop[0+1+1+1] x, \thickspace \dots \]}


\pnskip
\ger{Also schreibe ich statt \gdql $[x,\thickspace \xi,\thickspace \omop \xi]$\gdqr{}:}
\ogd{Therefore I write in place of ``$[x,\thickspace \xi,$ $\omop \xi]$'',}
\pmc{Therefore, instead of `$[x,\thickspace \xi,\thickspace \omop \xi]$',}

\pnskip
\ger{\[\text{\gdql}[\omop[0] x,\thickspace \omop[\nu] x,\thickspace \omop[\nu + 1] x]\text{\gdqr}.\]}
\ogd{\[``[\omop[0] x,\thickspace \omop[\nu] x,\thickspace \omop[\nu + 1] x]\text{''}.\]}
\pmc{\[\text{I write} \qquad \text{`}[\omop[0] x,\thickspace \omop[\nu] x,\thickspace \omop[\nu + 1] x]\text{'}.\]}

\pnskip
\ger{Und definiere:}
\ogd{And I define:}
\pmc{And I give the following definitions}

\pnskip
\ger{\sixzerotwostacktwogerman}
\ogd{\sixzerotwostacktwoogden}
\pmc{\sixzerotwostacktwopmc}

\pn{6.021}
\ger{Die Zahl ist der Exponent einer Operation.}
\ogd{A number is the exponent of an operation.}
\pmc{A number is the exponent of an operation.}

\pn{6.022}
\ger{Der Zahlbegriff ist nichts anderes als das Gemeinsame aller Zahlen, die allgemeine Form der Zahl.}
\ogd{The concept number is nothing else than that which is common to all numbers, the general form of a number.}
\pmc{The concept of number is simply what is common to all numbers, the general form of a number.}

\pnskip
\ger{Der Zahlbegriff ist die variable Zahl.}
\ogd{The concept number is the variable number.}
\pmc{The concept of number is the variable number.}

\pnskip
\ger{Und der Begriff der Zahlengleichheit ist die allgemeine Form aller speziellen Zahlengleichheiten.}
\ogd{And the concept of equality of numbers is the general form of all special equalities of numbers.}
\pmc{And the concept of numerical equality is the general form of all particular cases of numerical equality.}

\pn{6.03}
\ger{Die allgemeine Form der ganzen Zahl ist: $[0, \xi, \xi+1]$.}
\ogd{The general form of the cardinal number is: $[0, \xi, \xi+1]$.}
\pmc{The general form of an integer is $[0, \xi, \xi+1]$.}

\pn{6.031}
\ger{Die Theorie der Klassen ist in der Mathematik ganz {\"u}berfl{\"u}ssig.}
\ogd{The theory of classes is altogether superfluous in mathematics.}
\pmc{The theory of classes is completely superfluous in mathematics.}

\pnskip
\ger{Dies h{\"a}ngt damit zusammen, dass die Allgemeinheit, welche wir in der Mathematik brauchen, nicht die \germph{zuf{\"a}llige} ist.}
\ogd{This is connected with the fact that the generality which we need in mathematics is not the \emph{accidental} one.}
\pmc{This is connected with the fact that the generality required in mathematics is not \emph{accidental} generality.}

\pn{6.1}
\ger{Die S{\"a}tze der Logik sind Tautologien.}
\ogd{The propositions of logic are tautologies.}
\pmc{The propositions of logic are tautologies.}

\pn{6.11}
\ger{Die S{\"a}tze der Logik sagen also nichts. (Sie sind die analytischen S{\"a}tze.)}
\ogd{The propositions of logic therefore say nothing. (They are the analytical propositions.)}
\pmc{Therefore the propositions of logic say nothing. (They are the analytic propositions.)}

\pn{6.111}
\ger{Theorien, die einen Satz der Logik gehaltvoll erscheinen lassen, sind immer falsch. Man k{\"o}nnte z.~B.\ glauben, dass die Worte \gdql wahr\gdqr{} und \gdql falsch\gdqr{} zwei Eigenschaften unter anderen Eigenschaften bezeichnen, und da erschiene es als eine merkw{\"u}rdige Tatsache, dass jeder Satz eine dieser Eigenschaften besitzt. Das scheint nun nichts weniger als selbstverst{\"a}ndlich zu sein, ebensowenig selbstverst{\"a}ndlich, wie etwa der Satz: \gdql Alle Rosen sind entweder gelb oder rot\gdqr{} kl{\"a}nge, auch wenn er wahr w{\"a}re. Ja, jener Satz bekommt nun ganz den Charakter eines naturwissenschaftlichen Satzes, und dies ist das sichere Anzeichen daf{\"u}r, dass er falsch aufgefasst wurde.}
\ogd{Theories which make a proposition of logic appear substantial are always false. One could \emph{e.g.}\ believe that the words ``true'' and ``false'' signify two properties among other properties, and then it would appear as a remarkable fact that every proposition possesses one of these properties. This now by no means appears self-evident, no more so than the proposition ``All roses are either yellow or red'' would seem even if it were true. Indeed our proposition now gets quite the character of a proposition of natural science and this is a certain symptom of its being falsely understood.}
\pmc{All theories that make a proposition of logic appear to have content are false. One might think, for example, that the words `true' and `false' signified two properties among other properties, and then it would seem to be a remarkable fact that every proposition possessed one of these properties. On this theory it seems to be anything but obvious, just as, for instance, the proposition, `All roses are either yellow or red', would not sound obvious even if it were true. Indeed, the logical proposition acquires all the characteristics of a proposition of natural science and this is the sure sign that it has been construed wrongly.}

\pn{6.112}
\ger{Die richtige Erkl{\"a}rung der logischen S{\"a}tze muss ihnen eine einzigartige Stellung unter allen S{\"a}tzen geben.}
\ogd{The correct explanation of logical propositions must give them a peculiar position among all propositions.}
\pmc{The correct explanation of the propositions of logic must assign to them a unique status among all propositions.}

\pn{6.113}
\ger{Es ist das besondere Merkmal der logischen S{\"a}tze, dass man am Symbol allein erkennen kann, dass sie wahr sind, und diese Tatsache schlie{\ss}t die ganze Philosophie der Logik in sich. Und so ist es auch eine derwichtigsten Tatsachen, dass sich die Wahrheit oder Falschheit der nichtlogischen S{\"a}tze \germph{nicht} am Satz allein erkennen l{\"a}sst.}
\ogd{It is the characteristic mark of logical propositions that one can perceive in the symbol alone that they are true; and this fact contains in itself the whole philosophy of logic. And so also it is one of the most important facts that the truth or falsehood of non-logical propositions can \emph{not} be recognized from the propositions alone.}
\pmc{It is the peculiar mark of logical propositions that one can recognize that they are true from the symbol alone, and this fact contains in itself the whole philosophy of logic. And so too it is a very important fact that the truth or falsity of non-logical propositions \emph{cannot} be recognized from the propositions alone.}

\pn{6.12}
\ger{Dass die S{\"a}tze der Logik Tautologien sind, das \germph{zeigt} die formalen---logischen---Eigenschaften der Sprache, der Welt.}
\ogd{The fact that the propositions of logic are tautologies \emph{shows} the formal---logical---properties of language, of the world.}
\pmc{The fact that the propositions of logic are tautologies \emph{shows} the formal---logical---properties of language and the world.}

\pnskip
\ger{Dass ihre Bestandteile \germph{so} verkn{\"u}pft eine Tautologie ergeben, das charakterisiert die Logik ihrer Bestandteile.}
\ogd{That its constituent parts connected together \emph{in this way} give a tautology characterizes the logic of its constituent parts.}
\pmc{The fact that a tautology is yielded by \emph{this particular way} of connecting its constituents characterizes the logic of its constituents.}

\pnskip
\ger{Damit S{\"a}tze, auf bestimmte Art und Weise verkn{\"u}pft, eine Tautologie ergeben, dazu m{\"u}ssen sie bestimmte Eigenschaften der Struktur haben. Dass sie \germph{so} verbunden eine Tautologie ergeben, zeigt also, dass sie diese Eigenschaften der Struktur besitzen.}
\ogd{In order that propositions connected together in a definite way may give a tautology they must have definite properties of structure. That they give a tautology when \emph{so} connected shows therefore that they possess these properties of structure.}
\pmc{If propositions are to yield a tautology when they are connected in a certain way, they must have certain structural properties. So their yielding a tautology when combined \emph{in this way} shows that they possess these structural properties.}

\pn{6.1201}
\ger{Dass z.~B.\ die S{\"a}tze \gdql $p$\gdqr{} und \gdql $\rnot p$\gdqr{} in der Verbindung \gdql $\rnot (p \rand \rnot p)$\gdqr{} eine Tautologie ergeben, zeigt, dass sie einander widersprechen. Dass die S{\"a}tze \gdql $p \rimplies q$\gdqr{}, \gdql $p$\gdqr{} und \gdql $q$\gdqr{} in der Form \gdql $(p \rimplies q) \rand (p) \ddrimpliesdd (q)$\gdqr{} miteinander verbunden eine Tautologie ergeben, zeigt, dass $q$ aus $p$ und $p \rimplies q$ folgt. Dass \gdql $\ralld{x} f\negthinspace x \ddrimpliesdd f\negthinspace a$\gdqr{} eine Tautologie ist, dass $f\negthinspace a$ aus $\ralld{x} f\negthinspace x$ folgt.\ etc.\ etc.}
\ogd{That \emph{e.g.}\ the propositions ``$p$'' and ``$\rnot p$'' in the connexion ``$\rnot (p \rand \rnot p)$'' give a tautology shows that they contradict one another. That the propositions ``$p \rimplies q$'', ``$p$'' and ``$q$'' connected together in the form ``$(p \rimplies q) \rand (p) \ddrimpliesdd (q)$'' give a tautology shows that $q$ follows from $p$ and $p \rimplies q$. That ``$\ralld{x} f\negthinspace x \ddrimpliesdd f\negthinspace a$'' is a tautology shows that $f\negthinspace a$ follows from $\ralld{x} f\negthinspace x$, etc.\ etc.}
\pmc{For example, the fact that the propositions `$p$' and `$\rnot p$' in the combination `$\rnot (p \rand \rnot p)$' yield a tautology shows that they contradict one another. The fact that the propositions `$p \rimplies q$', `$p$', and `$q$', combined with one another in the form `$(p \rimplies q) \rand (p) \ddrimpliesdd (q)$', yield a tautology shows that $q$ follows from $p$ and $p \rimplies q$. The fact that `$\ralld{x} f\negthinspace x \ddrimpliesdd f\negthinspace a$' is a tautology shows that $f\negthinspace a$ follows from $\ralld{x} f\negthinspace x$. Etc.\ etc.}

\pn{6.1202}
\ger{Es ist klar, dass man zu demselben Zweck statt der Tautologien auch die Kontradiktionen verwenden k{\"o}nnte.}
\ogd{It is clear that we could have used for this purpose contradictions instead of tautologies.}
\pmc{It is clear that one could achieve the same purpose by using contradictions instead of tautologies.}

\pn{6.1203}
\ger{Um eine Tautologie als solche zu erkennen, kann man sich, in den F{\"a}llen, in welchen in der Tautologie keine Allgemeinheitsbezeichnung vorkommt, folgender anschaulichen Methode bedienen: Ich schreibe statt \gdql $p$\gdqr{}, \gdql $q$\gdqr{}, \gdql $r$\gdqr{} etc.\ \gdql $\mathrm{W}\thinspace p\thinspace \mathrm{F}$\gdqr{}, \gdql $\mathrm{W}\thinspace q\thinspace \mathrm{F}$\gdqr{}, \gdql $\mathrm{W}\thinspace r\thinspace \mathrm{F}$\gdqr{} etc. Die Wahrheitskombinationen dr{\"u}cke ich durch Klammern aus, z.~B.:}
\ogd{In order to recognize a tautology as such, we can, in cases in which no sign of generality occurs in the tautology, make use of the following intuitive method: I write instead of ``$p$'', ``$q$'', ``$r$'', etc., ``$\mathrm{T}\thinspace p\thinspace \mathrm{F}$'', ``$\mathrm{T}\thinspace q\thinspace \mathrm{F}$'', ``$\mathrm{T}\thinspace r\thinspace \mathrm{F}$'', etc. The truth-combinations I express by brackets, \emph{e.g.}:}
\pmc{In order to recognize an expression as a tautology, in cases where no generality-sign occurs in it, one can employ the following intuitive method: instead of `$p$', `$q$', `$r$', etc.\ I write `$\mathrm{T}\thinspace p\thinspace \mathrm{F}$', `$\mathrm{T}\thinspace q\thinspace \mathrm{F}$', `$\mathrm{T}\thinspace r\thinspace \mathrm{F}$', etc. Truth-combinations I express by means of brackets, e.g.}

\pnskip
\ger{\negpbk\abfigureonegerman}
\ogd{\negpbk\abfigureoneenglish}
\pmc{\negpbk\abfigureoneenglishpmc}

\pnskip
\ger{\negpbk und die Zuordnung der Wahr- oder Falschheit des ganzen Satzes und der Wahrheitskombinationen der Wahrheitsargumente durch Striche auf folgende Weise:}
\ogd{\negpbk and the co-ordination of the truth or falsity of the whole proposition with the truth-combinations of the truth-arguments by lines in the following way:}
\pmc{\negpbk and I use lines to express the correlation of the truth or falsity of the whole proposition with the truth-combinations of its truth-arguments, in the following way}

\pnskip
\ger{\negpbk\abfiguretwogerman}
\ogd{\negpbk\abfiguretwoenglish}
\pmc{\negpbk\abfiguretwoenglishpmc}

\pnskip
\ger{\negpbk Dies Zeichen w{\"u}rde also z.~B.\ den Satz $p \rimplies q$ darstellen. Nun will ich z.~B.\ den Satz $\rnot(p\rand \rnot p)$ (Gesetz des Widerspruchs) daraufhin untersuchen, ob er eine Tautologie ist. Die Form \gdql $\rnot \xi$\gdqr{} wird in unserer Notation}
\ogd{\negpbk This sign, for example, would therefore present the proposition $p \rimplies q$. Now I will proceed to inquire whether such a proposition as $\rnot(p \rand \rnot p)$ (The Law of Contradiction) is a tautology. The form ``$\rnot \xi$'' is written in our notation}
\pmc{\negpbk So this sign, for instance, would represent the proposition $p \rimplies q$. Now, by way of example, I wish to examine the proposition $\rnot (p \rand \rnot p)$ (the law of contradiction) in order to determine whether it is a tautology. In our notation the form `$\rnot \xi$' is written as}

\pnskip
\ger{\negpbk\abfigurethreegerman}
\ogd{\negpbk\abfigurethreeenglish}
\pmc{\negpbk\abfigurethreeenglishpmc}

\pnskip
\ger{\negpbk geschrieben; die Form \gdql $\xi \rand \eta$\gdqr{} so:}
\ogd{\negpbk the form ``$\xi \rand \eta$'' thus:---}
\pmc{\negpbk and the form `$\xi \rand \eta$' as}

\pnskip
\ger{\negpbk\abfigurefourgerman}
\ogd{\negpbk\abfigurefourenglish}
\pmc{\negpbk\abfigurefourenglishpmc}

\pnskip
\ger{\negpbk{}Daher lautet der Satz $\rnot (p \rand \rnot q)$ so:}
\ogd{\negpbk{}Hence the proposition $\rnot (p \rand \rnot q)$ runs thus:---}
\pmc{\negpbk Hence the proposition $\rnot (p \rand \rnot q)$ reads as follows}

\pnskip
\ger{\negpbk\abfigurefivegerman}
\ogd{\negpbk\abfigurefiveenglish}
\pmc{\negpbk\abfigurefiveenglishpmc}

\pnskip
\ger{\negpbk Setzen wir statt \gdql $q$\gdqr{} \gdql $p$\gdqr{} ein und untersuchen die Verbindung der {\"a}u{\ss}ersten W und F mit den innersten, so ergibt sich, dass die Wahrheit des ganzen Satzes \germph{allen} Wahrheitskombinationen seines Argumentes, seine Falschheit keiner der Wahrheitskombinationen zugeordnet ist.}
\ogd{\negpbk If here we put ``$p$'' instead of ``$q$'' and examine the combination of the outermost T and F with the innermost, it is seen that the truth of the whole proposition is co-ordinated with \emph{all} the truth-combinations of its argument, its falsity with none of the truth-combinations.}
\pmc{\negpbk If we here substitute `$p$' for `$q$' and examine how the outermost T and F are connected with the innermost ones, the result will be that the truth of the whole proposition is correlated with \emph{all} the truth-combinations of its argument, and its falsity with none of the truth-combinations.}

\pn{6.121}
\ger{Die S{\"a}tze der Logik demonstrieren die logischen Eigenschaften der S{\"a}tze, indem sie sie zu nichtssagenden S{\"a}tzen verbinden.}
\ogd{The propositions of logic demonstrate the logical properties of propositions, by combining them into propositions which say nothing.}
\pmc{The propositions of logic demonstrate the logical properties of propositions by combining them so as to form propositions that say nothing.}

\pnskip
\ger{Diese Methode k{\"o}nnte man auch eine Nullmethode nennen. Im logischen Satz werden S{\"a}tze miteinander ins Gleichgewicht gebracht und der Zustand des Gleichgewichts zeigt dann an, wie diese S{\"a}tze logisch beschaffen sein m{\"u}ssen.}
\ogd{This method could be called a zero-method. In a logical proposition propositions are brought into equilibrium with one another, and the state of equilibrium then shows how these propositions must be logically constructed.}
\pmc{This method could also be called a zero-method. In a logical proposition, propositions are brought into equilibrium with one another, and the state of equilibrium then indicates what the logical constitution of these propositions must be.}

\pn{6.122}
\ger{Daraus ergibt sich, dass wir auch ohne die logischen S{\"a}tze auskommen k{\"o}nnen, da wir ja in einer entsprechenden Notation die formalen Eigenschaften der S{\"a}tze durch das blo{\ss}e Ansehen dieser S{\"a}tze erkennen k{\"o}nnen.}
\ogd{Whence it follows that we can get on without logical propositions, for we can recognize in an adequate notation the formal properties of the propositions by mere inspection.}
\pmc{It follows from this that we can actually do without logical propositions; for in a suitable notation we can in fact recognize the formal properties of propositions by mere inspection of the propositions themselves.}

\pn{6.1221}
\ger{Ergeben z.~B.\ zwei S{\"a}tze \gdql $p$\gdqr{} und \gdql $q$\gdqr{} in der Verbindung \gdql $p \rimplies q$\gdqr{} eine Tautologie, so ist klar, dass $q$ aus $p$ folgt.}
\ogd{If for example two propositions ``$p$'' and ``$q$'' give a tautology in the connexion ``$p \rimplies q$'', then it is clear that $q$ follows from $p$.}
\pmc{If, for example, two propositions `$p$' and `$q$' in the combination `$p \rimplies q$' yield a tautology, then it is clear that $q$ follows from $p$.}

\pnskip
\ger{Dass z.~B.\ \gdql $q$\gdqr{} aus \gdql $p \rimplies q \rand p$\gdqr{} folgt, ersehen wir aus diesen beiden S{\"a}tzen selbst, aber wir k{\"o}nnen es auch \germph{so} zeigen, indem wir sie zu \gdql $p \rimplies q \rand p \ddrimpliesdd q$\gdqr{} verbinden und nun zeigen, dass dies eine Tautologie ist.}
\ogd{\emph{E.g.}\ that ``$q$'' follows from ``$p \rimplies q \rand p$'' we see from these two propositions themselves, but we can also show it by combining them to ``$p \rimplies q \rand p \ddrimpliesdd q$'' and then showing that this is a tautology.}
\pmc{For example, we see from the two propositions themselves that `$q$' follows from `$p \rimplies q \rand p$', but it is also possible to show it in \emph{this} way: we combine them to form `$p \rimplies q \rand p \ddrimpliesdd q$', and then show that this is a tautology.}

\pn{6.1222}
\ger{Dies wirft ein Licht auf die Frage, warum die logischen S{\"a}tze nicht durch die Erfahrung best{\"a}tigt werden k{\"o}nnen, ebensowenig wie sie durch die Erfahrung widerlegt werden k{\"o}nnen. Nicht nur muss ein Satz der Logik durch keine m{\"o}gliche Erfahrung widerlegt werden k{\"o}nnen, sondern er darf auch nicht durch eine solche best{\"a}tigt werden k{\"o}nnen.}
\ogd{This throws light on the question why logical propositions can no more be empirically established than they can be empirically refuted. Not only must a proposition of logic be incapable of being contradicted by any possible experience, but it must also be incapable of being established by any such.}
\pmc{This throws some light on the question why logical propositions cannot be confirmed by experience any more than they can be refuted by it. Not only must a proposition of logic be irrefutable by any possible experience, but it must also be unconfirmable by any possible experience.}

\pn{6.1223}
\ger{Nun wird klar, warum man oft f{\"u}hlte, als w{\"a}ren die \gdql logischen Wahrheiten\gdqr{} von uns zu \gdql \germph{fordern}\gdqr{}: Wir k{\"o}nnen sie n{\"a}mlich insofern fordern, als wir eine gen{\"u}gende Notation fordern k{\"o}nnen.}
\ogd{It now becomes clear why we often feel as though ``logical truths'' must be ``\emph{postulated}'' by us. We can in fact postulate them in so far as we can postulate an adequate notation.}
\pmc{Now it becomes clear why people have often felt as if it were for us to `\emph{postulate}' the `truths of logic'. The reason is that we can postulate them in so far as we can postulate an adequate notation.}

\pn{6.1224}
\ger{Es wird jetzt auch klar, warum die Logik die Lehre von den Formen und vom Schlie{\ss}en genannt wurde.}
\ogd{It also becomes clear why logic has been called the theory of forms and of inference.}
\pmc{It also becomes clear now why logic was called the theory of forms and of inference.}

\pn{6.123}
\ger{Es ist klar: Die logischen Gesetze d{\"u}rfen nicht selbst wieder logischen Gesetzen unterstehen.}
\ogd{It is clear that the laws of logic cannot themselves obey further logical laws.}
\pmc{Clearly the laws of logic cannot in their turn be subject to laws of logic.}

\pnskip
\ger{(Es gibt nicht, wie Russell meinte, f{\"u}r jede \gdql Type\gdqr{} ein eigenes Gesetz des Widerspruches, sondern Eines gen{\"u}gt, da es auf sich selbst nicht angewendet wird.)}
\ogd{(There is not, as Russell supposed, for every ``type'' a special law of contradiction; but one is sufficient, since it is not applied to itself.)}
\pmc{(There is not, as Russell thought, a special law of contradiction for each `type'; one law is enough, since it is not applied to itself.)}

\pn{6.1231}
\ger{Das Anzeichen des logischen Satzes ist \germph{nicht} die Allgemeing{\"u}ltigkeit.}
\ogd{The mark of logical propositions is not their general validity.}
\pmc{The mark of a logical proposition is \emph{not} general validity.}

\pnskip
\ger{Allgemein sein hei{\ss}t ja nur: zuf{\"a}lligerweise f{\"u}r alle Dinge gelten. Ein unverallgemeinerter Satz kann ja ebensowohl tautologisch sein als ein verallgemeinerter.}
\ogd{To be general is only to be accidentally valid for all things. An ungeneralized proposition can be tautologous just as well as a generalized one.}
\pmc{To be general means no more than to be accidentally valid for all things. An ungeneralized proposition can be tautological just as well as a generalized one.}

\pn{6.1232}
\ger{Die logische Allgemeing{\"u}ltigkeit k{\"o}nnte man wesentlich nennen, im Gegensatz zu jener zuf{\"a}lligen, etwa des Satzes: \gdql Alle Menschen sind sterblich\gdqr{}. S{\"a}tze wie Russells \gdql Axiom of Reducibility\gdqr{} sind nicht logische S{\"a}tze, und dies erkl{\"a}rt unser Gef{\"u}hl: Dass sie, wenn wahr, so doch nur durch einen g{\"u}nstigen Zufall wahr sein k{\"o}nnten.}
\ogd{Logical general validity, we could call essential as opposed to accidental general validity, \emph{e.g.}\ of the proposition ``all men are mortal''. Propositions like Russell's ``axiom of reducibility'' are not logical propositions, and this explains our feeling that, if true, they can only be true by a happy chance.}
\pmc{The general validity of logic might be called essential, in contrast with the accidental general validity of such propositions as `All men are mortal'. Propositions like Russell's `axiom of reducibility' are not logical propositions, and this explains our feeling that, even if they were true, their truth could only be the result of a fortunate accident.}

\pn{6.1233}
\ger{Es l{\"a}sst sich eine Welt denken, in der das Axiom of Reducibility nicht gilt. Es ist aber klar, dass die Logik nichts mit der Frage zu schaffen hat, ob unsere Welt wirklich so ist oder nicht.}
\ogd{We can imagine a world in which the axiom of reducibility is not valid. But it is clear that logic has nothing to do with the question whether our world is really of this kind or not.}
\pmc{It is possible to imagine a world in which the axiom of reducibility is not valid. It is clear, however, that logic has nothing to do with the question whether our world really is like that or not.}

\pn{6.124}
\ger{Die logischen S{\"a}tze beschreiben das Ger{\"u}st der Welt, oder vielmehr, sie stellen es dar. Sie \gdql handeln\gdqr{} von nichts. Sie setzen voraus, dass Namen Bedeutung, und Elementars{\"a}tze Sinn haben: Und dies ist ihre Verbindung mit der Welt. Es ist klar, dass es etwas {\"u}ber die Welt anzeigen muss, dass gewisse Verbindungen von Symbolen---welche wesentlich einen bestimmten Charakter haben---Tautologien sind. Hierin liegt das Entscheidende. Wir sagten, manches an den Symbolen, die wir gebrauchen, w{\"a}re willk{\"u}rlich, manches nicht. In der Logik dr{\"u}ckt nur dieses aus: Das hei{\ss}t aber, in der Logik dr{\"u}cken nicht \germph{wir} mit Hilfe der Zeichen aus, was wir wollen, sondern in der Logik sagt die Natur der naturnotwendigen Zeichen selbst aus: Wenn wir die logische Syntax irgendeiner Zeichensprache kennen, dann sind bereits alle S{\"a}tze der Logik gegeben.}
\ogd{The logical propositions describe the scaffolding of the world, or rather they present it. They ``treat'' of nothing. They presuppose that names have meaning, and that elementary propositions have sense. And this is their connexion with the world. It is clear that it must show something about the world that certain combinations of symbols---which essentially have a definite character---are tautologies. Herein lies the decisive point. We said that in the symbols which we use much is arbitrary, much not. In logic only this expresses: but this means that in logic it is not \emph{we} who express, by means of signs, what we want, but in logic the nature of the essentially necessary signs itself asserts. That is to say, if we know the logical syntax of any sign language, then all the propositions of logic are already given.}
\pmc{The propositions of logic describe the scaffolding of the world, or rather they represent it. They have no `subject-matter'. They presuppose that names have meaning and elementary propositions sense; and that is their connexion with the world. It is clear that something about the world must be indicated by the fact that certain combinations of symbols---whose essence involves the possession of a determinate character---are tautologies. This contains the decisive point. We have said that some things are arbitrary in the symbols that we use and that some things are not. In logic it is only the latter that express: but that means that logic is not a field in which \emph{we} express what we wish with the help of signs, but rather one in which the nature of the natural and inevitable signs speaks for itself. If we know the logical syntax of any sign-language, then we have already been given all the propositions of logic.}

\pn{6.125}
\ger{Es ist m{\"o}glich, und zwar auch nach der alten Auffassung der Logik, von vornherein eine Beschreibung aller \gdql wahren\gdqr{} logischen S{\"a}tze zu geben.}
\ogd{It is possible, even in the old logic, to give at the outset a description of all ``true'' logical propositions.}
\pmc{It is possible---indeed possible even according to the old conception of logic---to give in advance a description of all `true' logical propositions.}

\pn{6.1251}
\ger{Darum kann es in der Logik auch \germph{nie} {\"U}berraschungen geben.}
\ogd{Hence there can \emph{never} be surprises in logic.}
\pmc{Hence there can \emph{never} be surprises in logic.}

\pn{6.126}
\ger{Ob ein Satz der Logik angeh{\"o}rt, kann man berechnen, indem man die logischen Eigenschaften des \germph{Symbols} berechnet.}
\ogd{Whether a proposition belongs to logic can be calculated by calculating the logical properties of the \emph{symbol}.}
\pmc{One can calculate whether a proposition belongs to logic, by calculating the logical properties of the \emph{symbol}.}

\pnskip
\ger{Und dies tun wir, wenn wir einen logischen Satz \gdql beweisen\gdqr{}. Denn, ohne uns um einen Sinn und eine Bedeutung zu k{\"u}mmern, bilden wir den logischen Satz aus anderen nach blo{\ss}en \germph{Zeichenregeln}.}
\ogd{And this we do when we prove a logical proposition. For without troubling ourselves about a sense and a meaning, we form the logical propositions out of others by mere \emph{symbolic rules}.}
\pmc{And this is what we do when we `prove' a logical proposition. For, without bothering about sense or meaning, we construct the logical proposition out of others using only \emph{rules that deal with signs}.}

\pnskip
\ger{Der Beweis der logischen S{\"a}tze besteht darin, dass wir sie aus anderen logischen S{\"a}tzen durch successive Anwendung gewisser Operationen entstehen lassen, die aus den ersten immer wieder Tautologien erzeugen. (Und zwar \germph{folgen} aus einer Tautologie nur Tautologien.)}
\ogd{We prove a logical proposition by creating it out of other logical propositions by applying in succession certain operations, which again generate tautologies out of the first. (And from a tautology only tautologies \emph{follow}.)}
\pmc{The proof of logical propositions consists in the following process: we produce them out of other logical propositions by successively applying certain operations that always generate further tautologies out of the initial ones. (And in fact only tautologies \emph{follow} from a tautology.)}

\pnskip
\ger{Nat{\"u}rlich ist diese Art zu zeigen, dass ihre S{\"a}tze Tautologien sind, der Logik durchaus unwesentlich. Schon darum, weil die S{\"a}tze, von welchen der Beweis ausgeht, ja ohne Beweis zeigen m{\"u}ssen, dass sie Tautologien sind.}
\ogd{Naturally this way of showing that its propositions are tautologies is quite unessential to logic. Because the propositions, from which the proof starts, must show without proof that they are tautologies.}
\pmc{Of course this way of showing that the propositions of logic are tautologies is not at all essential to logic, if only because the propositions from which the proof starts must show without any proof that they are tautologies.}

\pn{6.1261}
\ger{In der Logik sind Prozess und Resultat {\"a}quivalent. (Darum keine {\"U}berraschung.)}
\ogd{In logic process and result are equivalent. (Therefore no surprises.)}
\pmc{In logic process and result are equivalent. (Hence the absence of surprise.)}

\pn{6.1262}
\ger{Der Beweis in der Logik ist nur ein mechanisches Hilfsmittel zum leichteren Erkennen der Tautologie, wo sie kompliziert ist.}
\ogd{Proof in logic is only a mechanical expedient to facilitate the recognition of tautology, where it is complicated.}
\pmc{Proof in logic is merely a mechanical expedient to facilitate the recognition of tautologies in complicated cases.}

\pn{6.1263}
\ger{Es w{\"a}re ja auch zu merkw{\"u}rdig, wenn man einen sinnvollen Satz \germph{logisch} aus anderen beweisen k{\"o}nnte, und einen logischen Satz \germph{auch}. Es ist von vornherein klar, dass der logische Beweis eines sinnvollen Satzes und der Beweis \germph{in} der Logik zwei ganz verschiedene Dinge sein m{\"u}ssen.}
\ogd{It would be too remarkable, if one could prove a significant proposition \emph{logically} from another, and a logical proposition \emph{also}. It is clear from the beginning that the logical proof of a significant proposition and the proof\thickspace \emph{in} logic must be two quite different things.}
\pmc{Indeed, it would be altogether too remarkable if a proposition that had sense could be proved \emph{logically} from others, and \emph{so too} could a logical proposition. It is clear from the start that a logical proof of a proposition that has sense and a proof \emph{in} logic must be two entirely different things.}

\pn{6.1264}
\ger{Der sinnvolle Satz sagt etwas aus, und sein Beweis zeigt, dass es so ist; in der Logik ist jeder Satz die Form eines Beweises.}
\ogd{The significant proposition asserts something, and its proof shows that it is so; in logic every proposition is the form of a proof.}
\pmc{A proposition that has sense states something, which is shown by its proof to be so. In logic every proposition is the form of a proof.}

\pnskip
\ger{Jeder Satz der Logik ist ein in Zeichen dargestellter modus ponens. (Und den modus ponens kann man nicht durch einen Satz ausdr{\"u}cken.)}
\ogd{Every proposition of logic is a modus ponens presented in signs. (And the modus ponens can not be expressed by a proposition.)}
\pmc{Every proposition of logic is a \emph{modus ponens} represented in signs. (And one cannot express the \emph{modus ponens} by means of a proposition.)}

\pn{6.1265}
\ger{Immer kann man die Logik so auffassen, dass jeder Satz sein eigener Beweis ist.}
\ogd{Logic can always be conceived to be such that every proposition is its own proof.}
\pmc{It is always possible to construe logic in such a way that every proposition is its own proof.}

\pn{6.127}
\ger{Alle S{\"a}tze der Logik sind gleichberechtigt, es gibt unter ihnen nicht wesentlich Grundgesetze und abgeleitete S{\"a}tze.}
\ogd{All propositions of logic are of equal rank; there are not some which are essentially primitive and others deduced from there.}
\pmc{All the propositions of logic are of equal status: it is not the case that some of them are essentially derived propositions.}

\pnskip
\ger{Jede Tautologie zeigt selbst, dass sie eine Tautologie ist.}
\ogd{Every tautology itself shows that it is a tautology.}
\pmc{Every tautology itself shows that it is a tautology.}

\pn{6.1271}
\ger{Es ist klar, dass die Anzahl der \gdql logischen Grundgesetze\gdqr{} willk{\"u}rlich ist, denn man k{\"o}nnte die Logik ja aus Einem Grundgesetz ableiten, indem man einfach z.\ B.\ aus Freges Grundgesetzen das logische Produkt bildet. (Frege w{\"u}rde vielleicht sagen, dass dieses Grundgesetz nun nicht mehr unmittelbar einleuchte. Aber es ist merkw{\"u}rdig, dass ein so exakter Denker wie Frege sich auf den Grad des Einleuchtens als Kriterium des logischen Satzes berufen hat.)}
\ogd{It is clear that the number of ``primitive propositions of logic'' is arbitrary, for we could deduce logic from one primitive proposition by simply forming, for example, the logical produce of Frege's primitive propositions. (Frege would perhaps say that this would no longer be immediately self-evident. But it is remarkable that so exact a thinker as Frege should have appealed to the degree of self-evidence as the criterion of a logical proposition.)}
\pmc{It is clear that the number of the `primitive propositions of logic' is arbitrary, since one could derive logic from a single primitive proposition, e.g.\ by simply constructing the logical product of Frege's primitive propositions. (Frege would perhaps say that we should then no longer have an immediately self-evident primitive proposition. But it is remarkable that a thinker as rigorous as Frege appealed to the degree of self-evidence as the criterion of a logical proposition.)}

\pn{6.13}
\ger{Die Logik ist keine Lehre, sondern ein Spiegelbild der Welt.}
\ogd{Logic is not a theory but a reflexion of the world.}
\pmc{Logic is not a body of doctrine, but a mirror-image of the world.}

\pnskip
\ger{Die Logik ist transzendental.}
\ogd{Logic is transcendental.}
\pmc{Logic is transcendental.}

\pn{6.2}
\ger{Die Mathematik ist eine logische Methode.}
\ogd{Mathematics is a logical method.}
\pmc{Mathematics is a logical method.}

\pnskip
\ger{Die S{\"a}tze der Mathematik sind Gleichungen, also Scheins{\"a}tze.}
\ogd{The propositions of mathematics are equations, and therefore pseudo-propositions.}
\pmc{The propositions of mathematics are equations, and therefore pseudo-propositions.}

\pn{6.21}
\ger{Der Satz der Mathematik dr{\"u}ckt keinen Gedanken aus.}
\ogd{Mathematical propositions express no thoughts.}
\pmc{A proposition of mathematics does not express a thought.}

\pn{6.211}
\ger{Im Leben ist es ja nie der mathematische Satz, den wir brauchen, sondern wir ben{\"u}tzen den mathematischen Satz \germph{nur}, um aus S{\"a}tzen, welche nicht der Mathematik angeh{\"o}ren, auf andere zu schlie{\ss}en, welche gleichfalls nicht der Mathematik angeh{\"o}ren.}
\ogd{In life it is never a mathematical proposition which we need, but we use mathematical propositions \emph{only} in order to infer from propositions which do not belong to mathematics to others which equally do not belong to mathematics.}
\pmc{Indeed in real life a mathematical proposition is never what we want. Rather, we make use of mathematical propositions \emph{only} in inferences from propositions that do not belong to mathematics to others that likewise do not belong to mathematics.}

\pnskip
\ger{(In der Philosophie f{\"u}hrt die Frage: \gdql Wozu gebrauchen wir eigentlich jenes Wort, jenen Satz\gdqr{} immer wieder zu wertvollen Einsichten.)}
\ogd{(In philosophy the question ``Why do we really use that word, that proposition?''\ constantly leads to valuable results.)}
\pmc{(In philosophy the question, `What do we actually use this word or this proposition for?' repeatedly leads to valuable insights.)}

\pn{6.22}
\ger{Die Logik der Welt, die die S{\"a}tze der Logik in den Tautologien zeigen, zeigt die Mathematik in den Gleichungen.}
\ogd{The logic of the world which the propositions of logic show in tautologies, mathematics shows in equations.}
\pmc{The logic of the world, which is shown in tautologies by the propositions of logic, is shown in equations by mathematics.}

\pn{6.23}
\ger{Wenn zwei Ausdr{\"u}cke durch das Gleichheitszeichen verbunden werden, so hei{\ss}t das, sie sind durch einander ersetzbar. Ob dies aber der Fall ist, muss sich an den beiden Ausdr{\"u}cken selbst zeigen.}
\ogd{If two expressions are connected by the sign of equality, this means that they can be substituted for one another. But whether this is the case must show itself in the two expressions themselves.}
\pmc{If two expressions are combined by means of the sign of equality, that means that they can be substituted for one another. But it must be manifest in the two expressions themselves whether this is the case or not.}

\pnskip
\ger{Es charakterisiert die logische Form zweier Ausdr{\"u}cke, dass sie durch einander ersetzbar sind.}
\ogd{It characterizes the logical form of two expressions, that they can be substituted for one another.}
\pmc{When two expressions can be substituted for one another, that characterizes their logical form.}

\pn{6.231}
\ger{Es ist eine Eigenschaft der Bejahung, dass man sie als doppelte Verneinung auffassen kann.}
\ogd{It is a property of affirmation that it can be conceived as double denial.}
\pmc{It is a property of affirmation that it can be construed as double negation.}

\pnskip
\ger{Es ist eine Eigenschaft von \gdql $1+1+1+1$\gdqr{}, dass man es als \gdql $(1+1)+(1+1)$\gdqr{} auffassen kann.}
\ogd{It is a property of ``$1+1+1+1$'' that it can be conceived as ``$(1+1)+(1+1)$''.}
\pmc{It is a property of `$1 + 1 + 1 + 1$' that it can be construed as `$(1 + 1) + (1 + 1)$'.}

\pn{6.232}
\ger{Frege sagt, die beiden Ausdr{\"u}cke haben dieselbe Bedeutung, aber verschiedenen Sinn.}
\ogd{Frege says that these expressions have the same meaning but different senses.}
\pmc{Frege says that the two expressions have the same meaning but different senses.}

\pnskip
\ger{Das Wesentliche an der Gleichung ist aber, dass sie nicht notwendig ist, um zu zeigen, dass die beiden Ausdr{\"u}cke, die das Gleichheitszeichen verbindet, dieselbe Bedeutung haben, da sich dies aus den beiden Ausdr{\"u}cken selbst ersehen l{\"a}sst.}
\ogd{But what is essential about equation is that it is not necessary in order to show that both expressions, which are connected by the sign of equality, have the same meaning: for this can be perceived from the two expressions themselves.}
\pmc{But the essential point about an equation is that it is not necessary in order to show that the two expressions connected by the sign of equality have the same meaning, since this can be seen from the two expressions themselves.}

\pn{6.2321}
\ger{Und, dass die S{\"a}tze der Mathematik bewiesen werden k{\"o}nnen, hei{\ss}t ja nichts anderes, als dass ihre Richtigkeit einzusehen ist, ohne dass das, was sie ausdr{\"u}cken, selbst mit den Tatsachen auf seine Richtigkeit hin verglichen werden muss.}
\ogd{And, that the propositions of mathematics can be proved means nothing else than that their correctness can be seen without our having to compare what they express with the facts as regards correctness.}
\pmc{And the possibility of proving the propositions of mathematics means simply that their correctness can be perceived without its being necessary that what they express should itself be compared with the facts in order to determine its correctness.}

\pn{6.2322}
\ger{Die Identit{\"a}t der Bedeutung zweier Ausdr{\"u}cke l{\"a}sst sich nicht \germph{behaupten}. Denn, um etwas von ihrer Bedeutung behaupten zu k{\"o}nnen, muss ich ihre Bedeutung kennen: und indem ich ihre Bedeutung kenne, wei{\ss} ich, ob sie dasselbe oder verschiedenes bedeuten.}
\ogd{The identity of the meaning of two expressions cannot be \emph{asserted}. For in order to be able to assert anything about their meaning, I must know their meaning, and if I know their meaning, I know whether they mean the same or something different.}
\pmc{It is impossible to \emph{assert} the identity of meaning of two expressions. For in order to be able to assert anything about their meaning, I must know their meaning, and I cannot know their meaning without knowing whether what they mean is the same or different.}

\pn{6.2323}
\ger{Die Gleichung kennzeichnet nur den Standpunkt, von welchem ich die beiden Ausdr{\"u}cke betrachte, n{\"a}mlich vom Standpunkte ihrer Bedeutungsgleichheit.}
\ogd{The equation characterizes only the standpoint from which I consider the two expressions, that is to say the standpoint of their equality of meaning.}
\pmc{An equation merely marks the point of view from which I consider the two expressions: it marks their equivalence in meaning.}

\pn{6.233}
\ger{Die Frage, ob man zur L{\"o}sung der mathematischen Probleme die Anschauung brauche, muss dahin beantwortet werden, dass eben die Sprache hier die n{\"o}tige Anschauung liefert.}
\ogd{To the question whether we need intuition for the solution of mathematical problems it must be answered that language itself here supplies the necessary intuition.}
\pmc{The question whether intuition is needed for the solution of mathematical problems must be given the answer that in this case language itself provides the necessary intuition.}

\pn{6.2331}
\ger{Der Vorgang des \germph{Rechnens} vermittelt eben diese Anschauung.}
\ogd{The process of calculation brings about just this intuition.}
\pmc{The process of \emph{calculating} serves to bring about that intuition.}

\pnskip
\ger{Die Rechnung ist kein Experiment.}
\ogd{Calculation is not an experiment.}
\pmc{Calculation is not an experiment.}

\pn{6.234}
\ger{Die Mathematik ist eine Methode der Logik.}
\ogd{Mathematics is a method of logic.}
\pmc{Mathematics is a method of logic.}

\pn{6.2341}
\ger{Das Wesentliche der mathematischen Methode ist es, mit Gleichungen zu arbeiten. Auf dieser Methode beruht es n{\"a}mlich, dass jeder Satz der Mathematik sich von selbst verstehen muss.}
\ogd{The essential of mathematical method is working with equations. On this method depends the fact that every proposition of mathematics must be self-intelligible.}
\pmc{It is the essential characteristic of mathematical method that it employs equations. For it is because of this method that every proposition of mathematics must go without saying.}

\pn{6.24}
\ger{Die Methode der Mathematik, zu ihren Gleichungen zu kommen, ist die Substitutionsmethode.}
\ogd{The method by which mathematics arrives at its equations is the method of substitution.}
\pmc{The method by which mathematics arrives at its equations is the method of substitution.}

\pnskip
\ger{Denn die Gleichungen dr{\"u}cken die Ersetzbarkeit zweier Ausdr{\"u}cke aus und wir schreiten von einer Anzahl von Gleichungen zu neuen Gleichungen vor, indem wir, den Gleichungen entsprechend, Ausdr{\"u}cke durch andere ersetzen.}
\ogd{For equations express the substitutability of two expressions, and we proceed from a number of equations to new equations, replacing expressions by others in accordance with the equations.}
\pmc{For equations express the substitutability of two expressions and, starting from a number of equations, we advance to new equations by substituting different expressions in accordance with the equations.}

\pn{6.241}
\ger{So lautet der Beweis des Satzes $2 \times 2=4$:}
\ogd{Thus the proof of the proposition $2 \times 2=4$ runs:}
\pmc{Thus the proof of the proposition $2 \times 2 = 4$ runs as follows:}

\pnskip
\ger{\begin{center}%
$\omopparen[\nu]{\mu} x = \omop[\nu \times \mu] x$ Def.\\
$\omop[2 \times 2] x = \omopparen[2]{2} x = \omopparen[2]{1+1} x = \omop[2]\omop[2] x$\\
$ = \omop[1+1]\omop[1+1] x = \mathop{(\Omega\text{'}\Omega)\text{'}}\mathop{(\Omega\text{'}\Omega)\text{'}} x$\\ $= \omop\omop\omop\omop x = \omop[1+1+1+1] x = \omop[4] x$.\\
\end{center}}
\ogd{\begin{center}%
$\omopparen[\nu]{\mu} x = \omop[\nu \times \mu] x$ Def.\\
$\omop[2 \times 2] x = \omopparen[2]{2} x = \omopparen[2]{1+1} x = \omop[2]\omop[2] x$\\
$ = \omop[1+1]\omop[1+1] x = \mathop{(\Omega\text{'}\Omega)\text{'}}\mathop{(\Omega\text{'}\Omega)\text{'}} x$\\ 
$= \omop\omop\omop\omop x = \omop[1+1+1+1] x = \omop[4] x$.\\
\end{center}}
\pmc{\begin{center}%
$\omopparen[\nu]{\mu} x = \omop[\nu \times \mu] x$ Def.\\
$\omop[2 \times 2] x = \omopparen[2]{2} x = \omopparen[2]{1+1} x = \omop[2]\omop[2] x$\\
$ = \omop[1+1]\omop[1+1] x = \mathop{(\Omega\text{'}\Omega)\text{'}}\mathop{(\Omega\text{'}\Omega)\text{'}} x$\\ $= \omop\omop\omop\omop x = \omop[1+1+1+1] x = \omop[4] x$.\\
\end{center}}

\pn{6.3}
\ger{Die Erforschung der Logik bedeutet die Erforschung \germph{aller Gesetzm{\"a}{\ss}igkeit}. Und au{\ss}erhalb der Logik ist alles Zufall.}
\ogd{Logical research means the investigation of \emph{all regularity}. And outside logic all is accident.}
\pmc{The exploration of logic means the exploration of \emph{everything that is subject to law}. And outside logic everything is accidental.}

\pn{6.31}
\ger{Das sogenannte Gesetz der Induktion kann jedenfalls kein logisches Gesetz sein, denn es ist offenbar ein sinnvoller Satz.---Und darum kann es auch kein Gesetz a priori sein.}
\ogd{The so-called law of induction cannot in any case be a logical law, for it is obviously a significant propositions.---And therefore it cannot be a law a priori either.}
\pmc{The so-called law of induction cannot possibly be a law of logic, since it is obviously a proposition with sense.---Nor, therefore, can it be an \emph{a priori} law.}

\pn{6.32}
\ger{Das Kausalit{\"a}tsgesetz ist kein Gesetz, sondern die Form eines Gesetzes.}
\ogd{The law of causality is not a law but the form of a law.*}
\pmc{The law of causality is not a law but the form of a law.}
% INTERLUDESTART
\end{parcolumns}
\footnotetext{* \kckaddition{[Ogden only]} \emph{I.e.}\ not the form of one particular law, but of any law of a certain sort (B.\ R.).}
\begin{parcolumns}[sloppy,%
                    rulebetween,
                    colwidths={1={.8in},2={2.9in},3={2.9in},4={2.9in}}%
                    ]{4}
% INTERLUDEEND

\pn{6.321}
\ger{\gdql Kausalit{\"a}tsgesetz\gdqr{}, das ist ein Gattungsname. Und wie es in der Mechanik, sagen wir, Minimum-Gesetze gibt---etwa der kleinsten Wirkung---so gibt es in der Physik Kausalit{\"a}tsgesetze, Gesetze von der Kausalit{\"a}tsform.}
\ogd{``Law of Causality'' is a class name. And as in mechanics there are, for instance, minimum-laws, such as that of least actions, so in physics there are causal laws, laws of the causality form.}
\pmc{`Law of causality'---that is a general name. And just as in mechanics, for example, there are `minimum-principles', such as the law of least action, so too in physics there are causal laws, laws of the causal form.}

\pn{6.3211}
\ger{Man hat ja auch davon eine Ahnung gehabt, dass es \germph{ein} \gdql Gesetz der kleinsten Wirkung\gdqr{} geben m{\"u}sse, ehe man genau wusste, wie es lautete. (Hier, wie immer, stellt sich das a priori Gewisse als etwas rein Logisches heraus.)}
\ogd{Men had indeed an idea that there must be \emph{a} ``law of least action'', before they knew exactly how it ran. (Here, as always, the a priori certain proves to be something purely logical.)}
\pmc{Indeed people even surmised that there must be a `law of least action' before they knew exactly how it went. (Here, as always, what is certain \emph{a priori} proves to be something purely logical.)}

\pn{6.33}
\ger{Wir \germph{glauben} nicht a priori an ein Erhaltungsgesetz, sondern wir \germph{wissen} a priori die M{\"o}glichkeit einer logischen Form.}
\ogd{We do not \emph{believe} a priori in a law of conservation, but we \emph{know} a priori the possibility of a logical form.}
\pmc{We do not have an \emph{a priori belief} in a law of conservation, but rather \emph{a priori knowledge} of the possibility of a logical form.}

\pn{6.34}
\ger{Alle jene S{\"a}tze, wie der Satz vom Grunde, von der Kontinuit{\"a}t in der Natur, vom kleinsten Aufwande in der Natur etc.\ etc., alle diese sind Einsichten a priori {\"u}ber die m{\"o}gliche Formgebung der S{\"a}tze der Wissenschaft.}
\ogd{All propositions, such as the law of causation, the law of continuity in nature, the law of least expenditure in nature, etc.\ etc., all these are a priori intuitions of possible forms of the propositions of science.}
\pmc{All such propositions, including the principle of sufficient reason, the laws of continuity in nature and of least effort in nature, etc.\ etc.---all these are \emph{a priori} insights about the forms in which the propositions of science can be cast.}

\pn{6.341}
\ger{Die Newtonsche Mechanik z.~B.\ bringt die Weltbeschreibung auf eine einheitliche Form. Denken wir uns eine wei{\ss}e Fl{\"a}che, auf der unregelm{\"a}{\ss}ige schwarze Flecken w{\"a}ren. Wir sagen nun: Was f{\"u}r ein Bild immer hierdurch entsteht, immer kann ich seiner Beschreibung beliebig nahe kommen, indem ich die Fl{\"a}che mit einem entsprechend feinen quadratischen Netzwerk bedecke und nun von jedem Quadrat sage, dass es wei{\ss} oder schwarz ist. Ich werde auf diese Weise die Beschreibung der Fl{\"a}che auf eine einheitliche Form gebracht haben. Diese Form ist beliebig, denn ich h{\"a}tte mit dem gleichen Erfolge ein Netz aus dreieckigen oder sechseckigen Maschen verwenden k{\"o}nnen. Es kann sein, dass die Beschreibung mit Hilfe eines Dreiecks-Netzes einfacher geworden w{\"a}re; das hei{\ss}t, dass wir die Fl{\"a}che mit einem gr{\"o}beren Dreiecks-Netz genauer beschreiben k{\"o}nnten als mit einem feineren quadratischen (oder umgekehrt) usw. Den verschiedenen Netzen entsprechen verschiedene Systeme der Weltbeschreibung. Die Mechanik bestimmt eine Form der Weltbeschreibung, indem sie sagt: Alle S{\"a}tze der Weltbeschreibung m{\"u}ssen aus einer Anzahl gegebener S{\"a}tze---den mechanischen Axiomen---auf eine gegebene Art und Weise erhalten werden. Hierdurch liefert sie die Bausteine zum Bau des wissenschaftlichen Geb{\"a}udes und sagt: Welches Geb{\"a}ude immer du auff{\"u}hren willst, jedes musst du irgendwie mit diesen und nur diesen Bausteinen zusammenbringen.}
\ogd{Newtonian mechanics, for example, brings the description of the universe to a unified form. Let us imagine a white surface with irregular black spots. We now say: Whatever kind of picture these make I can always get as near as I like to its description, if I cover the surface with a sufficiently fine square network and now say of every square that it is white or black. In this way I shall have brought the description of the surface to a unified form. This form is arbitrary, because I could have applied with equal success a net with a triangular or hexagonal mesh. It can happen that the description would have been simpler with the aid of a triangular mesh; that is to say we might have described the surface more accurately with a triangular, and coarser, than with the finer square mesh, or vice versa, and so on. To the different networks correspond different systems of describing the world. Mechanics determine a form of description by saying: All propositions in the description of the world must be obtained in a given way from a number of given propositions---the mechanical axioms. It thus provides the bricks for building the edifice of science, and says: Whatever building thou wouldst erect, thou shalt construct it in some manner with these bricks and these alone.}
\pmc{Newtonian mechanics, for example, imposes a unified form on the description of the world. Let us imagine a white surface with irregular black spots on it. We then say that whatever kind of picture these make, I can always approximate as closely as I wish to the description of it by covering the surface with a sufficiently fine square mesh, and then saying of every square whether it is black or white. In this way I shall have imposed a unified form on the description of the surface. The form is optional, since I could have achieved the same result by using a net with a triangular or hexagonal mesh. Possibly the use of a triangular mesh would have made the description simpler: that is to say, it might be that we could describe the surface more accurately with a coarse triangular mesh than with a fine square mesh (or conversely), and so on. The different nets correspond to different systems for describing the world. Mechanics determines one form of description of the world by saying that all propositions used in the description of the world must be obtained in a given way from a given set of propositions---the axioms of mechanics. It thus supplies the bricks for building the edifice of science, and it says, `Any building that you want to erect, whatever it may be, must somehow be constructed with these bricks, and with these alone.'}

\pnskip
\ger{(Wie man mit dem Zahlensystem jede beliebige Anzahl, so muss man mit dem System der Mechanik jeden beliebigen Satz der Physik hinschreiben k{\"o}nnen.)}
\ogd{(As with the system of numbers one must be able to write down any arbitrary number, so with the system of mechanics one must be able to write down any arbitrary physical proposition.)}
\pmc{(Just as with the number-system we must be able to write down any number we wish, so with the system of mechanics we must be able to write down any proposition of physics that we wish.)}

\pn{6.342}
\ger{Und nun sehen wir die gegenseitige Stellung von Logik und Mechanik. (Man k{\"o}nnte das Netz auch aus verschiedenartigen Figuren etwa aus Dreiecken und Sechsecken bestehen lassen.) Dass sich ein Bild, wie das vorhin erw{\"a}hnte, durch ein Netz von gegebener Form beschreiben l{\"a}sst, sagt {\"u}ber das Bild \germph{nichts} aus. (Denn dies gilt f{\"u}r jedes Bild dieser Art.) Das aber charakterisiert das Bild, dass es sich durch ein bestimmtes Netz von \germph{bestimmter} Feinheit \germph{vollst{\"a}ndig} beschreiben l{\"a}sst.}
\ogd{And now we see the relative position of logic and mechanics. (We could construct the network out of figures of different kinds, as out of triangles and hexagons together.) That a picture like that instanced above can be described by a network of a given form asserts \emph{nothing} about the picture. (For this holds of every picture of this kind.) But \emph{this} does characterize the picture, the fact, namely, that it can be \emph{completely} described by a definite net of \emph{definite} fineness.}
\pmc{And now we can see the relative position of logic and mechanics. (The net might also consist of more than one kind of mesh: e.g.\ we could use both triangles and hexagons.) The possibility of describing a picture like the one mentioned above with a net of a given form tells us \emph{nothing} about the picture. (For that is true of all such pictures.) But what \emph{does} characterize the picture is that it can be described \emph{completely} by a particular net with a \emph{particular} size of mesh.}

\pnskip
\ger{So auch sagt es nichts {\"u}ber die Welt aus, dass sie sich durch die Newtonsche Mechanik beschreiben l{\"a}sst; wohl aber, dass sie sich\ \germph{so}\ durch jene beschreiben l{\"a}sst, wie dies eben der Fall ist. Auch das sagt etwas {\"u}ber die Welt, dass sie sich durch die eine Mechanik einfacher beschreiben l{\"a}sst als durch die andere.}
\ogd{So too the fact that it can be described by Newtonian mechanics asserts nothing about the world; but \emph{this} asserts something, namely, that it can be described in that particular way in which as a matter of fact it is described. The fact, too, that it can be described more simply by one system of mechanics than by another says something about the world.}
\pmc{Similarly the possibility of describing the world by means of Newtonian mechanics tells us nothing about the world: but what does tell us something about it is the precise \emph{way} in which it is possible to describe it by these means. We are also told something about the world by the fact that it can be described more simply with one system of mechanics than with another.}

\pn{6.343}
\ger{Die Mechanik ist ein Versuch, alle \germph{wahren} S{\"a}tze, die wir zur Weltbeschreibung brauchen, nach Einem Plane zu konstruieren.}
\ogd{Mechanics is an attempt to construct according to a single plan all \emph{true} propositions which we need for the description of the world.}
\pmc{Mechanics is an attempt to construct according to a single plan all the \emph{true} propositions that we need for the description of the world.}

\pn{6.3431}
\ger{Durch den ganzen logischen Apparat hindurch sprechen die physikalischen Gesetze doch von den Gegenst{\"a}nden der Welt.}
\ogd{Through the whole apparatus of logic the physical laws still speak of the objects of the world.}
\pmc{The laws of physics, with all their logical apparatus, still speak, however indirectly, about the objects of the world.}

\pn{6.3432}
\ger{Wir d{\"u}rfen nicht vergessen, dass die Weltbeschreibung durch die Mechanik immer die ganz allgemeine ist. Es ist in ihr z.~B.\ nie von \germph{bestimmten} materiellen Punkten die Rede, sondern immer nur von \germph{irgend welchen}.}
\ogd{We must not forget that the description of the world by mechanics is always quite general. There is, for example, never any mention of \emph{particular} material points in it, but always only of \emph{some points or other}.}
\pmc{We ought not to forget that any description of the world by means of mechanics will be of the completely general kind. For example, it will never mention \emph{particular} point-masses: it will only talk about \emph{any point-masses whatsoever}.}

\pn{6.35}
\ger{Obwohl die Flecke in unserem Bild geometrische Figuren sind, so kann doch selbstverst{\"a}ndlich die Geometrie gar nichts {\"u}ber ihre tats{\"a}chliche Form und Lage sagen. Das Netz aber ist \germph{rein} geometrisch, alle seine Eigenschaften k{\"o}nnen a priori angegeben werden.}
\ogd{Although the spots in our picture are geometrical figures, geometry can obviously say nothing about their actual form and position. But the network is \emph{purely} geometrical, and all its properties can be given a priori.}
\pmc{Although the spots in our picture are geometrical figures, nevertheless geometry can obviously say nothing at all about their actual form and position. The network, however, is \emph{purely} geometrical; all its properties can be given \emph{a priori}.}

\pnskip
\ger{Gesetze wie der Satz vom Grunde, etc. handeln vom Netz, nicht von dem, was das Netz beschreibt.}
\ogd{Laws, like the law of causation, etc., treat of the network and not what the network describes.}
\pmc{Laws like the principle of sufficient reason, etc.\ are about the net and not about what the net describes.}

\pn{6.36}
\ger{Wenn es ein Kausalit{\"a}tsgesetz g{\"a}be, so k{\"o}nnte es lauten: \gdql Es gibt Naturgesetze\gdqr{}.}
\ogd{If there were a law of causality, it might run: ``There are natural laws''.}
\pmc{If there were a law of causality, it might be put in the following way: There are laws of nature.}

\pnskip
\ger{Aber freilich kann man das nicht sagen: es zeigt sich.}
\ogd{But that can clearly not be said: it shows itself.}
\pmc{But of course that cannot be said: it makes itself manifest.}

\pn{6.361}
\ger{In der Ausdrucksweise Hertz's  k{\"o}nnte man sagen: Nur \germph{gesetzm{\"a}{\ss}ige} Zusammenh{\"a}nge sind \germph{denkbar}.}
\ogd{In the terminology of Hertz we might say: Only \emph{uniform} connections are \emph{thinkable}.}
\pmc{One might say, using Hertz's terminology, that only connexions that are \emph{subject to law} are \emph{thinkable}.}%???

\pn{6.3611}
\ger{Wir k{\"o}nnen keinen Vorgang mit dem \gdql Ablauf der Zeit\gdqr{} vergleichen---diesen gibt es nicht---, sondern nur mit einem anderen Vorgang (etwa mit dem Gang des Chronometers).}
\ogd{We cannot compare any process with the ``passage of time''---there is no such thing---but only with another process (say, with the movement of the chronometer).}
\pmc{We cannot compare a process with `the passage of time'---there is no such thing---but only with another process (such as the working of a chronometer).}

\pnskip
\ger{Daher ist die Beschreibung des zeitlichen Verlaufs nur so m{\"o}glich, dass wir uns auf einen anderen Vorgang st{\"u}tzen.}
\ogd{Hence the description of the temporal sequence of events is only possible if we support ourselves on another process.}
\pmc{Hence we can describe the lapse of time only by relying on some other process.}

\pnskip
\ger{Ganz Analoges gilt f{\"u}r den Raum. Wo man z.~B.\ sagt, es k{\"o}nne keines von zwei Ereignissen (die sich gegenseitig ausschlie{\ss}en) eintreten, weil \germph{keine Ursache} vorhanden sei, warum das eine eher als das andere eintreten solle, da handelt es sich in Wirklichkeit darum, dass man gar nicht \germph{eines} der beiden Ereignisse beschreiben kann, wenn nicht irgend eine Asymmetrie vorhanden ist. Und \germph{wenn} eine solche Asymmetrie vorhanden \germph{ist}, so k{\"o}nnen wir diese als \germph{Ursache} des Eintreffens des einen und Nicht- Eintreffens des anderen auffassen.}
\ogd{It is exactly analogous for space. When, for example, we say that neither of two events (which mutually exclude one another) can occur, because there is \emph{no cause} why the one should occur rather than the other, it is really a matter of our being unable to describe \emph{one} of the two events unless there is some sort of asymmetry. And if there \emph{is} such an asymmetry, we can regard this as the \emph{cause} of the occurrence of the one and of the non-occurrence of the other.}
\pmc{Something exactly analogous applies to space: e.g. when people say that neither of two events (which exclude one another) can occur, because there is \emph{nothing to cause} the one to occur rather than the other, it is really a matter of our being unable to describe \emph{one} of the two events unless there is some sort of asymmetry to be found. And \emph{if} such an asymmetry \emph{is} to be found, we can regard it as the \emph{cause} of the occurrence of the one and the non-occurrence of the other.}

\pn{6.36111}
\ger{Das Kant'sche Problem von der rechten und linken Hand, die man nicht zur Deckung bringen kann, besteht schon in der Ebene, ja im eindimensionalen Raum, wo die beiden kongruenten Figuren $a$ und $b$ auch nicht zur Deckung gebracht werden k{\"o}nnen, ohne aus diesem Raum}
\ogd{The Kantian problem of the right and left hand which cannot be made to cover one another already exists in the plane, and even in one-dimensional space; where the two congruent figures $a$ and $b$ cannot be made to cover one another without \phantom{xxxxxxxxx}}
\pmc{Kant's problem about the right hand and the left hand, which cannot be made to coincide, exists even in two dimensions. Indeed, it exists in one-dimensional space}

\pnskip
\ger{\negpbk\theline}
\ogd{\negpbk\theline}
\pmc{\negpbk\theline}

\pnskip
\ger{\negpbk herausbewegt zu werden. Rechte und linke Hand sind tats{\"a}chlich vollkommen kongruent. Und dass man sie nicht zur Deckung bringen kann, hat damit nichts zu tun.}
\ogd{\negpbk moving them out of this space. The right and left hand are in fact completely congruent. And the fact that they cannot be made to cover one another has nothing to do with it.}
\pmc{\negpbk in which the two congruent figures, $a$ and $b$, cannot be made to coincide unless they are moved out of this space. The right hand and the left hand are in fact completely congruent. It is quite irrelevant that they cannot be made to coincide. }

\pnskip
\ger{Den rechten Handschuh k{\"o}nnte man an die linke Hand ziehen, wenn man ihn im vierdimensionalen Raum umdrehen k{\"o}nnte.}
\ogd{A right-hand glove could be put on a left hand if it could be turned round in four-dimensional space.}
\pmc{A right-hand glove could be put on the left hand, if it could be turned round in four-dimensional space.}

\pn{6.362}
\ger{Was sich beschreiben l{\"a}sst, das kann auch geschehen, und was das Kausalit{\"a}tsgesetz ausschlie{\ss}en soll, das l{\"a}sst sich auch nicht beschreiben.}
\ogd{What can be described can happen too, and what is excluded by the law of causality cannot be described.}
\pmc{What can be described can happen too: and what the law of causality is meant to exclude cannot even be described.}

\pn{6.363}
\ger{Der Vorgang der Induktion besteht darin, dass wir das \germph{einfachste} Gesetz annehmen, das mit unseren Erfahrungen in Einklang zu bringen ist.}
\ogd{The process of induction is the process of assuming the \emph{simplest} law that can be made to harmonize with our experience.}
\pmc{The procedure of induction consists in accepting as true the \emph{simplest} law that can be reconciled with our experiences.}

\pn{6.3631}
\ger{Dieser Vorgang hat aber keine logische, sondern nur eine psychologische Begr{\"u}ndung.}
\ogd{This process, however, has no logical foundation but only a psychological one.}
\pmc{This procedure, however, has no logical justification but only a psychological one.}

\pnskip
\ger{Es ist klar, dass kein Grund vorhanden ist, zu glauben, es werde nun auch wirklich der einfachste Fall eintreten.}
\ogd{It is clear that there are no grounds for believing that the simplest course of events will really happen.}
\pmc{It is clear that there are no grounds for believing that the simplest eventuality will in fact be realized.}

\pn{6.36311}
\ger{Dass die Sonne morgen aufgehen wird, ist eine Hypothese; und das hei{\ss}t: wir \germph{wissen} nicht, ob sie aufgehen wird.}
\ogd{That the sun will rise to-morrow, is an hypothesis; and that means that we do not \emph{know} whether it will rise.}
\pmc{It is an hypothesis that the sun will rise tomorrow: and this means that we do not \emph{know} whether it will rise.}

\pn{6.37}
\ger{Einen Zwang, nach dem Eines geschehen m{\"u}sste, weil etwas anderes geschehen ist, gibt es nicht. Es gibt nur eine \germph{logische} Notwendigkeit.}
\ogd{A necessity for one thing to happen because another has happened does not exist. There is only \emph{logical} necessity.}
\pmc{There is no compulsion making one thing happen because another has happened. The only necessity that exists is \emph{logical} necessity}

\pn{6.371}
\ger{Der ganzen modernen Weltanschauung liegt die T{\"a}uschung zugrunde, dass die sogenannten Naturgesetze die Erkl{\"a}rungen der Naturerscheinungen seien.}
\ogd{At the basis of the whole modern view of the world lies the illusion that the so-called laws of nature are the explanations of natural phenomena.}
\pmc{The whole modern conception of the world is founded on the illusion that the so-called laws of nature are the explanations of natural phenomena.}

\pn{6.372}
\ger{So bleiben sie bei den Naturgesetzen als bei etwas Unantastbarem stehen, wie die {\"A}lteren bei Gott und dem Schicksal.}
\ogd{So people stop short at natural laws as something unassailable, as did the ancients at God and Fate.}
\pmc{Thus people today stop at the laws of nature, treating them as something inviolable, just as God and Fate were treated in past ages.}

\pnskip
\ger{Und sie haben ja beide Recht, und Unrecht. Die Alten sind allerdings insofern klarer, als sie einen klaren Abschluss anerkennen, w{\"a}hrend es bei dem neuen System scheinen soll, als sei \germph{alles} erkl{\"a}rt.}
\ogd{And they are both right and wrong. but the ancients were clearer, in so far as they recognized one clear conclusion, whereas the modern system makes it appear as though \emph{everything} were explained.}
\pmc{And in fact both are right and both wrong: though the view of the ancients is clearer in so far as they have a clear and acknowledged terminus, while the modern system tries to make it look as if \emph{everything} were explained.}

\pn{6.373}
\ger{Die Welt ist unabh{\"a}ngig von meinem Willen.}
\ogd{The world is independent of my will.}
\pmc{The world is independent of my will.}

\pn{6.374}
\ger{Auch wenn alles, was wir w{\"u}nschen, gesch{\"a}he, so w{\"a}re dies doch nur, sozusagen, eine Gnade des Schicksals, denn es ist kein \germph{logischer} Zusammenhang zwischen Willen und Welt, der dies verb{\"u}rgte, und den angenommenen physikalischen Zusammenhang k{\"o}nnten wir doch nicht selbst wieder wollen.}
\ogd{Even if everything we wished were to happen, this would only be, so to speak, a favour of fate, for there is no \emph{logical} connexion between will and world, which would guarantee this, and the assumed physical connexion itself we could not again will.}
\pmc{Even if all that we wish for were to happen, still this would only be a favour granted by fate, so to speak: for there is no \emph{logical} connexion between the will and the world, which would guarantee it, and the supposed physical connexion itself is surely not something that we could will.}

\pn{6.375}
\ger{Wie es nur eine \germph{logische} Notwendigkeit gibt, so gibt es auch nur eine \germph{logische} Unm{\"o}glichkeit.}
\ogd{As there is only a \emph{logical} necessity, so there is only a \emph{logical} impossibility.}
\pmc{Just as the only necessity that exists is \emph{logical} necessity, so too the only impossibility that exists is \emph{logical} impossibility.}

\pn{6.3751}
\ger{Dass z.~B.\ zwei Farben zugleich an einem Ort des Gesichtsfeldes sind, ist unm{\"o}glich, und zwar logisch unm{\"o}glich, denn es ist durch die logische Struktur der Farbe ausgeschlossen.}
\ogd{For two colours, \emph{e.g.}\ to be at one place in the visual field, is impossible, logically impossible, for it is excluded by the logical structure of colour.}
\pmc{For example, the simultaneous presence of two colours at the same place in the visual field is impossible, in fact logically impossible, since it is ruled out by the logical structure of colour.}

\pnskip
\ger{Denken wir daran, wie sich dieser Widerspruch in der Physik darstellt: Ungef{\"a}hr so, dass ein Teilchen nicht zu gleicher Zeit zwei Geschwindigkeiten haben kann; das hei{\ss}t, dass es nicht zu gleicher Zeit an zwei Orten sein kann; das hei{\ss}t, dass Teilchen an verschiedenen Orten zu Einer Zeit nicht identisch sein k{\"o}nnen.}
\ogd{Let us consider how this contradiction presents itself in physics. Somewhat as follows: That a particle cannot at the same time have two velocities, \emph{i.e.}\ that at the same time it cannot be in two places, \emph{i.e.}\ that particles in different places at the same time cannot be identical.}
\pmc{Let us think how this contradiction appears in physics: more or less as follows---a particle cannot have two velocities at the same time; that is to say, it cannot be in two places at the same time; that is to say, particles that are in different places at the same time cannot be identical.}

\pnskip
\ger{(Es ist klar, dass das logische Produkt zweier Elementars{\"a}tze weder eine Tautologie noch eine Kontradiktion sein kann. Die Aussage, dass ein Punkt des Gesichtsfeldes zu gleicher Zeit zwei verschiedene Farben hat, ist eine Kontradiktion.)}
\ogd{It is clear that the logical product of two elementary propositions can neither be a tautology nor a contradiction. The assertion that a point in the visual field has two different colours at the same time, is a contradiction.}
\pmc{(It is clear that the logical product of two elementary propositions can neither be a tautology nor a contradiction. The statement that a point in the visual field has two different colours at the same time is a contradiction.)}

\pn{6.4}
\ger{Alle S{\"a}tze sind gleichwertig.}
\ogd{All propositions are of equal value.}
\pmc{All propositions are of equal value.}

\pn{6.41}
\ger{Der Sinn der Welt muss au{\ss}erhalb ihrer liegen. In der Welt ist alles, wie es ist, und geschieht alles, wie es geschieht; es gibt \germph{in} ihr keinen Wert---und wenn es ihn g{\"a}be, so h{\"a}tte er keinen Wert.}
\ogd{The sense of the world must lie outside the world. In the world everything is as it is and happens as it does happen. \emph{In} it there is no value---and if there were, it would be of no value.}
\pmc{The sense of the world must lie outside the world. In the world everything is as it is, and everything happens as it does happen: \emph{in} it no value exists---and if it did exist, it would have no value.}

\pnskip
\ger{Wenn es einen Wert gibt, der Wert hat, so muss er au{\ss}erhalb alles Geschehens und So-Seins liegen. Denn alles Geschehen und So-Sein ist zuf{\"a}llig.}
\ogd{If there is a value which is of value, it must lie outside all happening and being-so. For all happening and being-so is accidental.}
\pmc{If there is any value that does have value, it must lie outside the whole sphere of what happens and is the case. For all that happens and is the case is accidental.}

\pnskip
\ger{Was es nichtzuf{\"a}llig macht, kann nicht \germph{in} der Welt liegen, denn sonst w{\"a}re dies wieder zuf{\"a}llig.}
\ogd{What makes it non-accidental cannot lie \emph{in} the world, for otherwise this would again be accidental.}
\pmc{What makes it non-accidental cannot lie \emph{within} the world, since if it did it would itself be accidental.}

\pnskip
\ger{Es muss au{\ss}erhalb der Welt liegen.}
\ogd{It must lie outside the world.}
\pmc{It must lie outside the world.}

\pn{6.42}
\ger{Darum kann es auch keine S{\"a}tze der Ethik geben.}
\ogd{Hence also there can be no ethical propositions.}
\pmc{So too it is impossible for there to be propositions of ethics.}

\pnskip
\ger{S{\"a}tze k{\"o}nnen nichts H{\"o}heres ausdr{\"u}cken.}
\ogd{Propositions cannot express anything higher.}
\pmc{Propositions can express nothing that is higher.}

\pn{6.421}
\ger{Es ist klar, dass sich die Ethik nicht aussprechen l{\"a}sst.}
\ogd{It is clear that ethics cannot be expressed.}
\pmc{It is clear that ethics cannot be put into words.}

\pnskip
\ger{Die Ethik ist transzendental.}
\ogd{Ethics are transcendental.}
\pmc{Ethics is transcendental.}

\pnskip
\ger{(Ethik und {\"A}sthetik sind Eins.)}
\ogd{(Ethics and {\ae}sthetics are one.)}
\pmc{(Ethics and aesthetics are one and the same.)}

\pn{6.422}
\ger{Der erste Gedanke bei der Aufstellung eines ethischen Gesetzes von der Form \gdql Du sollst \ldots\gdqr{} ist: Und was dann, wenn ich es nicht tue? Es ist aber klar, dass die Ethik nichts mit Strafe und Lohn im gew{\"o}hnlichen Sinne zu tun hat. Also muss diese Frage nach den \germph{Folgen} einer Handlung belanglos sein.---Zum Mindesten d{\"u}rfen diese Folgen nicht Ereignisse sein. Denn etwas muss doch an jener Fragestellung richtig sein. Es muss zwar eine Art von ethischem Lohn und ethischer Strafe geben, aber diese m{\"u}ssen in der Handlung selbst liegen.}
\ogd{The first thought in setting up an ethical law of the form ``thou shalt \ldots'' is: And what if I do not do it? But it is clear that ethics has nothing to do with punishment and reward in the ordinary sense. This question as to the \emph{consequences} of an action must therefore be irrelevant. At least these consequences will not be events. For there must be something right in that formulation of the question. There must be some sort of ethical reward and ethical punishment, but this must lie in the action itself.}
\pmc{When an ethical law of the form, `Thou shalt \ldots' is laid down, one's first thought is, `And what if I do not do it?' It is clear, however, that ethics has nothing to do with punishment and reward in the usual sense of the terms. So our question about the \emph{consequences} of an action must be unimportant.---At least those consequences should not be events. For there must be something right about the question we posed. There must indeed be some kind of ethical reward and ethical punishment, but they must reside in the action itself.}

\pnskip
\ger{(Und das ist auch klar, dass der Lohn etwas Angenehmes, die Strafe etwas Unangenehmes sein muss.)}
\ogd{(And this is clear also that the reward must be something acceptable, and the punishment something unacceptable.)}
\pmc{(And it is also clear that the reward must be something pleasant and the punishment something unpleasant.)}

\pn{6.423}
\ger{Vom Willen als dem Tr{\"a}ger des Ethischen kann nicht gesprochen werden.}
\ogd{Of the will as the subject of the ethical we cannot speak.}
\pmc{It is impossible to speak about the will in so far as it is the subject of ethical attributes.}

\pnskip
\ger{Und der Wille als Ph{\"a}nomen interessiert nur die Psychologie.}
\ogd{And the will as a phenomenon is only of interest to psychology.}
\pmc{And the will as a phenomenon is of interest only to psychology.}

\pn{6.43}
\ger{Wenn das gute oder b{\"o}se Wollen die Welt {\"a}ndert, so kann es nur die Grenzen der Welt {\"a}ndern, nicht die Tatsachen; nicht das, was durch die Sprache ausgedr{\"u}ckt werden kann.}
\ogd{If good or bad willing changes the world, it can only change the limits of the world, not the facts; not the things that can be expressed in language.}
\pmc{If the good or bad exercise of the will does alter the world, it can alter only the limits of the world, not the facts---not what can be expressed by means of language.}

\pnskip
\ger{Kurz, die Welt muss dann dadurch {\"u}berhaupt eine andere werden. Sie muss sozusagen als Ganzes abnehmen oder zunehmen.}
\ogd{In brief, the world must thereby become quite another, it must so to speak wax or wane as a whole.}
\pmc{In short the effect must be that it becomes an altogether different world. It must, so to speak, wax and wane as a whole.}

\pnskip
\ger{Die Welt des Gl{\"u}cklichen ist eine andere als die des Ungl{\"u}cklichen.}
\ogd{The world of the happy is quite another than that of the unhappy.}
\pmc{The world of the happy man is a different one from that of the unhappy man.}

\pn{6.431}
\ger{Wie auch beim Tod die Welt sich nicht {\"a}ndert, sondern aufh{\"o}rt.}
\ogd{As in death, too, the world does not change, but ceases.}
\pmc{So too at death the world does not alter, but comes to an end.}

\pn{6.4311}
\ger{Der Tod ist kein Ereignis des Lebens. Den Tod erlebt man nicht.}
\ogd{Death is not an event of life. Death is not lived through.}
\pmc{Death is not an event in life: we do not live to experience death.}

\pnskip
\ger{Wenn man unter Ewigkeit nicht unendliche Zeitdauer, sondern Unzeitlichkeit versteht, dann lebt der ewig, der in der Gegenwart lebt.}
\ogd{If by eternity is understood not endless temporal duration but timelessness, then he lives eternally who lives in the present.}
\pmc{If we take eternity to mean not infinite temporal duration but timelessness, then eternal life belongs to those who live in the present.}

\pnskip
\ger{Unser Leben ist ebenso endlos, wie unser Gesichtsfeld grenzenlos ist.}
\ogd{Our life is endless in the way that our visual field is without limit.}
\pmc{Our life has no end in just the way in which our visual field has no limits.}

\pn{6.4312}
\ger{Die zeitliche Unsterblichkeit der Seele des Menschen, das hei{\ss}t also ihr ewiges Fortleben auch nach dem Tode, ist nicht nur auf keine Weise verb{\"u}rgt, sondern vor allem leistet diese Annahme gar nicht das, was man immer mit ihr erreichen wollte. Wird denn dadurch ein R{\"a}tsel gel{\"o}st, dass ich ewig fortlebe? Ist denn dieses ewige Leben dann nicht ebenso r{\"a}tselhaft wie das gegenw{\"a}rtige? Die L{\"o}sung des R{\"a}tsels des Lebens in Raum und Zeit liegt \germph{au{\ss}erhalb} von Raum und Zeit.}
\ogd{The temporal immortality of the human soul, that is to say, its eternal survival also after death, is not only in no way guaranteed, but this assumption in the first place will not do for us what we always tried to make it do. Is a riddle solved by the fact that I survive for ever? Is this eternal life not as enigmatic as our present one? The solution of the riddle of life in space and time lies \emph{outside} space and time.}
\pmc{Not only is there no guarantee of the temporal immortality of the human soul, that is to say of its eternal survival after death; but, in any case, this assumption completely fails to accomplish the purpose for which it has always been intended. Or is some riddle solved by my surviving for ever? Is not this eternal life itself as much of a riddle as our present life? The solution of the riddle of life in space and time lies \emph{outside} space and time.}

\pnskip
\ger{(Nicht Probleme der Naturwissenschaft sind ja zu l{\"o}sen.)}
\ogd{(It is not problems of natural science which have to be solved.)}
\pmc{(It is certainly not the solution of any problems of natural science that is required.)}

\pn{6.432}
\ger{\germph{Wie} die Welt ist, ist f{\"u}r das H{\"o}here vollkommen gleichg{\"u}ltig. Gott offenbart sich nicht \germph{in} der Welt.}
\ogd{\emph{How} the world is, is completely indifferent for what is higher. God does not reveal himself \emph{in} the world.}
\pmc{\emph{How} things are in the world is a matter of complete indifference for what is higher. God does not reveal himself \emph{in} the world.}

\pn{6.4321}
\ger{Die Tatsachen geh{\"o}ren alle nur zur Aufgabe, nicht zur L{\"o}sung.}
\ogd{The facts all belong only to the task and not to its performance.}
\pmc{The facts all contribute only to setting the problem, not to its solution.}

\pn{6.44}
\ger{Nicht \germph{wie} die Welt ist, ist das Mystische, sondern \germph{dass} sie ist.}
\ogd{Not \emph{how} the world is, is the mystical, but \emph{that} it is.}
\pmc{It is not \emph{how} things are in the world that is mystical, but \emph{that} it exists.}

\pn{6.45}
\ger{Die Anschauung der Welt sub specie aeterni ist ihre Anschauung als---begrenztes---Ganzes.}
\ogd{The contemplation of the world sub specie aeterni is its contemplation as a limited whole.}
\pmc{To view the world sub specie aeterni is to view it as a whole---a limited whole.}

\pnskip
\ger{Das Gef{\"u}hl der Welt als begrenztes Ganzes ist das mystische.}
\ogd{The feeling that the world is a limited whole is the mystical feeling.}
\pmc{Feeling the world as a limited whole---it is this that is mystical.}

\pn{6.5}
\ger{Zu einer Antwort, die man nicht aussprechen kann, kann man auch die Frage nicht aussprechen.}
\ogd{For an answer which cannot be expressed the question too cannot be expressed.}
\pmc{When the answer cannot be put into words, neither can the question be put into words.}

\pnskip
\ger{\germph{Das R{\"a}tsel} gibt es nicht. }
\ogd{\emph{The riddle} does not exist.}
\pmc{\emph{The riddle} does not exist.}

\pnskip
\ger{Wenn sich eine Frage {\"u}berhaupt stellen l{\"a}sst, so \germph{kann} sie auch beantwortet werden.}
\ogd{If a question can be put at all, then it \emph{can} also be answered.}
\pmc{If a question can be framed at all, it is also \emph{possible} to answer it.}

\pn{6.51}
\ger{Skeptizismus ist \germph{nicht} unwiderleglich, sondern offenbar unsinnig, wenn er bezweifeln will, wo nicht gefragt werden kann.}
\ogd{Scepticism is \emph{not} irrefutable, but palpably senseless, if it would doubt where a question cannot be asked.}
\pmc{Scepticism is \emph{not} irrefutable, but obviously nonsensical, when it tries to raise doubts where no questions can be asked.}

\pnskip
\ger{Denn Zweifel kann nur bestehen, wo eine Frage besteht; eine Frage nur, wo eine Antwort besteht, und diese nur, wo etwas \germph{gesagt} werden \germph{kann}.}
\ogd{For doubt can only exist where there is a question; a question only where there is an answer, and this only where something \emph{can} be \emph{said}.}
\pmc{For doubt can exist only where a question exists, a question only where an answer exists, and an answer only where something \emph{can be said}.}

\pn{6.52}
\ger{Wir f{\"u}hlen, dass, selbst wenn alle \germph{m{\"o}glichen} wissenschaftlichen Fragen beantwortet sind, unsere Lebensprobleme noch gar nicht ber{\"u}hrt sind. Freilich bleibt dann eben keine Frage mehr; und eben dies ist die Antwort.}
\ogd{We feel that even if \emph{all possible} scientific questions be answered, the problems of life have still not been touched at all. Of course there is then no question left, and just this is the answer.}
\pmc{We feel that even when \emph{all possible} scientific questions have been answered, the problems of life remain completely untouched. Of course there are then no questions left, and this itself is the answer.}

\pn{6.521}
\ger{Die L{\"o}sung des Problems des Lebens merkt man am Verschwinden dieses Problems.}
\ogd{The solution of the problem of life is seen in the vanishing of this problem.}
\pmc{The solution of the problem of life is seen in the vanishing of the problem.}

\pnskip
\ger{(Ist nicht dies der Grund, warum Menschen, denen der Sinn des Lebens nach langen Zweifeln klar wurde, warum diese dann nicht sagen konnten, worin dieser Sinn bestand?)}
\ogd{(Is not this the reason why men to whom after long doubting the sense of life became clear, could not then say wherein this sense consisted?)}
\pmc{(Is not this the reason why those who have found after a long period of doubt that the sense of life became clear to them have then been unable to say what constituted that sense?)}

\pn{6.522}
\ger{Es gibt allerdings Unaussprechliches. Dies \germph{zeigt} sich, es ist das Mystische.}
\ogd{There is indeed the inexpressible. This \emph{shows} itself; it is the mystical.}
\pmc{There are, indeed, things that cannot be put into words. They \emph{make themselves manifest}. They are what is mystical.}

\pn{6.53}
\ger{Die richtige Methode der Philosophie w{\"a}re eigentlich die: Nichts zu sagen, als was sich sagen l{\"a}sst, also S{\"a}tze der Naturwissenschaft---also etwas, was mit Philosophie nichts zu tun hat---, und dann immer, wenn ein anderer etwas Metaphysisches sagen wollte, ihm nachzuweisen, dass er gewissen Zeichen in seinen S{\"a}tzen keine Bedeutung gegeben hat. Diese Methode w{\"a}re f{\"u}r den anderen unbefriedigend---er h{\"a}tte nicht das Gef{\"u}hl, dass wir ihn Philosophie lehrten---aber \germph{sie} w{\"a}re die einzig streng richtige.}
\ogd{The right method of philosophy would be this: To say nothing except what can be said, \emph{i.e.}\ the propositions of natural science, \emph{i.e.}\ something that has nothing to do with philosophy: and then always, when someone else wished to say something metaphysical, to demonstrate to him that he had given no meaning to certain signs in his propositions. This method would be unsatisfying to the other---he would not have the feeling that we were teaching him philosophy---but it would be the only strictly correct method.}
\pmc{The correct method in philosophy would really be the following: to say nothing except what can be said, i.e.\ propositions of natural science---i.e.\ something that has nothing to do with philosophy---and then, whenever someone else wanted to say something metaphysical, to demonstrate to him that he had failed to give a meaning to certain signs in his propositions. Although it would not be satisfying to the other person---he would not have the feeling that we were teaching him philosophy---\emph{this} method would be the only strictly correct one.}

\pn{6.54}
\ger{Meine S{\"a}tze erl{\"a}utern dadurch, dass sie der, welcher mich versteht, am Ende als unsinnig erkennt, wenn er durch sie---auf ihnen---{\"u}ber sie hinausgestiegen ist. (Er muss sozusagen die Leiter wegwerfen, nachdem er auf ihr hinaufgestiegen ist.)}
\ogd{My propositions are elucidatory in this way: he who understands me finally recognizes them as senseless, when he has climbed out through them, on them, over them. (He must so to speak throw away the ladder, after he has climbed up on it.)}
\pmc{My propositions serve as elucidations in the following way: anyone who understands me eventually recognizes them as nonsensical, when he has used them---as steps---to climb up beyond them. (He must, so to speak, throw away the ladder after he has climbed up it.)}%???

\pnskip
\ger{Er muss diese S{\"a}tze {\"u}berwinden, dann sieht er die Welt richtig.}
\ogd{He must surmount these propositions; then he sees the world rightly.}
\pmc{He must transcend these propositions, and then he will see the world aright.}

\pn{7}
\ger{Wovon man nicht sprechen kann, dar{\"u}ber muss man schweigen.}
\ogd{Whereof one cannot speak, thereof one must be silent.}
\pmc{What we cannot speak about we must pass over in silence.}

%\end{mpxtabular} % ENDBODYCONVERT
\end{parcolumns}
\clearpage\phantomsection\addcontentsline{toc}{chapter}{Index}%
%%%%%%%%%%%%%%%%%%%%%%%%%%%%%%%%%%%%%%%%%%%%%%%%
%% start of Pears/McGuinness Index %%%%%%%%%%%%%
%%%%%%%%%%%%%%%%%%%%%%%%%%%%%%%%%%%%%%%%%%%%%%%%
\setlength{\columnseprule}{0.5pt}
\begin{multicols}{3}[\section*{Index (Pears/McGuinness)}]

\begin{center}
\kckaddition{[Original note by Pears and McGuinness.]}
\end{center}


% BEGININDEXNOTECONVERT
\noindent The translators' aim has been to include all the more interesting words, and, in each case, either to give all the occurrences of a word, or else to omit only a few unimportant ones. Paragraphs in the preface are referred to as P1, P2, etc. Propositions are indicated by numbers without points \kckaddition{[---the points have been restored for the side-by-side-by-side edition---]}; more than two consecutive propositions, by two numbers joined by an en-rule, as 202--2021.

In the translation it has sometimes been necessary to use different English expressions for the same German expression or the same English expression for different German expressions. The index contains various devices designed to make it an informative guide to the German terminology and, in particular, to draw attention to some important connexions between ideas that are more difficult to bring out in English than in German.

First, when a German expression is of any interest in itself, it is given in brackets after the English expression that translates it, e.g. \textbf{situation} [\textit{Sachlage}]; also, whenever an English expression is used to translate more than one German expression, each of the German expressions is given separately in numbered brackets, and is followed by the list of passages in which it is translated by the English expression, e.g.\ \textbf{reality} 1. [\textit{Realit{\"a}t}], 55561, etc. 2. [\textit{Wirklichkeit}], 206, etc.

Secondly, the German expressions given in this way sometimes have two
or more English translations in the text; and when this is so, if the alternative English translations are of interest, they follow the German expression inside the brackets, e.g.\ \textbf{proposition} [\textit{Satz}: law; principle].

The alternative translations recorded by these two devices are sometimes given in an abbreviated way. For a German expression need not actually be translated by the English expressions that it follows or precedes, as it is in the examples above. The relationship may be more complicated. For instance, the German expression may be only part of a phrase that is translated by the English expression, e.g.\ \textbf{stand in a relation to one another}; \textbf{are related}
[\textit{sich verhalten}: stand, how things; state of things].

Thirdly, cross-references have been used to draw attention to other important connexions between ideas, e.g. \textbf{true}, cf.\ correct; right: and \textbf{\textit{a priori}}, cf.\ advance, in.

In subordinate entries and cross-references the catchword is indicated by $\sim$, unless the catchword contains /, in which case the part preceding / is so indicated, e.g.\ \textbf{accident}; $\sim$\textbf{al} for \textbf{accident}; \textbf{accidental}, and \textbf{state of /affairs}; $\sim$ \textbf{things} for \textbf{state of affairs}; \textbf{state of things}. Cross-references relate to the last preceding entry or numbered bracket. When references are given both for a word in its own right and for a phrase containing it, occurrences of the latter are generally not also counted as occurrences of the former, so that both entries should be consulted.%
% ENDINDEXNOTECONVERT


\bigskip

\bigskip
\begin{itemize}
\raggedright
% BEGININDEXCONVERT
\indexentry{about [\textit{von etwas handeln}: concerned with; deal with; subject-matter], \indexref{3.24}, \indexref{5.44}, \indexref{6.35}; cf.\ mention; speak; talk.}

\indexentry{abstract, \indexref{5.5563}}

\indexentry{accident; $\sim$al [\textit{Zufall}], \indexref{2.012}, \indexref{2.0121}, \indexref{3.34}, \indexref{5.4733}, \indexref{6.031}, \indexref{6.1231}, \indexref{6.1232}, \indexref{6.3}, \indexref{6.41}}

\indexentry{action, \indexref{5.1362}, \indexref{6.422}}

\indexentry{activity, \indexref{4.112}}

\indexentry{addition, cf.\ logical.}

\indexentry{adjectiv/e; $\sim$al, \indexref{3.323}, \indexref{5.4733}}

\indexentry{advance, in [\textit{von vornherein}], \indexref{5.47}, \indexref{6.125}; cf.\ \textit{a priori}. aesthetics, \indexref{6.421}}

\indexentry{affirmation [\textit{Bejahung}], \indexref{4.064}, \indexref{5.124}, \indexref{5.1241}, \indexref{5.44}, \indexref{5.513}, \indexref{5.514}, \indexref{6.231}}

\indexentry{affix, [\textit{Index}], \indexref{4.0411}, \indexref{5.02}}

\indexentry{agreement}

   \indexsubentry{1. [\textit{stimmmen}: right; true], \indexref{5.512}}

   \indexsubentry{2. [\"Ubereinstimmmung], \indexref{2.21}, \indexref{2.222}, \indexref{4.2}, \indexref{4.4}, \indexref{4.42}--\indexref{4.431}, \indexref{4.462}}

\indexentry{analysis [\textit{Analyse}], \indexref{3.201}, \indexref{3.25}, \indexref{3.3442}, \indexref{4.1274}, \indexref{4.221}, \indexref{5.5562}; cf.\ anatomize; dissect; resolve.}

\indexentry{analytic, \indexref{6.11}}

\indexentry{anatomize [\textit{auseinanderlegen}], \indexref{3.261} ; cf.\ analysis.}

\indexentry{answer, \indexref{4.003}, \indexref{4.1274}, \indexref{5.4541}, \indexref{5.55}, \indexref{5.551}, \indexref{6.5}--\indexref{6.52}}

\indexentry{apparent, \indexref{4.0031}, \indexref{5.441}, \indexref{5.461}; cf.\ pseudo--.}

\indexentry{application [\textit{Anwendung}: employment], \indexref{3.262}, \indexref{3.5}, \indexref{5.2521}, \indexref{5.2523}, \indexref{5.32}, \indexref{5.5}, \indexref{5.5521}, \indexref{5.557}, \indexref{6.001}, \indexref{6.123}, \indexref{6.126}}

\indexentry{\textit{a priori}, \indexref{2.225}, \indexref{3.04}, \indexref{3.05}, \indexref{5.133}, \indexref{5.4541}, \indexref{5.4731}, \indexref{5.55}, \indexref{5.5541}, \indexref{5.5571}, \indexref{5.634}, \indexref{6.31}, \indexref{6.3211}, \indexref{6.33}, \indexref{6.34}, \indexref{6.35}; cf.\ advance, in.}

\indexentry{arbitrary, \indexref{3.315}, \indexref{3.322}, \indexref{3.342}, \indexref{3.3442}, \indexref{5.02}, \indexref{5.473}, \indexref{5.47321}, \indexref{5.554}, \indexref{6.124}, \indexref{6.1271} }

\indexentry{argument, \indexref{3.333}, \indexref{4.431}, \indexref{5.02}, \indexref{5.251}, \indexref{5.47}, \indexref{5.523}, \indexref{5.5351}; cf.\ truth-argument.}

   \indexsubentry{$\sim$-place, \indexref{2.0131}, \indexref{4.0411}, \indexref{5.5351}}

\indexentry{arithmetic, \indexref{4.4611}, \indexref{5.451}}

\indexentry{arrow, \indexref{3.144}, \indexref{4.461}}

\indexentry{articulated [\textit{artikuliert}] \indexref{3.141}, \indexref{3.251}; cf.\ segmented.}

\indexentry{ascribe [\textit{aussagen}: speak; state; statement; tell] \indexref{4.1241}}

\indexentry{assert}

   \indexsubentry{1. [behaupten],  \indexref{4.122}, \indexref{4.21}, \indexref{6.2322}}

   \indexsubentry{2. [zusprechen], \indexref{4.124}}

\indexentry{asymmetry, \indexref{6.3611}}

\indexentry{axiom, \indexref{6.341}}

   \indexsubentry{$\sim$ of infinity, \indexref{5.535}}

   \indexsubentry{$\sim$ of reducibility, \indexref{6.1232}, \indexref{6.1233}}

\indexgap

\indexentry{bad, \indexref{6.43}}

\indexentry{basis, \indexref{5.21}, \indexref{5.22}, \indexref{5.234}, \indexref{5.24}, \indexref{5.25}, \indexref{5.251}, \indexref{5.442}, \indexref{5.54}}

\indexentry{beautiful, \indexref{4.003}}

\indexentry{belief, \indexref{5.1361}, \indexref{5.1363}, \indexref{5.541}, \indexref{5.542}, \indexref{6.33}, \indexref{6.3631}}

\indexentry{bound; $\sim$ary [\textit{Grenze}: delimit; limit], \indexref{4.112}, \indexref{4.463}}

\indexentry{brackets, \indexref{4.441}, \indexref{5.46}, \indexref{5.461}}

\indexentry{build [\textit{Bau}: construction], \indexref{6.341}}

\indexgap

\indexentry{calculation, \indexref{6.126}, \indexref{6.2331}}

\indexentry{cardinal, cf.\ number.}

\indexentry{case, be the}

   \indexsubentry{1. [\textit{der Fall sein}], \indexref{1}, \indexref{1.12}, \indexref{1.21}, \indexref{2}, \indexref{2.024}, \indexref{3.342}, \indexref{4.024}, \indexref{5.1362}, \indexref{5.5151}, \indexref{5.541}, \indexref{5.5542}, \indexref{6.23}}

   \indexsubentry{2. [\textit{So-Sein}], \indexref{6.41}}

\indexentry{causality, \indexref{5.136}--\indexref{5.1362}, \indexref{6.32}, \indexref{6.321}, \indexref{6.36}, \indexref{6.3611}, \indexref{6.362}; cf.\ law.}

\indexentry{certainty [\textit{Gewi{\ss}heit}], \indexref{4.464}, \indexref{5.152}, \indexref{5.156}, \indexref{5.525}, \indexref{6.3211}}

\indexentry{chain, \indexref{2.03}; cf.\ concatenation.}

\indexentry{clarification, \indexref{4.112}}

\indexentry{class [\textit{Klasse}: set]. \indexref{3.311}, \indexref{3.315}, \indexref{4.1272}, \indexref{6.031}}

\indexentry{clear, \hyperlink{pref2}{P2}, \indexref{3.251}, \indexref{4.112}, \indexref{4.115}, \indexref{4.116}}

   \indexsubentry{make $\sim$ [\textit{erkl{\"a}ren}: definition; explanation], \indexref{5.452}}

\indexentry{colour, \indexref{2.0131}, \indexref{2.0232}, \indexref{2.0251}, \indexref{2.171}, \indexref{4.123}, \indexref{6.3751}}

   \indexsubentry{$\sim$-space, \indexref{2.0131}}

\indexentry{combination}

   \indexsubentry{1. [\textit{Kombination}], \indexref{4.27}, \indexref{4.28}, \indexref{5.46}; cf.\ rule, combinatory; truth-$\sim$.}

   \indexsubentry{2. [\textit{Verbindung}: connexion], \indexref{2.01}, \indexref{2.0121}, \indexref{4.0311}, \indexref{4.221}, \indexref{4.466}, \indexref{4.4661}, \indexref{5.131}, \indexref{5.451}, \indexref{5.515}, \indexref{6.12}, \indexref{6.1201}, \indexref{6.121}, \indexref{6.1221}, \indexref{6.124}, \indexref{6.23}, \indexref{6.232}; cf.\ sign.}

\indexentry{common, \indexref{2.022}, \indexref{2.16}, \indexref{2.17}, \indexref{2.18}, \indexref{2.2}, \indexref{3.31}, \indexref{3.311}, \indexref{3.317}, \indexref{3.321}, \indexref{3.322}, \indexref{3.333}, \indexref{3.341}, \indexref{3.3411}, \indexref{3.343}--\indexref{3.3441}, \indexref{4.014}, \indexref{4.12}, \indexref{5.11}, \indexref{5.143}, \indexref{5.152}, \indexref{5.24}, \indexref{5.47}, \indexref{5.4733}, \indexref{5.512}, \indexref{5.513}, \indexref{5.5261}, \indexref{6.022}}

\indexentry{comparison, \indexref{2.223}, \indexref{3.05}, \indexref{4.05}, \indexref{6.2321}, \indexref{6.3611}}

\indexentry{complete}

   \indexsubentry{1. [\textit{vollkommen}: folly], \indexref{5.156}}

   \indexsubentry{2. [\textit{vollst{\"a}dig}], \indexref{5.156};}

   \indexsubentry{analyse $\sim$ly, \indexref{3.201}, \indexref{3.25};}

   \indexsubentry{describe $\sim$ly, \indexref{2.0201}, \indexref{4.023}, \indexref{4.26}, \indexref{5.526}, \indexref{6.342}}

\indexentry{complex, \indexref{2.0201}, \indexref{3.1432}, \indexref{3.24}, \indexref{3.3442}, \indexref{4.1272}, \indexref{4.2211}, \indexref{4.441}, \indexref{5.515}, \indexref{5.5423}}

\indexentry{composite [\textit{zummmengesetzt}], \indexref{2.021}, \indexref{3.143}, \indexref{3.1431}, \indexref{3.3411}, \indexref{4.032}, \indexref{4.2211}, \indexref{5.47}, \indexref{5.5261}, \indexref{5.5421}, \indexref{5.55}}

\indexentry{compulsion, \indexref{6.37}}

\indexentry{concatenation [\textit{Verkettung}], \indexref{4.022}; cf.\ chain.}

\indexentry{concept [\textit{Begriff}: primitive idea], \indexref{4.063}, \indexref{4.126}--\indexref{4.1274}, \indexref{4.431}, \indexref{5.2523}, \indexref{5.521}, \indexref{5.555}, \indexref{6.022}; cf.\ formal $\sim$; pseudo-$\sim$.}

   \indexsubentry{$\sim$ual notation [\textit{Begriffsschrift}], \indexref{3.325}, \indexref{4.1272}, \indexref{4.1273}, \indexref{4.431}, \indexref{5.533}, \indexref{5.534}}

   \indexsubentry{$\sim$-word, \indexref{4.1272}}

\indexentry{concerned with [\textit{von etwas handeln}: about; deal with; subject-matter], \indexref{4.011}, \indexref{4.122}}

\indexentry{concrete, \indexref{5.5563}}

\indexentry{condition, \indexref{4.41}, \indexref{4.461}, \indexref{4.462}; cf.\ truth-$\sim$.}

\indexentry{configuration, \indexref{2.0231}, \indexref{2.0271}, \indexref{2.0272}, \indexref{3.21}}

\indexentry{connexion}

   \indexsubentry{1. [\textit{Verbindung}: combination], \indexref{6.124}, \indexref{6.232}}

   \indexsubentry{2. [\textit{Zusammenhang}: nexus], \indexref{2.0122}, \indexref{2.032}, \indexref{2.15}, \indexref{4.03}, \indexref{5.1311}, \indexref{5.1362}, \indexref{6.361}, \indexref{6.374}}

\indexentry{consequences, \indexref{6.422}}

\indexentry{conservation, cf.\ law.}

\indexentry{constant, \indexref{3.312}, \indexref{3.313}, \indexref{4.126}, \indexref{4.1271}, \indexref{5.501}, \indexref{5.522}; cf.\ logical $\sim$.}

\indexentry{constituent [\textit{Bestandteil}], \indexref{2.011}, \indexref{2.0201}, \indexref{3.24}, \indexref{3.315}, \indexref{3.4}, \indexref{4.024}, \indexref{4.025}, \indexref{5.4733}, \indexref{5.533}, \indexref{5.5423}, \indexref{6.12}}

\indexentry{construct [\textit{bilden}], \indexref{4.51}, \indexref{5.4733}, \indexref{5.475}, \indexref{5.501}, \indexref{5.503}, \indexref{5.512}, \indexref{5.514}, \indexref{5.5151}, \indexref{6.126}, \indexref{6.1271}}

\indexentry{construction}

   \indexsubentry{1. [\textit{Bau}: build], \indexref{4.002}, \indexref{4.014}, \indexref{5.45}, \indexref{5.5262}, \indexref{6.002}}

   \indexsubentry{2. [\textit{Konstruktion}], \indexref{4.023}, \indexref{4.5}, \indexref{5.233}, \indexref{5.556}, \indexref{6.343}}

\indexentry{contain [\textit{enthalten}], \indexref{2.014}, \indexref{2.203}, \indexref{3.02}, \indexref{3.13}, \indexref{3.24}, \indexref{3.332}, \indexref{3.333}, \indexref{5.121}, \indexref{5.122}, \indexref{5.44}, \indexref{5.47}}

\indexentry{content}

   \indexsubentry{1. [\textit{Gehalt}], \indexref{6.111}}

   \indexsubentry{2. [\textit{Inhalt}], \indexref{2.025}, \indexref{3.13}, \indexref{3.31}}

\indexentry{continuity, cf.\ law.}

\indexentry{contradiction}

   \indexsubentry{1. [\textit{Kontradiktion}], \indexref{4.46}--\indexref{4.4661} , \indexref{5.101}, \indexref{5.143}, \indexref{5.152}, \indexref{5.525}, \indexref{6.1202}, \indexref{6.3751}}

   \indexsubentry{2. [\textit{Widerspruch}], \indexref{3.032}, \indexref{4.1211}, \indexref{4.211}, \indexref{5.1241}, \indexref{6.1201}, \indexref{6.3751}; cf.\ law of $\sim$.}

\indexentry{convention}

   \indexsubentry{1. [\textit{Abmachung}], \indexref{4.002}}

   \indexsubentry{2. [\textit{{\"U}bereinkunft}], \indexref{3.315}, \indexref{5.02}}

\indexentry{co-ordinate, \indexref{3.032}, \indexref{3.41}, \indexref{3.42}, \indexref{5.64}}

\indexentry{copula, \indexref{3.323}}

\indexentry{correct [\textit{richtig}], \indexref{2.17}, \indexref{2.173}, \indexref{2.18}, \indexref{2.21}, \indexref{3.04}, \indexref{5.5302}, \indexref{5.62}, \indexref{6.2321}; cf.\ incorrect; true.}

\indexentry{correlate [\textit{zuordnen}], \indexref{2.1514}, \indexref{2.1515}, \indexref{4.43}, \indexref{4.44}, \indexref{5.526}, \indexref{5.542}, \indexref{6.1203}}

\indexentry{correspond [\textit{entsprechen}], \indexref{2.13}, \indexref{3.2}, \indexref{3.21}, \indexref{3.315}, \indexref{4.0621}, \indexref{4.063}, \indexref{4.28}, \indexref{4.441}, \indexref{4.466}, \indexref{5.5542}}

\indexentry{creation, \indexref{3.031}, \indexref{5.123}}

\indexentry{critique of language, \indexref{4.0031}}

\indexentry{cube, \indexref{5.5423}}

\indexgap

\indexentry{Darwin, \indexref{4.1122}}

\indexentry{deal with [\textit{von etwas handeln}: about; concerned with; subject-matter], \indexref{2.0121}}

\indexentry{death, \indexref{6.431}--\indexref{6.4312}}

\indexentry{deduce [\textit{folgern}], \indexref{5.132}--\indexref{5.134}; cf.\ infer.}

\indexentry{definition}

   \indexsubentry{1. [\textit{Definition}], \indexref{3.24}, \indexref{3.26}--\indexref{3.262}, \indexref{3.343}, \indexref{4.241}, \indexref{5.42}, \indexref{5.451}, \indexref{5.452}, \indexref{5.5302}, \indexref{6.02}}

   \indexsubentry{2. [\textit{Erkl{\"a}rung}: clear, make; explanation], \indexref{5.154}}

\indexentry{delimit [\textit{begrenzen}: bound; limit], \indexref{5.5262}}

\indexentry{depiction [\textit{Abbildung}: form, logico-pictorial; form, pictorial; pictorial], \indexref{2.16}--\indexref{2.172}, \indexref{2.18}, \indexref{2.19}, \indexref{2.2}, \indexref{2.201}, \indexref{4.013}, \indexref{4.014}, \indexref{4.015}, \indexref{4.016}, \indexref{4.041}}

\indexentry{derive [\textit{ableiten}], \indexref{4.0141}, \indexref{4.243}, \indexref{6.127}, \indexref{6.1271}; cf.\ infer.}

\indexentry{description [\textit{Beschreibung}], \indexref{2.0201}, \indexref{2.02331}, \indexref{3.144}, \indexref{3.24}, \indexref{3.317}, \indexref{3.33}, \indexref{4.016}, \indexref{4.023}, \indexref{4.0641}, \indexref{4.26}, \indexref{4.5}, \indexref{5.02}}

   \indexsubentry{$\sim$ of the world [\textit{Weltb.}], \indexref{6.341}, \indexref{6.343}, \indexref{6.3432}}

\indexentry{designate [\textit{bezeichnen}: sign; signify], \indexref{4.063}}

\indexentry{determin/ate [\textit{bestimmt}], \indexref{2.031}, \indexref{2.032}, \indexref{2.14}, \indexref{2.15}, \indexref{3.14}, \indexref{3.23}, \indexref{3.251}, \indexref{4.466}, \indexref{6.124}; cf.\ indeterminateness; undetermined.}

   \indexsubentry{$\sim$e, \indexref{1.11}, \indexref{1.12}, \indexref{2.0231}, \indexref{2.05}, \indexref{3.327}, \indexref{3.4}, \indexref{3.42}, \indexref{4.063}, \indexref{4.0641}, \indexref{4.431}, \indexref{4.463}}

\indexentry{difference [\textit{Verschiedenheit}], \indexref{2.0233}, \indexref{5.135}, \indexref{5.53}, \indexref{6.232}, \indexref{6.3751}}

\indexentry{display [\textit{aufweisen}], \indexref{2.172}, \indexref{4.121}; cf.\ show.}

\indexentry{dissect [\textit{zerlegen}], \indexref{3.26}; cf.\ analysis.}

\indexentry{doctrine [\textit{Lehre}: theory], \indexref{4.112}, \indexref{6.13}}

\indexentry{doubt, \indexref{6.51}, \indexref{6.521}}

\indexentry{dualism, \indexref{4.128}}

\indexentry{duration, \indexref{6.4311}}

\indexentry{dynamical model, \indexref{4.04}}

\indexgap

\indexentry{effort, least, cf.\ law.}

\indexentry{element, \indexref{2.13}--\indexref{2.14}, \indexref{2.15}, \indexref{2.151}, \indexref{2.1514}, \indexref{2.1515}, \indexref{3.14}, \indexref{3.2}, \indexref{3.201}, \indexref{3.24}, \indexref{3.42}}

   \indexsubentry{$\sim$ary proposition [\textit{Elementarsatz}], \indexref{4.21}--\indexref{4.221}, \indexref{4.23}, \indexref{4.24}, \indexref{4.243}--\indexref{4.26}, \indexref{4.28}--\indexref{4.42}, \indexref{4.431}, \indexref{4.45}, \indexref{4.46}, \indexref{4.51}, \indexref{4.52}, \indexref{5}, \indexref{5.01}, \indexref{5.101}, \indexref{5.134}, \indexref{5.152}, \indexref{5.234}, \indexref{5.3}--\indexref{5.32}, \indexref{5.41}, \indexref{5.47}, \indexref{5.5}, \indexref{5.524}, \indexref{5.5262}, \indexref{5.55}, \indexref{5.555}--\indexref{5.5571}, \indexref{6.001}, \indexref{6.124}, \indexref{6.3751}}

\indexentry{elucidation [\textit{Erl{\"a}uterung}], \indexref{3.263}, \indexref{4.112}, \indexref{6.54}}

\indexentry{empirical, \indexref{5.5561}}

\indexentry{employment}

   \indexsubentry{1. [\textit{Anwendung}: application], \indexref{3.202}, \indexref{3.323}, \indexref{5.452}}

   \indexsubentry{2. [\textit{Verwendung}: use], \indexref{3.327}}

\indexentry{enumeration, \indexref{5.501}}

\indexentry{equal value, of [\textit{gleichwertig}], \indexref{6.4}}

\indexentry{equality/, numerical [\textit{Zahlengleichheit}], \indexref{6.022}}

   \indexsubentry{sign of $\sim$ [\textit{Gleichheitszeichen}: identity, sign for], \indexref{6.23}, \indexref{6.232}}

\indexentry{equation [\textit{Gleichung}], \indexref{4.241}, \indexref{6.2}, \indexref{6.22}, \indexref{6.232}, \indexref{6.2323}, \indexref{6.2341}, \indexref{6.24}}

\indexentry{equivalent, cf.\ meaning, $\sim$ n.\ [\textit{{\"a}quivalent}], \indexref{5.232}, \indexref{5.2523}, \indexref{5.47321}, \indexref{5.514}, \indexref{6.1261}}

\indexentry{essence [\textit{Wesen}], \indexref{2.011}, \indexref{3.143}, \indexref{3.1431}, \indexref{3.31}, \indexref{3.317}, \indexref{3.34}--\indexref{3.3421}, \indexref{4.013}, \indexref{4.016}, \indexref{4.027}, \indexref{4.03}, \indexref{4.112}, \indexref{4.1121}, \indexref{4.465}, \indexref{4.4661}, \indexref{4.5}, \indexref{5.3}, \indexref{5.471}, \indexref{5.4711}, \indexref{5.501}, \indexref{5.533}, \indexref{6.1232}, \indexref{6.124}, \indexref{6.126}, \indexref{6.127}, \indexref{6.232}, \indexref{6.2341}}

\indexentry{eternity, \indexref{6.4311}, \indexref{6.4312}; cf.\ \textit{sub specie aeterni}.}

\indexentry{ethics, \indexref{6.42}--\indexref{6.423}}

\indexentry{everyday language [\textit{Umgangssprache}], \indexref{3.323}, \indexref{4.002}, \indexref{5.5563}}

\indexentry{existence}

   \indexsubentry{1. [\textit{Bestehen}: hold; subsist], \indexref{2}, \indexref{2.0121}, \indexref{2.04}--\indexref{2.06}, \indexref{2.062}, \indexref{2.11}, \indexref{2.201}, \indexref{4.1}, \indexref{4.122}, \indexref{4.124}, \indexref{4.125}, \indexref{4.2}, \indexref{4.21}, \indexref{4.25}, \indexref{4.27}, \indexref{4.3}, \indexref{5.131}, \indexref{5.135}}

   \indexsubentry{2. [\textit{Existenz}], \indexref{3.032}, \indexref{3.24}, \indexref{3.323}, \indexref{3.4}, \indexref{3.411}, \indexref{4.1274}, \indexref{5.5151}}

\indexentry{experience [\textit{Erfahrung}] \indexref{5.552}, \indexref{5.553}, \indexref{5.634}, \indexref{6.1222}, \indexref{6.363}}

\indexentry{explanation [\textit{Erkl{\"a}rung}: clear, make; definition], \indexref{3.263}, \indexref{4.02}, \indexref{4.021}, \indexref{4.026}, \indexref{4.431}, \indexref{5.5422}, \indexref{6.371}, \indexref{6.372}}

\indexentry{exponent, \indexref{6.021}}

\indexentry{expression [\textit{Ausdruck}: say], \hyperlink{pref3}{P3}, \indexref{3.1}, \indexref{3.12}, \indexref{3.13}, \indexref{3.142}, \indexref{3.1431}, \indexref{3.2}, \indexref{3.24}, \indexref{3.251}, \indexref{3.262}, \indexref{3.31}--\indexref{3.314}, \indexref{3.318}, \indexref{3.323}, \indexref{3.33}, \indexref{3.34}, \indexref{3.341}, \indexref{3.3441}, \indexref{4.002}, \indexref{4.013}, \indexref{4.03}, \indexref{4.0411}, \indexref{4.121}, \indexref{4.124}, \indexref{4.125}, \indexref{4.126}, \indexref{4.1272}, \indexref{4.1273}, \indexref{4.241}, \indexref{4.4}, \indexref{4.43}, \indexref{4.431}, \indexref{4.441}, \indexref{4.442}, \indexref{4.5}, \indexref{5.131}, \indexref{5.22}, \indexref{5.24}, \indexref{5.242}, \indexref{5.31}, \indexref{5.476}, \indexref{5.503}, \indexref{5.5151}, \indexref{5.525}, \indexref{5.53}, \indexref{5.5301}, \indexref{5.535}, \indexref{5.5352}, \indexref{6.124}, \indexref{6.1264}, \indexref{6.21}, \indexref{6.23}, \indexref{6.232}--\indexref{6.2323}, \indexref{6.24}}

    \indexsubentry{mode of $\sim$ [\textit{Ausdrucksweise}], \indexref{4.015}, \indexref{5.21}, \indexref{5.526}}

\indexentry{external, \indexref{2.01231}, \indexref{2.0233}, \indexref{4.023}, \indexref{4.122}, \indexref{4.1251}}

\indexgap

\indexentry{fact [Tatsache], \indexref{1.1}--\indexref{1.2}, \indexref{2}, \indexref{2.0121}, \indexref{2.034}, \indexref{2.06}, \indexref{2.1}, \indexref{2.141}, \indexref{2.16}, \indexref{3}, \indexref{3.14}, \indexref{3.142}, \indexref{3.143}, \indexref{4.016}, \indexref{4.0312}, \indexref{4.061}, \indexref{4.063}, \indexref{4.122}, \indexref{4.1221}, \indexref{4.1272}, \indexref{4.2211}, \indexref{4.463}, \indexref{5.156}, \indexref{5.43}, \indexref{5.5151}, \indexref{5.542}, \indexref{5.5423}, \indexref{6.2321}, \indexref{6.43}, \indexref{6.4321}; cf.\ negative $\sim$.}

\indexentry{fairy tale, \indexref{4.014}}

\indexentry{false [\textit{falsch}: incorrect], \indexref{2.0212}, \indexref{2.21}, \indexref{2.22}, \indexref{2.222}--\indexref{2.224}, \indexref{3.24}, \indexref{4.003}, \indexref{4.023}, \indexref{4.06}--\indexref{4.063}, \indexref{4.25}, \indexref{4.26}, \indexref{4.28}, \indexref{4.31}, \indexref{4.41}, \indexref{4.431}, \indexref{4.46}, \indexref{5.512}, \indexref{5.5262}, \indexref{5.5351}, \indexref{6.111}, \indexref{6.113}, \indexref{6.1203}; cf.\ wrong.}

\indexentry{fate, \indexref{6.372}, \indexref{6.374}}

\indexentry{feature [\textit{Zug}], \indexref{3.34}, \indexref{4.1221}, \indexref{4.126}}

\indexentry{feeling, \indexref{4.122}, \indexref{6.1232}, \indexref{6.45}}

\indexentry{finite, \indexref{5.32}}

\indexentry{follow, \indexref{4.1211}, \indexref{4.52}, \indexref{5.11}--\indexref{5.132}, \indexref{5.1363}--\indexref{5.142}, \indexref{5.152}, \indexref{5.43}, \indexref{6.1201}, \indexref{6.1221}, \indexref{6.126}}

\indexentry{foresee, \indexref{4.5}, \indexref{5.556}}

\indexentry{form [\textit{Form}], \indexref{2.0122}, \indexref{2.0141}, \indexref{2.022}--\indexref{2.0231}, \indexref{2.025}--\indexref{2.026}, \indexref{2.033}, \indexref{2.18}, \indexref{3.13}, \indexref{3.31}, \indexref{3.312}, \indexref{3.333}, \indexref{4.002}, \indexref{4.0031}, \indexref{4.012}, \indexref{4.063}, \indexref{4.1241}, \indexref{4.1271}, \indexref{4.241}, \indexref{4.242}, \indexref{4.5}, \indexref{5.131}, \indexref{5.156}, \indexref{5.231}, \indexref{5.24}, \indexref{5.241}, \indexref{5.2522}, \indexref{5.451}, \indexref{5.46}, \indexref{5.47}, \indexref{5.501}, \indexref{5.5351}, \indexref{5.542}, \indexref{5.5422}, \indexref{5.55}, \indexref{5.554}, \indexref{5.5542}, \indexref{5.555}, \indexref{5.556}, \indexref{5.6331}, \indexref{6}, \indexref{6.002}, \indexref{6.01}, \indexref{6.022}, \indexref{6.03}, \indexref{6.1201}, \indexref{6.1203}, \indexref{6.1224}, \indexref{6.1264}, \indexref{6.32}, \indexref{6.34}--\indexref{6.342}, \indexref{6.35}, \indexref{6.422}; cf.\ $\sim$al; general $\sim$; propositional $\sim$; series of $\sim$s.}

   \indexsubentry{logical $\sim$, \indexref{2.0233}, \indexref{2.18}, \indexref{2.181}, \indexref{2.2}, \indexref{3.315}, \indexref{3.327}, \indexref{4.12}, \indexref{4.121}, \indexref{4.128}, \indexref{5.555}, \indexref{6.23}, \indexref{6.33}}

   \indexsubentry{logico-pictorial $\sim$ [\textit{logische Form der Abbildung}], \indexref{2.2}}

   \indexsubentry{pictorial $\sim$ [\textit{Form der Abbildung}: depiction; pictorial], \indexref{2.15}, \indexref{2.151}, \indexref{2.17}, \indexref{2.172}, \indexref{2.181}, \indexref{2.22}}

   \indexsubentry{representational $\sim$ [\textit{Form der Darstellung}: present; represent], \indexref{2.173}, \indexref{2.174}}

\indexentry{formal [\textit{formal}], \indexref{4.122}, \indexref{5.501}}

   \indexsubentry{$\sim$ concept, \indexref{4.126}--\indexref{4.1273}}

   \indexsubentry{$\sim$ property, \indexref{4.122}, \indexref{4.124}, \indexref{4.126}, \indexref{4.1271}, \indexref{5.231}, \indexref{6.12}, \indexref{6.122}}

   \indexsubentry{$\sim$ relation [\textit{Relation}], \indexref{4.122}, \indexref{5.242}}

\indexentry{formulate [\textit{angeben}: give; say], \indexref{5.5563}}

\indexentry{free will, \indexref{5.1362}}

\indexentry{Frege, \hyperlink{pref6}{P6}, \indexref{3.143}, \indexref{3.318}, \indexref{3.325}, \indexref{4.063}, \indexref{4.1272}, \indexref{4.1273}, \indexref{4.431}, \indexref{4.442}, \indexref{5.02}, \indexref{5.132}, \indexref{5.4}, \indexref{5.42}, \indexref{5.451}, \indexref{5.4733}, \indexref{5.521}, \indexref{6.1271}, \indexref{6.232}}

\indexentry{fully [\textit{vollkommen}: complete], $\sim$ generalized, \indexref{5.526}, \indexref{5.5261}}

\indexentry{function [\textit{Funktion}], \indexref{3.318}, \indexref{3.333}, \indexref{4.126}, \indexref{4.1272}, \indexref{4.12721}, \indexref{4.24}, \indexref{5.02}, \indexref{5.2341}, \indexref{5.25}, \indexref{5.251}, \indexref{5.44}, \indexref{5.47}, \indexref{5.501}, \indexref{5.52}, \indexref{5.5301}; cf.\ truth-$\sim$.}

\indexentry{\textit{Fundamental Laws of Arithmetic} [\textit{Grundgesetze der Arithmetik}], \indexref{5.451}; cf.\ primitive proposition.}

\indexentry{future, \indexref{5.1361}, \indexref{5.1362}}

\indexgap

\indexentry{general [\textit{allgemein}], \indexref{3.3441}, \indexref{4.0141}, \indexref{4.1273}, \indexref{4.411}, \indexref{5.1311}, \indexref{5.156}, \indexref{5.242}, \indexref{5.2522}, \indexref{5.454}, \indexref{5.46}, \indexref{5.472}, \indexref{5.521}, \indexref{5.5262}, \indexref{6.031}, \indexref{6.1231}, \indexref{6.3432}}

   \indexsubentry{$\sim$ form, \indexref{3.312}, \indexref{4.1273}, \indexref{4.5}, \indexref{4.53}, \indexref{5.46}, \indexref{5.47}, \indexref{5.471}, \indexref{5.472}, \indexref{5.54}, \indexref{6}, \indexref{6.002}, \indexref{6.01}, \indexref{6.022}, \indexref{6.03}}

   \indexsubentry{$\sim$-ity-sign, \indexref{3.24}, \indexref{4.0411}, \indexref{5.522}, \indexref{5.523}, \indexref{6.1203}}

   \indexsubentry{$\sim$ validity, \indexref{6.1231}, \indexref{6.1232}}

\indexentry{generalization [\textit{verallgemeinerung}], \indexref{4.0411}, \indexref{4.52}, \indexref{5.156}, \indexref{5.526}, \indexref{5.5261}, \indexref{6.1231} ; cf.\ fully.}

\indexentry{geometry, \indexref{3.032}, \indexref{3.0321}, \indexref{3.411}, \indexref{6.35}}

\indexentry{give [\textit{angeben}: formulate; say], \indexref{3.317}, \indexref{4.5}, \indexref{5.4711}, \indexref{5.55}, \indexref{5.554}, \indexref{6.35}}

\indexentry{given [\textit{gegeben}], \indexref{2.0124}, \indexref{3.42}, \indexref{4.12721}, \indexref{4.51}, \indexref{5.442}, \indexref{5.524}, \indexref{6.002}, \indexref{6.124}}

\indexentry{God, \indexref{3.031}, \indexref{5.123}, \indexref{6.372}, \indexref{6.432}}

\indexentry{good, \indexref{4.003}, \indexref{6.43}}

\indexentry{grammar, cf.\ logical}

\indexgap

\indexentry{happy, \indexref{6.374}}

\indexentry{Hertz, \indexref{4.04}, \indexref{6.361}}

\indexentry{hierarchy, \indexref{5.252}, \indexref{5.556}, \indexref{5.5561}}

\indexentry{hieroglyphic script, \indexref{4.016}}

\indexentry{higher, \indexref{6.42}, \indexref{6.432}}

\indexentry{hold [\textit{bestehen}: existence; subsist], \indexref{4.014}}

\indexentry{how [\textit{wie}], \indexref{6.432}, \indexref{6.44}; cf.\ stand, $\sim$ things.}

   \indexsubentry{$\sim$) (what, \indexref{3.221}, \indexref{5.552}} % ??? Indeed, WHAT?}

\indexentry{hypothesis, \indexref{4.1122}, \indexref{5.5351}, \indexref{6.36311}}

\indexgap

\indexentry{idea, cf.\ primitive $\sim$.}

   \indexsubentry{1. [\textit{Gedanke}: thought], musical $\sim$, \indexref{4.014}}

   \indexsubentry{2. [\textit{Vorstellung}: present; represent], \indexref{5.631}}

\indexentry{idealist, \indexref{4.0412}}

\indexentry{identical [\textit{identisch}], \indexref{3.323}, \indexref{4.003}, \indexref{4.0411}, \indexref{5.473}, \indexref{5.4733}, \indexref{5.5303}, \indexref{5.5352}, \indexref{6.3751}; cf.\ difference.}

\indexentry{identity [\textit{Gleichheit}], \indexref{5.53}}

   \indexsubentry{sign for $\sim$ [\textit{Gleichheitszeiche}n: equality, sign of], \indexref{3.323}, \indexref{5.4733}, \indexref{5.53}, \indexref{5.5301}, \indexref{5.533}; cf.\ equation.}

\indexentry{illogical [\textit{unlogisch}], \indexref{3.03}, \indexref{3.031}, \indexref{5.4731}}

\indexentry{imagine [\textit{sich etwas denken}: think], \indexref{2.0121}, \indexref{2.022}, \indexref{4.01}, \indexref{6.1233}}

\indexentry{immortality, \indexref{6.4312}}

\indexentry{impossibility [\textit{Unm{\"o}glichkeit}], \indexref{4.464}, \indexref{5.525}, \indexref{5.5422}, \indexref{6.375}, \indexref{6.3751}}

\indexentry{incorrect}

   \indexsubentry{1. [\textit{falsch}: false], \indexref{2.17}, \indexref{2.173}, \indexref{2.18}}

   \indexsubentry{2. [\textit{unrichtig}], \indexref{2.21}}

\indexentry{independence [\textit{Selbst{\"a}ndigkeit}], \indexref{2.0122}, \indexref{3.261}}

\indexentry{independent [\textit{unabh{\"a}ngig}], \indexref{2.024}, \indexref{2.061}, \indexref{2.22}, \indexref{4.061}, \indexref{5.152}, \indexref{5.154}, \indexref{5.451}, \indexref{5.5261}, \indexref{5.5561}, \indexref{6.373}}

\indexentry{indetermlnateness [\textit{Unbestimmtheit}] \indexref{3.24}}

\indexentry{indicate}

   \indexsubentry{1. [\textit{anzeigen}], \indexref{3.322}, \indexref{6.121}, \indexref{6.124}}

   \indexsubentry{2. [\textit{auf etwas zeigen}: manifest; show], \indexref{2.02331}, \indexref{4.063}}

\indexentry{individuals, \indexref{5.553}}

\indexentry{induction, \indexref{6.31}, \indexref{6.363}}

\indexentry{infer [\textit{schlie{\ss}en}], \indexref{2.062}, \indexref{4.023}, \indexref{5.1311}, \indexref{5.132}, \indexref{5.135}, \indexref{5.1361}, \indexref{5.152}, \indexref{5.633}, \indexref{6.1224}, \indexref{6.211}; cf.\ deduce; derive.}

\indexentry{infinite, \indexref{2.0131}, \indexref{4.2211}, \indexref{4.463}, \indexref{5.43}, \indexref{5.535}, \indexref{6.4311}}

\indexentry{infinity, cf.\ axiom.}

\indexentry{inner, \indexref{4.0141}, \indexref{5.1311}, \indexref{5.1362}}

\indexentry{internal, \indexref{2.01231}, \indexref{3.24}, \indexref{4.014}, \indexref{4.023}, \indexref{4.122}--\indexref{4.1252}, \indexref{5.131}, \indexref{5.2}, \indexref{5.21}, \indexref{5.231}, \indexref{5.232}}

\indexentry{intuition [\textit{Anschauung}], \indexref{6.233}, \indexref{6.2331}}

\indexentry{intuitive [\textit{anschaulich}], \indexref{6.1203}}

\indexgap

\indexentry{judgement [\textit{Urteil}], \indexref{4.063}, \indexref{5.5422}}

   \indexsubentry{$\sim$-stroke [\textit{Urteilstrich}], \indexref{4.442}}

\indexentry{Julius Caesar, \indexref{5.02}}

\indexgap

\indexentry{Kant, \indexref{6.36111}}

\indexentry{know}

   \indexsubentry{1. [\textit{kennen}], \indexref{2.0123}, \indexref{2.01231}, \indexref{3.263}, \indexref{4.021}, \indexref{4.243}, \indexref{6.2322}; cf.\ theory of knowledge.}

   \indexsubentry{2. [\textit{wissen}], \indexref{3.05}, \indexref{3.24}, \indexref{4.024}, \indexref{4.461}, \indexref{5.1362}, \indexref{5.156}, \indexref{5.5562} \kckaddition{[---the previous entry was mistakenly listed as 5.562 in Pears and McGuinness's original index---]}, \indexref{6.3211}, \indexref{6.33}, \indexref{6.36311}}

\indexgap

\indexentry{language [\textit{Sprache}], \hyperlink{pref2}{P2}, \hyperlink{pref4}{P4}, \indexref{3.032}, \indexref{3.343}, \indexref{4.001}--\indexref{4.0031}, \indexref{4.014}, \indexref{4.0141}, \indexref{4.025}, \indexref{4.121}, \indexref{4.125}, \indexref{5.4731}, \indexref{5.535}, \indexref{5.6}, \indexref{5.62}, \indexref{6.12}, \indexref{6.233}, \indexref{6.43}; cf.\ critique of $\sim$; everyday $\sim$; sign-$\sim$.}

\indexentry{law}

   \indexsubentry{1. [\textit{Gesetz}: minimum-principle; primitive proposition], \indexref{3.031}, \indexref{3.032}, \indexref{3.0321}, \indexref{4.0141}, \indexref{5.501}, \indexref{6.123}, \indexref{6.3}--\indexref{6.3211}, \indexref{6.3431}, \indexref{6.35}, \indexref{6.361}, \indexref{6.363}, \indexref{6.422};}

   \indexsubsubentry{$\sim$ of causality [\textit{Kausalit{\"a}tsg.}], \indexref{6.32}, \indexref{6.321};}

   \indexsubsubentry{$\sim$ of conservation [\textit{Erhaltungsg.}], \indexref{6.33};}

   \indexsubsubentry{$\sim$ of contradiction [\textit{G.\ des Widerspruchs}], \indexref{6.1203}, \indexref{6.123};}

   \indexsubsubentry{$\sim$ of least action [\textit{G.\ der kleinsten Wirkung}], \indexref{6.321}, \indexref{6.3211};}
   \indexsubsubentry{$\sim$ of nature [\textit{Naturg.}], \indexref{5.154}, \indexref{6.34}, \indexref{6.36}, \indexref{6.371}, \indexref{6.372}}

   \indexsubentry{2. [\textit{Satz}: principle of sufficient reason; proposition], \indexref{6.34};}

   \indexsubsubentry{$\sim$ of continuity [\textit{S.\ von der Kontinuit{\"a}t}], \indexref{6.34};}

   \indexsubsubentry{$\sim$ of least effort [\textit{S.\ von kleinsten Aufwande}], \indexref{6.34}}

\indexentry{life, \indexref{5.621}, \indexref{6.4311}, \indexref{6.4312}, \indexref{6.52}, \indexref{6.521}}

\indexentry{limit [\textit{Grenze}: bound; delimit], \hyperlink{pref3}{P3}, \hyperlink{pref4}{P4}, \indexref{4.113}, \indexref{4.114}, \indexref{4.51}, \indexref{5.143}, \indexref{5.5561}, \indexref{5.6}--\indexref{5.62}, \indexref{5.632}, \indexref{5.641}, \indexref{6.4311}, \indexref{6.45}}

\indexentry{logic; $\sim$al, \indexref{2.012}, \indexref{2.0121}, \indexref{3.031}, \indexref{3.032}, \indexref{3.315}, \indexref{3.41}, \indexref{3.42}, \indexref{4.014}, \indexref{4.015}, \indexref{4.023}, \indexref{4.0312}, \indexref{4.032}, \indexref{4.112}, \indexref{4.1121}, \indexref{4.1213}, \indexref{4.126}, \indexref{4.128}, \indexref{4.466}, \indexref{5.02}, \indexref{5.1362}, \indexref{5.152}, \indexref{5.233}, \indexref{5.42}, \indexref{5.43}, \indexref{5.45}--\indexref{5.47}, \indexref{5.472}--\indexref{5.4731}, \indexref{5.47321}, \indexref{5.522}, \indexref{5.551}--\indexref{5.5521}, \indexref{5.555}, \indexref{5.5562}--\indexref{5.557}, \indexref{5.61}, \indexref{6.1}--\indexref{6.12}, \indexref{6.121}, \indexref{6.122}, \indexref{6.1222}--\indexref{6.2}, \indexref{6.22}, \indexref{6.234}, \indexref{6.3}, \indexref{6.31}, \indexref{6.3211}, \indexref{6.342}, \indexref{6.3431}, \indexref{6.3631}, \indexref{6.37}, \indexref{6.374}--\indexref{6.3751}; cf.\ form, $\sim$al; illogical.}

   \indexsubentry{$\sim$al addition, \indexref{5.2341}}

   \indexsubentry{$\sim$al constant, \indexref{4.0312}, \indexref{5.4}, \indexref{5.441}, \indexref{5.47}}

   \indexsubentry{$\sim$al grammar, \indexref{3.325}}

   \indexsubentry{$\sim$al multiplication, \indexref{5.2341}}

   \indexsubentry{$\sim$al object, \indexref{4.441}, \indexref{5.4}}

   \indexsubentry{$\sim$al picture, \indexref{2.18}--\indexref{2.19}, 3, \indexref{4.03}}

   \indexsubentry{$\sim$al place, \indexref{3.41}--\indexref{3.42}, \indexref{4.0641}}

   \indexsubentry{$\sim$al product, \indexref{3.42}, \indexref{4.465}, \indexref{5.521}, \indexref{6.1271}, \indexref{6.3751}}

   \indexsubentry{$\sim$al space, \indexref{1.13}, \indexref{2.11}, \indexref{2.202}, \indexref{3.4}, \indexref{3.42}, \indexref{4.463}}

   \indexsubentry{$\sim$al sum, \indexref{3.42}, \indexref{5.521}}

   \indexsubentry{$\sim$al syntax, \indexref{3.325}, \indexref{3.33}, \indexref{3.334}, \indexref{3.344}, \indexref{6.124}}

   \indexsubentry{$\sim$o-pictorial, cf.\ form.}

   \indexsubentry{$\sim$o-syntactical, \indexref{3.327}}

\indexgap

\indexentry{manifest [\textit{sich zeigen}: indicate; show], \indexref{4.122}, \indexref{5.24}, \indexref{5.4}, \indexref{5.513}, \indexref{5.515}, \indexref{5.5561}, \indexref{5.62}, \indexref{6.23}, \indexref{6.36}, \indexref{6.522}}

\indexentry{material, \indexref{2.0231}, \indexref{5.44}}

\indexentry{mathematics, \indexref{4.04}--\indexref{4.0411}, \indexref{5.154}, \indexref{5.43}, \indexref{5.475}, \indexref{6.031}, \indexref{6.2}--\indexref{6.22}, \indexref{6.2321}, \indexref{6.233}, \indexref{6.234}--\indexref{6.24}}

\indexentry{Mauthner, \indexref{4.0031}}

\indexentry{mean [\textit{meinen}], \indexref{3.315}, \indexref{4.062}, \indexref{5.62}}

\indexentry{meaning [\textit{Bedeutung}: signify], \indexref{3.203}, \indexref{3.261}, \indexref{3.263}, \indexref{3.3}, \indexref{3.314}, \indexref{3.315}, \indexref{3.317}, \indexref{3.323}, \indexref{3.328}--\indexref{3.331}, \indexref{3.333}, \indexref{4.002}, \indexref{4.026}, \indexref{4.126}, \indexref{4.241}--\indexref{4.243}, \indexref{4.466}, \indexref{4.5}, \indexref{5.02}, \indexref{5.31} , \indexref{5.451}, \indexref{5.461}, \indexref{5.47321}, \indexref{5.4733}, \indexref{5.535}, \indexref{5.55}, \indexref{5.6}, \indexref{5.62}, \indexref{6.124}, \indexref{6.126}, \indexref{6.232}, \indexref{6.2322}, \indexref{6.53}}

   \indexsubentry{equivalent in $\sim$ [\textit{Bedeutungsgleichheit}], \indexref{4.243}, \indexref{6.2323}}

   \indexsubentry{$\sim$ful [\textit{bedeutungsvoll}], \indexref{5.233}}

   \indexsubentry{$\sim$less [\textit{bedeutungslos}], \indexref{3.328}, \indexref{4.442}, \indexref{4.4661}, \indexref{5.47321}}

\indexentry{mechanics, \indexref{4.04}, \indexref{6.321}, \indexref{6.341}--\indexref{6.343}, \indexref{6.3432}}

\indexentry{mention [\textit{von etwas reden}: talk about], \indexref{3.24}, \indexref{3.33}, \indexref{4.1211}, \indexref{5.631}, \indexref{6.3432}; cf.\ about.}

\indexentry{metaphysical, \indexref{5.633}, \indexref{5.641}, \indexref{6.53}}

\indexentry{method, \indexref{3.11}, \indexref{4.1121}, \indexref{6.121}, \indexref{6.2}, \indexref{6.234}--\indexref{6.24}, \indexref{6.53}; cf.\ projection, $\sim$ of; zero-$\sim$.}

\indexentry{microcosm, \indexref{5.63}}

\indexentry{minimum-principle [\textit{Minimum-Gesetz}: law], \indexref{6.321}}

\indexentry{mirror, \indexref{4.121}, \indexref{5.511}, \indexref{5.512}, \indexref{5.514}, \indexref{6.13}}

   \indexsubentry{$\sim$-image [\textit{Spiegelbild}: picture], \indexref{6.13}}

\indexentry{misunderstanding, \hyperlink{pref2}{P2}}

\indexentry{mode, cf.\ expression; signification.}

\indexentry{model, \indexref{2.12}, \indexref{4.01}, \indexref{4.463}; cf.\ dynamical $\sim$.}

\indexentry{\textit{modus ponens}, \indexref{6.1264}}

\indexentry{monism, \indexref{4.128}}

\indexentry{Moore, \indexref{5.541}}

\indexentry{multiplicity, \indexref{4.04}--\indexref{4.0412}, \indexref{5.475}}

\indexentry{music, \indexref{3.141}, \indexref{4.011}, \indexref{4.014}, \indexref{4.0141}}

\indexentry{mystical, \indexref{6.44}, \indexref{6.45}, \indexref{6.522}}

\indexgap

\indexentry{name}

   \indexsubentry{1. [\textit{Name}], \indexref{3.142}, \indexref{3.143}, \indexref{3.144}, \indexref{3.202}, \indexref{3.203}, \indexref{3.22}, \indexref{3.26}, \indexref{3.261}, \indexref{3.3}, \indexref{3.314}, \indexref{3.3411}, \indexref{4.0311}, \indexref{4.126}, \indexref{4.1272}, \indexref{4.22}, \indexref{4.221}, \indexref{4.23}, \indexref{4.24}, \indexref{4.243}, \indexref{4.5}, \indexref{5.02}, \indexref{5.526}, \indexref{5.535}, \indexref{5.55}, \indexref{6.124}; cf.\ variable $\sim$.}

   \indexsubsubentry{general $\sim$ [\textit{Gattungsn.}], \indexref{6.321}}

   \indexsubsubentry{proper $\sim$ of a person [\textit{Personenn.}], \indexref{3.323}}

   \indexsubentry{2. [\textit{benennen}; \textit{nennen}], \indexref{3.144}, \indexref{3.221}}

\indexentry{natur/e, \indexref{2.0123}, \indexref{3.315}, \indexref{5.47}, \indexref{6.124}; cf.\ law of $\sim$e.}

   \indexsubsubentry{$\sim$al phenomena, \indexref{6.371}}

   \indexsubsubentry{$\sim$al science, \indexref{4.11}, \indexref{4.111}, \indexref{4.1121}--\indexref{4.113}, \indexref{6.111}, \indexref{6.4312}, \indexref{6.53}}

\indexentry{necessary, \indexref{4.041}, \indexref{5.452}, \indexref{5.474}, \indexref{6.124}; cf.\ unnecessary.}

\indexentry{negation}

   \indexsubentry{1. [\textit{Negation}], \indexref{5.5}, \indexref{5.502}}

   \indexsubentry{2. [\textit{Verneinung}], \indexref{3.42}, \indexref{4.0621}, \indexref{4.064}, \indexref{4.0641}, \indexref{5.1241}, \indexref{5.2341}, \indexref{5.254}, \indexref{5.44}, \indexref{5.451}, \indexref{5.5}, \indexref{5.512}, \indexref{5.514}, \indexref{6.231}}

\indexentry{negative [\textit{negativ}], \indexref{4.463}, \indexref{5.513}, \indexref{5.5151}}

   \indexsubentry{$\sim$ fact, \indexref{2.06}, \indexref{4.063}, \indexref{5.5151}}

\indexentry{network, \indexref{5.511}, \indexref{6.341}, \indexref{6.342}, \indexref{6.35}}

\indexentry{Newton, \indexref{6.341}, \indexref{6.342}}

\indexentry{nexus}

   \indexsubentry{1. [\textit{Nexus}], \indexref{5.136}, \indexref{5.1361}}

   \indexsubentry{2. [\textit{Zusammenhang}: connexion], \indexref{3.3}, \indexref{4.22}, \indexref{4.23}}

\indexentry{non-proposition, \indexref{5.5351}}

\indexentry{nonsense [\textit{Unsinn}], \hyperlink{pref4}{P4}, \indexref{3.24}, \indexref{4.003}, \indexref{4.124}, \indexref{4.1272}, \indexref{4.1274}, \indexref{4.4611}, \indexref{5.473}, \indexref{5.5303}, \indexref{5.5351}, \indexref{5.5422}, \indexref{5.5571}, \indexref{6.51}, \indexref{6.54}; cf.\ sense, have no.}

\indexentry{notation, \indexref{3.342}, \indexref{3.3441}, \indexref{5.474}, \indexref{5.512}--\indexref{5.514}, \indexref{6.1203}, \indexref{6.122}, \indexref{6.1223}; cf.\ conceptual $\sim$.}

\indexentry{number}

   \indexsubentry{1. [\textit{Anzahl}], \indexref{4.1272}, \indexref{5.474}--\indexref{5.476}, \indexref{5.55}, \indexref{5.553}, \indexref{6.1271}}

   \indexsubentry{2. [\textit{Zahl}: integer], \indexref{4.1252}, \indexref{4.126}, \indexref{4.1272}, \indexref{4.12721}, \indexref{4.128}, \indexref{5.453}, \indexref{5.553}, \indexref{6.02}, \indexref{6.022}; cf.\ equality, numerical; privileged $\sim$s; series of $\sim$s; variable $\sim$.}

   \indexsubsubentry{cardinal $\sim$, \indexref{5.02};}

   \indexsubsubentry{$\sim$-system, \indexref{6.341}}

\indexgap

\indexentry{object [\textit{Gegenstand}], \indexref{2.01}, \indexref{2.0121}, \indexref{2.0123}--\indexref{2.0124}, \indexref{2.0131}--\indexref{2.02}, \indexref{2.021}, \indexref{2.023}--\indexref{2.0233}, \indexref{2.0251}--\indexref{2.032}, \indexref{2.13}, \indexref{2.15121}, \indexref{3.1431}, \indexref{3.2}, \indexref{3.203}--\indexref{3.221}, \indexref{3.322}, \indexref{3.3411}, \indexref{4.023}, \indexref{4.0312}, \indexref{4.1211}, \indexref{4.122}, \indexref{4.123}, \indexref{4.126}, \indexref{4.127}, \indexref{4.1272}, \indexref{4.12721}, \indexref{4.2211}, \indexref{4.431}, \indexref{4.441}, \indexref{4.466}, \indexref{5.02}, \indexref{5.123}, \indexref{5.1511}, \indexref{5.4}, \indexref{5.44}, \indexref{5.524}, \indexref{5.526}, \indexref{5.53}--\indexref{5.5302}, \indexref{5.541}, \indexref{5.542}, \indexref{5.5561}, \indexref{6.3431} ; cf.\ thing.}

\indexentry{obvious [\textit{sich von selbst verstehen}: say; understand], \indexref{6.111}; cf.\ self-evidence.}

\indexentry{Occam, \indexref{3.328}, \indexref{5.47321}}

\indexentry{occur [\textit{vorkommen}], \indexref{2.012}--\indexref{2.0123}, \indexref{2.0141}, \indexref{3.24}, \indexref{3.311}, \indexref{4.0621}, \indexref{4.1211}, \indexref{4.23}, \indexref{4.243}, \indexref{5.25}, \indexref{5.451}, \indexref{5.54}, \indexref{5.541}, \indexref{6.1203}}

\indexentry{operation, \indexref{4.1273}, \indexref{5.21}--\indexref{5.254}, \indexref{5.4611}, \indexref{5.47}, \indexref{5.5}, \indexref{5.503}, \indexref{6.001}--\indexref{6.01}, \indexref{6.021}, \indexref{6.126}; cf.\ sign for a logical $\sim$; truth-$\sim$.}

\indexentry{oppos/ed; $\sim$ite [\textit{entgegengesetzt}], \indexref{4.0621}, \indexref{4.461}, \indexref{5.1241}, \indexref{5.513}}

\indexentry{order, \indexref{4.1252}, \indexref{5.5563}, \indexref{5.634}}

\indexgap

\indexentry{paradox, Russell's, \indexref{3.333}}

\indexentry{particle, \indexref{6.3751}}

\indexentry{perceive, \indexref{3.1}, \indexref{3.11}, \indexref{3.32}, \indexref{5.5423}}

\indexentry{phenomenon, \indexref{6.423}; cf.\ natural $\sim$.}

\indexentry{philosophy, \hyperlink{pref2}{P2}, \hyperlink{pref5}{P5}, \indexref{3.324}, \indexref{3.3421}, \indexref{4.003}, \indexref{4.0031}, \indexref{4.111}--\indexref{4.115}, \indexref{4.122}, \indexref{4.128}, \indexref{5.641}, \indexref{6.113}, \indexref{6.211}, \indexref{6.53}}

\indexentry{physics, \indexref{3.0321}, \indexref{6.321}, \indexref{6.341}, \indexref{6.3751}}

\indexentry{pictorial}

   \indexsubentry{1. [\textit{abbilden}: depict; form, logico-$\sim$], \indexref{2.15}, \indexref{2.151}, \indexref{2.1513}, \indexref{2.1514}, \indexref{2.17}, \indexref{2.172}, \indexref{2.181}, \indexref{2.22}; cf.\ form, $\sim$.}

   \indexsubentry{2. [\textit{bildhaftig}], \indexref{4.013}, \indexref{4.015}}

\indexentry{picture [\textit{Bild}: mirror-image; \textit{tableau vivant}], \indexref{2.0212}, \indexref{2.1}--\indexref{2.1512}, \indexref{2.1513}--\indexref{3.01}, \indexref{3.42}, \indexref{4.01}--\indexref{4.012}, \indexref{4.021}, \indexref{4.03}, \indexref{4.032}, \indexref{4.06}, \indexref{4.462}, \indexref{4.463}, \indexref{5.156}, \indexref{6.341}, \indexref{6.342}, \indexref{6.35}; cf.\ logical $\sim$; prototype.}

\indexentry{place [\textit{Ort}], \indexref{3.411}, \indexref{6.3751}; cf.\ logical $\sim$.}

\indexentry{point-mass [\textit{materieller Punkt}], \indexref{6.3432}}

\indexentry{positive, \indexref{2.06}, \indexref{4.063}, \indexref{4.463}, \indexref{5.5151}}

\indexentry{possible, \indexref{2.012}, \indexref{2.0121}, \indexref{2.0123}--\indexref{2.0141}, \indexref{2.033}, \indexref{2.15}, \indexref{2.151}, \indexref{2.201}--\indexref{2.203}, \indexref{3.02}, \indexref{3.04}, \indexref{3.11}, \indexref{3.13}, \indexref{3.23}, \indexref{3.3421}, \indexref{3.3441}, \indexref{3.411}, \indexref{4.015}, \indexref{4.0312}, \indexref{4.124}, \indexref{4.125}, \indexref{4.2}, \indexref{4.27}--\indexref{4.3}, \indexref{4.42}, \indexref{4.45}, \indexref{4.46}, \indexref{4.462}, \indexref{4.464}, \indexref{4.5}, \indexref{5.252}, \indexref{5.42}, \indexref{5.44}, \indexref{5.46}, \indexref{5.473}, \indexref{5.4733}, \indexref{5.525}, \indexref{5.55}, \indexref{5.61}, \indexref{6.1222}, \indexref{6.33}, \indexref{6.34}, \indexref{6.52}; cf.\ impossibility; truth-possibility.}

\indexentry{postulate [\textit{Forderung}: requirement], \indexref{6.1223}}

\indexentry{predicate, cf.\ subject.}

\indexentry{present}

   \indexsubentry{1. [\textit{darstellen}: represent], \indexref{3.312}, \indexref{3.313}, \indexref{4.115}}

   \indexsubentry{2. [\textit{vorstellen}: Idea; represent], \indexref{2.11}, \indexref{4.0311}}

\indexentry{presuppose [\textit{voraussetzen}], \indexref{3.31}, \indexref{3.33}, \indexref{4.1241}, \indexref{5.515}, \indexref{5.5151}, \indexref{5.61}, \indexref{6.124}}

\indexentry{primitive idea [\textit{Grundbegriff}] \indexref{4.12721}, \indexref{5.451}, \indexref{5.476}}

\indexentry{primitive proposition [\textit{Grundgesetz}], \indexref{5.43}, \indexref{5.452}, \indexref{6.127}, \indexref{6.1271}; cf.\ \textit{Fundamental Laws of Arithmetic}; law.}

\indexentry{primitive sign [\textit{Urzeichen}], \indexref{3.26}, \indexref{3.261}, \indexref{3.263}, \indexref{5.42}, \indexref{5.45}, \indexref{5.451}, \indexref{5.46}, \indexref{5.461}, \indexref{5.472}}

\indexentry{\textit{Principia Mathematica}, \indexref{5.452}}

\indexentry{principle of sufficient reason [\textit{Satz vom Grunde}: law; proposition], \indexref{6.34}, \indexref{6.35}}

\indexentry{\textit{Principles of Mathematics}, \indexref{5.5351}}

\indexentry{privileged [\textit{ausgezeichnet}], $\sim$ numbers, \indexref{4.128}, \indexref{5.453}, \indexref{5.553}}

\indexentry{probability, \indexref{4.464}, \indexref{5.15}--\indexref{5.156}}

\indexentry{problem}

   \indexsubentry{1. [\textit{Fragestellung}: question], \hyperlink{pref2}{P2}, \indexref{5.62}}

   \indexsubentry{2. [\textit{Problem}], \hyperlink{pref2}{P2}, \indexref{4.003}, \indexref{5.4541}, \indexref{5.535}, \indexref{5.551}, \indexref{5.5563}, \indexref{6.4312}, \indexref{6.521}}

\indexentry{product, cf.\ logical.}

\indexentry{project/ion; $\sim$ive, \indexref{3.11}--\indexref{3.13}, \indexref{4.0141}}

   \indexsubentry{method of $\sim$ion, \indexref{3.11}}

\indexentry{proof [\textit{Beweis}], \indexref{6.126}, \indexref{6.1262}, \indexref{6.1263}--\indexref{6.1265}, \indexref{6.2321}, \indexref{6.241}}

\indexentry{proper, cf.\ name.}

\indexentry{property [\textit{Eigenschaft}], \indexref{2.01231}, \indexref{2.0231}, \indexref{2.0233}, \indexref{2.02331}, \indexref{4.023}, \indexref{4.063}, \indexref{4.122}--\indexref{4.1241}, \indexref{5.473}, \indexref{5.5302}, \indexref{6.111}, \indexref{6.12}, \indexref{6.121}, \indexref{6.126}, \indexref{6.231}, \indexref{6.35}; cf.\ formal $\sim$.}

\indexentry{proposition [\textit{Satz}: law; principle], \indexref{2.0122}, \indexref{2.0201}, \indexref{2.0211}, \indexref{2.0231}, \indexref{3.1} (\&\ passim thereafter); cf.\ non-$\sim$; primitive $\sim$; pseudo-$\sim$; variable, $\sim$al; variable $\sim$.}

   \indexsubentry{$\sim$al form, \indexref{3.312}, \indexref{4.0031}, \indexref{4.012}, \indexref{4.5}, \indexref{4.53}, \indexref{5.131}, \indexref{5.1311}, \indexref{5.156}, \indexref{5.231}, \indexref{5.24}, \indexref{5.241}, \indexref{5.451}, \indexref{5.47}, \indexref{5.471}, \indexref{5.472}, \indexref{5.54}--\indexref{5.542}, \indexref{5.5422}, \indexref{5.55}, \indexref{5.554}, \indexref{5.555}, \indexref{5.556}, \indexref{6}, \indexref{6.002}}

   \indexsubentry{$\sim$al sign, \indexref{3.12}, \indexref{3.14}, \indexref{3.143}, \indexref{3.1431}, \indexref{3.2}, \indexref{3.21}, \indexref{3.332}, \indexref{3.34}, \indexref{3.41}, \indexref{3.5}, \indexref{4.02}, \indexref{4.44}, \indexref{4.442}, \indexref{5.31}}

\indexentry{prototype [\textit{Urbild}], \indexref{3.24}, \indexref{3.315}, \indexref{3.333}, \indexref{5.522}, \indexref{5.5351}; cf.\ picture.}

\indexentry{pseudo-, cf.\ apparent.}

   \indexsubentry{$\sim$-concept, \indexref{4.1272}}

   \indexsubentry{$\sim$-proposition, \indexref{4.1272}, \indexref{5.534}, \indexref{5.535}, \indexref{6.2}}

   \indexsubentry{$\sim$relation, \indexref{5.461}}

\indexentry{psychology, \indexref{4.1121}, \indexref{5.541}, \indexref{5.5421}, \indexref{5.641}, \indexref{6.3631}, \indexref{6.423}}

\indexentry{punishment, \indexref{6.422}}

\indexgap

\indexentry{question [\textit{Frage}: problem], \indexref{4.003}, \indexref{4.1274}, \indexref{5.4541}, \indexref{5.55}, \indexref{5.551}, \indexref{5.5542}, \indexref{6.5}--\indexref{6.52}}

\indexgap

\indexentry{range [\textit{Spielraum}], \indexref{4.463}, \indexref{5.5262}; cf.\ space.}

\indexentry{real [\textit{wirklich}], \indexref{2.022}, \indexref{4.0031}, \indexref{5.461}}

\indexentry{realism, \indexref{5.64}}

\indexentry{reality}

   \indexsubentry{1. [\textit{Realit{\"a}t}], \indexref{5.5561}, \indexref{5.64}}

   \indexsubentry{2. [\textit{Wirklichkeit}], \indexref{2.06}, \indexref{2.063}, \indexref{2.12}, \indexref{2.1511}, \indexref{2.1512}, \indexref{2.1515}, \indexref{2.17}, \indexref{2.171}, \indexref{2.18}, \indexref{2.201}, \indexref{2.21}, \indexref{2.222}, \indexref{2.223}, \indexref{4.01}, \indexref{4.011}, \indexref{4.021}, \indexref{4.023}, \indexref{4.05}, \indexref{4.06}, \indexref{4.0621}, \indexref{4.12}, \indexref{4.121}, \indexref{4.462}, \indexref{4.463}, \indexref{5.512}}

\indexentry{reducibility, cf.\ axiom.}

\indexentry{relation}

   \indexsubentry{1. [\textit{Beziehung}], \indexref{2.1513}, \indexref{2.1514}, \indexref{3.12}, \indexref{3.1432}, \indexref{3.24}, \indexref{4.0412}, \indexref{4.061}, \indexref{4.0641}, \indexref{4.462} \indexref{4.4661}, \indexref{5.131}, \indexref{5.1311}, \indexref{5.2}--\indexref{5.22}, \indexref{5.42}, \indexref{5.461}, \indexref{5.4733}, \indexref{5.5151}, \indexref{5.5261}, \indexref{5.5301}; cf.\ pseudo-.}

   \indexsubentry{2. [\textit{Relation}], \indexref{4.122}, \indexref{4.123}, \indexref{4.125}, \indexref{4.1251}, \indexref{5.232}, \indexref{5.42}, \indexref{5.5301}, \indexref{5.541}, \indexref{5.553}, \indexref{5.5541} ; cf.\ formal.}

   \indexsubentry{3. stand in a $\sim$ to one another; are related [\textit{sich verhalten}: stand, how things; state of things], \indexref{2.03}, \indexref{2.14}, \indexref{2.15}, \indexref{2.151}, \indexref{3.14}, \indexref{5.5423}}

\indexentry{represent}

   \indexsubentry{1. [\textit{darstellen}: present], \indexref{2.0231}, \indexref{2.173}, \indexref{2.174}, \indexref{2.201}--\indexref{2.203}, \indexref{2.22}, \indexref{2.221}, \indexref{3.032}, \indexref{3.0321}, \indexref{4.011}, \indexref{4.021}, \indexref{4.031}, \indexref{4.04}, \indexref{4.1}, \indexref{4.12}, \indexref{4.121}, \indexref{4.122}, \indexref{4.124}, \indexref{4.125}, \indexref{4.126}, \indexref{4.1271}, \indexref{4.1272}, \indexref{4.24}, \indexref{4.31}, \indexref{4.462}, \indexref{5.21}, \indexref{6.1203}, \indexref{6.124}, \indexref{6.1264}; cf.\ form, $\sim$ational.}

   \indexsubentry{2. [\textit{vorstellen}: idea; present], \indexref{2.15}}

\indexentry{representative, be the $\sim$ of [\textit{vertreten}], \indexref{2.131}, \indexref{3.22}, \indexref{3.221}, \indexref{4.0312}, \indexref{5.501}}

\indexentry{requirement [\textit{Forderung}: postulate], \indexref{3.23}}

\indexentry{resolve, cf.\ analysis.}

   \indexsubentry{1. [\textit{aufl{\"o}sen}], \indexref{3.3442}}

   \indexsubentry{2. [\textit{zerlegen}], \indexref{2.0201}}

\indexentry{reward, \indexref{6.422}}

\indexentry{riddle, \indexref{6.4312}, \indexref{6.5}}

\indexentry{right [\textit{stimmen}: agreement; true], \indexref{3.24}}

\indexentry{rule [\textit{Regel}], \indexref{3.334}, \indexref{3.343}, \indexref{3.344}, \indexref{4.0141}, \indexref{5.47321}, \indexref{5.476}, \indexref{5.512}, \indexref{5.514}}

   \indexsubentry{combinatory $\sim$ [\textit{Kombinationsr.}], \indexref{4.442}}

   \indexsubentry{$\sim$ dealing with signs [\textit{Zeichenr.}] \indexref{3.331}, \indexref{4.241}, \indexref{6.02}, \indexref{6.126}}

\indexentry{Russell, \hyperlink{pref6}{P6}, \indexref{3.318}, \indexref{3.325}, \indexref{3.331}, \indexref{3.333}, \indexref{4.0031}, \indexref{4.1272}--\indexref{4.1273}, \indexref{4.241}, \indexref{4.442}, \indexref{5.02}, \indexref{5.132}, \indexref{5.252}, \indexref{5.4}, \indexref{5.42}, \indexref{5.452}, \indexref{5.4731}, \indexref{5.513}, \indexref{5.521}, \indexref{5.525}, \indexref{5.5302}, \indexref{5.532}, \indexref{5.535}, \indexref{5.5351}, \indexref{5.541}, \indexref{5.5422}, \indexref{5.553}, \indexref{6.123}, \indexref{6.1232}}

\indexgap

\indexentry{say}

   \indexsubentry{1. [\textit{angeben}: give], \indexref{5.5571}}

   \indexsubentry{2. [\textit{audr{\"u}cksen}: expression], \indexref{5.5151}}

   \indexsubentry{3. [\textit{aussprechen}: words, put into], $\sim$ clearly, \indexref{3.262}}

   \indexsubentry{4. [\textit{sagen}], can be said, \hyperlink{pref3}{P3}, \indexref{3.031}, \indexref{4.115}, \indexref{4.1212}, \indexref{5.61} \indexref{5.62}, \indexref{6.36}, \indexref{6.51}, \indexref{6.53};}

   \indexsubsubentry{said) (shown, \indexref{4.022}, \indexref{4.1212}, \indexref{5.535}, \indexref{5.62}, \indexref{6.36};}

   \indexsubsubentry{$\sim$ nothing, \indexref{4.461}, \indexref{5.142}, \indexref{5.43}, \indexref{5.4733}, \indexref{5.513}, \indexref{5.5303}, \indexref{6.11}, \indexref{6.121}, \indexref{6.342}, \indexref{6.35}}

   \indexsubentry{5. [\textit{sich von selbst verstehen}: obvious; understand], $\sim$ing,}

   \indexsubsubentry{go without, \indexref{3.334}, \indexref{6.2341}}

\indexentry{scaffolding, \indexref{3.42}, \indexref{4.023}, \indexref{6.124}}

\indexentry{scepticism, \indexref{6.51}}

\indexentry{schema, \indexref{4.31}, \indexref{4.43}, \indexref{4.441}, \indexref{4.442}, \indexref{5.101}, \indexref{5.151}, \indexref{5.31}}

\indexentry{science, \indexref{6.34}, \indexref{6.341}, \indexref{6.52}; cf.\ natural $\sim$.}

\indexentry{scope, \indexref{4.0411}}

\indexentry{segmented [\textit{gegliedert}], \indexref{4.032}; cf.\ articulated.}

\indexentry{self, the [\textit{das Ich}], \indexref{5.64}, \indexref{5.641}}

\indexentry{self-evidence [\textit{Einleuchten}], \indexref{5.1363}, \indexref{5.42}, \indexref{5.4731}, \indexref{5.5301}, \indexref{6.1271}; cf.\ obvious.}

\indexentry{sense [\textit{Sinn}; \textit{sinnvoll}], \hyperlink{pref2}{P2}, \indexref{2.0211}, \indexref{2.221}, \indexref{2.222}, \indexref{3.11}, \indexref{3.13}, \indexref{3.142}, \indexref{3.1431}, \indexref{3.144}, \indexref{3.23}, \indexref{3.3}, \indexref{3.31}, \indexref{3.326}, \indexref{3.34}, \indexref{3.341}, \indexref{3.4}, \indexref{4.002}, \indexref{4.011}, \indexref{4.014}, \indexref{4.02}--\indexref{4.022}, \indexref{4.027}--\indexref{4.031}, \indexref{4.032}, \indexref{4.061}, \indexref{4.0621}--\indexref{4.064}, \indexref{4.1211}, \indexref{4.122}, \indexref{4.1221}, \indexref{4.1241}, \indexref{4.126}, \indexref{4.2}, \indexref{4.243}, \indexref{4.431}, \indexref{4.465}, \indexref{4.52}, \indexref{5.02}, \indexref{5.122}, \indexref{5.1241}, \indexref{5.2341}, \indexref{5.25}, \indexref{5.2521}, \indexref{5.4}, \indexref{5.42}, \indexref{5.44}, \indexref{5.46}, \indexref{5.4732}, \indexref{5.4733}, \indexref{5.514}, \indexref{5.515}, \indexref{5.5302}, \indexref{5.5542}, \indexref{5.631}, \indexref{5.641}, \indexref{6.124}, \indexref{6.126}, \indexref{6.232}, \indexref{6.41}, \indexref{6.422}, \indexref{6.521}}

   \indexsubentry{have the same $\sim$ [\textit{gleichsinnig}], \indexref{5.515}}

   \indexsubentry{have no $\sim$; lack $\sim$; without $\sim$ [\textit{sinnlos}], \indexref{4.461}, \indexref{5.132}, \indexref{5.1362}, \indexref{5.5351}; cf.\ nonsense.}

   \indexsubentry{$\sim$ of touch [\textit{Tastsinn}], \indexref{2.0131}}

\indexentry{series [\textit{Reihe}], \indexref{4.1252}, \indexref{4.45}, \indexref{5.1}, \indexref{5.232}, \indexref{6.02}}

   \indexsubentry{$\sim$ of forms [\textit{Formenr}.], \indexref{4.1252}, \indexref{4.1273}, \indexref{5.252}, \indexref{5.2522}, \indexref{5.501}}

   \indexsubentry{$\sim$ of numbers [\textit{Zahlenr}.], \indexref{4.1252}}

\indexentry{set [\textit{Klasse}: class], \indexref{3.142}}

\indexentry{show [\textit{zeigen}: indicate; manifest], \indexref{3.262}, \indexref{4.022}, \indexref{4.0621}, \indexref{4.0641}, \indexref{4.121}--\indexref{4.1212}, \indexref{4.126}, \indexref{4.461}, \indexref{5.1311}, \indexref{5.24}, \indexref{5.42}, \indexref{5.5261}, \indexref{5.5421}, \indexref{5.5422}, \indexref{5.631}, \indexref{6.12}, \indexref{6.1201}, \indexref{6.1221}, \indexref{6.126}, \indexref{6.127}, \indexref{6.22}, \indexref{6.232}; cf.\ display; say.}

\indexentry{sign [\textit{Zeichen}], \indexref{3.11}, \indexref{3.12}, \indexref{3.1432}, \indexref{3.201}--\indexref{3.203}, \indexref{3.21}, \indexref{3.221}, \indexref{3.23}, \indexref{3.261}--\indexref{3.263}, \indexref{3.315}, \indexref{3.32}--\indexref{3.322}, \indexref{3.325}--\indexref{3.334}, \indexref{3.3442}, \indexref{4.012}, \indexref{4.026}, \indexref{4.0312}, \indexref{4.061}, \indexref{4.0621}, \indexref{4.126}, \indexref{4.1271}, \indexref{4.1272}, \indexref{4.241}--\indexref{4.243}, \indexref{4.431}--\indexref{4.441}, \indexref{4.466}, \indexref{4.4661}, \indexref{5.02}, \indexref{5.451}, \indexref{5.46}, \indexref{5.473}, \indexref{5.4732}--\indexref{5.4733}, \indexref{5.475}, \indexref{5.501}, \indexref{5.512}, \indexref{5.515}, \indexref{5.5151}, \indexref{5.53}, \indexref{5.5541}, \indexref{5.5542}, \indexref{6.02}, \indexref{6.1203}, \indexref{6.124}, \indexref{6.126}, \indexref{6.1264}, \indexref{6.53}; cf.\ primitive $\sim$; propositional $\sim$; rule dealing with $\sim$s; simple $\sim$.}

   \indexsubentry{be a $\sim$ for [\textit{bezeichnen}: designate; signify], \indexref{5.42}}

   \indexsubentry{combination of $\sim$s [\textit{Zeichenverbindung}], \indexref{4.466}, \indexref{5.451}}

   \indexsubentry{$\sim$ for a logical operation [\textit{logisches Operationsz}.], \indexref{5.4611}}

   \indexsubentry{$\sim$-language [\textit{Zeichensprache}], \indexref{3.325}, \indexref{3.343}, \indexref{4.011}, \indexref{4.1121}, \indexref{4.1213}, \indexref{4.5}, \indexref{6.124}}

\indexentry{signif/y}

   \indexsubentry{1. [\textit{bedeuten}: meaning], \indexref{4.115}}

   \indexsubentry{2. [\textit{bezeichnen}: designate: sign], \indexref{3.24}, \indexref{3.261}, \indexref{3.317}, \indexref{3.321}, \indexref{3.322}, \indexref{3.333}, \indexref{3.334}, \indexref{3.3411}, \indexref{3.344}, \indexref{4.012}, \indexref{4.061}, \indexref{4.126}, \indexref{4.127}, \indexref{4.1272}, \indexref{4.243}, \indexref{5.473}, \indexref{5.4733}, \indexref{5.476}, \indexref{5.5261}, \indexref{5.5541}, \indexref{6.111};}

   \indexsubsubentry{mode of $\sim$ication [\textit{Bezeichnungsweise}], \indexref{3.322}, \indexref{3.323}, \indexref{3.325}, \indexref{3.3421}, \indexref{4.0411}, \indexref{5.1311}}

\indexentry{similarity, \indexref{4.0141}, \indexref{5.231}}

\indexentry{simple, \indexref{2.02}, \indexref{3.24}, \indexref{4.21}, \indexref{4.24}, \indexref{4.51}, \indexref{5.02}, \indexref{5.4541}, \indexref{5.553}, \indexref{5.5563}, \indexref{6.341}, \indexref{6.342}, \indexref{6.363}, \indexref{6.3631};}

   \indexsubentry{$\sim$ sign, \indexref{3.201}, \indexref{3.202}, \indexref{3.21}, \indexref{3.23}, \indexref{4.026}}

\indexentry{\textit{simplex sigillum veri}, \indexref{5.4541}}

\indexentry{situation [\textit{Sachlage}], \indexref{2.0121}, \indexref{2.014}, \indexref{2.11}, \indexref{2.202}, \indexref{2.203}, \indexref{3.02}, \indexref{3.11}, \indexref{3.144}, \indexref{3.21}, \indexref{4.021}, \indexref{4.03}, \indexref{4.031}, \indexref{4.032} \indexref{4.04}, \indexref{4.124}, \indexref{4.125}, \indexref{4.462}, \indexref{4.466}, \indexref{5.135}, \indexref{5.156}, \indexref{5.525}}

\indexentry{Socrates, \indexref{5.473}, \indexref{5.4733}}

\indexentry{solipsism, \indexref{5.62}, \indexref{5.64}}

\indexentry{solution, \hyperlink{pref8}{P8}, \indexref{5.4541}, \indexref{5.535}, \indexref{6.4312}, \indexref{6.4321}, \indexref{6.521}}

\indexentry{soul, \indexref{5.5421}, \indexref{5.641}, \indexref{6.4312}}

\indexentry{space [\textit{Raum}], \indexref{2.0121}, \indexref{2.013}, \indexref{2.0131}, \indexref{2.0251}, \indexref{2.11}, \indexref{2.171}, \indexref{2.182}, \indexref{2.202}, \indexref{3.032}--\indexref{3.0321}, \indexref{3.1431}, \indexref{4.0412}, \indexref{4.463}, \indexref{6.3611}, \indexref{6.36111}, \indexref{6.4312}; cf.\ colour-$\sim$; logical $\sim$; range.}

\indexentry{speak/ about [\textit{von etwas sprechen}], \indexref{3.221}, \indexref{6.3431}, \indexref{6.423}, \indexref{7}; cf.\ about.}

   \indexsubentry{$\sim$ for itself [\textit{aussagen}: ascribe; state; statement; tell], \indexref{6.124}}

\indexentry{stand/, how things [\textit{sich verhalten}: relation; state of things], \indexref{4.022}, \indexref{4.023}, \indexref{4.062}, \indexref{4.5}}

   \indexsubentry{$\sim$ for [\textit{f{\"u}r etwas stehen}], \indexref{4.0311}, \indexref{5.515}}

\indexentry{state [\textit{aussagen}: ascribe; speak; statement; tell], \indexref{3.317}, \indexref{4.03}, \indexref{4.242}, \indexref{4.442}, \indexref{6.1264}}

\indexentry{statement [\textit{Aussage}], \indexref{2.0201}, \indexref{6.3751}}

   \indexsubentry{make a $\sim$ [\textit{aussagen}: ascribe; speak; state; tell], \indexref{3.332}, \indexref{5.25}}

\indexentry{state of/ affairs [\textit{Sachverhalt}: $\sim$ things], \indexref{2}--\indexref{2.013}, \indexref{2.014}, \indexref{2.0272}- \indexref{2.062}, \indexref{2.11}, \indexref{2.201}, \indexref{3.001}, \indexref{3.0321}, \indexref{4.023}, \indexref{4.0311}, \indexref{4.1}, \indexref{4.122}, \indexref{4.2}, \indexref{4.21}, \indexref{4.2211}, \indexref{4.25}, \indexref{4.27}, \indexref{4.3}}

   \indexsubentry{$\sim$ things}

   \indexsubsubentry{1. [\textit{Sachverhalt}: $\sim$ affairs], \indexref{2.01}}

   \indexsubsubentry{2. [\textit{sich verhalten}: relation; stand, how things], \indexref{5.552}}

\indexentry{stipulate [\textit{festsetzen}], \indexref{3.316}, \indexref{3.317}, \indexref{5.501}}

\indexentry{structure [\textit{Struktur}], \indexref{2.032}--\indexref{2.034}, \indexref{2.15}, \indexref{4.1211}, \indexref{4.122}, \indexref{5.13}, \indexref{5.2}, \indexref{5.22}, \indexref{6.12}, \indexref{6.3751}}

\indexentry{subject}

   \indexsubentry{1. [\textit{Subjekt}], \indexref{5.5421}, \indexref{5.631}--\indexref{5.633}, \indexref{5.641};}

   \indexsubsubentry{$\sim$-predicate propositions, \indexref{4.1274}}

   \indexsubentry{2. [\textit{Tr{\"a}ger}], \indexref{6.423}}

   \indexsubentry{3. $\sim$-matter [\textit{von etwas handeln}: about; concerned with; deal with], \indexref{6.124}}

\indexentry{subsistent [\textit{bestehen}: existence; hold], \indexref{2.024}, \indexref{2.027}, \indexref{2.0271}}

\indexentry{\textit{sub specie aeterni}, \indexref{6.45}; cf.\ eternity.}

\indexentry{substance [\textit{Substanz}], \indexref{2.021}, \indexref{2.0211}, \indexref{2.0231}, \indexref{2.04}}

\indexentry{substitut/e, \indexref{3.344}, \indexref{3.3441}, \indexref{4.241}, \indexref{6.23}, \indexref{6.24}}

   \indexsubentry{$\sim$ion, method of, \indexref{6.24}}

\indexentry{successor [\textit{Nachfolger}], \indexref{4.1252}, \indexref{4.1273}}

\indexentry{sum, cf.\ logical.}

\indexentry{sum-total [gesamt: totality; whole], \indexref{2.063}}

\indexentry{superstition, \indexref{5.1361}}

\indexentry{supposition [\textit{Annahme}], \indexref{4.063}}

\indexentry{survival [\textit{Fortleben}], \indexref{6.4312}}

\indexentry{symbol [\textit{Symbol}], \indexref{3.24}, \indexref{3.31}, \indexref{3.317}, \indexref{3.32}, \indexref{3.321}, \indexref{3.323}, \indexref{3.325}, \indexref{3.326}, \indexref{3.341}, \indexref{3.3411}, \indexref{3.344}, \indexref{4.126}, \indexref{4.24}, \indexref{4.31}, \indexref{4.465}, \indexref{4.4661}, \indexref{4.5}, \indexref{5.1311}, \indexref{5.473}, \indexref{5.4733}, \indexref{5.513}--\indexref{5.515}, \indexref{5.525}, \indexref{5.5351}, \indexref{5.555}, \indexref{6.113}, \indexref{6.124}, \indexref{6.126}}

   \indexsubentry{$\sim$ism [\textit{Symbolismus}], \indexref{4.461}, \indexref{5.451}}

\indexentry{syntax, cf.\ logical.}

\indexentry{system, \indexref{5.475}, \indexref{5.555}, \indexref{6.341}, \indexref{6.372}; cf.\ number-$\sim$.}

\indexgap

\indexentry{\textit{tableau vivant} [\textit{lebendes Bild}: picture], \indexref{4.0311}}

\indexentry{talk about [\textit{von etwas reden}: mention], \hyperlink{pref2}{P2}, \indexref{5.641}, \indexref{6.3432}; cf.\ about.}

\indexentry{tautology, \indexref{4.46}--\indexref{4.4661}, \indexref{5.101}, \indexref{5.1362}, \indexref{5.142}, \indexref{5.143}, \indexref{5.152}, \indexref{5.525}, \indexref{6.1}, \indexref{6.12}--\indexref{6.1203}, \indexref{6.1221}, \indexref{6.1231}, \indexref{6.124}, \indexref{6.126}, \indexref{6.1262}, \indexref{6.127}, \indexref{6.22}, \indexref{6.3751}}

\indexentry{tell [\textit{aussagen}: ascribe; speak; state; statement], \indexref{6.342}}

\indexentry{term [\textit{Glied}], \indexref{4.1273}, \indexref{4.442}, \indexref{5.232}, \indexref{5.252}, \indexref{5.2522}, \indexref{5.501}}

\indexentry{theory}

   \indexsubentry{1. [\textit{Lehre}: doctrine], \indexref{6.1224};}

   \indexsubsubentry{$\sim$ of probability, \indexref{4.464}}

   \indexsubentry{2. [\textit{Theorie}], \indexref{4.1122}, \indexref{5.5422}, \indexref{6.111};}

   \indexsubsubentry{$\sim$ of classes, \indexref{6.031};}

   \indexsubsubentry{$\sim$ of knowledge, \indexref{4.1121}, \indexref{5.541};}

   \indexsubsubentry{$\sim$ of types, \indexref{3.331}, \indexref{3.332}}

\indexentry{thing, cf.\ object; state of affairs; state of $\sim$s.}

   \indexsubentry{1. [\textit{Ding}], \indexref{1.1}, \indexref{2.01}--\indexref{2.0122}, \indexref{2.013}, \indexref{2.02331}, \indexref{2.151}, \indexref{3.1431}, \indexref{4.0311}, \indexref{4.063}, \indexref{4.1272}, \indexref{4.243}, \indexref{5.5301}, \indexref{5.5303}, \indexref{5.5351}, \indexref{5.5352}, \indexref{5.553}, \indexref{5.634}, \indexref{6.1231}}

   \indexsubentry{2. [\textit{Sache}], \indexref{2.01}, \indexref{2.15}, \indexref{2.1514}, \indexref{4.1272}}

\indexentry{think [\textit{denken}: imagine], \hyperlink{pref3}{P3}, \indexref{3.02}, \indexref{3.03}, \indexref{3.11}, \indexref{3.5}, \indexref{4.114}, \indexref{4.116}, \indexref{5.4731}, \indexref{5.541}, \indexref{5.542}, \indexref{5.61}, \indexref{5.631}}

   \indexsubentry{$\sim$able [\textit{denkbar}] \hyperlink{pref3}{P3}, \indexref{3} \indexref{3.001}, \indexref{3.02}, \indexref{6.361}; cf.\ unthinkable.}

\indexentry{thought [\textit{Gedanke}: idea], \hyperlink{pref3}{P3}, \indexref{3}, \indexref{3.01}, \indexref{3.02}, \indexref{3.04}--\indexref{3.1}, \indexref{3.12}, \indexref{3.2}, \indexref{3.5}, \indexref{4}, \indexref{4.002}, \indexref{4.112}, \indexref{6.21}}

   \indexsubentry{$\sim$-process [\textit{Denkproze{\ss}}], \indexref{4.1121}}

\indexentry{time, \indexref{2.0121}, \indexref{2.0251}, \indexref{6.3611}, \indexref{6.3751}, \indexref{6.4311}, \indexref{6.4312}}

\indexentry{totality [\textit{Gesamtheit}: sum-total; whole], \indexref{1.1}, \indexref{1.12}, \indexref{2.04}, \indexref{2.05}, \indexref{3.01}, \indexref{4.001}, \indexref{4.11}, \indexref{4.52}, \indexref{5.5262}, \indexref{5.5561}}

\indexentry{transcendental, \indexref{6.13}, \indexref{6.421}}

\indexentry{translation, \indexref{3.343}, \indexref{4.0141}, \indexref{4.025}, \indexref{4.243}}

\indexentry{tru/e}

   \indexsubentry{1. [\textit{Faktum}], \indexref{5.154}}

   \indexsubentry{2. [\textit{wahr}], \indexref{2.0211}, \indexref{2.0212}, \indexref{2.21}, \indexref{2.22}, \indexref{2.222}--\indexref{2.225}, \indexref{3.01}, \indexref{3.04}, \indexref{3.05}, \indexref{4.022}--\indexref{4.024}, \indexref{4.06}--\indexref{4.063}, \indexref{4.11}, \indexref{4.25}, \indexref{4.26}, \indexref{4.28}, \indexref{4.31}, \indexref{4.41}, \indexref{4.43}, \indexref{4.431}, \indexref{4.442}, \indexref{4.46}, \indexref{4.461}, \indexref{4.464}, \indexref{4.466}, \indexref{5.11}, \indexref{5.12}, \indexref{5.123}, \indexref{5.13}, \indexref{5.131}, \indexref{5.1363}, \indexref{5.512}, \indexref{5.5262}, \indexref{5.5352}, \indexref{5.5563}, \indexref{5.62}, \indexref{6.111}, \indexref{6.113}, \indexref{6.1203}, \indexref{6.1223}, \indexref{6.1232}, \indexref{6.125}, \indexref{6.343}; cf.\ correct; right.}

   \indexsubentry{come $\sim$e [\textit{stimmmen}: agreement; right], \indexref{5.123}}

   \indexsubentry{$\sim$th-argument, \indexref{5.01}, \indexref{5.101}, \indexref{5.152}, \indexref{6.1203}}

   \indexsubentry{$\sim$th-combination, \indexref{6.1203}}

   \indexsubentry{$\sim$th-condition, \indexref{4.431}, \indexref{4.442}, \indexref{4.45}--\indexref{4.461}, \indexref{4.463}}

   \indexsubentry{$\sim$th-function, \indexref{3.3441}, \indexref{5}, \indexref{5.1}, \indexref{5.101}, \indexref{5.234}, \indexref{5.2341}, \indexref{5.3}, \indexref{5.31}, \indexref{5.41}, \indexref{5.44}, \indexref{5.5}, \indexref{5.521}, \indexref{6}}

   \indexsubentry{$\sim$th-ground, \indexref{5.101}--\indexref{5.121}, \indexref{5.15}}

   \indexsubentry{$\sim$th-operation, \indexref{5.234}, \indexref{5.3}, \indexref{5.32}, \indexref{5.41}, \indexref{5.442}, \indexref{5.54}}

   \indexsubentry{$\sim$th-possibility, \indexref{4.3}--\indexref{4.44}, \indexref{4.442}, \indexref{4.45}, \indexref{4.46}, \indexref{5.101}}

   \indexsubentry{$\sim$th-value, \indexref{4.063}}

\indexentry{type, \indexref{3.331}, \indexref{3.332}, \indexref{5.252}, \indexref{6.123}; cf.\ prototype.}

\indexgap

\indexentry{unalterable [\textit{fest}], \indexref{2.023}, \indexref{2.026}--\indexref{2.0271}}

\indexentry{understand [\textit{verstehen}: obvious; say], \indexref{3.263}, \indexref{4.002}, \indexref{4.003}, \indexref{4.02}, \indexref{4.021}, \indexref{4.024}, \indexref{4.026}, \indexref{4.243}, \indexref{4.411}, \indexref{5.02}, \indexref{5.451}, \indexref{5.521}, \indexref{5.552}, \indexref{5.5562}, \indexref{5.62}; cf.\ misunderstanding.}

   \indexsubentry{make oneself understood [\textit{sich verst{\"a}ndigen}], \indexref{4.026}, \indexref{4.062}}

\indexentry{undetermined [\textit{nicht bestimmt}], \indexref{3.24}, \indexref{4.431}}

\indexentry{unit, \indexref{5.155}, \indexref{5.47321}}

\indexentry{unnecessary, \indexref{5.47321}}

\indexentry{unthinkable, \indexref{4.123}}

\indexentry{use}

   \indexsubentry{1. [\textit{Gebrauch}], \indexref{3.326}, \indexref{4.123}, \indexref{4.1272}, \indexref{4.241}, \indexref{6.211};}

   \indexsubsubentry{$\sim$less [\textit{nicht gebraucht}], \indexref{3.328}}

   \indexsubentry{2. [\textit{Verwendung}: employment], \indexref{3.325}, \indexref{4.013}, \indexref{6.1202}}

\indexgap

\indexentry{validity, \indexref{6.1233}; cf.\ general $\sim$.}

\indexentry{value [\textit{Wert}], \indexref{6.4}, \indexref{6.41} ; cf.\ truth-$\sim$.}

   \indexsubentry{$\sim$ of a variable, \indexref{3.313}, \indexref{3.315}--\indexref{3.317}, \indexref{4.127}, \indexref{4.1271}, \indexref{5.501}, \indexref{5.51}, \indexref{5.52}}

\indexentry{variable, \indexref{3.312}--\indexref{3.317}, \indexref{4.0411}, \indexref{4.1271}, \indexref{4.1272}, \indexref{4.1273}, \indexref{4.53}, \indexref{5.24}, \indexref{5.242}, \indexref{5.2522}, \indexref{5.501}, \indexref{6.022}}

   \indexsubentry{propositional $\sim$ [\textit{Satzvariable}], \indexref{3.313}, \indexref{3.317}, \indexref{4.126}, \indexref{4.127}, \indexref{5.502}}

   \indexsubentry{$\sim$ name, \indexref{3.314}, \indexref{4.1272}}

   \indexsubentry{$\sim$ number, \indexref{6.022}}

   \indexsubentry{$\sim$ proposition [\textit{variabler Satz}], \indexref{3.315}}

\indexentry{visual field, \indexref{2.0131}, \indexref{5.633}, \indexref{5.6331}, \indexref{6.3751}, \indexref{6.4311}}

\indexgap

\indexentry{Whitehead, \indexref{5.252}, \indexref{5.452}}

\indexentry{whole [\textit{gesamt}: sum-total; totality], \indexref{4.11}, \indexref{4.12}}

\indexentry{will [\textit{Wille}; \textit{wollen}] \indexref{5.1362}, \indexref{5.631}, \indexref{6.373}, \indexref{6.374}, \indexref{6.423}, \indexref{6.43}}

\indexentry{wish [\textit{w{\"u}nschen}], \indexref{6.374}}

\indexentry{word [\textit{Wort}], \indexref{2.0122}, \indexref{3.14}, \indexref{3.143}, \indexref{3.323}, \indexref{4.002}, \indexref{4.026}, \indexref{4.243}, \indexref{6.211}; cf.\ concept-$\sim$.}

   \indexsubentry{put into $\sim$s [\textit{aussprechen}; \textit{unausprechlich}: say], \indexref{3.221}, \indexref{4.116}, \indexref{6.421}, \indexref{6.5}, \indexref{6.522}}

\indexentry{world, \indexref{1}--\indexref{1.11}, \indexref{1.13}, \indexref{1.2}, \indexref{2.021}--\indexref{2.022}, \indexref{2.0231}, \indexref{2.026}, \indexref{2.063}, \indexref{3.01}, \indexref{3.12}, \indexref{3.3421}, \indexref{4.014}, \indexref{4.023}, \indexref{4.12}, \indexref{4.2211}, \indexref{4.26}, \indexref{4.462}, \indexref{5.123}, \indexref{5.4711}, \indexref{5.511}, \indexref{5.526}--\indexref{5.5262}, \indexref{5.551}, \indexref{5.5521}, \indexref{5.6}--\indexref{5.633}, \indexref{5.641}, \indexref{6.12}, \indexref{6.1233}, \indexref{6.124}, \indexref{6.22}, \indexref{6.342}, \indexref{6.3431}, \indexref{6.371}, \indexref{6.373}, \indexref{6.374}, \indexref{6.41}, \indexref{6.43}, \indexref{6.431}, \indexref{6.432}, \indexref{6.44}, \indexref{6.45}, \indexref{6.54}; cf.\ description of the $\sim$.}

\indexentry{wrong [\textit{nicht stimmen}: agreement; true], \indexref{3.24}; cf.\ false.}

\indexgap

\indexentry{zero-method, \indexref{6.121}}
% ENDINDEXCONVERT
\end{itemize}
\end{multicols}
%%%%%%%%%%%%%%%%%%%%%%%%%%%%%%%%%%%%%%%%%%%%%%
%% End of Pears/McGuinness Index %%%%%%%%%%%%%
%%%%%%%%%%%%%%%%%%%%%%%%%%%%%%%%%%%%%%%%%%%%%%

\phantomsection\addcontentsline{toc}{chapter}{Edition notes}\bigskip

\noindent\hrulefill

\phantom{xx}

%\noindent {\Huge \ccPublicDomain}\ Ludwig Wittgenstein's \textit{Tractatus Logico-Philosophicus} is in the \textbf{Public Domain}.
\noindent {\Huge \copyright}\ Ludwig Wittgenstein's \textit{Tractatus Logico-Philosophicus} is in the \textbf{Public Domain}.

\noindent See \url{http://creativecommons.org/licenses/publicdomain/}


\bigskip

%\noindent {\Huge \ccLogo\ccAttribution\ccShareAlike}\ This typesetting (including \LaTeX\ code), by Kevin C.\ Klement, is licensed under a \textbf{Creative Commons Attribution---Share Alike 3.0 United States License}.
\noindent This typesetting (including \LaTeX\ code), by Kevin C.\ Klement, is licensed under a \textbf{Creative Commons Attribution---Share Alike 3.0 United States License}.

\noindent See \url{http://creativecommons.org/licenses/by-sa/3.0/us/}

\end{document}
