%&pdflatex
\documentclass[11pt]{article}
\usepackage{amsmath,amsthm,amscd}
\usepackage{amssymb}
\usepackage[all]{xy}

\textwidth = 6.5 in
\textheight = 9 in
\oddsidemargin = 0.0 in
\evensidemargin = 0.0 in
\topmargin = 0.0 in
\headheight = 0.0 in
\headsep = 0.0 in
\parskip = 0.2cm
\parindent = 0.0in

\listfiles
\newtheorem{thm}{Theorem}

\title{\huge 
	Representing Homology Classes by Locally Flat
	Surfaces of Minimum Genus%
\thanks{This is an excerpt from a paper published under the same title
in the American Journal of Mathematics \textbf{119} (1997), 1119--1137.
Typeset by the authors using \LaTeX\ 
with packages from \AmS\ and \Xy-pic.}
}
\author{Ronnie Lee and Dariusz M.~Wilczy\'{n}ski\\
\small\scshape Yale University\\
\small\scshape Utah State University}
\date{}
\begin{document}

\maketitle
\thispagestyle{empty}

\section{Introduction}

A necessary and sufficient condition will be given for a nontrivial homology class
of a simply connected \text{4-manifold} to be represented by a simple, topologically
locally flat embedding of a compact Riemann surface.

\section{Splittings of Hermitian Modules}

We begin with an algebraic result.

\begin{thm}
The following is a commutative diagram of pointed hermitian modules.\\
 %\CompileMatrices
 \[\xymatrix{
 (M,h,z) \ar[dd]^{\pi_0} \ar[dr]^\alpha_\cong \ar[rr]^{\pi_1}
 && (M_1,h_1,0) \ar'[d]^-{\pi_{1d}}[dd] \ar[dr]^{\alpha_1}_\cong
 \\
 & (M',h',z')\oplus H(\Lambda^k) \ar[dd]^<(.25){\pi_0} \ar[rr]^<(.25){\pi_1}
 && (M'_1,h'_1,0)\oplus H(\Lambda_1^k) \ar[dd]^{\pi_{1d}}
 \\
 (M_0,h_0,z_0) \ar@{=}[dd] \ar[dr]^{\alpha_0}_\cong \ar'[r]^<(.6){\pi_{0d}}[rr]
 && (M_d,h_d,0) \ar@{=}'[d][dd] \ar[dr]^{\alpha_d}_\cong
 \\
 & (M'_0,h'_0,z'_0)\oplus H(\Lambda_0^k) \ar[dd]^<(.25){\beta'_0\oplus\text{id}}_<(.25)\cong
\ar[rr]^<(.25){\pi_{0d}}
 && (M'_d,h'_d,0)\oplus H(\Lambda_d^k) \ar[dd]^{\beta'_d\oplus\text{id}}_\cong
 \\
 (M_0,h_0,z_0) \ar[dr]^{\beta_0}_\cong \ar'[r]^<(.6){\pi_{0d}}[rr]
 && (M_d,h_d,0) \ar[dr]^{\beta_d}_\cong
 \\
 & (L,\lambda,x)\oplus H(\Lambda_0^k) \ar[rr]^{\pi_{0d}}
 && (L_d,\lambda_d,0)\oplus H(\Lambda_d^k)
 }\]
\end{thm}

\vphantom{X}
\end{document}