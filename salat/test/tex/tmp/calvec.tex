%An interactive document that uses JavaScript to 
%perform simple vector operations. 
%
%Faro, Portugal, 05/04/2003 at 22:00h 
%Written by Orlando Camargo Rodr�guez
%
%In order to compile this document the insdljs.sty and 
%dljscc.def files are required. Those files should be 
%available at http://www.math.uakron.edu/~dpstory/webeq.html
%
\newcommand{\va}{\mbox{$\mathbf{a}$}} 
\newcommand{\vb}{\mbox{$\mathbf{b}$}} 
\newcommand{\ve}{\mbox{$\mathbf{e}$}} 

\documentclass[12pt]{article} 

\pagestyle{empty}

\usepackage{graphicx}
\usepackage[pdfpagemode=FullView]{hyperref}
\usepackage[pdftex]{insdljs} 

\begin{insDLJS}[vecsum]{vecsum}{My Private vector sum}
function vecsum()
{
    var output1 = this.getField("Output.First");
    var output2 = this.getField("Output.Second"); 
    var output3 = this.getField("Output.Third");
    output1.value = v1x.value + v2x.value ; 
    output2.value = v1y.value + v2y.value ;
    output3.value = v1z.value + v2z.value ;
} 

\end{insDLJS}
\begin{insDLJS}[innerpro]{innerpro}{My Private vector inner product}
function innerpro()
{   
    var absv1 = 0 ; 
    var absv2 = 0 ; 
    var costheta = 0 ; 
    
    var output4 = this.getField("Output.Fourth");
    var output5 = this.getField("Output.Fifth"); 
    
    output4.value = v1x.value*v2x.value + v1y.value*v2y.value + v1z.value*v2z.value ; 
    
    absv1 = Math.sqrt( v1x.value*v1x.value + v1y.value*v1y.value + v1z.value*v1z.value );
    absv2 = Math.sqrt( v2x.value*v2x.value + v2y.value*v2y.value + v2z.value*v2z.value );
    costheta = output4.value/( absv1*absv2 ) ; 
    
    output5.value = Math.acos( costheta )*180/Math.PI ; 
} 
\end{insDLJS}

\begin{insDLJS}[outerpro]{outerpro}{My Private vector outer product}
function outerpro()
{
  var output6 = this.getField("Output.Sixth");
  var output7 = this.getField("Output.Seventh");
  var output8 = this.getField("Output.Eighth");
  output6.value = v1y.value*v2z.value - v1z.value*v2y.value; 
  output7.value = v1z.value*v2x.value - v1x.value*v2z.value;
  output8.value = v1x.value*v2y.value - v1y.value*v2x.value;  
} 
\end{insDLJS}

%Let's go: 
\begin{document}

\begin{Form}
\hrule
\vskip3mm
\hrule
\vskip5mm 
\centerline{\em Insira as componentes dos vectores:}
\vskip15mm
\centerline{
\begin{tabular}{ccccc}
\hskip2mm$\va$\hskip2mm & = & 
\TextField[width=30mm,name=Inputv1.First,borderstyle=I,value={},align=1,
validate={v1x=this.getField("Inputv1.First");}]{ }{\ $\ve_x+$} & 
\TextField[width=30mm,name=Inputv1.Second,borderstyle=I,value={},align=1,
validate={v1y=this.getField("Inputv1.Second");}]{ }{\ $\ve_y+$} & 
\TextField[width=30mm,name=Inputv1.Third,borderstyle=I,value={},align=1,
validate={v1z=this.getField("Inputv1.Third");}]{ }{\ $\ve_z$} \\ 
\ \\ 
\hskip2mm$\vb$\hskip2mm & = & 
\TextField[width=30mm,name=Inputv2.First,borderstyle=I,value={},align=1,
validate={v2x=this.getField("Inputv2.First");}]{ }{\ $\ve_x+$} & 
\TextField[width=30mm,name=Inputv2.Second,borderstyle=I,value={},align=1,
validate={v2y=this.getField("Inputv2.Second");}]{ }{\ $\ve_y+$} & 
\TextField[width=30mm,name=Inputv2.Third,borderstyle=I,value={},align=1,
validate={v2z=this.getField("Inputv2.Third");}]{ }{\ $\ve_z$} \\ 
\ \\ 
\ \\ 
\ \\ 
\multicolumn{5}{c}{\em Clique em cada bot\~ao para executar o c\'alculo respectivo:} \\ 
\ \\ 
\PushButton[name=pb1,borderstyle=B,borderwidth=1.5,onclick={vecsum();}]
{\hskip2mm$\va+\vb$\hskip2mm}
& = & 
\TextField[width=30mm,name=Output.First,borderstyle=I,value={},align=1,
validate={output1=this.getField("Output.First");}]{ }{\ $\ve_x+$} & 
\TextField[width=30mm,name=Output.Second,borderstyle=I,value={},align=1,
validate={output2=this.getField("Output.Second");}]{ }{\ $\ve_y+$} & 
\TextField[width=30mm,name=Output.Third,borderstyle=I,value={},align=1,
validate={output3=this.getField("Output.Third");}]{ }{\ $\ve_z$}
\end{tabular}
           }
\vskip15mm
\centerline{
\PushButton[name=pb2,borderstyle=B,borderwidth=1.5,onclick={innerpro();}]
{\hskip2mm$\va\cdot\vb$\hskip2mm} = 
\TextField[width=40mm,name=Output.Fourth,borderstyle=I,value={0},
align=1,validate={output4=this.getField("Output.Fourth");},
calculate={output4.value}]{ }
           }
\vskip15mm
\centerline{
\PushButton[name=pb3,borderstyle=B,borderwidth=1.5,onclick={}]
{\hskip2mm$\theta$\hskip2mm} = 
\TextField[width=40mm,name=Output.Fifth,borderstyle=I,value={0},
align=1,validate={output5=this.getField("Output.Fifth");}]{ }\hskip2mm$^\circ$
           }	   
\vskip15mm
\centerline{
\PushButton[name=pb4,borderstyle=B,borderwidth=1.5,onclick={outerpro();}]
{\hskip2mm$\va\times\vb$\hskip2mm} = 
\TextField[width=30mm,name=Output.Sixth,borderstyle=I,value={0},
align=1,validate={output6=this.getField("Output.Fifth");},
calculate={output1.value}]{ }{\ $\ve_x+$}
\TextField[width=30mm,name=Output.Seventh,borderstyle=I,value={0},
align=1,validate={output7=this.getField("Output.Sixth");},
calculate={output1.value}]{ }{\ $\ve_y+$}
\TextField[width=30mm,name=Output.Eighth,borderstyle=I,value={0},
align=1,validate={output8=this.getField("Output.Eighth");},
calculate={output1.value}]{ }{\ $\ve_z$}
           }	   
\vskip5mm
\hrule
\vskip3mm
\hrule 	   	   
\end{Form}

\end{document}
