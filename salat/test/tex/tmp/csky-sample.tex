\chapter{Explosive and Whirling Nebul{\ae}}

\MyDroppedCaps O{\sc ne} of the most surprising triumphs of celestial photography was 
\index{Spiral nebul{\ae}}\index{Keeler, Prof., discovery}Professor Keeler's discovery, in {\os 1899}, that the great majority of the
nebul{\ae} have a distinctly spiral form. This form, previously known
in Lord Rosse's great \index{Rosse, Lord, Whirlpool Nebula}\index{Whirlpool Nebula, Rosse's}\quote{Whirlpool Nebula,} had been supposed to\MN{nebulae-m51}
be exceptional; now the photographs, far excelling telescopic views in
the revelation of nebular forms, showed the spiral to be the typical
shape. Indeed, it is a question whether all nebul{\ae} are not to some
extent spiral. The extreme importance of this discovery is shown in\MN{nebulae-percent-spiral}
the effect that it has had upon hitherto prevailing views of solar and
planetary evolution. For more than three-quarters of a century
\index{Laplace, Prof., theory}Laplace's celebrated hypothesis of the manner of origin of the solar
system from a rotating and contracting nebula surrounding the sun had
guided speculation on that subject, and had been tentatively extended
to cover the evolution of systems in general. The apparent forms of
some of the nebul{\ae} which the telescope had revealed were regarded,
and by some are still regarded, as giving visual evidence in favor of
this theory. There is a \quote{ring nebula} in \index{Lyra}Lyra with a central\MN{nebulae-planetary}
star, and a \quote{planetary nebula} in \index{Gemini}Gemini bearing no little
resemblance to the planet Saturn with its rings, both of which appear
to be practical realizations of \index{Laplace, Prof., theory}Laplace's idea, and the elliptical
rings surrounding the central condensation of the \index{Andromeda Nebula}\index{Nebula+Andromeda}Andromeda Nebula may
be cited for the same kind of proof.

\placegraphic[here]{\Messier{51} -- Lord Rosse's Nebula}{\externalfigure[m0051]}

But since Keeler's discovery there has been a decided turning away of
speculation another way. The form of the spiral nebul{\ae} seems to
be entirely inconsistent with the theory of an originally globular or
disk-shaped nebula condensing around a sun and throwing or leaving off
rings, to be subsequently shaped into planets. Some astronomers,
indeed, now reject \index{Laplace, Prof., theory}Laplace's hypothesis {\em in toto,} preferring to
think that even our solar system originated from a spiral
nebula. Since the spiral type prevails among the existing nebul{\ae},\MN{nebulae-systems-from-spirals}
we must make any mechanical theory of the development of stars and
planetary systems from them accord with the requirements which that
form imposes. A glance at the extraordinary variations upon the spiral
which Professor Keeler's photographs reveal is sufficient to convince
one of the difficulty of the task of basing a general theory upon
them. In truth, it is much easier to criticize \index{Laplace, Professor, theory}Laplace's hypothesis 
than to invent a satisfactory substitute for it. If the spiral
nebul{\ae} seem to oppose it there are other nebul{\ae} which appear to
support it, and it may be that no one fixed theory can account for all
the forms of stellar evolution in the universe. Our particular
planetary system may have originated very much as the great French
mathematician supposed, while others have undergone, or are now
undergoing, a different process of development. There is always a too
strong tendency to regard an important new discovery and the theories
and speculations based upon it as revolutionizing knowledge, and
displacing or overthrowing everything that went before. Upon the plea
that \quote{Laplace only made a guess} more recent guesses have been
driven to extremes and treated by injudicious exponents as \quote{the
solid facts at last.}

Before considering more recent theories than La\-place's, let us see
what the nature of the photographic revelations is. The vast celestial
maelstrom discovered by Lord Rosse in the \quote{Hunting Dogs} may be
taken as the leading type of the spiral nebul{\ae}, although there are
less conspicuous objects of the kind which, perhaps, better illustrate
some of their peculiarities. \index{Rosse, Lord, Whirlpool Nebula}\index{Whirlpool Nebula, Rosse's}Lord Rosse's nebula appears far more
wonderful in the photographs than in his drawings made with the aid of
his giant reflecting telescope at Parsonstown, for the photographic
plate records details that no telescope is capable of showing. Suppose
we look at the photograph of this object as any person of common sense
would look at any great and strange natural phenomenon. What is the
first thing that strikes the mind? It is certainly the appearance of
violent whirling motion. One would say that the whole glowing mass had
been spun about with tremendous velocity, or that it had been set
rotating so rapidly that it had become the victim of
\quote{centrifugal force,} one huge fragment having broken loose and
started to gyrate off into space. Closer inspection shows that in
addition to the principal focus there are various smaller
condensations scattered through the mass. These are conspicuous in the
spirals. Some of them are stellar points, and but for the significance
of their location we might suppose them to be stars which happen to
lie in a line between us and the nebula. But when we observe how many
of them follow most faithfully the curves of the spirals we cannot but\MN{nebulae-stellar-points}
conclude that they form an essential part of the phenomenon; it is not
possible to believe that their presence in such situations is merely
fortuitous. One of the outer spirals has at least a dozen of these
star-like points strung upon it; some of them sharp, small, and
distinct, others more blurred and nebulous, suggesting different
stages of condensation. Even the part which seems to have been flung
loose from the main mass has, in addition to its central condensation,
at least one stellar point gleaming in the half-vanished spire
attached to it. Some of the more distant stars scattered around the
\quote{whirlpool} look as if they too had been shot out of the mighty
vortex, afterward condensing into unmistakable solar bodies. There are
at least two curved rows of minute stars a little beyond the periphery
of the luminous whirl which clearly follow lines concentric with those
of the nebulous spirals. Such facts are simply dumbfounding for anyone
who will bestow sufficient thought upon them, for these are {\em
suns,} though they may be small ones; and what a birth is that for a
sun!

Look now again at the glowing spirals. We observe that hardly have
they left the central mass before they begin to coagulate. In some
places they have a \quote{ropy} aspect; or they are like peascods
filled with growing seeds, which eventually will become stars. The
great focus itself shows a similar tendency, especially around its
circumference. The sense that it imparts of a tremendous shattering
force at work is overwhelming. There is probably more matter in that
whirling and bursting nebula than would suffice to make a hundred
solar systems! It must be confessed at once that there is no
confirmation of the \index{Laplace, Prof., theory}Laplacean hypothesis here; but what hypothesis
will fit the facts? There is one which it has been claimed does so,
but we shall come to that later. In the meanwhile, as a preparation,
fix in the memory the appearance of that second spiral mass spinning
beside its master which seems to have spurned it away.

For a second example of the spiral nebul{\ae} look at the one in the 
constellation \index{Triangulum}Triangulum. {\em God, how hath the imagination of puny\MN{nebulae-m33}
man failed to comprehend Thee!} Here is creation through destruction
with a vengeance! The spiral form of the nebula is unmistakable, but\MN{nebulae-tornadic}
it is half obliterated amid the turmoil of flying masses hurled away
on all sides with tornadic fury. The focus itself is splitting asunder
under the intolerable strain, and in a little while, as time is
reckoned in the Cosmos, it will be gyrating into stars. And then look
at the cyclonic rain of already finished stars whirling round the
outskirts of the storm. Observe how scores of them are yet involved in
the fading streams of the nebulous spirals; see how they have been
thrown into vast loops and curves, of a beauty that half redeems the
terror of the spectacle enclosed within their lines~-- like iridescent
cirri hovering about the edges of a hurricane. And so again are suns
born!

\placegraphic[here]{\Messier{33} -- Wonderful spiral in Triangulum}{\externalfigure[m0033]}

Let us turn to the exquisite spiral in \index{Ursa Major}Ursa Major; how different its\MN{nebulae-m81} 
aspect from that of the other! One would say that if the terrific coil
in \index{Triangulum}Triangulum has all but destroyed itself in its fury, this one on
the contrary has just begun its self-demolition. As one gazes one
seems to see in it the smooth, swift, accelerating motion that
precedes catastrophe. The central part is still intact, dense, and
uniform in texture. How graceful are the spirals that smoothly rise
from its oval rim and, gemmed with little stars, wind off into the
darkness until they have become as delicate as threads of gossamer!
But at bottom the story told here is the same~-- creation by gyration!

\placegraphic[here]{\Messier{81} -- Spiral in Ursa Major}{\externalfigure[m0081]}

Compare with the above the curious mass in \index{Cetus}\index{Nebula+in Cetus}Cetus. Here the plane of\MN{nebulae-ngc253} 
the whirling nebula nearly coincides with our line of sight and we see
the object at a low angle. It is far advanced and torn to shreds, and
if we could look at it perpendicularly to its plane it is evident that
it would closely resemble the spectacle in \index{Triangulum}Triangulum. 

\placegraphic[here]{{\sc ngc}{\os 253} in Sculptor}{\externalfigure[cetus]}

Then take the famous \index{Andromeda Nebula}\index{Nebula+Andromeda}Andromeda Nebula (see Frontispiece), which is so
vast that notwithstanding its immense distance even the naked eye
perceives it as an enigmatical wisp in the sky. Its image on the
sensitive plate is the masterpiece of astronomical photography; for
wild, incomprehensible beauty there is nothing that can be compared
with it. Here, if anywhere, we look upon the spectacle of creation in
one of its earliest stages. The \index{Andromeda Nebula}\index{Nebula+Andromeda}Andromeda Nebula is apparently less
advanced toward transformation into stellar bodies than is that in
\index{Triangulum}Triangulum. The immense crowd of stars sprinkled over it and its
neighborhood seem in the main to lie this side of the nebula, and
consequently to have no connection with it. But incipient stars (in
some places clusters of them) are seen in the nebulous rings, while
one or two huge masses seem to give promise of transformation into
stellar bodies of unusual magnitude. I say \quote{rings} because
although the loops encompassing the \index{Andromeda Nebula}\index{Nebula+Andromeda}Andromeda Nebula have been called
spirals by those who wish utterly to demolish \index{Laplace, Prof., theory}Laplace's hypothesis,\MN{nebulae-rings} 
yet they are not manifestly such, as can be seen on comparing them
with the undoubted spirals of the \index{Rosse, Lord, Whirlpool Nebula}\index{Whirlpool Nebula, Rosse's}Lord Rosse Nebula. They look quite
as much like circles or ellipses seen at an angle of, say, fifteen or
twenty degrees to their plane. If they are truly elliptical they
accord fairly well with \index{Laplace, Prof., theory}Laplace's idea, except that the scale of
magnitude is stupendous, and if the \index{Andromeda Nebula}\index{Nebula+Andromeda}Andromeda Nebula is to become a
solar system it will surpass ours in grandeur beyond all possibility
of comparison.

There is one circumstance connected with the spiral nebul{\ae}, and
conspicuous in the \index{Andromeda Nebula}\index{Nebula+Andromeda}Andromeda Nebula on account of its brightness,
which makes the question of their origin still more puzzling; they all
show continuous spectra, which, as we have before remarked, indicate
that the mass from which the light comes is either solid or liquid, or
a gas under heavy pressure. Thus nebul{\ae} fall into two classes: the
\quote{white} nebul{\ae}, giving a continuous spectrum; and the
\quote{green} nebul{\ae} whose spectra are distinctly gaseous. The
\index{Andromeda Nebula}\index{Nebula+Andromeda}Andromeda Nebula is the great representative of the former class and
the \index{Nebula+Orion}Orion Nebula of the latter. The spectrum of the \index{Andromeda Nebula}\index{Nebula+Andromeda}Andromeda Nebula 
has been interpreted to mean that it consists not of luminous gas, but
of a flock of stars so distant that they are separately\MN{nebulae-truth}
indistinguishable even with powerful telescopes, just as the component
stars of the \index{Milky Way, the}Milky Way are indistinguishable with the naked eye; and
upon this has been based the suggestion that what we see in Andromeda\MN{nebulae-outer}
is an outer universe whose stars form a series of elliptical garlands
surrounding a central mass of amazing richness. But this idea is 
unacceptable if for no other reason than that, as just said, all the
spiral nebul{\ae} possess the same kind of spectrum, and probably no
one would be disposed to regard them all as outer universes. As we
shall see later, the peculiarity of the spectra of the spiral
nebul{\ae} is appealed to in support of a modern substitute for
\index{Laplace, Prof., theory}Laplace's hypothesis. 

Finally, without having by any means exhausted the variety exhibited
by the spiral nebul{\ae}, let us turn to the great representative of
the other species, the \index{Nebula+Orion}Orion Nebula. In some ways this is even more
marvelous than the others. The early drawings with the telescope failed
to convey an adequate conception either of its sublimity or of its
complication of structure. It exists in a nebulous region of space,
since photographs show that nearly the whole constellation is
interwoven with faintly luminous coils. To behold the entry of the
great nebula into the field even of a small telescope is a startling
experience which never loses its novelty. As shown by the photographs,
it is an inscrutable chaos of perfectly amazing extent, where spiral
bands, radiating streaks, dense masses, and dark yawning gaps are
strangely intermingled without apparent order. In one place four
conspicuous little stars, better seen in a telescope than in the\MN{nebulae-trapezium}
photograph on account of the blurring produced by over-exposure, are
suggestively situated in the midst of a dark opening, and no observer
has ever felt any doubt that these stars have been formed from the
substance of the surrounding nebula. There are many other stars
scattered over its expanse which manifestly owe their origin to the
same source. But compare the general appearance of this nebula with
the others that we have studied, and remark the difference. If the
unmistakably spiral nebul{\ae} resemble bursting fly-wheels or
grindstones from whose perimeters torrents of sparks are flying, the
\index{Nebula+Orion}Orion Nebula rather recalls the aspect of a cloud of smoke and
fragments produced by the explosion of a shell. This idea is enforced
by the look of the outer portion farthest from the bright half of the
nebula, where sharply edged clouds with dark spaces behind seem to be
billowing away as if driven by a wind blowing from the center.

\placegraphic[here]{The Orion Nebula}{\externalfigure[orion]}

Next let us consider what scientific speculation has done in the
effort to explain these mysteries. \index{Laplace, Prof., theory}Laplace's hypothesis can certainly
find no standing ground either in the \index{Nebula+Orion}Orion Nebula or in those of a
spiral configuration, whatever may be its situation with respect to
the grand \index{Andromeda Nebula}\index{Nebula+Andromeda}Nebula of Andromeda, or the \quote{ring} and
\quote{planetary} nebul{\ae}. Some other hypothesis more consonant with
the appearances must be found. Among the many that have been proposed
the most elaborate is the \index{Planetesimal Hypothesis}\quote{Planetesimal Hypothesis} of
\index{Chamberlin, Prof., theory}Professors Chamberlin and \index{Moulton, Prof., theory}Moulton. It is to be remarked that it
applies to the spiral nebul{\ae} distinctively, and not to an
apparently chaotic mass of gas like the vast luminous cloud in
\index{Nebula+Orion}Orion. The gist of the theory is that these curious objects are
probably the result of close approaches to each other of two\MN{nebulae-detailed}
independent suns, reminding us of what was said on this subject when
we were dealing with temporary stars. Of the previous history of these
appulsing suns the theory gives us no account; they are simply
supposed to arrive within what may be called an effective
tide-producing distance, and then the drama begins. Some of the
probable consequences of such an approach have been noticed in Chapter
{\os 5}; let us now consider them a little more in detail. 

\index{Tides in couple}Tides always go in couples; if there is a tide on one side of a globe
there will be a corresponding tide on the other side. The cause is to
be found in the law that the force of gravitation varies inversely as
the square of the distance; the attraction on the nearest surface of
the body exercised by another body is greater than on its center, and
greater yet than on its opposite surface. If two great globes attract
each other, each tends to draw the other out into an ellipsoidal
figure; they must be more rigid than steel to resist this~-- and even
then they cannot altogether resist. If they are liquid or gaseous they
will yield readily to the force of distortion, the amount of which
will depend upon their distance apart, for the nearer they are the
greater becomes the tidal strain. If they are encrusted without and
liquid or gaseous in the interior, the internal mass will strive to
assume the figure demanded by the \index{Tidal explosion}tidal force, and will, if it can,
burst the restraining envelope. Now this is virtually the predicament
of the body we call a sun when in the immediate presence of another
body of similarly great mass. Such a body is presumably gaseous
throughout, the component gases being held in a state of rigidity by
the compression produced by the tremendous gravitational force of
their own aggregate mass. At the surface such a body is enveloped in a
shell of relatively cool matter. Now suppose a great attracting body,
such as another sun, to approach near enough for the difference in its
attraction on the two opposite sides of the body and on its center to
become very great; the consequence will be a tidal deformation of the
whole body, and it will lengthen out along the line of the
gravitational pull and draw in at the sides, and if its shell offers
considerable resistance, but not enough to exercise a complete
restraint, it will be violently burst apart, or blown to atoms, and
the internal mass will leap out on the two opposite sides in great
fiery spouts. In the case of a sun further advanced in cooling than
ours the interior might be composed of molten matter while the
exterior crust had become rigid like the shell of an egg; then the
force of the \index{Tidal explosion}\quote{tidal explosion} produced by the appulse of
another sun would be more violent in consequence of the greater
resistance overcome. Such, then, is the mechanism of the first phase
in the history of a spiral nebula according to the \index{Planetesimal Hypothesis}Planetesimal
Hypothesis. Two suns, perhaps extinguished ones, have drawn near
together, and an explosive outburst has occured in one or both. The
second phase calls for a more agile exercise of the imagination.

To simplify the case, let us suppose that only one of the tugging suns
is seriously affected by the strain. Its vast wings produced by the
outburst are twisted into spirals by their rotation and the contending
attractions exercised upon them, as the two suns, like battleships in
desperate conflict, curve round each other, concentrating their
destructive energies. Then immense quantities of d{\'e}bris are scattered
about in which eddies are created, and finally, as the sun that caused
the damage goes on its way, leaving its victim to repair its injuries
as it may, the dispersed matter cools, condenses, and turns into
streams of solid particles circling in elliptical paths about their
parent sun. These particles, or fragments, are the
\quote{planetesimals} of the theory. In consequence of the inevitable
intersection of the orbits of the planetesimals, nodes are formed
where the flying particles meet, and at these nodes large masses are
gradually accumulated. The larger the mass the greater its attraction,
and at last the nodal points become the nuclei of great aggregations
from which planets are shaped. 

This, in very brief form, is the \index{Planetesimal Hypothesis}Planetesimal Hypothesis which we are
asked to substitute for that based on \index{Laplace, Prof., theory}Laplace's suggestion as an
explanation of the mode of origin of the solar system; and the
phenomena of the spiral nebul{\ae} are appealed to as offering evident
support to the new hypothesis. We are reminded that they are
elliptical in outline, which accords with the hypothesis; that their
spectra are not gaseous, which shows that they may be composed of
solid particles like the planetesimals; and that their central masses
present an oval form, which is what would result from the tidal
effects, as just described. We also remember that some of them, like
the \index{Rosse, Lord, Whirlpool Nebula}\index{Whirlpool Nebula, Rosse's}Lord Rosse and the 
\index{Andromeda Nebula}\index{Nebula+Andromeda}Andromeda nebul{\ae}, are visually double, and
in these cases we might suppose that the two masses represent the
tide-burst suns that ventured into too close proximity. It may be
added that the authors of the theory do not insist upon the appulse of
two suns as the {\em only} way in which the planetesimals may have
originated, but it is the only supposition that has been worked out.

But serious questions remain. It needs, for instance, but a glance at
the \index{Triangulum}Triangulum monster to convince the observer that it cannot be a
solar system which is being evolved there, but rather a swarm of
stars. Many of the detached masses are too vast to admit of the
supposition that they are to be transformed into planets, in our sense
of planets, and the distances of the stars which appear to have been
originally ejected from the focal masses are too great to allow us to
liken the assemblage that they form to a solar system. Then, too, no
nodes such as the hypothesis calls for are visible. Moreover, in most
of the spiral nebul{\ae} the appearances favor the view that the
supposititious encountering suns have not separated and gone each
rejoicing on its way, after having inflicted the maximum possible
damage on its opponent, but that, on the contrary, they remain in
close association like two wrestlers who cannot escape from each
other's grasp. And this is exactly what the law of gravitation
demands; stars cannot approach one another with impunity, with regard
either to their physical make-up or their future independence of
movement. The theory undertakes to avoid this difficulty by assuming
that in the case of our system the approach of the foreign body to the
sun was not a close one~-- just close enough to produce the tidal
extrusion of the relatively insignificant quantity of matter needed to
form the planets. But even then the effect of the appulse would be to
change the direction of flight, both of the sun and of its visitor,
and there is no known star in the sky which can be selected as the
sun's probable partner in their ancient {\em pas deux.} That there are
unconquered difficulties in \index{Laplace, Prof., theory}Laplace's hypothesis no one would deny,
but in simplicity of conception it is incomparably more satisfactory,
and with proper modifications could probably be made more consonant
with existing facts in our solar system than that which is offered to
replace it. Even as an explanation of the spiral nebul{\ae}, not as
solar systems in process of formation, but as the birthplaces of
stellar clusters, the \index{Planetesimal Hypothesis}Planetesimal Hypothesis would be open to many
objections. Granting its assumptions, it has undoubtedly a strong
mathematical framework, but the trouble is not with the mathematics
but with the assumptions. Laplace was one of the ablest mathematicians
that ever lived, but he had never seen a spiral nebula; if he had, he
might have invented a hypothesis to suit its phenomena. His actual
hypothesis was intended only for our solar system, and he left it in
the form of a \quote{note} for the consideration of his successors,
with the hope that they might be able to discover the full truth,
which he confessed was hidden from him. It cannot be said that that
truth has yet been found, and when it is found the chances are that
intuition and not logic will have led to it.

The spiral nebul{\ae}, then, remain among the greatest riddles of the
universe, while the gaseous nebul{\ae}, like that of \index{Nebula+Orion}Orion, are no less
mysterious, although it seems impossible to doubt that both forms give\MN{nebulae-summary}
birth to stars. It is but natural to look to them for light on the
question of the origin of our planetary system; but we should not
forget that the scale of the phenomena in the two cases is vastly
different, and the forces in operation may be equally different. A
hill may have been built up by a glacier, while a mountain may be the
product of volcanic forces or of the upheaval of the strata of the
planet. 









