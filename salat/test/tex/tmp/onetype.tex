%One Typeface, MANY FONTS
%
%First, set up margins, text block and page size:
\special{papersize=2.75in,4.25in}
%the above special may be specific to the NeXT implementation of TeX
%
%set up text block to classical sizes/proportions for eighth letter page
\hsize=2in
\vsize=3.125in
%
%set up margins to match
\hoffset=-.5in
%make left margin equal to .47 inches (1 inch less offset)
\voffset=-.3125in
%ditto
\nopagenumbers
\frenchspacing
%
%set up fonts
\font\pad=padr9alt at 8pt
\pad
\font\lr=padr9o at 11pt
\font\lsc=padrc9o at 11pt
\font\sc=padrc9o at 8pt
\font\it=padi9alt at 8pt
\font\alt=padra at 8pt
\font\altit=padria at 8pt
\font\padr=padr9t at 8pt
\font\padrno=padrno at 8pt
%
%input needed files
%\input eplain.tex
\input hangingpunctuation.tex
%
\hyphenchar\pad=128
%Cover
\topglue
\vfill

\hfill {\lr One Typeface,}

\hfill {\lsc many fonts\thinspace}
\vfill

\hfill A Guide to Roman\thinspace

\hfill {\it \& Italic Designs\thinspace}

\vfill

\hfill {\it {\altit W}\kern+\pnt5ptilliam {\altit A}dams}

\vfill

\break
\parindent=0pt
First, a few corrections and additions:

{\sc colophon}---(1) The trade emblem or device
of a printer or publisher. (2) A page sometimes found at the
end of a book, listing details pertaining
to production of the book and/or the
printer's imprint.

{\sc cross stroke}---horizontal stroke.

{\sc etaoin shrdlu}---A typographer's sign to
indicate a mistake. Originally the first
line of a Linotype keyboard (which was
arranged by letter frequency) these keys
would be struck in the event of an error
in setting the line to fill it out so that it
might be cast and discarded.

{\sc gutter}---in binding, the blank space where
two pages meet. Also, the blank space
between columns of type.
\vfill

{\padrno c} 1997 William F. Adams\hfill

\break
{\sc margin}---the unprinted area around the
edges of a page. The margins as
designated in book specifications refer
to the remaining margins after the book
has been trimmed.

{\sc small capitals}---capitals redrawn and sized
to match the proportions of lower-case
letters Usually the same height as
the x-height, or only slightly taller. A full-size
capital shrunk to this size is too thin
and light. Used for abbreviations within
text, sub-titles \& c.

{\sc typography}---the art or craft of setting type
to improve understanding of the text.

\quad ref. Rauri McLean, 

\quad\quad{\it Thames \& Hudson Manual of Typography}, 

\quad Robert Bringhurst,

\quad\quad{\it The Elements of Typographic Style}.

\vfill

{Corrections courtesy Mac McGrew}

\quad \& Dr. Richard McClintock

\break
\parskip=0pt
Originally, a typeface design was a thing
unto itself, with texts being set in
roman, or italic (or Fraktur, Rotunda
or Schwabacher), but never mixing either.
Italics originally used upright capitals however,
which provided a useful contrast at
need. In the 16th century, typographers
began using italics in roman texts for
emphasis, or to pick out foreign words, a
practice which continues to this day,
despite certain efforts to the contrary.

\parindent=1em

Other languages, naturally, have other
conventions, German being notable for
having two separate fonts as well, Fraktur,
literally {\it broken script\/}, and Schwabacher,
{\it rounded script\/}, which were used to good effect in older
texts to differentiate language usage.
A single typeface family (as opposed to
superfamily, such as Lucida or Stone)
may contain the following:

\break

\font\tc=padd8a at 29pt
\font\lfi=padri9t at 8pt
\font\lf=padr9t at 8pt
\font\exp=padr8x at 8pt
{\leftskip = 6pt
\centerline {\tc TITLING}
\centerline {\tc \kern 3ptCAPITALS}
{\obeylines 
ROMAN CAPITALS
\quad \& LINING FIGURES {\lf 0123456789}
{\sc roman small capitals}
roman lowercase letters
\quad {\exp \&} old-style figures 0123456789
alternat{\alt e} roma{\alt n} characters
{\it SLOPED/ITALIC CAPITALS}
\quad {\altit \&} {\it Lining Figures} {\lfi 0123456789}
{\it {\altit A}lternate {\altit S}wash {\altit C}haracters}
{\it italic lowercase letters}
\quad {\it \&} {\it italic old-style figures 0123456789}
\quad \quad {\exp \&} ornaments {\alt 1 2} {\altit 1)
{\alt 1111111111111}
}}}

\break

Some typefaces will also have italic
small capitals, and in certain instances, an
obliqued or slanted roman as well as a true italic. 
This latter convention is most
appropriate to fonts intended for setting
mathematics, but is all-too often 
done in ignorance of the true nature of italic. 
It bears noting that
an italic is not such simply because of its
slant, but because of its structure, which is
derived from handwriting.

Shown above, but not specifically
referenced, were ligatures. Most roman type
designs have {\it f}~s which kern, or hang over
into the boundaries of the following character. 
Normally, this is not a difficulty, but
some collisions do occur, hence, the
ligatures ff , fi , fl , ffi and ffl. Non-kerning
{\it f}~s do exist, with Linotype being noted (or
notorious) for making them, their rationale
being that it facilitates letterspacing
lowercase text.

Other ligatures include the ampersand, \kern -2pt{\altit \&}
%{\it and per se and\/} ``and as itself and'' 
a ligature of the Latin word for and, {\it et\/}), the
German double-s, {\it eszett\/}, or sharp-s, \ss,
which grew out of the long-s which was
used in the middle of words, and the purely
decorative ct and {\it st\/} ligatures,
hold\-overs from Chancery calligraphy.

A typeface will also include a number of
characters which are not intuitively available
from a typewriter keyboard. As any good
style manual will indicate (e. g. {\it The Chicago
Manual of Style\/} prior to the 13th edition) these must be used. Two
hyphens do not make an em-dash---nor is
a single hyphen suitable to stand in for an
en-dash. Most applications will automatically 
place apostrophes and quotation
marks (but be certain to use an apostrophe
to indicate omission {\it 'struth\/}) but few will
correctly use prime marks (i.e. {\padrno ' "})
%$' ''$
to indicate
units of measure.

Similarly, the 
%multiply 
sign, 
% $\times$
%
{\padrno x}
, not an
{\it x\/} should be used, 
%Likewise
when indicating dimensions, or
multiplication. Fractions should be built
using a solidus and superscript and subscript 
numbers from an expert font 
(e.g. {\padrno D}, {\padrno H}, {\padrno L}),
%1/4, 1/2 and 3/4
not alluded to with lining figures
%on the baseline 
separated by a slash.

Two recent font technologies have attempted to address these issues: 
Apple's QuickDraw/{\sc gx} and Adobe/Micro\-soft's 
OpenType. Apple's effort is to be revived
in their nascent Mac~OS~X as {\sc atsui} (Apple Typographic System for
Unicode Information), while Microsoft's attempt
seems typical of work produced by committee. OpenType is notable
however, for having enlisted the aid of
Prof. Hermann Zapf in creating a new
version of Palatino to be distributed as the
first OpenType font. 
\vfill
\end

