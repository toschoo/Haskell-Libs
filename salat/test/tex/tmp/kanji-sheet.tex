%% Kanji writing practice sheets

%% Note: This is a ConTeXt file, to be compiled using
%% texexec --xtx sheet.tex

\setuplayout[top=1cm,header=0pt,bottom=1cm,footer=0pt]
%% Select the font to use here (end of line):
\definetypeface[base][rm][Xserif][STKaiti]
\setupbodyfont[base, 12pt]
\setuppagenumber[state=stop]
\setupindenting[none]

%% Change for different size:
\def\kanjisize{15mm}
\definefont[bigfont][Regular at \kanjisize]

\startuseMPgraphic{line}
  pickup pencircle scaled .7pt;
  picture c, box;
  c=\sometxt{\MPvar{c}};
  draw unitsquare scaled \kanjisize\space withcolor 0.25white;
  pickup pencircle scaled 0.5pt;
  draw .5[llcorner currentpicture,lrcorner currentpicture] --
    .5[ulcorner currentpicture,urcorner currentpicture]
    withcolor 0.5white;
  draw .5[llcorner currentpicture,ulcorner currentpicture] --
    .5[lrcorner currentpicture,urcorner currentpicture]
    withcolor 0.5white;
  box := currentpicture;
  setbounds box to boundingbox box enlarged -1pt;
  currentpicture := nullpicture;
  
  numeric imax;
  imax = 180mm/bbwidth(box);
  for i = 0 step 1 until imax:
    draw c shifted ((center box)-(center c) + (i*bbwidth(box), 0pt))
      withcolor (0.5+i/(imax+15))*white;
    draw box shifted (i*bbwidth(box), 0pt);
  endfor;
\stopuseMPgraphic

\def\line#1{{%
  \par
  \leavevmode
  \hskip-1.75cm
  \bigfont
  \useMPgraphic{line}{c=#1}%
  \par
}}

\starttext
有り難う – あ\cdot り\cdot がと\cdot う – Danke
\line{有}
\line{難}
下さい – くだ\cdot さい – Bitte (geben Sie mir)
\line{下}
お願いします – お\cdot ねが\cdot いします – Bitte (um etwas)
\line{願}
大丈夫 – だい\cdot じょう\cdot ぶ – Alles in Ordnung
\line{大}
\line{丈}
\line{夫}
駄目 – だ\cdot め – Schlecht, nicht in Ordnung
\line{駄}
\line{目}
乾杯 – かん\cdot ぱい – Prost
\line{乾}
\line{杯}
お腹いっぱい – お\cdot なか\cdot いっぱい – satt, nicht hungrig
\line{腹}

\stoptext

