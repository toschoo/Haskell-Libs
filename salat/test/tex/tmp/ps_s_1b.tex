% Christopher Allen
% callen@alum.dartmouth.org

\documentclass[letterpaper,10pt]{article}

\oddsidemargin=39pt
\evensidemargin=39pt
\marginparwidth=68pt
\textwidth=390pt

\usepackage{pstricks}
\usepackage{pst-coil}
\usepackage{pst-3d}
\usepackage{amstext}
\usepackage{amsmath}
\usepackage{amssymb}

\begin{document}

\center{\textbf{\LARGE{AP Physics B Problem Sheet: Spring \#1}}}\\

\begin{enumerate}

\item A long straight wire carries a constant current horizontally. A wire of diameter $d$ and resistivity $\rho$ is shaped into a rectangular loop with sides of lengths $l$ and $w$. A small battery of voltage $V$ is attached to the wire loop. The mass of
 the loop and battery together is $m$. This loop is placed so that the long straight wire is directly above its top by a height $h$ and so that they lie in the same plane, as shown below. When released, the loop remains suspended in mid-air.
\begin{center}
\begin{pspicture}(0,.3)(12,2)
% Loop
\psline(6.1,.2)(7.5,.2)(7.5,1.2)(7.5,1.2)(4.5,1.2)(4.5,.2)(5.9,.2)
% Battery
\psline[linewidth=0.4mm](5.9,.1)(5.9,.3)
\psline[linewidth=0.4mm](6.1,-.1)(6.1,.5)
% Labels
\rput(5.8,.5){$V$}
\psline[linestyle=dashed,dash=2pt 2pt]{<->}(4.3,.2)(4.3,1.2)
\rput(4.1,.7){$w$}
\psline[linestyle=dashed,dash=2pt 2pt]{<->}(4.3,2.2)(4.3,1.2)
\rput(4.1,1.7){$h$}
\psline[linestyle=dashed,dash=2pt 2pt]{<->}(4.5,1.4)(7.5,1.4)
\rput(6,1.6){$l$}
% Long straight wire
\psline(0,2.2)(12,2.2)
\end{pspicture}
\end{center}
\begin{enumerate}
\item How much current flows through the loop? (4 pts)
\item What is the direction of the magnetic field from the long straight wire at the loop's location? (1 pt)
\item In what direction does current flow through the long straight wire? (1 pts)
\item What is the magnitude of the current in the long straight wire? (4 pts)
\end{enumerate}

\item Two capacitor plates are set up to launch a proton from rest into a uniform magnetic field, as shown below. The voltage between the plates is 100 V. The magnetic field has a strength of 0.05 T and is directed into the page.
\begin{center}
\begin{pspicture}(0,.3)(4.4,3.9)
% Capacitor plates
\psline(.1,0)(.1,1.95)
\psline(.1,2.05)(.1,4)
\psline(1.1,0)(1.1,1.95)
\psline(1.1,2.05)(1.1,4)
% Magnetic field
\multirput(1.4,.5)(0,1){4}{\multirput(0,0)(1,0){4}{$\times$}}
% Proton
\psdots(0,2)
% Possible paths
\psline{->}(0.2,2)(1.1,2)
\psarc{->}(1.1,3.5){1.5}{-85}{20}
\rput(2.75,3.8){$a$}
\psarcn{->}(1.1,.5){1.5}{85}{-20}
\rput(2.75,.2){$b$}
\end{pspicture}
\end{center}
\begin{enumerate}
\item At what speed does the proton enter the magnetic field? (3 pts)
\item Will the proton follow path $a$ or path $b$? (1 pt)
\item What will the radius of this path be? (3 pts)
\item How long after it enters the magnetic field will the proton hit the back of the capacitor plate? (3 pts)
\end{enumerate}

\item Two long straight wires of length $L$ and mass $M$ hang side-by-side from very light strings of length $l$, with $l\ll L$. These wires each carry a current of the same magnitude. Each wire's strings make an angle $\theta$ to the vertical.
\begin{center}
\begin{pspicture}(0,-.45)(4.8,2.8)
\psset{viewpoint=2 -2 1}
% Backs of wires
\ThreeDput[normal=0 -1 0](0,5,0){\pscustom[fillstyle=solid,fillcolor=white]{\pscircle(0,0){.2}}}
\ThreeDput[normal=0 -1 0](0,5,0){\pscustom[fillstyle=solid,fillcolor=white]{\pscircle(2,0){.2}}}
% Sides of wires
\ThreeDput[normal=1 0 1](1.911,0,.179){\pscustom[fillstyle=solid,fillcolor=white]{\psline(5,-.4)(0,-.4)(0,0)(5,0)}}
\ThreeDput[normal=1 0 1](-.089,0,.179){\pscustom[fillstyle=solid,fillcolor=white]{\psline(5,-.4)(0,-.4)(0,0)(5,0)}}
% Fronts of wires
\ThreeDput[normal=0 -1 0]{\pscustom[fillstyle=solid,fillcolor=white]{\pscircle(0,0){.2}}}
\ThreeDput[normal=0 -1 0]{\pscustom[fillstyle=solid,fillcolor=white]{\pscircle(2,0){.2}}}
% Strings
\ThreeDput[normal=0 -1 0](0,.5,0){\psline(.089,.179)(1,2)(1.911,.179)}
\ThreeDput[normal=0 -1 0](0,4.5,0){\psline(.089,.179)(1,2)(1.911,.179)}
% Support
\ThreeDput[normal=1 0 0](1,0,0){\psline[linewidth=.6mm](0,2)(5,2)}
% Angles
\ThreeDput[normal=0 -1 0](0,4.5,0){\psline[linestyle=dashed](1,0)(1,1.8)}
\ThreeDput[normal=0 -1 0](0,4.5,0){\psarc(1,2){.7}{243.4}{296.6}}
\rput(3.65,1.7){$\theta$}
\rput(4.05,1.7){$\theta$}
% Wire length label
\ThreeDput[normal=1 0 0](2.8,0,0){\psline[linestyle=dashed]{<->}(5,0)}
\rput(4.2,-.2){$L$}
% String length label
\ThreeDput[normal=0 -1 0](0,-.5,0){\psline[linestyle=dashed]{<->}(1,2)}
\rput(-.3,.7){$l$}
\end{pspicture}
\end{center}
\begin{enumerate}
\item Using an end-on view, draw a free body diagram for each wire. (3 pts)
\item Find the magnitude of the magnetic force between the wires. (2 pts)
\item Do the currents run parallel or anti-parallel? (1 pt)
\item Find the magnitude of the current in the wires. (4 pts)
\end{enumerate}

\item Two resistors and an inductor are attached to a battery, as shown below. The switch is originally open. The switch is then closed and remains closed.
\begin{center}
\begin{pspicture}(0,.2)(8,1.8)
% Wires and switch
\psline(1,1)(1,1.8)(2.2,1.8)
\psline(1,.8)(1,0)(2.2,0)
\psline(2.8,1.8)(7,1.8)(7,1.2)
\psline(4,0)(4,.6)
\psline(4,1.2)(4,1.8)
\psline(2.29,.3)(2.8,0)(7,0)(7,.6)
\psdots(2.2,0)(2.8,0)(4,0)(4,1.8)
% Upper resistor
\pszigzag[coilarm=.01,coilwidth=.18,linewidth=0.4mm](2.2,1.8)(2.8,1.8)
\rput(2.5,1.5){4000 $\Omega$}
% Inductor
\pscoil[coilarm=.01,coilwidth=.18](7,.6)(7,1.2)
\rput(7.7,.9){3 mH}
% Middle resistor
\pszigzag[coilarm=.01,coilwidth=.18,linewidth=0.4mm](4,.6)(4,1.2)
\rput(4.8,.9){2000 $\Omega$}
% Battery
\psline[linewidth=0.4mm](.7,1)(1.3,1)
\psline[linewidth=0.4mm](.9,.8)(1.1,.8)
\rput(.2,.9){12 V}
\end{pspicture}
\end{center}
\begin{enumerate}
\item How much current flows through each resistor and the inductor immediately after the switch is closed? \mbox{(4 pts)}
\item How much current flows through each resistor and the inductor a long time later? \mbox{(4 pts)}
\item How much energy is stored in the inductor a long time later? \mbox{(2 pts)}
\end{enumerate}

\item An inclined plane is set up with two conductive rails running along its sides. The rails are electrically connected at the top of the incline. A bar, of mass $m$ and length $l$, slides down the frictionless rails, making electrical contact with them
. The net resistance of the circuit is $R$. The plane is inclined by an angle $\theta$ to the horizontal. There is a uniform magnetic field, of magnitude $B$, directed straight downward.
\begin{center}
\begin{pspicture}(0,-.1)(5,3.6)
\psset{viewpoint=2 -4 1}
% Inside base line of back
\ThreeDput[normal=1 0 0](.24,0,0){\psline(.3,0)(2.7,0)}
% Third triangle
\ThreeDput[normal=0 -1 0](0,2.7,0){\pscustom[fillstyle=solid,fillcolor=white]{\psline(.24,0)(.24,2.82)(4,0)(.24,0)}}
% Second triangle
\ThreeDput[normal=0 -1 0](0,.3,0){\pscustom[fillstyle=solid,fillcolor=white]{\psline(.24,0)(.24,2.82)(4,0)(.24,0)}}
% Front triangle
\ThreeDput[normal=0 -1 0]{\pscustom[fillstyle=solid,fillcolor=white]{\psline(0,0)(0,3)(4,0)(0,0)}}
% Top
\ThreeDput[normal=3 0 4](4,0,0){\pscustom[fillstyle=solid,fillcolor=white]{\psline(.3,0)(.3,4.7)(2.7,4.7)(2.7,0)(3,0)(3,5)(0,5)(0,0)(.3,0)}}
% Angle and its label
\ThreeDput[normal=0 -1 0]{\psarc(4,0){.75}{143}{180}}
\rput(2.7,0){$\theta$}
% Bar's front end
\ThreeDput[normal=0 -1 0]{\pscustom[fillstyle=solid,fillcolor=white]{\psline(2,1.5)(2.2,1.35)(2.35,1.55)(2.15,1.7)(2,1.5)}}
%Bar's front
\ThreeDput[normal=4 0 -3](2.2,0,1.35){\pscustom[fillstyle=solid,fillcolor=white]{\psline(0,0)(3,0)(3,.25)(0,.25)(0,0)}}
% Bar's top
\ThreeDput[normal=3 0 4](2.35,0,1.55){\pscustom[fillstyle=solid,fillcolor=white]{\psline(0,0)(3,0)(3,.25)(0,.25)(0,0)}}
% Velocity and its label
\ThreeDput[normal=0 -1 0](0,1.5,0){\psline{->}(2.475,1.3)(3.075,.85)}
\rput(3,.9){$\mathbf{v}$}
% Length and its label
\ThreeDput[normal=1 0 0](-.2,0,.15){\psline[linestyle=dashed]{<->}(0,3)(3,3)}
\rput(.4,3.6){$l$}
%Magnetic field and its label
\ThreeDput[normal=0 -1 0](0,1.5,0){\psline{->}(4.5,3.5)(4.5,2.5)}
\rput(5,2.85){$\mathbf{B}$}
\end{pspicture}
\end{center}
\begin{enumerate}
\item Using an end-on view, draw a free body diagram for the bar. (2 pts)
\item Find the current through the bar when it reaches terminal velocity. (4 pts)
\item Determine the terminal velocity of the bar. (4 pts)
\end{enumerate}

\end{enumerate}

\end{document}