\documentclass[a4paper]{article}
%\usepackage{bookman}
%\usepackage{extsizes}
\usepackage{geometry}
\geometry{hmargin={0mm,0mm}}
\geometry{vmargin={-17mm,-50mm}}
\usepackage{qtree}
\pagestyle{empty}
\usepackage{color}
\usepackage{chicago}
\usepackage{amssymb}
\usepackage{ulem}
\usepackage{gb4e}

\begin{document}
\pagecolor{black}
\hyphenation{non-semantic derivation}
\parindent=0in
\parskip=-1mm

\colorbox{white}{\parbox[b][3.5cm]{6cm}{\normalcolor{~}}}
\hspace{1cm}
\colorbox{white}{\parbox[b][3.5cm]{9cm}{{~}\\\centerline{\huge{\textsf{\textbf{On
          Having Arguments and}}}}
\\\centerline{\huge{\textsf{\textbf{Agreeing: Semantic EPP}}}}\\{~}}}
\hspace{1cm}
\colorbox{blue}{\parbox[b][3.5cm]{4cm}{~}}

\hspace{17.67cm}
\colorbox{blue}{\parbox[b][1cm]{4cm}{~}}

\colorbox{white}{\parbox[b][2cm]{6cm}{{~}}}
\hspace{1cm} 
\colorbox{red}{\parbox[b][2cm]{9cm}{\centerline{\Large{\textsf{Jonny
          Butler, University of York}}}\\
\centerline{\Large{\textsf{jrcb100@york.ac.uk}}}\\
\centerline{\Large{\textsf{www-users.york.ac.uk/{\~}jrcb100}}}\\{~}}}
\hspace{1cm}
\colorbox{blue}{\parbox[b][2cm]{4cm}{\textcolor{blue}{text}}}

\colorbox{white}{\parbox[b][1cm]{6cm}{{~}}}
          

\colorbox{white}{\parbox[b][3.5cm]{6cm}{{~}}}
\hspace{1cm}
\colorbox{white}{\parbox[b][3.5cm]{6cm}{\footnotesize{\textsf{\begin{quote}
          \textbf{So:}\glt [$\Lambda$] $\ldots$ [{\sc Id}]\glt
          \textcolor{white}{text}\hookrightarrow \glt
          \textbf{maps to} \glt \textcolor{white}{text}} \downarrow
          \glt $\lambda$ 
          $\ldots$ \texttt{x} \glt
          \textcolor{white}{tei}\hookrightarrow \glt\end{quote}}}}}
\hspace{1cm}
\colorbox{white}{\parbox[b][3.5cm]{6cm}{\footnotesize{\textsf{\begin{quote}
          V --- introduces $\theta$ (= [{\sc Id}]) \glt \textit{v}
          --- introduces [EPP] (= [$\Lambda$]) \glt [$\Lambda$] binds
          [{\sc Id}] (= predicate) \glt Predicate satisfied by
          DP (= argument) \glt = \textit{v}P \glt (derivation steps 1
          \& 2)
           \end{quote}}}}}

\colorbox{white}{\parbox[b][1cm]{6cm}{{~}}}

\colorbox{white}{\parbox[b][3.5cm]{6cm}{\footnotesize{\textsf{
          \begin{quote}\textbf{Introduction:} \glt `EPP-features are...\
          nonsemantic...\ though the configuration they establish has
          effects for interpretation' (Chomsky 2000)
          \glt But if they have semantic effects, why are they
          nonsemantic? \glt \textit{...\ What if ..?}\glt
          --- EPP-features \textit{are} semantic?\glt --- like this:\glt [EPP] =
          [$\Lambda$] = $\lambda$ \glt --- And they bind argument
          variables?\glt --- which are like this: \glt $\theta$ =
          [{\sc Id}] = \texttt{x}\glt (cf.\ Adger \& Ramchand
          2003)\glt Which is to say: \glt \textbf{EPP-features
          instantiate predication} (Williams 1980; Rothstein 1983;
          Heycock 1991; \r{A}farli \& Eide 2001)\glt --- \textbf{by
          means of predicate ($\lambda$) abstraction} (Heim \&
          Kratzer 1998; Nissenbaum 1998; Sauerland 1998)\glt ---
          \textbf{which is represented in the syntax by two features, [$\Lambda$]
          and [{\sc Id}]} (Adger \& Ramchand 2003).\end{quote}}}}}
\hspace{1cm}
\colorbox{yellow}{\parbox[b][3.5cm]{13.5cm}{\footnotesize\textsf{\begin{quote}
          T --- introduces [EPP] (= [$\Lambda$])\glt
          T$_{\mathrm{[\Lambda]}}$ forms a dependency with
          \textit{v}$_{\mathrm{[\Lambda]}}$ via AGREE --- so
          \glt T$_{\mathrm{[\Lambda]}}$ ends up abstracting over
          [{\sc Id}] too \glt DP is Remerged~~
          ($\longrightarrow$ INTERPRET EVERYTHING --- well, everything
          interpretable (derivation step 4) --- cf. Sportiche
          2002)\glt = TP \glt{~} 
          \end{quote}}}}\vspace{1cm} 

\colorbox{white}{\parbox[2cm]{21cm}{\footnotesize{\textsf{\begin{quote}
    \textbf{A Derivation:}\glt
    \qtreecenterfalse 
    \Tree [.\textit{v}P \textit{v}$_{\mathrm{[\Lambda]}}$ 
    \qroof{laugh$_{\mathrm{[ID]}}$}.VP ] 
    %\glt = $\lambda$.~laugh (\texttt{x})
    \Tree
    [.\textit{v}P Arthur [.\textit{v}$\acute$ \textit{v}$_{\mathrm{[\Lambda]}}$ 
    \qroof{laugh$_{\mathrm{[ID]}}$}.VP ]] 
    \Tree [.TP T$_{\mathrm{[\Lambda]}}$
    [.\textit{v}P Arthur [.\textit{v}$\acute$ \textit{v}$_{\mathrm{[\Lambda]}}$ 
    \qroof{laugh$_{\mathrm{[ID]}}$}.VP ]]]
    \Tree [.TP Arthur [.T$\acute$ T$_{\mathrm{[\Lambda]}}$
    [.\textit{v}P Arthur [.\textit{v}$\acute$ \textit{v}$_{\mathrm{[\Lambda]}}$ 
    \qroof{laugh$_{\mathrm{[ID]}}$}.VP ]]]]
\end{quote}}}}}

\colorbox{white}{\parbox[5mm]{21cm}{\footnotesize{\textsf{
          \textcolor{white}{textttttttttt}= $\lambda$.~laugh~(\texttt{x})
          \textcolor{white}{textt}
          =~$\lambda$.~[laugh~(\textit{x})]~(Arthur)
          \textcolor{white}{textttttt}
          =~$\lambda$.~Arthur~laugh~(\texttt{x})
          \textcolor{white}{textttttttttttttttt} =~$\lambda$.~[Arthur laugh
          (\texttt{x})]~(Arthur) }}}} 

\colorbox{white}{\parbox[5mm]{21cm}{\footnotesize{\textsf{
          \textcolor{white}{textttttttttttttttttttttttttitttttttt} =~Arthur laughs
          \textcolor{white}{texttttttttttttttttttttttittttttttttttttttttttttttttttttttttttttttt}
          =~Arthur is such that
          Arthur laughs}}}} 
% \colorbox{white}{\parbox[5cm]{4cm}{text}}
% \colorbox{white}{\parbox[5cm]{4cm}{text}}
% \colorbox{white}{\parbox[5cm]{4cm}{text}}
\vspace{1cm}

\colorbox{red}{\parbox[b][1.5cm]{6cm}{\textcolor{red}{text}}}
\hspace{1cm}
\colorbox{white}{\parbox[b][1.5cm]{6cm}{{~}}}
\hspace{1cm}
\colorbox{yellow}{\parbox[b][1.5cm]{6cm}{\textcolor{yellow}{~}}}}

\hspace{7.3cm}
\colorbox{white}{\parbox[b][1cm]{6cm}{{~}}}
\hspace{1cm}
\colorbox{yellow}{\parbox[b][1cm]{6cm}{\textcolor{yellow}{~}}}

\colorbox{white}{\parbox[b][3cm]{6cm}{\textcolor{white}{text}}}
\hspace{1cm}
\colorbox{white}{\parbox[b][3cm]{6cm}{\footnotesize{\textsf{\begin{quote}
         \textbf{No [$\Lambda$]?} Don't worry --- we have 
         many other binders to meet your needs:\glt
         GEN $\ldots$ [{\sc Id}] = PRO$_{\mathrm{ARB}}$\glt
         \textcolor{white}{tttttt}  \hookrightarrow\glt
         CONTROL $\ldots$ [{\sc Id}] = PRO$_{\mathrm{Control}}$\glt
         \textcolor{white}{ttttttttttttt}  \hookrightarrow \glt
         $\exists$ $\ldots$ [{\sc Id}] = Passive subject \glt
         \textcolor{white}{tt}  \hookrightarrow
         \glt{~} \glt{~}\glt{~}\glt{~} \end{quote}}}}}
\hspace{1cm}
\colorbox{yellow}{\parbox[b][3cm]{6cm}{\scriptsize{\textsf{\begin{quote}
          \textbf{References} Adger \& Ramchand 2003.\
          `Merge and Move: wh-dependencies revisited' ms; Chomsky
          2000.\ `Minimalist Inquiries' in \textit{Step by Step};
          Heim \& Kratzer 1998.\ \textit{Semantics in Generative
            Grammar}; Heycock 1991.\ \textit{Layers of Predication};
          Nissenbaum 1998.\ `Movement and derived predicates' MITWPL 25;
          Rothstein 1983.\ \textit{The Syntactic Forms of Predication};
          Sauerland 1998.\ \textit{The Meaning of Chains}; Sportiche
          2002.\ `Movement types and triggers' TiLT; Williams 1980.\
          `Predication' LI 11; \r{A}farli \& Eide 2001.\ `Predication at
          the Interface' ZASPiL 26\glt{~}\glt{~} \glt{~}\end{quote}}}}}



%\colorbox{yellow}{mondrian}


% \rule{21cm}{1cm}

% \parbox[t][5cm]{4cm}{~}
% \rule{1cm}{12cm}



\end{document}